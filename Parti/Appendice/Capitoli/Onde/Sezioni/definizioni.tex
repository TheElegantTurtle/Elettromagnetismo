\begin{defn}[Onda]
    Si definisce onda una perturbazione che nasce da una sorgente e si propaga nel tempo e
    nello spazio.
\end{defn}

\begin{obs}
    Un'onda di ampiezza costante è descritta da una funzione che goda della proprietà
    \[
        f(x,t)=f(\xi(x,t))\equiv f(\xi)
    \]
    con $\xi(x,t)=x\mp vt$, $v$ costante positiva.
\end{obs}
\begin{proof}
    Se si considera la funzione $f(\xi)$ questa ha un ben definito profilo che rappresenta la perturbazione
    generata dalla sorgente. Questo profilo corrisponde al profilo di $f(x)$, con $t$ fissato.
    Si consideri un certo valore $\xi_0=x_0\mp v t_0$. Ci si chiede per quale valore $x_0+\Delta x$
    all'istante $t_0+\Delta t$ si abbia ancora il valore $\xi_0$. Si deve quindi risolvere l'equazione
    $x_0 \mp v t_0 = (x_0+\Delta x) \mp v(t_0+\Delta t)$ che porta alla condizione $\Delta x \mp \Delta t=0$
    ovvero $\Delta x / \Delta t = \mp v$: $f(\xi(x,t))$ rappresenta dunque una perturbazione che viaggia nello
    spazio con velocità $v$.
\end{proof}
$\xi$ viene detto fase dell'onda, $v$ è la velocità con cui si muove la fase dell'onda: l'onda resta costante nel tempo
se la si osserva da un sistema di riferimento con velocità $v$. Per questo motivo $v$ è detta velocità di fase.
L'onda si dice positiva o regressiva
a seconda del segno che compare nell'espressione di $\xi$.

\begin{defn}[Fronte d'onda]
    Si definisce fronte d'onda il luogo dei punti in cui $\xi$ assume lo stesso valore.
\end{defn}
Si parla di  onde piane, sferiche etc... in base alla forma dei fronti d'onda.

\begin{defn}[Onde periodiche]
    Si parla di onda periodica quando si ha a che fare con un'onda descritta da una
    funzione periodica in $\xi$.
\end{defn}
Una classe particolarmente importante di onde periodiche sono le onde sinusoidali.
\begin{defn}[Onde sinusoidali]
    Si definiscono onde sinusoidali le onde periodiche in cui $f(\xi)$ assume una delle seguenti
    espressioni fra loro equivalenti
    \[
        \begin{split}
            & A\sin\Biggl[\frac{2\pi}{\lambda}(x-vt)+\phi\Biggr]\\
            & A\sin\Biggl[\frac{2\pi}{T}\Biggl(\frac{x}{v}-t\Biggr)+\phi\Biggr]\\
            & A\sin\Biggl[2\pi\Biggl(\frac{x}{\lambda}-\frac{t}{T}\Biggr)+\phi\Biggr]\\
            & A\sin(kx-\omega t + \phi)\\
        \end{split}
    \]
    dove:
    \begin{itemize}
            \item $A$: ampiezza
            \item $\phi$: fase iniziale
            \item $\lambda$: lunghezza d'onda, distanza fra due picchi
            \item T: periodo, tempo necessario affinchè nel punto $x$ fissato l'onda assuma nuovamente lo stesso valore
            \item $\omega$: pulsazione, $\omega=2\pi/T$
            \item $k$: numero d'onda, $\k=2\pi/\lambda$
            \item $v=\lambda/T=\omega/k$
    \end{itemize}
\end{defn}
