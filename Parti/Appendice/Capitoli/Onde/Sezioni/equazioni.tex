Le equazioni differenziali che danno come soluzione un'onda sono nella forma $\square\vb{F}=0$,
dove $\square$ è l'operatore dalembertiano
\begin{equation}
    \label{app:eqn:dalembertiano}
    \square=\laplacian - \rec{v^2}\pdv[2]{t}= \pdv[2]{x} + \pdv[2]{y} + \pdv[2]{z} - \rec{v^2}\pdv[2]{t}
\end{equation}
Chiaramente il dalemebertiano è un'operatore lineare e dunque le equazioni delle
onde sono equazioni differenziali lineari e omogenee. Vale perciò il principio di sovrapposizione:
ogni combinazione lineare di soluzioni dell'equazione delle onde è anch'essa una soluzione dell'equazione
delle onde.

Le soluzioni possono essere risolte sviluppando le soluzioni in serie di Fourier: per il principio di
sovrapposizione non si perde di generalità limitandosi a considerare solo soluzioni sinusoidali.
