I concetti esposti saranno formulati per concretezza nei conduttori metallici,
ma potranno essere estesi senza sforzo a qualsiasi altro tipo di conduttore.

Un conduttore metallico può essere visto come una struttura reticolare tridimensionale di atomi
con un gran numero di elettroni liberi. Per avere una stima di questo numero si consideri del rame,
con densità $\rho=8,9 g/cm^3$ e $A=63,5$ da cui si ricava immediatamente un valore di $8\,10^{22}$ elettroni liberi per centimetro cubo.
Le dimensioni degli elettroni sono molto più piccole delle sensibilità sperimentali oggi raggiunte.
Si possono vedere quindi gli elettroni liberi come un gas contenuto in un recipiente chiuso (il reticolo atomico).
Questi elettroni si muovono disordinatamente e urtano con gli ioni del reticolo portandosi
in equilibrio termico con questi ultimi. Applicando una differenza di potenziale si genera un campo elettrico.
L'effetto di questo campo è che un elettrone che dopo l'urto ha velocità $\vb{v}_T$ viene accelerato fino ad una velocità $\vb{v}'_T$.
Si ha
\[
\Delta\vb{v}=\frac{\vb{f}}{m}\Delta t=\frac{-e\vb{E}}{m}\Delta t
\]
dove l'intervallo di tempo è quello che intercorre fra due urti consecutivi. Complessivamente l'elettrone tra i due urti acquista una velocità di deriva
data dal valor medio di $\Delta\vb{v}$
\[
\vb{v}_d=\frac{\Delta\vb{v}}{2}=\Biggl(\frac{-e\Delta t}{2m} \Biggr)\vb{E}
\]
In linea di principio l'intervallo di tempo è dipendente dal campo elettrico.
In realtà si ha $v_d<<v_T$ (frazioni di millimetri al secondo contro centinaia di chilometri al secondo),
per cui $v_T\simeq v'_T$, ovvero la velocità termica non dipende sensibilmente dal campo elettrico.
Chiamando ora $l$ il libero cammino medio si ha $\Delta t\simeq l/v_T$ che non dipende quindi dal campo elettrico.
Ne segue che la velocità di deriva è costante e proporzionale al campo\footnote{In realtà la conduzione è un fenomeno sostanzialmente quantistico.
Analizzandolo però in ottica classica si può comunque avere un'idea qualitativa dei fenomeni che lo governano.}.

Come si è visto nei conduttori sono gli elettroni di conduzione le cariche che si muovono,
ma per ragioni storiche il fenomeno viene descritto dal punto di vista delle cariche positive fittizie che si muovono in direzione opposta.

\begin{defn} [Corrente elettrica]
Considerato un conduttore nel quale si abbia un movimento ordinato di
cariche, si definisce corrente elettrica
\[
I=\dv{Q}{t}
\]
\end{defn}
Nel sistema internazionale l'unità di misura è l'ampère.
