\begin{defn}[densità di corrente]
Si definisce densità di corrente, detto $n$ il numero di portatori di carica $q$ per unità di volume, il vettore
\[
\vb{J}=nq\vb{v}_d
\]
\end{defn}
Si ha che
\[
[J]=\rec{m^3}C\frac{m}{s}=\frac{C}{s}\rec{m^2}=\frac{A}{m^2}
\]
È importante osservare come la densità di corrente risulti sempre parallela al campo elettrico, infatti
le velcità di deriva costituiscono un campo vettoriale parallelo o antiparallelo ad $\vb{E}$ a seconda che
la carica sia positiva o negativa e di conseguenza il prodotto $q\vb{v}_d$ è sempre parallelo ad $\vb{E}$.
In via del tutto generale fra il campo elettrico e la densità di corrente sussiste una relazione del tipo $\vb{J}=\vb{F}(\vb{E})$.
Nel caso particolare dei conduttori detti \textit{lineari} questa relazione diventa
\begin{equation}
    \label{eqn:rel_J_E}
    \vb{J}=\norm{\sigma}\vb{E}
\end{equation}
dove $\norm{\sigma}$ è una matrice detta \textit{tensore di conducibilità} i cui elementi dipendono in linea generale
non dall'intensità del campo elettrico ma dalla sua direzione. I materiali per cui vale esplicitamente questa relazione
sono detti \textit{anisotropi}. Si dicono invece \textit{isotropi}
i materiali per cui questa relazione non dipende dalla direzione del campo elettrico.

L'introduzione del vettore polarizzazione è giustificata dal seguente risultato
\begin{thm}
    \label{thm:I_flusso_J}
    Data una sezione $S$ del conduttore attraversata da corrente si ha che
    \[
        I=\int_S\vb{J}\vdot \dd{\vb{S}}=\Phi_S(\vb{J})
    \]
\end{thm}
\begin{proof}
    Dato un conduttore di sezione $S$ al cui interno sia presente un campo elettrico, le velocità di deriva delle cariche libere $\vb{v}_d$
    costituiscono un campo vettoriale. Si consideri un tubo di flusso elementare di questo campo con sezione $\dd{\vb{S}}$.
    Allora nel tempo $\dd{t}$ si ha che la carica che attraversa il tubo di flusso è
    \[
        \dd{q}=nq\vb{v}_d\vdot\dd{\vb{S}}\dd{t}=\vb{J}\vdot\dd{\vb{S}}\dd{t}
    \]
    Dalla definizione si ha che la corrente elettrica in questa porzione infinitesima di conduttore vale
    \[
        \dd{I}=\dv{q}{t}=\vb{J}\vdot\dd{\vb{S}}
    \]
    Integrando entrabi i membri dell'equazione si ha la tesi.
\end{proof}

\begin{cor}[Equazione di continuità della corrente]
    \begin{equation}
        \label{eqn:continuità}
        \div{\vb{J}}+\pdv{\rho}{t}=0
    \end{equation}
    Che dal punto di vista fisico significa che se in un conduttore si ha una variazione di carica nel tempo,
    questa deve essere dovuta alla carica che fluisce attraverso la superficie che racchiude il volume.
\end{cor}
\begin{proof}
    Si consideri una superficie chiusa $S$ in un conduttore in cui all'istante $t$ sia racchiusa la carica $Q(t)$.
    Se dopo un certo tempo la superficie racchiude una carica $Q(t)-\dd{Q}$ per la conservazione della carica,
    questa deve essere fluita fuori dalla superficie. Quindi, per il teorema appena dimostrato e per il teorema della divergenza
    \[
        -\dv{Q}{t}=\int_S\vb{J}\vdot\dd{\vb{S}}=\int_V\div{\vb{J}}\dd{\tau}
    \]
    D'altra parte $Q(t)=\int_V \rho(x,y,z,t)\dd{\tau}$. È possibile passare con la derivata sotto al segno di integrale, ottenendo
    \[
        \dv{Q}{t}=\int_V\pdv{\rho}{t}\dd{\tau}
    \]
    Uguagliando le espressioni ottenute e ricordando che il ragionamento fatto vale indipendentemente dalla scelta del volume $V$,
    si ha l'uguaglianza degli integrandi, ovvero la tesi.
\end{proof}
