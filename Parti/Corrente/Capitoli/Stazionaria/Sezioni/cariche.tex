Con gli strumenti fin qui presentati è possibile studiare la configurazione delle cariche
in un conduttore percorso da corrente stazionaria.

La differenza principale fra un conduttore nel caso statico ed un conduttore percorso da
corrente è la migrazione sistematica delle cariche microscopiche con una velocità media
diversa da $0$. Di conseguenza all'interno del conduttore si ha un campo elettrico non nullo
e dunque un potenziale non uniforme: la superficie del conduttore non è equipotenziale
e dunque il campo elettrico non è ortogonale alla superficie.

\begin{thm}
Nel caso stazionario, in un conduttore ohmico omogeneo le cariche si dispongono solo sulla
superficie, ovvero la densità di carica volumica $\rho_c$ è $0$.
\end{thm}
\begin{proof}
    Inserendo l'equazione \eqref{eqn:rel_J_E} nella prima equazione di Maxwell\footnote{l'uso
    di $\epsilon$ al posto di $\epsilon_0$ è giustificato dal fatto che, sebbene nei metalli
    la polarizzazione sia $0$, neiconduttori non metallici questo può non essere vero} si ottiene
    $\rho_c=\epsilon\div{\vb{E}}=\epsilon\rho\div{\vb{J}}$, dove è stato possibile portare
    la resistività fuori dalla divergenza proprio per l'ipotesi di conduttore ohmico omogeneo (in
    questo caso infatti $\rho$ si riduce ad una costante).  $\div{\vb{J}}=0$ per
    l'equazione di continuità nel casos stazionario, dimostrando la tesi.
\end{proof}
L'ipotesi che il conduttore sia omogeneo è fondamentale, infatti qualora il conduttore presenti
ad esempio una giunzione fra materiali diversi, su quest'ultima su possono avere degli addensamenti
di carica. Inoltre, il fatto che la densità di carica di volume sia nulla non implica che la
corrente si localizzi sulla superficie: all'interno del conduttore si ha una corrente dovuta
al moto di deriva dei portatori e nonostante questo il bilancio netto fra cariche positive e
negative è nullo.

La densità di carica superficiale $\sigma_c$ è legata al campo elettrico interno al conduttore in
maniera complessa. Per vederlo si consideri un conduttore ohmico omogenero filiforme, non necessariamente
rettilineo nè di sezione costante. Si consideri anche un cilindretto a cavallo fra interno ed esterno del
conduttore, con le basi ortogonali alla superficie del conduttore e altezza infinitesima. Siccome
$\div{\vb{J}}=0$ il flusso di $\vb{J}$ attraverso il cilindretto deve essere nullo. Mandando a $0$ l'altezza
del cilindro, questo implica che la componente normale $J_n$ della densità di corrente si conserva, passando
dall'interno all'esterno del conduttore. È naturale ipotizzare che esternamente al conduttore non ci sia corrente
e da questo segue che anche internamente al conduttore $J_n=0$, ovvero $\vb{J}=J\vu{t}$.
Siccome il conduttore è ohmico, per la prima legge di Ohm la corrente è la stessa lungo tutto il filo e quindi
anche $\vb{J}$ non varia: $I=\Phi_S(\vb{J})=JS$ da cui segue $J=I/S$.
È possibile a questo punto per la \eqref{eqn:rel_J_E} calcolare il campo elettrico interno al conduttore
$\vb{E}=\rho I/S \vu{t}$. In assenza del conduttore il campo sarebbe determianto esclusivamente dalla
differenza di potenziale; introducendo il conduttore invece, il campo diventa tangente al filo e ne
segue quindi il percorso. Questa differenza di comportamento non può che essere dovuta alla
densità di carica superficiale, che somma il suo effetto a quello degli elettrodi che generano la ddp.
Il calcolo effettivo di $\sigma_c$ è complesso. Qualitativamente, se la forma del circuito è regolare,
si ha un addensamento di cariche positive vicino all'elettrodo positivo e di cariche negative vicino
all'elettrodo negativo.
