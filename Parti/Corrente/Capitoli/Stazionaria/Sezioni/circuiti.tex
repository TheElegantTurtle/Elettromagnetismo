In questo paragrafo si vuole sviluppare qualche primo elemento di teoria dei circuiti lineari,
ovvero costituiti da elementi che abbiano una caratteristica lineare.
Il discorso verrà sviluppato facendo riferiento al modello di \textit{circuito a costanti concentrate}:
il circuito viene considerato come costituito da elementi ideali concentrati in determinate zone
e collegati da linee a potenziale costante -ovvero, fili a resistenza nulla. Gli elementi circuitali
sono prevalentemente bipolari, quindi hanno solo due morsetti (uno d'entrata e uno d'uscita). Questi
elementi circuitali sono caratterizzati completamente una volta nota la relazione che impongono fra potenziale
e corrente ai loro capi.

Si vuole anzitutto generalizzare la prima legge di Ohm. Per farlo si osservi inzialmente come
il campo elettrico totale $\vb{E}$ è dato dalla somma fra il campo $\vb{E}_s$ che si instaura fra
i morsetti dei generatori, il campo elettromotore $\vb{E}_e$ ed eventuali campi contro-elettromotori
dovuti ad eventuali carichi non ohmici $\vb{E}_{ce}$.
\begin{thm}[legge di Ohm generalizzata]
    Su un ramo del circuito vale la seguente relazione
    \[
        (V_A-V_B)+\sum f_i+\sum f_{ci}=R_{AB}I
    \]
    con $R_{AB}$ la resistenza totale del ramo.
\end{thm}
\begin{proof}
    La legge di Ohm locale correla il
    campo totale con la densità di corrente $\vb{E}=\rho\vb{J}$. Si ottiene quindi l'equazione
    \[
        \vb{E}_s+\vb{E}_e+\vb{E}_{ce}=\rho\vb{J}
    \]
    Integrandola lungo un ramo del circuito che collega i punti $A$ e $B$, le tre componenti del campo
    danno
    \[
        \begin{split}
            &\int_A^B \vb{E}_s\vdot \dd{\vb{l}}=V_A-V_B \\
            &\int_A^B \vb{E}_e\vdot \dd{\vb{l}}=\sum f_i \\
            &\int_A^B \vb{E}_e\vdot \dd{\vb{l}}=\sum f_{ci} \\
        \end{split}
    \]
    dove le forze elettromotrici sul ramo sono prese positive se il generatore, agendo da solo, tenderebbe
    a spostare le cariche nel verso scelto come positivo; per le forze elettromotrici è vero il contrario.
    Si ottiene quindi
    \[
        (V_A-V_B)+\sum f_i+\sum f_{ci}=\int_A^B \rho\vb{J}\vdot\dd{\vb{l}}
    \]
    Supponendo che le sezioni $S$ del ramo fra $A$ e $B$ siano sufficientemente piccole affinchè
    $\vb{J}$ sia costante su ciascuna sezione si ha $I=\Phi_S(\vb{J})=JS$. Ricordando che la corrente
    in ognu punto del ramo ha lo stesso valore, l'integrale a secondo membro diventa
    \[
        \int_A^B\rho\vb{J}\vdot\dd{\vb{l}}=\int_A^B\rho\frac{I}{S}\dd{\vb{l}}=I\int_A^B\rho\frac{\dd{l}}{s}=IR_{AB}
    \]
    L'ultima uguaglianza è giustificata dal fatto che se $\rho$ fosse costante su tutto il ramo si otterrebbe
    prorpio la seconda legge di Ohm.
\end{proof}
Il seguente corollario ha dimostrazione immediata
\begin{cor}
In assenza di carichi non ohmici la legge di Ohm assume la forma
    \[
        (V_A-V_B)+\sum f_i=I\sum R_i
    \]
\end{cor}
L'andamento del potenziale lungo il ramo è quindi dato da una sequenza di tratti lineari in cui il passaggio da un capo all'altro
della resistenza $R_i$\footnote{Questa resistenza può anche essere la resistenza interna ad un generatore},
nel verso della corrente, porta ad una caduta di potenziale $IR_i$, mentre il passaggio da un capo all'altro di un generatore
porta ad un salto di potenziale positivo o negativo a seconda del verso nel quale il generatore spinge la corrente.
