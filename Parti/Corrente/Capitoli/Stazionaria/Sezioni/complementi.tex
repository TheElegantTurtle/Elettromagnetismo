\label{par:corrente_correntestazionaria_complementi}
L'obiettivo di questo paragrafo è mettere in evidenza alcuni aspetti delicati riguardanti l'applicazione
del metodo dei lavori virtuali e studiare come si comporta la forza fra le maglie di un condensatore.
A tal proposito, si condiseri un condensatore con capacità $C=\epsilon_0 S/x$, con $x$ la distanza fra le armature, ed
energia elettrostatica $U_C$. Usando il metodo dei lavori virtuali, ovvero immaginando di spostare le armature di una
distanza infinitesima $\delta x$, si ricava che la forza tra le armature vale
\[
R_x=\pdv{U_C}{x}
\]

Dalla \ref{eqn:U_condensatore}  si hanno due possibili espressioni per $U_C$
\[
U_C=\rec{2} \frac{Q^2}{C}=\rec{2} C (\Delta V)^2
\]

Derivando la prima di queste a $Q$ costante, ovvero consideando il condensatore isolato, si ottiene
\[
R_x=-\rec{2}\frac{Q^2}{\epsilon_0 S}
\]
che essendo negativa rappresenta una forza attrattiva.

Derivando la seconda espressione a $\Delta V=fem$ costante, che equivale a considerare il condensatore
collegato ad un generatore di forza elettromotrice costante, si ha
\[
R_x=\rec{2}\frac{\epsilon_0 S (\Delta V)^2}{x^2}=\rec{2}\frac{C^2 (\Delta V)^2}{\epsilon_0 S}=\rec{2}\frac{Q^2}{\epsilon_0 S}
\]
che ha lo stesso modulo ma verso opposto a quella già determinta. Sperimentalmente inoltre si verifica che
è la forza attrattiva ad essere quella fisicamente corretta.
Per comprendere l'origine di questa incongruenza, bisogna affidarsi ad alcune considerazioni di carattere termodinamico.
In questo caso prima di applicare il metodo dei lavori virtuali ci si trova in una situazione di equilibro, ovvero di minimo
per l'energia. Lo spostamento infinitesimo porta il sistema fuori dall'equilibrio e quindi ad un aumento dell'energia:
ma questo è proibito dal secondo principio della termodinamica. Condizione necessaria per poter usare il metodo dei
lavori virtuali è quindi considerare tutto il sistema suscettibile a variazioni di energia e includere fra queste ultime
anche le variazioni di calore. Nel caso del condensatore isolato questa condizione è soddisfatta, mentre non è così
quando si considera il condensatore legato ad un generatore. Si condieri infatti la forza esterna $F_e$ che allontana le
maglie del condensatore della quantità $\delta x$. Questa deve essere appena maggiore della forza elettrostatica $R_x$
e di segno opposto. L'allontanamento delle armature comporta una diminuzione della capacità ($\delta C<0$) e quindi dell'energia
elettrostatica pari a
\[
\delta U_C =\delta\Biggl[\rec{2}C(x) fem^2\Biggr]=\rec{2} \delta C(x) fem^2<0
\]
%\rec{2}\dv{C(x)}{x}\delta x fem^2=\rec{2}\Biggl(-\frac{\epsilon_0 S}{x^2} \Biggr)\delta x fem^2

Siccome $\Delta V=Ex$, se le armature si allontanano ($x$ aumenta) $E$ deve diminuire al fine di tenere costante la
differenza di potenziale. Ma allora deve diminuire la carica sulle armature: in pratica una quantità $\delta Q$
si sposta dall'armatura positiva a quella negativa passando per il generatore in verso opposto al campo elettromotore,
compiendo un lavoro $\delta U_g=-fem \delta Q$. In contemporanea alla diminuzione di energia nel condensatore
si ha un aumento di energia nel generatore, entrambe dovute al lavoro delle forze esterne. Ma siccome
$\delta Q=\Delta V \delta C=fem \Delta C$ si ha
\[
\delta U_g=-fem^2 \delta C=-2 U_C>0
\]
In conclusione $\delta L_e=F_e\delta x \simeq -R_x\delta x= \delta U_C +\delta U_g=-\delta U_C$, ovvero
\[
R_x=\pdv{U_C}{x}=--\rec{2}\frac{\epsilon_0 S (\Delta V)^2}{x^2}=-\rec{2}\frac{C^2 (\Delta V)^2}{\epsilon_0 S}=-\rec{2}\frac{Q^2}{\epsilon_0 S}
\]
che è proprio l'espressione corretta.
