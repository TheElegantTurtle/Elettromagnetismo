Per quanto visto nella sezione precedente, il passaggio di corrente in un conduttore implica un trasferimento di energia.
È necessario allora, affichè la corrente sia stazionaria, che sia presente un generatore per fornire con continuità l'energia persa.
\begin{obs}
    Il campo totale $\vb{E}$ presente in un circuito dotato di generatore non è conservativo.
\end{obs}
\begin{proof}
    Per la prima legge di Kirchoff in tutto il circuito compreso di generatore il flusso di cariche è costante.
    Ogni punto della linea chiusa che costituisce il circuito può infatti essere assimilato ad un nodo raggiunto da due soli rami.
    Siccome nel circuito avvengono fenomeni come l'effetto Joule che riducono l'energia delle cariche in moto,
    è allora necessario che il lavoro che viene trasferito dall'esterno sui portatori di carica non sia nullo.
    Dividendo il lavoro per la carica totale:
    \[
        \oint \vb{E}\vdot\dd{\vb{l}}\neq 0
    \]
\end{proof}
Siccome il campo elettrico che mette in moto le cariche è conservativo,
è necessario che vi siano delle azioni
i cui effetti siano assimilabili a quelli di un campo elettrico non conservativo.
Si può pensare inizialmente che tali azioni siano localizzate all'interno del generatore.
Si indicherà con $\vb{E}_s$ il campo che mette in moto i portatori di carica, presente sia all'esterno che
all'interno del generatore,
e con $\vb{E}_e$ il campo non conservativo interno al generatore -quest'ultimo viene chiamato \textit{campo elettromotore}.
Si può immaginare il generatore come una zona ben definita e localizzata del circuito che a livello pratico,
è caratterizzato da due morsetti su cui si accumulano le cariche che generano il campo elettrico $\vb{E}_s$:
l'anodo, caricato positivamente, ed il catodo, caricato negativamente.
Compito del generatore è ripristinare continuamente gli accumuli di carica sui morsetti in modo da
mantenere un flusso di cariche nel circuito costante nel tempo.
In questo caso quindi il campo $\vb{E}_e$ spinge le cariche negative verso il catodo e le cariche positive verso l'anodo.
Si ha quindi che i versi di $\vb{E}_s$ ed $\vb{E}_e$ sono opposti.

\begin{defn}[Forza elettromotrice\footnote{È importante sottolineare come,
    nonstante il nome, la forza elettromotrice abbia le dimensioni di un potenziale} con generatore localizzato]
    Si definisce forza elettromotrice
    il lavoro che il campo $\vb{E}_e$ compie per postare la carica positiva unitaria dal catodo all'anodo.
    \[
        fem=\int_B^A \vb{E}_e \vdot \dd{\vb{l}}
    \]
\end{defn}

\begin{thm}
    La forza elettromotrice coincide con la differenza di potenziale presente fra anodo e catodo a circuito aperto.
\end{thm}
\begin{proof}
    Si consideri un generatore collegato ad un circuito aperto. In tale situazione $\vb{E}_e$
    accumula cariche positive e negative su anodo e catodo, le quali generano il campo $\vb{E}_s$ contrario al primo
    via via crescente finchè non si raggiunge la situazione di equilibrio $\vb{E}_e=-\vb{E}_s$.
    Si ha quindi, indicando con $A$ la posizione dell'anodo e con $B$ la posizione del catodo:
    \[
        fem=\int_B^A\vb{E}_e\vdot \dd{\vb{l}}=\int_B^A-\vb{E}_s\vdot \dd{\vb{l}}=\int_A^B\vb{E}_s\vdot \dd{\vb{l}}
    \]
    Siccome il campo $\vb{E}_s$ è conservativo, l'ultimo termine corrisponde per definizione alla differenza di potenziale fra $A$ e $B$
\end{proof}
Questo risultato permette agevolmente di generalizzare la definizione di forza elettromotrice al caso in cui il
campo elettromotore non sia localizzato.
\begin{defn}[forza elettromotrice con generatore non localizzato]
    Indicato con $\vb{E}$ il campo totale, somma di $\vb{E}_e$ ed $\vb{E}_s$
    \[
        fem=\oint \vb{E} \vdot \dd{\vb{l}}
    \]
\end{defn}
Il fatto che questa definizione sia una generalizzazione della precedente è evidente.
Infatti, il campo $\vb{E}_s$ è conservativo e quindi il suo integrale su tutto il circuito è nullo;
inoltre il campo $\vb{E}_e$ è nullo su tutto il circuito ad eccezione della zona in cui è presente il generatore.

Ci si chiede ora come vari la differenza di potenziale ai capi del generatore nel momento
in cui si chiude il circuito ed iniza a circolare corrente. Di fatto, questa cambia e le cause sono due:
\begin{enumerate}
    \item La connessione elettrica interna al generatore ha resistenza non nulla e questo, per la prima legge di Ohm, implica una riduzione del potenziale;
    \item Il campo elettromotore varia con la corrente circolante nel generatore.
\end{enumerate}
Questi due effetti vengono studiati mediante la \textit{curva caratteristica del generatore},
ovvero il grafico della differenza di potenziale interna al generatore in funzione della corrente circolante.
Per $I=0$ si ha ovviamente $\Delta V=fem$, segue poi un tratto approssimativamente lineare
e infine la differenza di potenziale decresce più velocemente di un polinomio del primo ordine.
Per le applicazioni pratiche ci si limita spesso al tratto lineare dove vale un'equazione del tipo
\begin{equation}
    \label{eqn:curva_caratteristica}
    \Delta V=fem - rI
\end{equation}
$r$ ha le dimensioni di una resistenza.
Si chiama \textit{generatore ideale} un generatore la cui differenza di potenziale sia sempre uguale a $fem$,
anche quando il circuito viene chiuso.
\begin{obs}
    Nella zona lineare della curva caratteristica, il circuito è equivalente ad un circuito con un generatore ideale messo in serie ad una resistenza $R=r$
\end{obs}
\begin{proof}
    Si consideri un circuito dotato di generatore a cui sia collegata una resistenza $R'$.
    Per la prima legge di Ohm e per la \eqref{eqn:curva_caratteristica} si ha
    \[
        \Delta V=fem-Ir=R'I
    \]
    da cui segue che
    \[
        fem=(r+R')I
    \]
    Ovvero, il circuito si comporta come un circuito in cui la differenza di poteziale del generatore sia pari alla $fem$,
    nel quale però oltre alla resistenza $R'$ sia presente in serie anche una resistenza $r$.
\end{proof}

\begin{thm}
    La potenza erogata dal generatore vale
    \begin{equation}
        W_g=Ifem= I\Delta V+W^{(d)}
    \end{equation}
    dove $W^{(d)}$ è la potenza dissipata all'interno del generatore
\end{thm}
\begin{proof}
    Per la definizione di forza elettromotrice il lavoro elementare fatto dal generatore quando è attraversato da carica $\dd{q}=I\dd{t}$ è
    \[
        \dd{L_g}=fem\,\dd{q}=fem\,I\dd{t}
    \]
    e quindi la potenza vale
    \[
        W_g=\dv{L_g}{t}=fem\,I
    \]
    Inoltre per la conservazione dell'energia questa potenza deve essere pari alla potenza trasferita dal generatore
    alla corrente più la potenza dissipata dal generatore. Facendo riferimento alla \eqref{eqn:potenza_stazionaria} si ha la tesi.
\end{proof}
\begin{cor}
    La differenza di poteziale ai capi di un generatore reale vale
    \begin{equation}
        \Delta V=fem-\frac{W^{(d)}}{I}
    \end{equation}
\end{cor}
\begin{proof}
    La dimostrazione è immediata: basta dividere per I il risultato precedente.
\end{proof}
L'importanza di questo corollario è mostrare di quanto effettivamente venga ridotta la differenza
di potenziale ai capi del generatore una volta chiuso il circuito. A circuito aperto, con $I=0$,
si deve riottenere $\Delta V=fem$: questo implica che la potenza dissipata tende a 0 col tendere a 0
della corrente più velocemente di quanto non faccia la corrente stessa.
Quanto visto finora ci permette di affermare che per qualificare completamente il comportamento di un generatore
è sufficiente la conoscenza di $fem$ ed $r$, entrambe determinabili facendo misure all'esterno del generatore:
la $fem$ come differenza di potenziale a circuito aperto; la $r$,
una volta nota $fem$ si ottiene misurando la corrente che passa nel circuito con una resistenza nota $R'$.


Nel caso di carichi sul circuito di natura non ohmica, la conversione dell'energia elettrica
avviene in favore di forme di energia diverse dal calore. Qualsiasi sia la forma di energia,
questa azione realizza sempre una sottrazione di energia alla corrente circolante, che è schematizzabile con dei campi elettrici
in verso opposto a quello della corrente. Questi campi elettrici, solitamente localizzati nei carichi,
vengono detti \textit{campi controelettromotori} $\vb{E}_{ce}$.
\begin{defn}[Forza controelettromotrice]
    Si definisce forza controelettromotrice
    \[
        f_c=\int_D^C \vb{E}_{ce}\vdot\dd{\vb{l}}
    \]
    dove $C$ e $D$ sono i terminali del carico non ohmico.
\end{defn}
\begin{obs}
    L'energia elettrica totale fornita al carico è uguale all'energia trasformata dalla forza controelettromotrice
    aumentata dell'energia dissipata sottoforma di calore dal carico stesso.
\end{obs}
