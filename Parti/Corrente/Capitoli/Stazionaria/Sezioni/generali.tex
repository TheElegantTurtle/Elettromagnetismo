\begin{defn}[Corrente stazionaria]
    Si dice che un conduttore è in regime di corrente stazionaria quando la corrente è costante nel tempo.
\end{defn}

Il passaggio di corrente elettrica implica uno spostamento delle cariche ad opera del campo elettrico che compie quindi un lavoro.
\begin{thm}
    La potenza sviluppata da un campo elettrico che induce una corrente stazionaria $I$ in un tratto di conduttore compreso fra due punti $A$ e $B$ è
    \begin{equation}
        \label{eqn:potenza_stazionaria}
        W=I(V_A-V_B)
    \end{equation}
\end{thm}
\begin{proof}
    (???????)
    Si supponga un volume di conduttore cilindrico con basi poste nei punti $A$ e $B$
    soggetto ad un potenziale costante nel tempo. Nell'intervallo di tempo $\dd{t}$ le cariche
    si spostano di un tratto $\dd{l}$ in modo tale che le cariche che inizialmente occupavano lo spazio $AB$ ora occupano
    lo spazio $A'B'$ ottenuto traslando $AB$ di $\dd{l}$. Dalla definizione di corrente si ha che questo equivale a spostare
    la carica $\dd{q}=I\dd{t}$ dal punto a potenziale $V_A$ al punto a potenziale $V_B$. Nel fare questo il campo compie un lavoro
    \[
        \dd{L}=\dd{q}(V_A-V_B)=I\dd{t}(V_A-V_B)
    \]
    Da cui si ha immediatamente la tesi.
\end{proof}

\begin{thm}
    Nel caso stazionario, data una qualunque superficie chiusa $S$, il campo $\vb{J}$ è solenoidale su tale superficie.
\end{thm}
\begin{proof}
    La dimotrazione segue immediatamente dall'equazione \eqref{eqn:continuità}
    osservando che per definizione di stazionarietà $\rho$ non può variare nel tempo.
\end{proof}
