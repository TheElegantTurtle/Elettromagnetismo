Dalla  \eqref{eqn:potenza_stazionaria} emerge che la presenza di un campo elettrico che metta in moto
le cariche libere di un conduttore implica la produzione di energia.
Nel caso stazionario, questa energia non può essere nè energia cinetica nè energia potenziale.
L'unica possibilità è quindi che si trasformi in energia termica, ovvero moto di agitazione disordinato.
Questa potenza viene dissipata nell'ambiente sottoforma di calore o luce e il fenomeno prende il nome di \textit{effetto Joule}.
\begin{thm}[legge di Joule]
    In un conduttore ohmico si ha che
    \begin{equation}
        \label{eqn:effetto_joule}
        W=RI^2
    \end{equation}
\end{thm}
\begin{proof}
    Dalla \eqref{eqn:potenza_stazionaria} e dalla seconda legge di Ohm si ha immediatamente la tesi
    \[
        W=I\Delta V=I (RI)
    \]
\end{proof}

\begin{thm}[Legge di Joule locale]
    Si ha che la potenza trasferita dal campo elettrico alla corrente nell'unità di volume è
    \begin{equation}
        w=\vb{E}\vdot\vb{J} \quad\quad\quad w=\frac{\dd{L}}{\dd{\tau}\dd{t}}
    \end{equation}
\end{thm}
\begin{proof}
    Nella porzione di volume elementare del conduttore $\dd{\tau}$ sono presenti $n\dd{\tau}$
    portatori con carica $q$. Indicando con $\dd{\vb{l}}=\vb{v}_d\dd{t}$
    lo spostamento medio dei portatori di carica sottoposti al campo nel tempo $\dd{t}$,
    si ha che il lavoro infinitesimo è
    \[
        \dd{L}=\vb{F}\vdot\dd{\vb{l}}=n\dd{\tau}q\vb{E}\vdot \dd{\vb{l}}=\vb{E}\vdot (nq\vb{v}_d)\dd{t}\dd{\tau}
    \]
    da cui la tesi.
\end{proof}
