\begin{obses}[Prima legge di Ohm]
    Per determinati materiali e per un ampio intervallo di valori di $\Delta V$\footnote{Ovvero,
    finchè la differenza di potenziale non è tale da generare scariche che danneggiano il materiale}
    sussiste una relazione di proporzionalità diretta fra la differenza di potenziale e la corrente
    che fluisce fra due punti $A$ e $B$ di un conduttore
    \begin{equation}
        V_A-V_B=\Delta V=RI
    \end{equation}
\end{obses}
La costante di proporzionalità si chiama \textit{resistenza}. L'unità di misura della resistenza è l'\textit{ohm} $\Omega=V/A$
Un materiale che rispetta la legge di ohm è detto \textit{ohmico}. Tutti i conduttori ohmici sono omogenei ed isotropi.
È evidente come la caratteristica, ovvero la curva $I(\Delta V)$ di un conduttore ohmico sia una retta con pendenza $1/R=G$, detta \textit{conduttanza}.
La resistenza dipende sia dalle condizioni fisiche in cui è posto un materiale che dalla sua geometria.

Si riportano i due seguenti risultati che permettono di trattare resistenze in serie e in parallelo,
di cui viene omessa la dimostrazione perchè analoga a quella riportata per i condensatori.
\begin{thm}[Resistenze in serie]
    Dati $n$ conduttori collegati in serie la resistenza del sistema è equivalente a quella di un unico conduttore con resistenza
    \[
        R=\sum_{i=1}^n R_i
    \]
\end{thm}
\begin{thm}[Resistenze in parallelo]
    Dati $n$ conduttori collegati in parallelo la resistenza del sistema è equivalente a quella di un unico conduttore con resistenza
    \[
        \frac{1}{R}=\sum_{i=1}^n\frac{1}{R_i}
    \]
\end{thm}


\begin{obses}[Seconda legge di Ohm]
    Per un tratto di conduttore di lunghezza $l$, omogeneo e a sezione $S$ costante vale la relazione
    \begin{equation}
        R=\rho\frac{l}{S}=\rec{\sigma}\frac{l}{S}
    \end{equation}
\end{obses}
Dove $\rho$ è detto \textit{resistività elettrica} $\sigma$ è detto \textit{conducibilità elettrica} e non vanno confusi
con le densità di carica.

Con quanto appena introdotto si può dimostrare il seguente risultato.
\begin{thm}[Legge di Ohm locale]
    In un conduttore lineare ed isotropo sussiste la seguente relazione
    \begin{equation}
        \vb{J}=\sigma\vb{E}
    \end{equation}
\end{thm}
\begin{proof}
    Si consideri all'interno di un conduttore un cilindretto di lunghezza $\dd{l}$ e superficie $\dd{S}$,
    dove le quantità possono essere studiate in forma scalare per via dell'ipotesi di isotropia.
    Per la prima e la seconda legge di Ohm si ha che
    \[
        \dd{V}=E\dd{l}=R\dd{I}=\rec{\sigma}\frac{\dd{l}}{\dd{S}}J\dd{S}
    \]
    Da cui si ricava che
    \[
        E\dd{l}=\rec{\sigma}\dd{l}J
    \]
    da cui si ottiene la tesi in forma scalare. La forma vettoriale è ottenibile banalmente.
\end{proof}
Quanto ottenuto è di fatto un caso particolare della relazione \eqref{eqn:rel_J_E}, che mostra come
i conduttori ohmici siano isotropi. La legge di
Ohm locale è più generale della legge di Ohm che invece è riferita a porzioni estese di conduttore e può
essere sfruttata per calcolare la resistenza di conduttori ohmici con geometrie complesse.
Inoltre, può essere usata per studiare dielettrici non perfetti, grazie al seguente risultato.
\begin{cor}
    Dato un condensatore, le cui maglie siano conduttori ohmici, riempito con un dielettrico
    non perfettamente isolante di resistività $\rho$ e costante dielettrica $\epsilon$,
    capacità e resistenza del condensatore sono legate dalla relazione
    \[
        CR=\rho\epsilon
    \]
\end{cor}
\begin{proof}
    Si considerino due conduttori ohmici, $1$ e $2$, rispettivamente a potenziale $V_1$ e $V_2$.
    La configurazione del campo elettrico è univocamente determinata grazie al teorema di unicità.
    Sia $S$ una superficie chiusa che intercetti tutte le linee di forza del campo elettrico. Si vuole
    calcolare il flusso del campo nel caso in cui: lo spazio fra le maglie del condensatore sia interamente
    riempito con un conduttore omogeneo; lo spazio sia riempito con un dielettrico omogeneo.
    Nel primo caso, per la legge di Ohm locale $\Phi_S(\vb{E})=\rho\Phi_S(\vb{J})=\rho I=\rho\Delta V/R$,
    dove si è tenuto conto della definizione di $I$ e della prima legge di Ohm.
    Nel secondo caso invece per il teorema di Gauss e la definizione di capacità
    $\Phi_S(\vb{E})=Q/\epsilon=C\Delta V/\epsilon$. Uguagliando le due equazioni si ottiene la relazione
    fra la capacità del condensatore riempito col dielettrico e la resistenza che si presenta fra le due
    armature qualora il condensatore fosse riempito con un conduttore. Dato che un dielettrico non perfettamente
    isolante può essere visto come un conduttore, segue la tesi.
\end{proof}

La resistenza di un conduttore dipende dalla geometria del materiale ma anche dalle condizioni fisiche e
in particolare dalla temperatura. Questa dipendenza intorno alla temperatura ambiente può essere
espressa sviluppando al primo ordinte l'andamento di $\rho$:
\[
    \rho(t)=\rho_0(1+\alpha t)
\]
con $t$ la temperatura in gradi celsius e $\alpha$ un coefficiente detto \textit{coefficiente di temperatura}, che assume
valori positivi per i metalli e negativi per gli elettroliti e alcuni materiali non metallici come il carbonio.
