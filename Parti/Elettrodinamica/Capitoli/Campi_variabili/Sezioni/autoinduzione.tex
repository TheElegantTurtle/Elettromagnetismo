L'analisi proposta in questo e nel prossimo paragrafo è fatta per materiali omogenei ed isotropi in cui $\mu$ sia costante.

Si consideri un circuito in condizioni quasi-stazionarie, il che implica che la corrente abbia
lo stesso valore lungo tutto il circuito, percorso da una corrente variabile nel tempo $I(t)$.
Questa corrente genera nello spazio circostante un campo di induzione magnetica $\vb{B}(t)$ il cui
flusso concatenato al circuito è in generale diverso da zero e variabile.
Per la legge di Faraday-Neumann si genera nel circuito una forza elettromotrice detta \textit{autoindotta}.
\begin{obs}
    \begin{equation}
        \label{eqn:induttanza}
        \Phi(\vb{B})=LI
    \end{equation}
\end{obs}
\begin{proof}
    per la \eqref{eqn:dB} il campo di induzione magnetica è proporzionale alla corrente che percorre il circuito.
    Poichè il flusso elementare di $\vb{B}$ attraverso ogni elemento di superficie $\dd{\vb{S}}$ è
    proporzionale a $\vb{B}$, anche il flusso concatenato deve essere proporzionale ad $I$.
\end{proof}
\begin{defn}[Induttanza]
    La costante di proporzionalità che compare nell'osservazione precedente si chiama induttanza.
\end{defn}
L'unità di misura per l'induttanza è detta \textit{henry}
\[
    [L]=\frac{w}{A}=\frac{V \vdot s}{A}=\Omega \vdot s = H
\]

\begin{obs}
    L'induttanza per un solenoide nell'approssimazione di solenoide infinito vale
    \[
        L=\mu n^2 lS=\mu N^2 \frac{S}{l}
    \]
\end{obs}
\begin{proof}
    Il campo di induzione magnetica per un solenoide vale in modulo
    \[
        B=\mu nI
    \]
    ed è diretto assialmente. Il flusso concatenato ad una singola spira è $=SB$
    e quindi il flusso concatenato a tutto il circuito vale $\Phi(\vb{B})=NS\mu nI$.
    Ma allora l'induttanza vale
    \[
        L=\frac{\Phi(\vb{B})}{I}=\mu n^2 lS=\mu N^2 \frac{S}{l}
    \]
\end{proof}
Nelle ipotesi\footnote{Quasi-stazionarietà e materiale omogeneo e isotropo}, dalla legge di Faraday-Neumann si ha immediatamente
l'espressione per la forza elettromotrice autoindotta: $f_a=-L\dd{I}/\dd{t}$.
Se il circuito con resistenza $R$ è sede di una corrente variabile nel tempo, l'equazione del circuito è
\begin{equation}
    \label{eqn:RL}
    RI=f+f_a=f-L\dv{I}{t}
\end{equation}
Quando la geometria del circuito è semplice il termine $f_a$ risulta trascurabile, riducendosi quindi al caso stazionario.
Quando questo termine aggiuntivo non è trascurabile (ad esempio, per la presenza nel circuito di un solenoide) si indica
esplicitamente la presenza di un componente autoinduttivo.
\begin{example}
    Dopo aver chiuso  un circuito LR (circuito RL in fase di carica) la corrente nel tempo varia secondo la legge
    \[
        I=I_m\Bigl(1-e^{-\frac{t}{\tau}}\Bigr)
    \]
    con $I_m=f/R$ e $\tau=L/R$.
    Una volta chiuso il circuito quindi la corrente aumenta passando da $0$ a $I_m$, ovvero il valore
    che verrebbe raggiungo istantaneamente in assenza di induttanza.

    Per arrivare a questo risultato bisogna semplicemente risolvere l'equazione del circuito con $f$ costante
    e condizione iniziale $I(0)=0$.
    Con la notazione introdotta l'equazione può essere riscritta come
    \[
        I=I_m-\tau \dv{I}{t}
    \]
    Con la sostituzione $I\rightarrow I_m-I=u$ si ha $\dv*{I}{t}=-\dv*{u}{t}$ e $u(0)=I_m$. Di conseguenza
    \[
        \frac{\dv{u}{t}}{u}=-\rec{\tau}
    \]
    Integrando ambo i membri da $0$ a $t$
    \[
        \ln(u)-\ln(u(0))=-\frac{t}{\tau}
    \]
    Mediante sostituzione inversa si ottiene l'equazione cercata.
    Se ora in \eqref{eqn:RL} si isola $f_a$ e si inserisce l'andamento appena trovato per la corrente, si ottiene che
    la differenza di potenziale ai capi dell'induttanza vale $f_a=-fe^{-\frac{t}{\tau}}$, ovvero
    $f_a$ si oppone quindi ad $f$ ma diminusce col tempo fino ad azzerarsi.
\end{example}

\begin{thm}[Legge di Felici]
    \label{teo:felici}
    Sia data una spira con area $S$ e costituita da $N$ avvolgimenti, di resistenza $R$ e immersa in un campo
    di induzione magnetica costante nel tempo. Detta $Q$ la carica totale che attraversa il circuito
    \[
        Q=\frac{\Phi_i-\Phi_f}{R}
    \]
    ovvero la carica totale che attraversa il circuito dipende solo dallo stato iniziale e dallo stato finale.
    In particolare, se la spira viene portata in una zona priva di campo magnetico
    \[
        Q=\frac{NS[B]}{R}
    \]
    con $[B]$ il valor medio del campo di induzione magnetica sulla superficie della spira.
\end{thm}
\begin{proof}
    Dimostriamo la prima parte dell'asserto.
    In assenza di generatori, l'unica forza elettromotrice nel circuito è quella autoindotta la cui espressione
    è data dalla legge di Faraday-Neumann. Per la prima legge di Ohm la corrente nel circuito allora vale
    \[
        I=\rec{R}f_i=-\rec{R}\pdv{\Phi_S(\vb{B})}{t}
    \]
    La carica totale che scorre nel circuito dall'istante iniziale al tempo $t$ allora vale
    \[
        Q=\int_0^t I\dd{t}=-\rec{R}\int_0^t\pdv{\Phi_S(\vb{B})}{t}\dd{t}=\frac{\Phi_i-\Phi_f}{R}
    \]
    Per la seconda parte, si ha che la spira è inizialmente posta in quiete nel campo
    \[
        \Phi_i(\vb{B})=N\int_S \vb{B}\vdot\dd{\vb{S}}=NS\Biggl(\rec{S}\int_S \vb{B}\vdot\dd{\vb{S}} \Biggr)=NS[B]
    \]
    Se si porta la spira in una regione in cui il campo di induzione magnetica è nullo $\Phi_f(\vb{B})=0$ e quindi
    per quanto mostrato
    \[
        Q=\frac{NS[B]}{R}
    \]
\end{proof}
