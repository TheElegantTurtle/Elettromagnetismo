Nel caso statico non si è affrontato il discorso relativo all'energia magnetica. Questo perchè la creazione
di una configurazione stazionaria di correnti (e campi magnetici associati) richiede che esista
un periodo iniziale di transizione in cui le correnti si vanno da $0$ al loro valore finale. Per quanto
visto in questo capitolo, in questa fase sul sistema vengono indotte delle forze elettromotrici.
Siccome l'energia posseduta dal campo magnetico è l'energia che è stata necessaria alla sua produzione
questo contributo non può essere tralasciato.

L'oggetto attraverso il quale si passa per studiare l'energia magnetica è l'induttanza.
Si consideri quindi un circuito $RL$, descritto dall'equazione \eqref{eqn:RL}. Moltiplicando per la quantità $\dd{Q}=I\dd{t}$ si ottiene
\[
    f\dd{Q}=RI^2\dd{t}+LI\dd{I}
\]
Dalla definizione di forza elettromotrice emerge che il primo membro dell'equazione è l'energia erogata dal
generatore nell'intervallo di tempo $\dd{t}$. Per la legge sull'effetto Joule
\eqref{eqn:effetto_joule} il primo termine del secondo membro rappresenta l'energia dissipata nella resistenza.
Il secondo termine al secondo membro è quello invece che non ci sarebbe se non ci fosse induttanza:
lo si può quindi interpretare come l'energia da fornire all'induttanza affinchè la corrente circolante
in essa si porti nel tempo $\dd{t}$ dal valore $I$ al valore $I+\dd{I}$.
L'energia necessaria allora affinchè la corrente circolante nell'induttanza si porti da $0$ ad $I$ è
\begin{equation}
    U_L=\int_0^I LI\dd{I}=\rec{2}LI^2
    \label{eqn:U_induttore}
\end{equation}
Questa interpretazione è confermata dal seguente calcolo
\begin{obs}
    L'energia $U_L$ è proprio l'energia posseduta da un'induttanza $L$ percorsa da corrente $I$
\end{obs}
\begin{proof}
    Si consideri un circuito percorso da una corrente $I_m=f/R$. All'instante iniziale,
    il generatore viene staccato dal circuito il quale, viene chiuso in corto circuito. L'equazione del circuito è
    diventa quindi
    \[
        RI=-L\dv{I}{t}, \quad\quad I(0)=I_m
    \]
    la cui soluzione è
    \[
        I=I_me^{t/\tau}
    \]
    con $\tau=L/R$. La potenza dissipata per effetto Joule è $RI^2$ e quindi l'energia totale dissipata dalla resistenza è
    \[
        U_R=\int_0^\infty RI^2\dd{t}=\int_0^\infty \frac{f^2}{R}e^{-2t/\tau}\dd{t}=\frac{f^2}{2R^2}L=\rec{2}LI_m^2
    \]
    Per conservazione dell'energia, questa non può che essere l'energia posseduta dall'induttanza prima di chiudere il circuito.
\end{proof}
$U_L$ rappresenta l'energia potenziale immagazzinata nel dispositivo -ovvero il singolo circuito.
Si consideri ora un solenoide percorso da corrente costante all'interno del quale sia parzialmente inserito un nucleo di ferro.
Si verifica sperimentalmente che il nucleo viene risucchiato all'interno del solenoide. Questi fenomeno è in apparenza contraddittorio:
a nucleo completamente inserito infatti l'induttanza è maggiore\footnote{Si ricordi che l'induttanza di un solenoide è
$L=\mu_r\mu_0 N^2S/l$ e che per il ferro $\mu_r>0$.} e quindi lo è anche l'energia del solenoide - ovvero, apparentemente
il sistema si è spostato spontaneamente da una configurazione con energia minore ad una con energia maggiore. La spiegazione
si basa su un discorso analogo a quello svolto nel paragrafo \ref{par:corrente_correntestazionaria_complementi}
sulla forza fra le maglie di un condensatore: l'ipotesi che la corrente sia costante richiede implicitamente
che al circuito sia collegato un generatore in grado di soddisfare questa richiesta.

Si vuole ora calcolare l'energia potenziale $U_M$ immagazzinata in un insieme di più circuiti, soggetti quindi
a mutua induzione.
\begin{example}
    Si consideri una regione di spazio uniformemente riempita con un materiale
    con $B/H=\mu$ costante
    in cui sia presente un campo $\vb{B}$.
    Si consideri un circuito $RL$ costituito da un solenoide con una lughezza molto maggiore del suo diametro in fase di carica,
    descritto dall'equazione \eqref{eqn:RL}.
    Il campo interno al solenoide è uniforme e coincide col campo medio. Tenuto conto della \eqref{eqn:induttanza}
    e della legge di Felici \ref{teo:felici} si ottiene quindi
    \[
        L\dv{I}{t}=NS\dv{B}{t}
    \]
    Sostituendo quindi nella \eqref{eqn:RL} e moltiplicando per $I\dd{t}$, si ricava
    \[
        fI\dd{t}=RI^2\dd{t}+INS\dd{B}
    \]
    Per le considerazioni fatte all'inizio del paragrafo, l'energia fornita all'induttanza è, indicando con $n=N/l$
    \[
        \dd{U}_L=INS\dd{B}=SlnI\dd{B}=nVI\dd{B}
    \]
    dove $V=Sl$è il volume del solenoide. Dividendo per $V$ in modo da ricavare la densità di energia
    e tenendo presente la \eqref{eqn:H_solenoide}, si ottiene
    \[
        \dd{u}_L=\frac{\dd{U}_L}{V}=nI\dd{B}=H\dd{B}
    \]
    Fin'ora, non si è fatto uso dell'ipotesi di $\mu$ costante, quindi i risultati ottenuti sono indipendenti dal fatto che il
    materiale sia paramagnetico, ferromagnetico o diamagnetico\footnote{
        si noti invece l'ipotesi fortemente restrittiva di avere a che fare con un solenoide, ovvero con un campo uniforme su un volume
        di geometria semplice}.
    Considerando anche questa ipotesi
    \[
        u_M=u_L(B)=\int_0^B \dd{u}_L=\int_0^B H\dd{B}=\int_0^B \rec{\mu}B\dd{B}=\rec{2\mu}B^2=\rec{2}BH=\rec{2}\mu H^2
    \]


    la densità di energia magnetica vale in conclusione
    \begin{equation}
        u_M=\rec{2}\mu H^2
    \end{equation}
\end{example}

\begin{thm}
    L'energia magnetica per un numero $N$ di circuiti è
    \begin{equation}
        U_M=\rec{2}\sum_{i,j=1}^N M_{ij}I_iI_j
    \end{equation}
    con $M_{ii}=L_i$, o equivalentemente
    \begin{equation}
        \label{eqn:UM_flussi}
        U_M=\rec{2}\sum_{i=1}^N I_i\Phi_i
    \end{equation}
    dove $\Phi_i$ è il flusso di tutti i campi di induzione magnetica presenti nel sistema attraverso il circuito i-esimo.
\end{thm}
\begin{proof}
    Si parte dal caso più semplice di due soli circuiti e si arriva ad una generalizzazione ad $N$ circuiti.
    Il sistema di due circuiti è descritto da
    \[
        \begin{cases}
            &  f_1=L_1\dv{I_1}{t}+M_{12}\dv{I_2}{t}+R_1I_1 \\
            &  f_2=L_2\dv{I_2}{t}+M_{21}\dv{I_1}{t}+R_2I_2 \\
        \end{cases}
    \]
    Moltiplicando la prima equazione per $\dd{Q_1}=I_1\dd{t}$ e la seconda per $\dd{Q_2}=I_2\dd{t}$,
    ricordando che $M_{21}=M_{12}$, poi sommando membro a membro si ottiene
    \[
        (f_1I_1\dd{t}+f_2I_2\dd{t})=(I_1^2R_1+I_2^2R_2)\dd{t}+[L_1I_1\dd{I_1}+L_2I_2\dd{I_2}+M_{12}(I_1\dd{I_2}+I_2\dd{I_1})]
    \]
    Per confronto, il membro fra parentesi quadre rappresenta il differenziale di energia magnetica, ovvero l'energia da fornire
    per incrementare di $\dd{I}$ le correnti $I_1$ e $I_2$.
    \[
        \dd{U_M}=L_1I_1\dd{I_1}+L_2I_2\dd{I_2}+M_{12}(I_1\dd{I_2}+I_2\dd{I_1})=\dd(\rec{2}L_1I_1^2+\rec{2}L_2I_2^2+M_{12}I_1I_2)
    \]
    Integrando da $0$ ai valori finali $I_1$, $I_2$
    \[
        U_M=\rec{2}L_1I_1^2+\rec{2}L_1I_1^2+M_{12}I_2I_2
    \]
    dove i primi due termini rappresentano l'energia dei signoli circuiti, mentre il terzo rappresenta l'energia di mututa induzione.
    Per l'uguaglianza fra i coefficenti di mutua induzione è possibile scrivere $M_{12}I_1I_2=1/2(M_{12}I_1I_2+M_{21}I_2I_1)$.
    Chiamando quindi $L_1=M_{11}$ ed $L_2=M_{22}$ la formula ottenuta può essere scritta come
    \[
        U_M=\rec{2}(M_{11}I_1^2+M_{12}I_1I_2+M_{21}I_2I_1+M_{22}I_2^2)=\rec{2}\sum_{i,j=1}^2M_{ij}I_iI_j
    \]

    Raccogliendo $I_1$ per i primi due termini del secondo membro e $I_2$ per gli altri due, si ha,
    dalle definizioni di induttanza e di coefficente di mutua induttanza:
    \[
        U_M=\rec{2}I_1(M_{11}I_1+M_{12}I_2)+\rec{2}I_2(M_{21}I_1+M_{22}I_2)=\rec{2}(I_1\Phi_1+I_2\Phi_2)=\rec{2}\sum_{i=1}^2I_i\Phi_i
    \]

    Entrambe le espressioni trovate per $U_M$ sono immediatamente generalizzabili al caso di $N$ circuiti.
\end{proof}


\begin{thm}
    Le forze agenti su un circuito percorso da corrente $I$ immerso in un campo magnetico esterno costante
    $\vb{B}_{ext}$ sono date dal gradiente (non cambiato di segno) dell'energia potenziale, che può essere scritta
    nel caso di un singolo circuito come
    \[
        U=I\Phi(\vb{B}_{ext})
    \]
\end{thm}
\begin{proof}
    Il sistema può essere schematizzato come costituio da
    $N$ circuiti, $N-1$ dei quali sono le sorgenti di $\vb{B}_{ext}$. Sia il circuito k-esimo percoro da corrente $I_k$ e abbia
    resistenza complessiva $R_k$. Si immagini di applicare una traslazione virtuale a questo circuito
    mediante una forza esterna $\vb{F}^{(k)}$ che imprima sul circuito una veloctià $\vb{v}^{(k)}$
    piccola, in modo tale che sia trascurabile l'energia cinetica associata a questo movimento.
    L'energia del circuito, ovvero l'energia magnetica\footnote{L'energia magnetica è l'energia potenziale totale
    immagazzinata nel circuito in quanto non sono presenti condensatori.}, varierà come conseguenza di tre contributi:
    $\vb{F}^{(k)}\vdot\dd{\vb{x}}$ lavoro compiuto
    dalla forza esterna; $\dd{L}_G$ lavoro compiuto dai generatori nel circuito; $\dd{L}_{R}$ lavoro
    dissipato (e quindi negativo) per effetto Joule. Per la conservazione dell'energia, rapportando tutto all'unità di tempo
    \[
        \dv{U_M}{t}=\vb{F}^{(k)}\vdot\dd{\vb{v}} + \dv{L_G}{t} - \dv{L_R}{t}
    \]
    Ricordando che
    \[
        \begin{split}
            &\dv{L_G}{t}=\sum I_j f_j=\sum I_j(I_j R_j + \dv{\Phi_j}{t} ) \\
            &\dv{L_R}{t}=\sum I^2_j R_j \\
            &\dv{U_M}{t}=\dv{t} \Biggl(\rec{2} \sum I_j \Phi_j \Biggr)=
            \rec{2} \sum I_j \dv{\Phi_j}{t}
        \end{split}
    \]
    Si ha
    \[
        \vb{F}^{(k)}\vdot\dd{\vb{v}}=-\rec{2}\sum I_j\dv{\Phi_j}{t}=-\dv{U_M}{t}
    \]
    Per garantire una traslazione virtuale a velocità trascurabile la forza esterna deve essere
    uguale in modulo e opposta alla forza $\vb{f}^{(k)}$ che il campo magentico esercita sul
    circuito k-esimo. Ne segue che $\vb{f}^{(k)}\vdot\dd{\vb{x}}=\dd{U_M}$, ovvero la forza è data
    dal gradiente di $U_M$.

    Svolgendo esplicitamente il gradiente dell'energia magnetica, se è costante la corrente ed il circuito è
    indeformabile (in modo che $L$ sia costante), le derivate dell'energia magnetica totale coincidono con le sole derivate dell'
    energia di accoppiamento, ovvero dell'energia dovuta all'interazione del circuito k-esimo con gli altri $N-1$ circuiti
    \[
        U_{acc}=\sum_{j\neq k} I_k \,M_{kj}I_j=I_k\Phi^{(k)}_{acc}(\vb{B})
    \]
    dove $\Phi^{(k)}_{acc}(\vb{B})$ è il flusso concatenato col k-esimo circuito, prodotto
    da tutti gli altri circuiti. Ma questo è proprio il flusso del campo esterno attraverso il
    circuito k-esimo, ovvero la tesi.
\end{proof}

Quelle ottenute sono espressioni di tipo integrale. È interessante cercare di trovare delle relazioni locali:
queste forme si riveleranno utili sia per generalizzare quanto visto fin'ora al caso non stazionario,
sia per trovare la densità di energia nella forma il più generale possibile.
\begin{thm}
    Dato sistema costituito da $N$ circuiti e un volume $\tau$ che racchiuda tutti i circuiti, l'energia magnetica del sistema è:
    \begin{equation}
        \label{eqn:UM_J_A}
        U_M= \rec{2}\int_\tau \Biggl(\vb{J}+\pdv{\vb{D}}{t}\Biggr)\vdot\vb{A}\dd{\tau}
    \end{equation}
\end{thm}
\begin{proof}
    In un sistema di $N$ circuiti si consideri nel dettaglio il circuito k-esimo.
    Questo sarà costituito da un conduttore di sezione non nulla $\sigma_k$ chiuso su se stesso
    e percorso da una corrente $I_k=\int_{\sigma_k}\vb{J}_k\vdot\dd{\vb{\sigma}_k}$.
    Si consideri la superficie $S_k$ che abbia $l_k$ come contorno. Per la \eqref{eqn:UM_flussi}
    \[
        \begin{split}
            U_M &=\rec{2}\sum_k I_k\Phi_k=\rec{2}\sum_k\Biggl[\int_{\sigma_k}\vb{J}_k\vdot\dd{\vb{\sigma}_k} \int_{S_k}\vb{B}\vdot\dd{\vb{S}_k} \Biggr]\\
            &=\rec{2}\sum_k\Biggl[\int_{\sigma_k}\vb{J}_k\vdot\dd{\vb{\sigma}_k} \int_{S_k}(\curl{\vb{A}})\vdot\dd{\vb{S}_k} \Biggr]=\\
            &=\rec{2}\sum_k\Biggl[\int_{\sigma_k}\vb{J}_k\vdot\dd{\vb{\sigma}_k} \oint_{l_k}\vb{A}\vdot\dd{\vb{l}_k} \Biggr]
        \end{split}
    \]
    Siccome $\dd{\vb{l}_k}$, $\vb{J}_k$, $\dd{\vb{\sigma}_k}$ sono paralleli, indicando con $\vu{l}_k$
    la direzioni di questi vettori, la relazione può essere riscritta come
    \[
        \begin{split}
            U_M&=\rec{2}\sum_k\Biggl[\int_{\sigma_k}J_k\dd{\sigma_k} \oint_{l_k}\vb{A}\vdot\vu{l}_k\dd{l_k} \Biggr]
            =\rec{2}\sum_k \int_{\sigma_k}\oint_{l_k}J_k\vu{l}_k \vdot \vb{A}\dd{\sigma_k}\dd{l_k} \\
            &=\rec{2}\sum_k \int_{\tau_k} \vb{J}_k \vdot \vb{A}\dd{\tau_k}
        \end{split}
    \]
    Siccome la densità di corrente è nulla al di fuori dei circuiti, si può elminiare la sommatoria
    trasformando l'integrale su $\tau_k$ in un integrale su un qualsiasi volume $\tau$ che racchiuda tutti i circuiti,
    ottenendo la tesi nel caso stazionario.

    Sostituendo la densità di corrente con la densità di corrente
    generalizzata, si ottiene la tesi nel caso generale.
\end{proof}

\begin{thm}
    Data una densità di corrente $J$ l'energia magnetica vale
    \begin{equation}
        \label{eqn:UM}
        U_M=\rec{2} \int_{S} (\vb{H}\cp\vb{A}) \vdot \dd{\vb{S}} +\rec{2} \int_{\tau}\vb{H}\vdot\vb{B}\dd{\tau}
    \end{equation}
\end{thm}
\begin{proof}
    Per la quarta equazione di Maxwell nel caso non stazionario, l'integrando nella \eqref{eqn:UM_J_A} può essere riscritto come
    \[
        \Biggl(\vb{J}+\pdv{\vb{D}}{t}\Biggr)\vdot\vb{A}=(\curl{\vb{H}})\vdot\vb{A}
    \]
    Tenuto conto dell'identità \eqref{app:eqn:div_cp}
    \[
        (\curl{\vb{H}})\vdot\vb{A}=\div(\vb{H}\cp\vb{A})+\vb{H}\vdot(\curl{\vb{A}})
    \]
    E indicando con $S$ la superficie che racchiude il volume $\tau$, $U_M$ può essere riscritta come
    \[
        \begin{split}
            U_M&=\rec{2} \int_{\tau} \div(\vb{H}\cp\vb{A})\dd{\tau} +\rec{2} \int_{\tau}\vb{H}\vdot(\curl{\vb{A}})\dd{\tau}\\
            &=\rec{2} \int_{S} (\vb{H}\cp\vb{A}) \vdot \dd{\vb{S}} +\rec{2} \int_{\tau}\vb{H}\vdot\vb{B}\dd{\tau}
        \end{split}
    \]
\end{proof}
L'equazione ottenuta è formalmente analoga alla \eqref{eqn:UE}: valgono tutte le considerazioni fatte in quel caso
e in particolare
\begin{cor}
    Se si prende in considerazione tutto lo spazio, allora l'energia vale
    \begin{equation}
        \label{eqn:uM_H_B}
        U_M=\int u\dd{\tau} \quad\quad\quad \text{ con } u=\frac{\vb{H}\vdot\vb{B}}{2}
    \end{equation}
\end{cor}
Un'osservazione su quanto ottenuto. Sia l'integranda della \eqref{eqn:UM_J_A} che la \eqref{eqn:uM_H_B}
godono della proprietà che il loro integrale esteso a tutto lo spazio sia pari all'energia magnetica.
Di fatto però, le due funzioni non sono necessariamente uguali punto per punto. Sono allora considerazioni di carattere puramente fisico
che portano ad interpretare la seconda come l'effettiva densità di energia.

Si conclude la sezione con un confronto fra l'energia magnetica e l'energia elettrostatica.
La prima analogia è fra le formule per l'energia di un'induttore \eqref{eqn:U_induttore} e quella per un condensatore \eqref{eqn:U_condensatore}.
In entrambi i casi la struttura della formula è quella di un prodotto fra l'integrale della densità della sorgente del campo
e la caratteristica del circuito. Inoltre, sia nel calcolo delle forze meccaniche dovute all'energia magnetica che di quelle dovute
all'energia elettrostatica si prendono le derivate positive dell'energia, contrariamente a quanto si fa nel caso meccanico.
