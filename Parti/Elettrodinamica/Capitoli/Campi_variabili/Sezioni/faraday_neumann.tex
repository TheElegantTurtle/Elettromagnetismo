Per tenere conto delle osservazioni sperimentali, si introduce la seguente legge.
\begin{obses} [Legge di Faraday-Neumann]
    Dato un circuito immerso in un campo di induzione magnetica $\vb{B}$ e detto $\Phi(\vb{B})$
    il flusso con del campo concatenato al circuito, allora nel circuito si genera una forza elettromotrice indotta
    \begin{equation}
        f_i=-\dv{\Phi(\vb{B})}{t}
    \end{equation}
\end{obses}
La derivata totale non può essere portata all'interno dell'intergale che esprime il flusso
in quanto la forma del circuito, e quindi il cammino di integrazione, dipende dal tempo.
\begin{obs}
    La forza elettromotrice indotta implica la presenza di un campo elettromotore indotto $\vb{E}_i$
    \begin{equation}
        \label{eqn:fi}
        \vb{E}_i=\vb{E}+\vb{v}_T\cp\vb{B}
    \end{equation}
    Dove $\vb{E}$ e $\vb{B}$ sono i campi elettrico e di induzione magnetica in cui il circuito è immerso
    e $\vb{v}_T$ è la velocità di trascinamento con cui si muove ciascun elemento infinitesimo di circuito.
\end{obs}
\begin{proof}
    Dalla definizione $f_i=\oint_l \vb{E}_i\vdot\dd{\vb{l}}$, che mostra la presenza del campo elettromotore indotto.
    Nel fenomeno dell'induzione i portatori di carica $q$ che danno orgine alla corrente nel circuito sono messi in moto
    da una forza
    \[
        \vb{F}=q\vb{E}+q\vb{v}_{tot}\cp\vb{B}
    \]
    Il campo elettromotore può quindi essere visto come rapporto fra questa forza e la carica dei portatori, ovvero
    \[
        \vb{E}_i=\vb{E}+\vb{v}_{tot}\cp\vb{B}=\vb{E}+(\vb{v}_T+\vb{v}_d)\cp\vb{B}
    \]
    dove $\vb{v}_T$ è la velocità di trascinamento e $\vb{v}_d$ la velocità di deriva delle cariche.
    Si ha quindi, considerando che la velocità di deriva è parallela punto per punto all'elemento $\dd{\vb{l}}$
    di volta in volta considerato
    \[
        f_i=\oint_l \vb{E}_i\vdot\dd{\vb{l}}=\oint_l(\vb{E}+\vb{v}_T\cp\vb{B})\vdot\dd{\vb{l}}
    \]
    Siccome la forza elettromotrice indotta è l'unica osservabile attraverso la quale sia possibile individuare
    il campo elettromotore indotto, allora si può tranqullamente identificare quest'ultimo con il secondo integrando.
\end{proof}


La corrente che circola nel circuito a sua volta genera un campo magnetico indotto $\vb{B}_i$.
\begin{cor}[Legge di Lenz]
    Il campo magnetico indotto $\vb{B}_i$ è tale che il suo verso si opponga
    a quello del campo magnetico che genera la forza elettromotrice indotta.
\end{cor}
\begin{proof}
    La dimostrazione è immediata, per il fatto che nella legge di Faraday-Neumann compare il segno meno
\end{proof}
