L'obiettivo di questo paragrafo è duplice: dare un significato fisico più
concreto alla legge di Faraday-Neumann attraverso tre esempi significativi
e esporre il concetto di flusso tagliato, utile per i risultati
espressi nei prossimi paragrafi.

\begin{defn}[Flusso tagliato]
    Il flusso attraverso la superficie spazzata da un circuito in movimento è chiamato flusso tagliato.
\end{defn}


\subsubsection{Circuito variabile, B costante nel tempo}
$\vb{B}$ costante nel tempo significa che le caratteristiche delle sorgenti del campo non cambiano nel tempo
e che queste sorgenti sono in quiete nel sistema di riferimento scelto per descrivere il fenomeno.
\begin{thm}
    Dato un circuito di forma generica, il cui generico elemento $\dd{l}$ sia in moto con velocità $\vb{v}_T$,
    immerso in un campo $\vb{B}$ costante nel tempo, la variazione del flusso concatenato al circuito è
    pari in modulo e opposta in segno al flusso tagliato.
\end{thm}
\begin{proof}
    Detta $\Sigma$ la superficie orientata attraverso la quale viene calcolato il flusso concatenato all'istante $t_0$
    (indicato con $\Phi_i$), $\Sigma'$ la superficie attraverso cui viene calcolato il flusso $\Phi_f$ all'istante
    $t+\dd{t}$ e $\dd{\Sigma}$ la superficie spazzata dal circuito nel suo spostamento, si ha che l'unione di queste tre superfici
    è una superficie chiusa: il flusso totale -ovvero la somma dei flussi uscenti- deve essere nullo\footnote{Questo è vero solo
    nell'ipotesi di campo costante nel tempo, in quanto le superfici $\Sigma$ e $\Sigma'$ sono appoggiate al circuito
    in tempi diversi}. Si scelga convenzionalmente come verso di percorrenza positivo quello per cui la normale alla superficie del
    circuito sia parallela alla velocità complessiva del circuito. Allora
    \[
        -\Phi_i+\Phi_f+\Phi_{\dd{\Sigma}}=0
    \]
    dove il segno meno al primo termine è dovuto al fatto che con la convenzione scelta $\Phi_i$ è
    il flusso entrante nella superficie chiusa, mentre la somma è nulla quando riferita ai flussi uscenti.
    Si ha quindi che
    \[
        \dd{\Phi} \equiv \Phi_f-\Phi_i=-\Phi_{\dd{\Sigma}}
    \]
    ovvero la tesi.
\end{proof}
L'utilità di questo risultato risiede nel fatto che è possibile trovare un'espressione esplicita per il flusso
tagliato come mostrato nel seguente teorema.
\begin{thm}
    Dato un circuito di forma generica, il cui generico elemento $\dd{l}$ sia in moto con velocità $\vb{v}_T$,
    immerso in un campo $\vb{B}$ costante nel tempo, il flusso tagliato vale
    \[
        \Phi_{\dd{\Sigma}}=\dd{t}\oint_l(\vb{v}_T\cp\vb{B})\vdot\dd{l}
    \]
\end{thm}
\begin{proof}
    Si vuole calcolare $\Phi_{\dd{\Sigma}}$. Per farlo, siccome $\vb{B}$ è noto per ipotesi, bisogna esprimere
    $\dd{\vb{S}}$ in termini di grandezze note. Questo si ottiene considerando che
    ogni elemento di circuito compie uno spostamento $\dd{\vb{s}}=\vb{v}_T\dd{t}$ e nel
    farlo spazza la superficie $\dd{\vb{S}}=\dd{\vb{l}}\cp\dd{\vb{s}}=\dd{\vb{l}}\cp\vb{v}_T\dd{t}$.
    \[
        \Phi_{\dd{\Sigma}}=\int_{\dd{\Sigma}}\vb{B}\vdot\dd{\vb{S}}=\oint_l\vb{B}\vdot(\dd{\vb{l}}\cp\vb{v}_T\dd{t})
    \]
    Usando l'identità \eqref{app:eqn:vdot_cp} si ha la tesi.
    % \[
    %     \oint_l\vb{B}\vdot(\dd{\vb{l}}\cp\vb{v}_T\dd{t})=\oint_l(\vb{v}_T\dd{t}\cp\vb{B})\vdot\dd{\vb{l}}
    % \]
\end{proof}

\begin{cor}
    Nel caso di circuito in moto con velocità $\vb{v}_T$ in un campo $\vb{B}$ costante, il campo elettromotore indotto è
    \[
        \vb{E}_i=\vb{v}_T\cp\vb{B}
    \]
\end{cor}
\begin{proof}
    Per i due risultati appena dimostrati $\dv*{\Phi}{t}=-\Phi_{\dd{\Sigma}}/\dd{t}=-\oint_l(\vb{v}_T\dd{t}\cp\vb{B})\vdot\dd{\vb{l}}$.
    Confrontando il primo e l'ultimo membro dell'uguaglianza con la legge di Faraday-Neumann si ha la tesi.
\end{proof}
L'interpretazione fisica del fenomeno dell'induzione elettromagnetica è quindi immediata nelle ipotesi del corollario:
le cariche costrette a muoversi nel campo di induzione magnetica per via del moto del circuito, sono sottoposte alla forza di Lorentz.
Nonostante la forza di Lorentz non compia lavoro la forza elettromotice può essere diversa da $0$:
il lavoro dissipato dalla corrente nel circuito è compiuto dalla forza esterna che mantiene $\vb{v}_T$ costante
o è compiuto a spese dell'energia cinetica, per cui il circuito rallenta fino a fermarsi.



\subsubsection{Circuito rigido, sorgenti di B stazionarie in moto}
Si consideri un circuito $C$ in quiete nel sistema di riferimento inerziale $sr=Oxyz$. Supponendo che le
sorgenti di $\vb{B}$ siano in uno stato stazionario, ovvero con caratteristiche costanti nel tempo,
una variazione del flusso può avvenire solo se queste cariche si muovono rispetto a $sr$ con una velocità $\vb{v}$.
Nell'ipotesi che ci sia una sola sorgente del campo il cui moto sia
di pura traslazione con $v$ costante, è possibile considerare un sistema di riferimento inerziale
$sr'=O'x'y'z'$ rispetto al quale la sorgente è in quiete. $sr$ si muove rispetto ad $sr'$ con
velocità $\vb{v}'=-\vb{v}$. $sr'$ si trova allora nelle stesse condizioni discusse nel caso del circuito variabile e
campo costante nel tempo: le cariche risentono della forza di Lorentz ($\vb{F}'_l=q\vb{v}'\cp\vb{B}'$) come nel caso precedente.
In $sr$, dove il circuito è fermo, non si ha nessuna forza di Lorentz. A partire da considerazioni relativistiche però si può dedurre che
il moto delle sorgenti di $\vb{B}$ provochi l'insorgere di un campo $\vb{E}_l$ che esercita sui
portatori di carica del circuito una forza $\vb{F}_l=q\vb{E}_l$ equivalente alla forza di Lorentz $\vb{F}'_l$.
Per $v<<c$ si ha $\vb{B}\simeq\vb{B}'$, da cui $\vb{E}_l=\vb{v}'\cp\vb{B}'=-\vb{v}\cp\vb{B}$.

In virtù dell'additività di $\vb{B}$ il ragionamento fatto è estendibile al caso di più sorgenti.



\subsubsection{Circuito rigido, sorgenti di B ferme non stazionarie}
Il circuito $C$ e le sorgenti del campo di induzione magnetica non sono in moto relativo fra loro.
Le sorgenti di $\vb{B}$ non stazionarie sono dei circuiti nei quali le correnti di alimentazione
non sono costanti. L'effetto è che in ogni punto di $C$ il campo $\vb{B}$ sia variabile nel tempo,
esattamente come visto del caso del circuito in moto relativo con le sorgenti del campo.
È ragionevole perciò aspettarsi quindi che anche in questo caso si abbia come effetto un campo elettromotore,
sebbene questo non sia causato dalla forza di Lorenz. In effetti questo campo è presente ed è una
conseguenza della generalizzazione della terza equazione di Maxwell al caso non stazionario che
verrà presentata nel prossimo paragrafo.
