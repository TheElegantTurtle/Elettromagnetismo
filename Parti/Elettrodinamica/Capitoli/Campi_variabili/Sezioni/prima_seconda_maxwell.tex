Vista la generalizzazione della terza equazione di Maxwell al caso non stazionario,
ci si pone come obiettivo quello di generalizzare le altre equazioni.
\begin{obses}
    Per quanto concerne la prima e la seconda equazione di Maxwell,
    si verifica sperimentalmente che la generalizzazione
    si ottiene banalmente introducendo la dipendenza dal tempo nella densità di carica elettrica.
    \[
        \begin{split}
            &\div{\vb{E}}=\frac{\rho(x,y,z,t)}{\epsilon_0}\\
            &\div{\vb{B}}=0
        \end{split}
    \]
\end{obses}

La prima di queste due equazioni può portare a conseguenze all'apparenza paradossali.
Infatti è immediato verificare che il teorema di Gauss continua a valere anche in condizioni non stazionarie,
ma il teorema di Gauss collega fra loro la carica presente all'interno di una superficie
ed il flusso sulla superficie: due quantità poste in posizioni diverse sono collegate istantaneamente.
La spiegazione del paradosso emerge considerando la legge di conservazione della carica:
la variazione della quantità di carica interna infatti può essere dovuta esclusivamente al passaggio di cariche
attracerso la superficie, perciò la variazione di flusso è un fenomeno essenzialmente locale

La conseguenza della seconda è che anche nel caso non stazionario
il flusso del campo magnetico attraverso una superficie chiusa è ad ogni istante nullo.
