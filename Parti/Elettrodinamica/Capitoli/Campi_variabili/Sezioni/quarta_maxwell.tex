I procedimenti usati nei paragrafi precedenti non possono essere usati per generalizzare anche la quarta equazione di Maxwell, infatti

\begin{obs}
    L'estensione al caso non stazionario della quarta equazione di Maxwell non può essere
    ottenuto rendendo dipendenti dal tempo le grandezze che in essa compaiono.
\end{obs}
\begin{proof}
    Si applichi l'operatore divergenza alla quarta equazione di Maxwell
    \[
        \div(\curl{\vb{B}})=\div(\mu_0\vb{J})=\mu_0\div{\vb{J}}
    \]
    La divergenza del rotore è sempre nulla, quindi segue che $\div{\vb{J}}=0$.
    Dall'equazione di coninutià \eqref{eqn:continuità} si ottiene che questo è vero se e solo se
    \[
        \pdv{\rho}{t}=0
    \]
    Ovvero se e solo se la variazione della densità di carica è nulla
    e questo succede se e solo se ci si trova nel caso stazionario.
\end{proof}

La generalizzazione viene, di fatto, fatta un po' a culo.
\begin{defn}[Densità di corrente di spostamento]
    Si definisce densità di corrente di spostamento il vettore
    \[
        \vb{J}_S=\epsilon_0\pdv{\vb{E}}{t}
    \]
\end{defn}
\begin{defn}[Corrente di spostamento]
    Si definisce corrente di spostamento il flusso della densità di corrente di spostamento attraverso una superficie.
\end{defn}
Si osservi come dimensionalmente questa sia effettivamente una densità di corrente, infatti dalla definizione di campo
elettrico $[\epsilon_0 E]=[q/r^2]=C/m^2$ e quindi $[\epsilon_0\pdv*{E}{t}]=A/m^2=[J]$.
\begin{defn}[Densità di corrente totale generalizzata]
    Si definisce densità totale di corrente generalizzata $\vb{J}_t$
    \[
        \vb{J}_t=\vb{J}+\epsilon_0\pdv{\vb{E}}{t}
    \]
\end{defn}
La seguente osservazione si rivela fondamentale e giustifica l'introduzione della corrente di spostamento.
\begin{obs}
    La densità di corrente totale generalizzata si riduce alla densità di corrente nel caso stazionario e ha sempre divergenza nulla.
\end{obs}
\begin{proof}
    La dimostrazione della prima parte dell'asserto è banale e si ottiene semplicemente considerando che nel caso stazionario
    la variazione del campo elettrico è nulla. Per quanto riguarda la seconda parte invece, si consideri l'equazione di continuità
    \eqref{eqn:continuità} e usando le equazioni di Maxwell per introdurre i campi, ovvero sostituendo al posto della densità di carica
    l'espressione fornita dalla prima equazione di Maxwell, si ottiene
    \[
        \div{\vb{J}}+\pdv{\rho}{t}=\div{\vb{J}}+\epsilon_0\pdv{t}(\div{\vb{E}})=0
    \]
    Per il teorema di Schwartz è possibile scambiare la derivata parziale con la divergenza:
    \[
        0=\div{\vb{J}}+\div{\epsilon_0\pdv{\vb{E}}{t}}=\div(\vb{J}+\epsilon_0\pdv{\vb{E}}{t})
    \]
    Quello fra parentesi è proprio la densità di corrente totale generalizzata. Si è ottenuta così la tesi.
\end{proof}
Questi due risultati dicono che la densità di corrente totale generalizzata è un buon candidato per
la generalizzazione della quarta equazione di Maxwell che può essere ottenuta pertanto sostituendo
la densità di corrente usuale con la densità di corrente totale generalizzata:
\begin{equation}
    \curl{\vb{B}}=\mu_0\Biggl(\vb{J}+\epsilon_0\pdv{\vb{E}}{t} \Biggr)
\end{equation}

La validità di questa legge è confermata dall'accordo degli esperimenti coi risultati della teoria che
si sviluppa usando questa equazione. Il seguente esempio mostra come essa sia in grado di risolvere una
situazione all'apparenza paradossale.
\begin{example}[Scarica del condensatore]
    Si consideri un circuito RC aperto, con il condensatore inizialmente carico con carica $Q_0$.
    All'istante $t=0$ viene chiuso
    l'interruttore e il condensatore inizia a scaricarsi secondo la legge
    \[
        Q(t)=Q_0e^{-\frac{t}{RC}}
    \]
    Il campo elettrico vale quindi
    \[
        E=\rec{\epsilon_0}\frac{Q(t)}{S}
    \]
    e la corrente
    \[
        I(t)=\frac{Q_0}{RC}e^{-\frac{t}{RC}}
    \]
    Considero a questo punto un circuito di Ampère $c$ tra le maglie del condensatore e due superfici con questo circuito
    come contorno: $S_1$ interseca il filo del circuito collegato ad una maglia del condensatore; $S_2$ all'interno
    del volume delimitato dalle maglie del condensatore. Se valesse la "vecchia" quarta equazione di Maxwell, per il
    teorema di Ampère la circuitazione di $\vb{B}$ su $c$ varrebbe $I$ considerando $S_1$ e $0$ considerando $S_2$. Questa aporia viene
    risolta proprio dall'equazione di Maxewell generalizzata appena introdotta. Nel caso in cui si consideri $S_1$ infatti
    il campo elettrico attraverso la superficie è nullo e quidi continua a valere il teorema di Ampère. Nel caso in cui
    si consideri $S_2$ invece $\vb{J}=0$, ma $J_S=\dv*{E}{t}=I(t)/S$ da cui si ottiene una corrente in modulo uguale a $I$.
\end{example}

La generalizzazione di questa equazione ai materiali è insidiosa
ma porta all'intuibile risultato
\begin{equation}
    \curl{\vb{H}}=\vb{J}+\pdv{\vb{D}}{t}
\end{equation}
valido per qualsiasi tipo di materiale, compresi i ferromagneti.

Vengono esposte ora due conseguenze di quanto detto in questo paragrafo.

\begin{obs}[Prima legge di Kirchoff generalizzata]
    La legge di Kirchoff dei nodi è estendibile al caso non stazionario a patto di considerare sia la corrente di
    conduzione che quella di spostamento.
\end{obs}
\begin{proof}
    La dimostrazione è analoga a quella della legge di Kirchoff nel caso stazionario considerando che
    siccome la densità di corrente totale generalizzata ha divergenza nulla, è nullo il suo flusso attraverso qualsiasi
    superficie chiusa.
\end{proof}

\begin{obs}[Teorema della circuitazione di Ampère generalizzato]
    Il teorema della circuitazione di Ampère è estendibile al caso non stazionario a patto di considerare sia la corrente di
    conduzione che quella di spostamento.
\end{obs}
\begin{proof}
    Anche in questo caso la dimostrazione è analoga al caso stazionario, pur di prendere la quarta equazione di
    Maxwell generalizzata.
\end{proof}
Al fine del calcolo dei campi però, l'utilità pratica di questo teorema è limitata al caso quasi-stazionario,
ovvero al caso in cui le dimensioni del sistema siano tali per cui il tempo impiegato dai segnali
elettromagnetici per attraversarlo sia molto piccolo rispetto al tempo che caratterizza le variazioni di
$\rho$ e $\vb{J}$. In questo caso infatti la corrente di spostamento è con buona approssimazione trascurabile all'interno
dei buoni conduttori fino a frequenze nell'ordine delle microonde.
