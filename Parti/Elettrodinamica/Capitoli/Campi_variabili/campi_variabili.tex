\begin{obses}
  Si consideri un circuito costituito da una linea chiusa di materiale conduttore, in serie al quale sia disposto un galvanometro. Nei seguenti casi, il galvanometro indica il passaggio di corrente elettrica nel circuito:
  \begin{enumerate}
    \item il circuito si trova in prossimità di un altro circuito percorso da corrente variabile nel tempo;
    \item il circuito si trova in moto relativo rispetto ad un altro circuito percorso da corrente costante nel tempo;
    \item il circuito si trova in moto relativo rispetto ad un magnete permanente;
    \item il circuito viene deformato all'interno di un campo di induzione magnetica.
  \end{enumerate}
\end{obses}

\section{Legge di Faraday-Neumann}
Per tenere conto delle osservazioni sperimentali, si introduce la seguente legge.
\begin{obses} [Legge di Faraday-Neumann]
    Dato un circuito immerso in un campo di induzione magnetica $\vb{B}$ e detto $\Phi(\vb{B})$
    il flusso con del campo concatenato al circuito, allora nel circuito si genera una forza elettromotrice indotta
    \begin{equation}
        f_i=-\dv{\Phi(\vb{B})}{t}
    \end{equation}
\end{obses}
La derivata totale non può essere portata all'interno dell'intergale che esprime il flusso
in quanto la forma del circuito, e quindi il cammino di integrazione, dipende dal tempo.
\begin{obs}
    La forza elettromotrice indotta implica la presenza di un campo elettromotore indotto $\vb{E}_i$
    \begin{equation}
        \label{eqn:fi}
        \vb{E}_i=\vb{E}+\vb{v}_T\cp\vb{B}
    \end{equation}
    Dove $\vb{E}$ e $\vb{B}$ sono i campi elettrico e di induzione magnetica in cui il circuito è immerso
    e $\vb{v}_T$ è la velocità di trascinamento con cui si muove ciascun elemento infinitesimo di circuito.
\end{obs}
\begin{proof}
    Dalla definizione $f_i=\oint_l \vb{E}_i\vdot\dd{\vb{l}}$, che mostra la presenza del campo elettromotore indotto.
    Nel fenomeno dell'induzione i portatori di carica $q$ che danno orgine alla corrente nel circuito sono messi in moto
    da una forza
    \[
        \vb{F}=q\vb{E}+q\vb{v}_{tot}\cp\vb{B}
    \]
    Il campo elettromotore può quindi essere visto come rapporto fra questa forza e la carica dei portatori, ovvero
    \[
        \vb{E}_i=\vb{E}+\vb{v}_{tot}\cp\vb{B}=\vb{E}+(\vb{v}_T+\vb{v}_d)\cp\vb{B}
    \]
    dove $\vb{v}_T$ è la velocità di trascinamento e $\vb{v}_d$ la velocità di deriva delle cariche.
    Si ha quindi, considerando che la velocità di deriva è parallela punto per punto all'elemento $\dd{\vb{l}}$
    di volta in volta considerato
    \[
        f_i=\oint_l \vb{E}_i\vdot\dd{\vb{l}}=\oint_l(\vb{E}+\vb{v}_T\cp\vb{B})\vdot\dd{\vb{l}}
    \]
    Siccome la forza elettromotrice indotta è l'unica osservabile attraverso la quale sia possibile individuare
    il campo elettromotore indotto, allora si può tranqullamente identificare quest'ultimo con il secondo integrando.
\end{proof}


La corrente che circola nel circuito a sua volta genera un campo magnetico indotto $\vb{B}_i$.
\begin{cor}[Legge di Lenz]
    Il campo magnetico indotto $\vb{B}_i$ è tale che il suo verso si opponga
    a quello del campo magnetico che genera la forza elettromotrice indotta.
\end{cor}
\begin{proof}
    La dimostrazione è immediata, per il fatto che nella legge di Faraday-Neumann compare il segno meno
\end{proof}


\section{Il flusso tagliato}
L'obiettivo di questo paragrafo è duplice: dare un significato fisico più
concreto alla legge di Faraday-Neumann attraverso tre esempi significativi
e esporre il concetto di flusso tagliato, utile per i risultati
espressi nei prossimi paragrafi.

\begin{defn}[Flusso tagliato]
    Il flusso attraverso la superficie spazzata da un circuito in movimento è chiamato flusso tagliato.
\end{defn}


\subsubsection{Circuito variabile, B costante nel tempo}
$\vb{B}$ costante nel tempo significa che le caratteristiche delle sorgenti del campo non cambiano nel tempo
e che queste sorgenti sono in quiete nel sistema di riferimento scelto per descrivere il fenomeno.
\begin{thm}
    Dato un circuito di forma generica, il cui generico elemento $\dd{l}$ sia in moto con velocità $\vb{v}_T$,
    immerso in un campo $\vb{B}$ costante nel tempo, la variazione del flusso concatenato al circuito è
    pari in modulo e opposta in segno al flusso tagliato.
\end{thm}
\begin{proof}
    Detta $\Sigma$ la superficie orientata attraverso la quale viene calcolato il flusso concatenato all'istante $t_0$
    (indicato con $\Phi_i$), $\Sigma'$ la superficie attraverso cui viene calcolato il flusso $\Phi_f$ all'istante
    $t+\dd{t}$ e $\dd{\Sigma}$ la superficie spazzata dal circuito nel suo spostamento, si ha che l'unione di queste tre superfici
    è una superficie chiusa: il flusso totale -ovvero la somma dei flussi uscenti- deve essere nullo\footnote{Questo è vero solo
    nell'ipotesi di campo costante nel tempo, in quanto le superfici $\Sigma$ e $\Sigma'$ sono appoggiate al circuito
    in tempi diversi}. Si scelga convenzionalmente come verso di percorrenza positivo quello per cui la normale alla superficie del
    circuito sia parallela alla velocità complessiva del circuito. Allora
    \[
        -\Phi_i+\Phi_f+\Phi_{\dd{\Sigma}}=0
    \]
    dove il segno meno al primo termine è dovuto al fatto che con la convenzione scelta $\Phi_i$ è
    il flusso entrante nella superficie chiusa, mentre la somma è nulla quando riferita ai flussi uscenti.
    Si ha quindi che
    \[
        \dd{\Phi} \equiv \Phi_f-\Phi_i=-\Phi_{\dd{\Sigma}}
    \]
    ovvero la tesi.
\end{proof}
L'utilità di questo risultato risiede nel fatto che è possibile trovare un'espressione esplicita per il flusso
tagliato come mostrato nel seguente teorema.
\begin{thm}
    Dato un circuito di forma generica, il cui generico elemento $\dd{l}$ sia in moto con velocità $\vb{v}_T$,
    immerso in un campo $\vb{B}$ costante nel tempo, il flusso tagliato vale
    \[
        \Phi_{\dd{\Sigma}}=\dd{t}\oint_l(\vb{v}_T\cp\vb{B})\vdot\dd{l}
    \]
\end{thm}
\begin{proof}
    Si vuole calcolare $\Phi_{\dd{\Sigma}}$. Per farlo, siccome $\vb{B}$ è noto per ipotesi, bisogna esprimere
    $\dd{\vb{S}}$ in termini di grandezze note. Questo si ottiene considerando che
    ogni elemento di circuito compie uno spostamento $\dd{\vb{s}}=\vb{v}_T\dd{t}$ e nel
    farlo spazza la superficie $\dd{\vb{S}}=\dd{\vb{l}}\cp\dd{\vb{s}}=\dd{\vb{l}}\cp\vb{v}_T\dd{t}$.
    \[
        \Phi_{\dd{\Sigma}}=\int_{\dd{\Sigma}}\vb{B}\vdot\dd{\vb{S}}=\oint_l\vb{B}\vdot(\dd{\vb{l}}\cp\vb{v}_T\dd{t})
    \]
    Usando l'identità \eqref{app:eqn:vdot_cp} si ha la tesi.
    % \[
    %     \oint_l\vb{B}\vdot(\dd{\vb{l}}\cp\vb{v}_T\dd{t})=\oint_l(\vb{v}_T\dd{t}\cp\vb{B})\vdot\dd{\vb{l}}
    % \]
\end{proof}

\begin{cor}
    Nel caso di circuito in moto con velocità $\vb{v}_T$ in un campo $\vb{B}$ costante, il campo elettromotore indotto è
    \[
        \vb{E}_i=\vb{v}_T\cp\vb{B}
    \]
\end{cor}
\begin{proof}
    Per i due risultati appena dimostrati $\dv*{\Phi}{t}=-\Phi_{\dd{\Sigma}}/\dd{t}=-\oint_l(\vb{v}_T\dd{t}\cp\vb{B})\vdot\dd{\vb{l}}$.
    Confrontando il primo e l'ultimo membro dell'uguaglianza con la legge di Faraday-Neumann si ha la tesi.
\end{proof}
L'interpretazione fisica del fenomeno dell'induzione elettromagnetica è quindi immediata nelle ipotesi del corollario:
le cariche costrette a muoversi nel campo di induzione magnetica per via del moto del circuito, sono sottoposte alla forza di Lorentz.
Nonostante la forza di Lorentz non compia lavoro la forza elettromotice può essere diversa da $0$:
il lavoro dissipato dalla corrente nel circuito è compiuto dalla forza esterna che mantiene $\vb{v}_T$ costante
o è compiuto a spese dell'energia cinetica, per cui il circuito rallenta fino a fermarsi.



\subsubsection{Circuito rigido, sorgenti di B stazionarie in moto}
Si consideri un circuito $C$ in quiete nel sistema di riferimento inerziale $sr=Oxyz$. Supponendo che le
sorgenti di $\vb{B}$ siano in uno stato stazionario, ovvero con caratteristiche costanti nel tempo,
una variazione del flusso può avvenire solo se queste cariche si muovono rispetto a $sr$ con una velocità $\vb{v}$.
Nell'ipotesi che ci sia una sola sorgente del campo il cui moto sia
di pura traslazione con $v$ costante, è possibile considerare un sistema di riferimento inerziale
$sr'=O'x'y'z'$ rispetto al quale la sorgente è in quiete. $sr$ si muove rispetto ad $sr'$ con
velocità $\vb{v}'=-\vb{v}$. $sr'$ si trova allora nelle stesse condizioni discusse nel caso del circuito variabile e
campo costante nel tempo: le cariche risentono della forza di Lorentz ($\vb{F}'_l=q\vb{v}'\cp\vb{B}'$) come nel caso precedente.
In $sr$, dove il circuito è fermo, non si ha nessuna forza di Lorentz. A partire da considerazioni relativistiche però si può dedurre che
il moto delle sorgenti di $\vb{B}$ provochi l'insorgere di un campo $\vb{E}_l$ che esercita sui
portatori di carica del circuito una forza $\vb{F}_l=q\vb{E}_l$ equivalente alla forza di Lorentz $\vb{F}'_l$.
Per $v<<c$ si ha $\vb{B}\simeq\vb{B}'$, da cui $\vb{E}_l=\vb{v}'\cp\vb{B}'=-\vb{v}\cp\vb{B}$.

In virtù dell'additività di $\vb{B}$ il ragionamento fatto è estendibile al caso di più sorgenti.



\subsubsection{Circuito rigido, sorgenti di B ferme non stazionarie}
Il circuito $C$ e le sorgenti del campo di induzione magnetica non sono in moto relativo fra loro.
Le sorgenti di $\vb{B}$ non stazionarie sono dei circuiti nei quali le correnti di alimentazione
non sono costanti. L'effetto è che in ogni punto di $C$ il campo $\vb{B}$ sia variabile nel tempo,
esattamente come visto del caso del circuito in moto relativo con le sorgenti del campo.
È ragionevole perciò aspettarsi quindi che anche in questo caso si abbia come effetto un campo elettromotore,
sebbene questo non sia causato dalla forza di Lorenz. In effetti questo campo è presente ed è una
conseguenza della generalizzazione della terza equazione di Maxwell al caso non stazionario che
verrà presentata nel prossimo paragrafo.


\section{Equazioni di Maxwell -  caso non stazionario}
\subsection{Terza equazione di Maxwell}
Quanto visto evidenzia come in condizioni non stazionarie non sia garantita la conservatività del campo elettrico.
Si consideri infatti il caso in cui le sorgenti di $\vb{B}$ non siano stazionarie ma $\vb{v}_T=0$: $\vb{E}_i=\vb{E}$
e quindi $\oint \vb{E}_i=f_i=-dv{\Phi}{t}\neq 0$. Si rende perciò necessaria una generalizzazione della
terza equazione di Maxwell alla quale si perviene nel tenteativo di trovare una forma locale della legge di Faraday-Neumann.
\begin{thm}
    \begin{equation}
        \curl{\vb{E}}=-\pdv{\vb{B}}{t}
    \end{equation}
\end{thm}
\begin{proof}
    %    La dimostrazione viene presentata prima nel caso di circuito rigido in quiete e poi generalizzata.
    %    Si cosideri quindi un circuito fermo, la cui forma non cambia nel tempo.
    %    La variazione del flusso di $\vb{B}$ è quindi dovta alla variazione del campo stesso.
    %    La derivata temporale nella legge di Faraday-Newmann può quindi essere portata sotto
    %    al segno di intergale diventando derivata parziale. Per il teorema del rotore si ha quindi che
    %    \[
    %        -\int_S \pdv{\vb{B}}{t}\vdot\vu{n}\dd{S}=\oint_l \vb{E}_i\vdot\dd{\vb{l}}=\int_S(\curl{\vb{E}})\vdot\vu{n}\dd{S}
    %    \]
    %    dove nel secondo passaggio si è fatto uso, oltre al teorema del rotore, del fatto che per la $\eqref{eqn:fi}$ $\vb{E}_i=\vb{E}$
    %    essendo $\vb{v}_T=0$.
    %    Dato che la catena di uguaglianze vale qualsiasi sia la forma del circuito e qualsiasi sia la superficie di integrazione, si ha la tesi.

    Sia il circuito in moto e non rigido. Esplicitando la derivata e sviluppando al primo ordine $\vb{B}$ nella legge di Faraday-Newmann 
    si ottiene
    \[
        \begin{split}
            \oint_l \vb{E}_i\vdot\dd{\vb{l}}
            =&-\rec{\dd{t}}\Biggl[\int_{S(t+\dd{t})}\vb{B}(t+\dd{t})\vdot\dd{\vb{S}}-\int_{S(t)}\vb{B}(t)\vdot\dd{\vb{S}} \Biggr] \\
            =&-\rec{\dd{t}}\Biggl[\int_{S(t+\dd{t})}\vb{B}(t)\vdot\dd{\vb{S}}
            +\int_{S(t+\dd{t})}\pdv{\vb{B}(t)}{t}\dd{t}\vdot\dd{\vb{S}} -\int_{S(t)}\vb{B}(t)\vdot\dd{\vb{S}} \Biggr] \\
            =&-\rec{\dd{t}}\Biggl[\int_{S(t+\dd{t})}\vb{B}(t)\vdot\dd{\vb{S}}
            -\int_{S(t)}\vb{B}(t)\vdot\dd{\vb{S}} \Biggr] - \int_{S(t+\dd{t})}\pdv{\vb{B}(t)}{t}\vdot\dd{\vb{S}}\\
            =&\oint_l(\vb{v}\cp\vb{B})\vdot\dd{\vb{l}} - \int_{S(t+\dd{t})}\pdv{\vb{B}(t)}{t}\vdot\dd{\vb{S}}\\
        \end{split}
    \]
    dove nell'utlimo passaggio si è usato il fatto che il termine fra parentesi rappresenta la variazione di flusso
    relativa esclusivamente al moto del circuito ed è quindi possibile sfruttare i risultati ottenuti
    nel paragrafo sul flusso tagliato. Nel limite $\dd{t}\to 0$ si ha infine
    \[
        \oint_l \vb{E}_i\vdot\dd{\vb{l}}=\oint_l(\vb{v}\cp\vb{B})\vdot\dd{\vb{l}} - \int_{S(t)}\pdv{\vb{B}(t)}{t}\vdot\dd{\vb{S}}\\
    \]
    Il secondo termine rappresenta la variazione di flusso dovuta esclusivamente alla variazione
    nel tempo del campo di induzione magnetica. Per la linearità dell'intergale
    \[
        \oint_l (\vb{E}_i-\vb{v}\cp\vb{B})\vdot\dd{\vb{l}}= - \int_{S(t)}\pdv{\vb{B}(t)}{t}\vdot\dd{\vb{S}}\\
    \]
    Per la \eqref{eqn:fi} l'integrando a primo membro è proprio $\vb{E}$. 
    Per il teorema del rotore l'integrando a primo membro è uguale a $\int_{S(t)}\vb{E}\vdot\dd{\vb{S}}$.
    Siccome la procedura fin qui segutia vale qualunque sia il dominio di integrazione, l'uguaglianza ottenuta implica
    l'uguaglianza delle integrande, ovvero la tesi.
\end{proof}
Il risultato ottenuto si estende anche al caso in cui non siano presenti circuiti: nel vuoto, nei dielettrici o nei
conduttori, se il campo magnetico cambia nel tempo allora è presente un campo elettrico non conservativo.


\subsection{Prima e seconda equazione di Maxwell}
Vista la generalizzazione della terza equazione di Maxwell al caso non stazionario,
ci si pone come obiettivo quello di generalizzare le altre equazioni.
\begin{obses}
    Per quanto concerne la prima e la seconda equazione di Maxwell,
    si verifica sperimentalmente che la generalizzazione
    si ottiene banalmente introducendo la dipendenza dal tempo nella densità di carica elettrica.
    \[
        \begin{split}
            &\div{\vb{E}}=\frac{\rho(x,y,z,t)}{\epsilon_0}\\
            &\div{\vb{B}}=0
        \end{split}
    \]
\end{obses}

La prima di queste due equazioni può portare a conseguenze all'apparenza paradossali.
Infatti è immediato verificare che il teorema di Gauss continua a valere anche in condizioni non stazionarie,
ma il teorema di Gauss collega fra loro la carica presente all'interno di una superficie
ed il flusso sulla superficie: due quantità poste in posizioni diverse sono collegate istantaneamente.
La spiegazione del paradosso emerge considerando la legge di conservazione della carica:
la variazione della quantità di carica interna infatti può essere dovuta esclusivamente al passaggio di cariche
attracerso la superficie, perciò la variazione di flusso è un fenomeno essenzialmente locale

La conseguenza della seconda è che anche nel caso non stazionario
il flusso del campo magnetico attraverso una superficie chiusa è ad ogni istante nullo.


\subsection{Quarta equazione di Maxwell}
I procedimenti usati nei paragrafi precedenti non possono essere usati per generalizzare anche la quarta equazione di Maxwell, infatti

\begin{obs}
    L'estensione al caso non stazionario della quarta equazione di Maxwell non può essere
    ottenuto rendendo dipendenti dal tempo le grandezze che in essa compaiono.
\end{obs}
\begin{proof}
    Si applichi l'operatore divergenza alla quarta equazione di Maxwell
    \[
        \div(\curl{\vb{B}})=\div(\mu_0\vb{J})=\mu_0\div{\vb{J}}
    \]
    La divergenza del rotore è sempre nulla, quindi segue che $\div{\vb{J}}=0$.
    Dall'equazione di coninutià \eqref{eqn:continuità} si ottiene che questo è vero se e solo se
    \[
        \pdv{\rho}{t}=0
    \]
    Ovvero se e solo se la variazione della densità di carica è nulla
    e questo succede se e solo se ci si trova nel caso stazionario.
\end{proof}

La generalizzazione viene, di fatto, fatta un po' a culo.
\begin{defn}[Densità di corrente di spostamento]
    Si definisce densità di corrente di spostamento il vettore
    \[
        \vb{J}_S=\epsilon_0\pdv{\vb{E}}{t}
    \]
\end{defn}
\begin{defn}[Corrente di spostamento]
    Si definisce corrente di spostamento il flusso della densità di corrente di spostamento attraverso una superficie.
\end{defn}
Si osservi come dimensionalmente questa sia effettivamente una densità di corrente, infatti dalla definizione di campo
elettrico $[\epsilon_0 E]=[q/r^2]=C/m^2$ e quindi $[\epsilon_0\pdv*{E}{t}]=A/m^2=[J]$.
\begin{defn}[Densità di corrente totale generalizzata]
    Si definisce densità totale di corrente generalizzata $\vb{J}_t$
    \[
        \vb{J}_t=\vb{J}+\epsilon_0\pdv{\vb{E}}{t}
    \]
\end{defn}
La seguente osservazione si rivela fondamentale e giustifica l'introduzione della corrente di spostamento.
\begin{obs}
    La densità di corrente totale generalizzata si riduce alla densità di corrente nel caso stazionario e ha sempre divergenza nulla.
\end{obs}
\begin{proof}
    La dimostrazione della prima parte dell'asserto è banale e si ottiene semplicemente considerando che nel caso stazionario
    la variazione del campo elettrico è nulla. Per quanto riguarda la seconda parte invece, si consideri l'equazione di continuità
    \eqref{eqn:continuità} e usando le equazioni di Maxwell per introdurre i campi, ovvero sostituendo al posto della densità di carica
    l'espressione fornita dalla prima equazione di Maxwell, si ottiene
    \[
        \div{\vb{J}}+\pdv{\rho}{t}=\div{\vb{J}}+\epsilon_0\pdv{t}(\div{\vb{E}})=0
    \]
    Per il teorema di Schwartz è possibile scambiare la derivata parziale con la divergenza:
    \[
        0=\div{\vb{J}}+\div{\epsilon_0\pdv{\vb{E}}{t}}=\div(\vb{J}+\epsilon_0\pdv{\vb{E}}{t})
    \]
    Quello fra parentesi è proprio la densità di corrente totale generalizzata. Si è ottenuta così la tesi.
\end{proof}
Questi due risultati dicono che la densità di corrente totale generalizzata è un buon candidato per
la generalizzazione della quarta equazione di Maxwell che può essere ottenuta pertanto sostituendo
la densità di corrente usuale con la densità di corrente totale generalizzata:
\begin{equation}
    \curl{\vb{B}}=\mu_0\Biggl(\vb{J}+\epsilon_0\pdv{\vb{E}}{t} \Biggr)
\end{equation}

La validità di questa legge è confermata dall'accordo degli esperimenti coi risultati della teoria che
si sviluppa usando questa equazione. Il seguente esempio mostra come essa sia in grado di risolvere una
situazione all'apparenza paradossale.
\begin{example}[Scarica del condensatore]
    Si consideri un circuito RC aperto, con il condensatore inizialmente carico con carica $Q_0$.
    All'istante $t=0$ viene chiuso
    l'interruttore e il condensatore inizia a scaricarsi secondo la legge
    \[
        Q(t)=Q_0e^{-\frac{t}{RC}}
    \]
    Il campo elettrico vale quindi
    \[
        E=\rec{\epsilon_0}\frac{Q(t)}{S}
    \]
    e la corrente
    \[
        I(t)=\frac{Q_0}{RC}e^{-\frac{t}{RC}}
    \]
    Considero a questo punto un circuito di Ampère $c$ tra le maglie del condensatore e due superfici con questo circuito
    come contorno: $S_1$ interseca il filo del circuito collegato ad una maglia del condensatore; $S_2$ all'interno
    del volume delimitato dalle maglie del condensatore. Se valesse la "vecchia" quarta equazione di Maxwell, per il
    teorema di Ampère la circuitazione di $\vb{B}$ su $c$ varrebbe $I$ considerando $S_1$ e $0$ considerando $S_2$. Questa aporia viene
    risolta proprio dall'equazione di Maxewell generalizzata appena introdotta. Nel caso in cui si consideri $S_1$ infatti
    il campo elettrico attraverso la superficie è nullo e quidi continua a valere il teorema di Ampère. Nel caso in cui
    si consideri $S_2$ invece $\vb{J}=0$, ma $J_S=\dv*{E}{t}=I(t)/S$ da cui si ottiene una corrente in modulo uguale a $I$.
\end{example}

La generalizzazione di questa equazione ai materiali è insidiosa
ma porta all'intuibile risultato
\begin{equation}
    \curl{\vb{H}}=\vb{J}+\pdv{\vb{D}}{t}
\end{equation}
valido per qualsiasi tipo di materiale, compresi i ferromagneti.

Vengono esposte ora due conseguenze di quanto detto in questo paragrafo.

\begin{obs}[Prima legge di Kirchoff generalizzata]
    La legge di Kirchoff dei nodi è estendibile al caso non stazionario a patto di considerare sia la corrente di
    conduzione che quella di spostamento.
\end{obs}
\begin{proof}
    La dimostrazione è analoga a quella della legge di Kirchoff nel caso stazionario considerando che
    siccome la densità di corrente totale generalizzata ha divergenza nulla, è nullo il suo flusso attraverso qualsiasi
    superficie chiusa.
\end{proof}

\begin{obs}[Teorema della circuitazione di Ampère generalizzato]
    Il teorema della circuitazione di Ampère è estendibile al caso non stazionario a patto di considerare sia la corrente di
    conduzione che quella di spostamento.
\end{obs}
\begin{proof}
    Anche in questo caso la dimostrazione è analoga al caso stazionario, pur di prendere la quarta equazione di
    Maxwell generalizzata.
\end{proof}
Al fine del calcolo dei campi però, l'utilità pratica di questo teorema è limitata al caso quasi-stazionario,
ovvero al caso in cui le dimensioni del sistema siano tali per cui il tempo impiegato dai segnali
elettromagnetici per attraversarlo sia molto piccolo rispetto al tempo che caratterizza le variazioni di
$\rho$ e $\vb{J}$. In questo caso infatti la corrente di spostamento è con buona approssimazione trascurabile all'interno
dei buoni conduttori fino a frequenze nell'ordine delle microonde.



\section{L'autoinduzione}
L'analisi proposta in questo e nel prossimo paragrafo è fatta per materiali omogenei ed isotropi in cui $\mu$ sia costante.

Si consideri un circuito in condizioni quasi-stazionarie, il che implica che la corrente abbia
lo stesso valore lungo tutto il circuito, percorso da una corrente variabile nel tempo $I(t)$.
Questa corrente genera nello spazio circostante un campo di induzione magnetica $\vb{B}(t)$ il cui
flusso concatenato al circuito è in generale diverso da zero e variabile.
Per la legge di Faraday-Neumann si genera nel circuito una forza elettromotrice detta \textit{autoindotta}.
\begin{obs}
    \begin{equation}
        \label{eqn:induttanza}
        \Phi(\vb{B})=LI
    \end{equation}
\end{obs}
\begin{proof}
    per la \eqref{eqn:dB} il campo di induzione magnetica è proporzionale alla corrente che percorre il circuito.
    Poichè il flusso elementare di $\vb{B}$ attraverso ogni elemento di superficie $\dd{\vb{S}}$ è
    proporzionale a $\vb{B}$, anche il flusso concatenato deve essere proporzionale ad $I$.
\end{proof}
\begin{defn}[Induttanza]
    La costante di proporzionalità che compare nell'osservazione precedente si chiama induttanza.
\end{defn}
L'unità di misura per l'induttanza è detta \textit{henry}
\[
    [L]=\frac{w}{A}=\frac{V \vdot s}{A}=\Omega \vdot s = H
\]

\begin{obs}
    L'induttanza per un solenoide nell'approssimazione di solenoide infinito vale
    \[
        L=\mu n^2 lS=\mu N^2 \frac{S}{l}
    \]
\end{obs}
\begin{proof}
    Il campo di induzione magnetica per un solenoide vale in modulo
    \[
        B=\mu nI
    \]
    ed è diretto assialmente. Il flusso concatenato ad una singola spira è $=SB$
    e quindi il flusso concatenato a tutto il circuito vale $\Phi(\vb{B})=NS\mu nI$.
    Ma allora l'induttanza vale
    \[
        L=\frac{\Phi(\vb{B})}{I}=\mu n^2 lS=\mu N^2 \frac{S}{l}
    \]
\end{proof}
Nelle ipotesi\footnote{Quasi-stazionarietà e materiale omogeneo e isotropo}, dalla legge di Faraday-Neumann si ha immediatamente
l'espressione per la forza elettromotrice autoindotta: $f_a=-L\dd{I}/\dd{t}$.
Se il circuito con resistenza $R$ è sede di una corrente variabile nel tempo, l'equazione del circuito è
\begin{equation}
    \label{eqn:RL}
    RI=f+f_a=f-L\dv{I}{t}
\end{equation}
Quando la geometria del circuito è semplice il termine $f_a$ risulta trascurabile, riducendosi quindi al caso stazionario.
Quando questo termine aggiuntivo non è trascurabile (ad esempio, per la presenza nel circuito di un solenoide) si indica
esplicitamente la presenza di un componente autoinduttivo.
\begin{example}
    Dopo aver chiuso  un circuito LR (circuito RL in fase di carica) la corrente nel tempo varia secondo la legge
    \[
        I=I_m\Bigl(1-e^{-\frac{t}{\tau}}\Bigr)
    \]
    con $I_m=f/R$ e $\tau=L/R$.
    Una volta chiuso il circuito quindi la corrente aumenta passando da $0$ a $I_m$, ovvero il valore
    che verrebbe raggiungo istantaneamente in assenza di induttanza.

    Per arrivare a questo risultato bisogna semplicemente risolvere l'equazione del circuito con $f$ costante
    e condizione iniziale $I(0)=0$.
    Con la notazione introdotta l'equazione può essere riscritta come
    \[
        I=I_m-\tau \dv{I}{t}
    \]
    Con la sostituzione $I\rightarrow I_m-I=u$ si ha $\dv*{I}{t}=-\dv*{u}{t}$ e $u(0)=I_m$. Di conseguenza
    \[
        \frac{\dv{u}{t}}{u}=-\rec{\tau}
    \]
    Integrando ambo i membri da $0$ a $t$
    \[
        \ln(u)-\ln(u(0))=-\frac{t}{\tau}
    \]
    Mediante sostituzione inversa si ottiene l'equazione cercata.
    Se ora in \eqref{eqn:RL} si isola $f_a$ e si inserisce l'andamento appena trovato per la corrente, si ottiene che
    la differenza di potenziale ai capi dell'induttanza vale $f_a=-fe^{-\frac{t}{\tau}}$, ovvero
    $f_a$ si oppone quindi ad $f$ ma diminusce col tempo fino ad azzerarsi.
\end{example}

\begin{thm}[Legge di Felici]
    \label{teo:felici}
    Sia data una spira con area $S$ e costituita da $N$ avvolgimenti, di resistenza $R$ e immersa in un campo
    di induzione magnetica costante nel tempo. Detta $Q$ la carica totale che attraversa il circuito
    \[
        Q=\frac{\Phi_i-\Phi_f}{R}
    \]
    ovvero la carica totale che attraversa il circuito dipende solo dallo stato iniziale e dallo stato finale.
    In particolare, se la spira viene portata in una zona priva di campo magnetico
    \[
        Q=\frac{NS[B]}{R}
    \]
    con $[B]$ il valor medio del campo di induzione magnetica sulla superficie della spira.
\end{thm}
\begin{proof}
    Dimostriamo la prima parte dell'asserto.
    In assenza di generatori, l'unica forza elettromotrice nel circuito è quella autoindotta la cui espressione
    è data dalla legge di Faraday-Neumann. Per la prima legge di Ohm la corrente nel circuito allora vale
    \[
        I=\rec{R}f_i=-\rec{R}\pdv{\Phi_S(\vb{B})}{t}
    \]
    La carica totale che scorre nel circuito dall'istante iniziale al tempo $t$ allora vale
    \[
        Q=\int_0^t I\dd{t}=-\rec{R}\int_0^t\pdv{\Phi_S(\vb{B})}{t}\dd{t}=\frac{\Phi_i-\Phi_f}{R}
    \]
    Per la seconda parte, si ha che la spira è inizialmente posta in quiete nel campo
    \[
        \Phi_i(\vb{B})=N\int_S \vb{B}\vdot\dd{\vb{S}}=NS\Biggl(\rec{S}\int_S \vb{B}\vdot\dd{\vb{S}} \Biggr)=NS[B]
    \]
    Se si porta la spira in una regione in cui il campo di induzione magnetica è nullo $\Phi_f(\vb{B})=0$ e quindi
    per quanto mostrato
    \[
        Q=\frac{NS[B]}{R}
    \]
\end{proof}


\section{L'induzione mutua}
Si considerino due circuiti $C_1$ e $C_2$ in condizioni quasi-stazionarie, immersi in un mezzo
omogeneo e isotropo,
e si supponga che $C_1$ sia percorso all'istante $t$ da una corrente $I_1(t)$.
Il circuito $C_1$ è quindi sorgente di un campo $\vb{B}_1(t)$ che per la \eqref{eqn:induttanza}
è proporzionale a $I_1(t)$. Allora anche il flusso di $\vb{B}_1$ concatenato a $C_2$
è proporzionale a $I_1$, ovvero $\Phi_2(\vb{B}_1)=M_{12}I_1$.
\begin{defn}[Induttanza mutua]
    Si definisce induttanza mutua il coefficente di proporzionalità nell'equazione precedente
\end{defn}
Ovviamente, mediante un ragionamento analogo si può definire il coefficente $M_{21}$.
Il seguente teorema garantisce la non ambiguità della definizione.
\begin{thm}
    \[
        M_{12}=M_{21}
    \]
\end{thm}
\begin{proof}
    Ricordando l'equazione definitoria del potenziale vettore \eqref{eqn:def_potenziale_vettore} si può scrivere
    \[
        \Phi_2(\vb{B}_1)=\int_{S_2}\vb{B_1}\vdot\dd{\vb{S}_2}=\int_{S_2}\curl{\vb{A}_1}\vdot\dd{\vb{S}_1}=\oint_{l_2}\vb{A}_1(\vb{r}_2)\vdot\dd{\vb{l}_2}
    \]
    Sostituendo la \eqref{eqn:potenziale_vettore_circuito_lineare}
    \[
        \Phi_2(\vb{B}_1)=I_1\frac{\mu}{4\pi}\oint_{l_2}\oint_{l_1}\frac{\dd{\vb{l}_1}\vdot\dd{\vb{l}_2}}{\abs{\vb{r}_2-\vb{r}_1}}
    \]
    Confrontando ora quanto ottenuto con la definizione di induzione mutua si ha
    \[
        M_{12}=\frac{\mu}{4\pi}\oint_{l_2}\oint_{l_1}\frac{\dd{\vb{l}_1}\vdot\dd{\vb{l}_2}}{\abs{\vb{r}_2-\vb{r}_1}}
    \]
    L'espressione è indipendente dall'ordine degli indici, si ha quindi la tesi.
\end{proof}


\section{L'energia magnetica}
Nel caso statico non si è affrontato il discorso relativo all'energia magnetica. Questo perchè la creazione
di una configurazione stazionaria di correnti (e campi magnetici associati) richiede che esista
un periodo iniziale di transizione in cui le correnti si vanno da $0$ al loro valore finale. Per quanto
visto in questo capitolo, in questa fase sul sistema vengono indotte delle forze elettromotrici.
Siccome l'energia posseduta dal campo magnetico è l'energia che è stata necessaria alla sua produzione
questo contributo non può essere tralasciato.

L'oggetto attraverso il quale si passa per studiare l'energia magnetica è l'induttanza.
Si consideri quindi un circuito $RL$, descritto dall'equazione \eqref{eqn:RL}. Moltiplicando per la quantità $\dd{Q}=I\dd{t}$ si ottiene
\[
    f\dd{Q}=RI^2\dd{t}+LI\dd{I}
\]
Dalla definizione di forza elettromotrice emerge che il primo membro dell'equazione è l'energia erogata dal
generatore nell'intervallo di tempo $\dd{t}$. Per la legge sull'effetto Joule
\eqref{eqn:effetto_joule} il primo termine del secondo membro rappresenta l'energia dissipata nella resistenza.
Il secondo termine al secondo membro è quello invece che non ci sarebbe se non ci fosse induttanza:
lo si può quindi interpretare come l'energia da fornire all'induttanza affinchè la corrente circolante
in essa si porti nel tempo $\dd{t}$ dal valore $I$ al valore $I+\dd{I}$.
L'energia necessaria allora affinchè la corrente circolante nell'induttanza si porti da $0$ ad $I$ è
\begin{equation}
    U_L=\int_0^I LI\dd{I}=\rec{2}LI^2
    \label{eqn:U_induttore}
\end{equation}
Questa interpretazione è confermata dal seguente calcolo
\begin{obs}
    L'energia $U_L$ è proprio l'energia posseduta da un'induttanza $L$ percorsa da corrente $I$
\end{obs}
\begin{proof}
    Si consideri un circuito percorso da una corrente $I_m=f/R$. All'instante iniziale,
    il generatore viene staccato dal circuito il quale, viene chiuso in corto circuito. L'equazione del circuito è
    diventa quindi
    \[
        RI=-L\dv{I}{t}, \quad\quad I(0)=I_m
    \]
    la cui soluzione è
    \[
        I=I_me^{t/\tau}
    \]
    con $\tau=L/R$. La potenza dissipata per effetto Joule è $RI^2$ e quindi l'energia totale dissipata dalla resistenza è
    \[
        U_R=\int_0^\infty RI^2\dd{t}=\int_0^\infty \frac{f^2}{R}e^{-2t/\tau}\dd{t}=\frac{f^2}{2R^2}L=\rec{2}LI_m^2
    \]
    Per conservazione dell'energia, questa non può che essere l'energia posseduta dall'induttanza prima di chiudere il circuito.
\end{proof}
$U_L$ rappresenta l'energia potenziale immagazzinata nel dispositivo -ovvero il singolo circuito.
Si consideri ora un solenoide percorso da corrente costante all'interno del quale sia parzialmente inserito un nucleo di ferro.
Si verifica sperimentalmente che il nucleo viene risucchiato all'interno del solenoide. Questi fenomeno è in apparenza contraddittorio:
a nucleo completamente inserito infatti l'induttanza è maggiore\footnote{Si ricordi che l'induttanza di un solenoide è
$L=\mu_r\mu_0 N^2S/l$ e che per il ferro $\mu_r>0$.} e quindi lo è anche l'energia del solenoide - ovvero, apparentemente
il sistema si è spostato spontaneamente da una configurazione con energia minore ad una con energia maggiore. La spiegazione
si basa su un discorso analogo a quello svolto nel paragrafo \ref{par:corrente_correntestazionaria_complementi}
sulla forza fra le maglie di un condensatore: l'ipotesi che la corrente sia costante richiede implicitamente
che al circuito sia collegato un generatore in grado di soddisfare questa richiesta.

Si vuole ora calcolare l'energia potenziale $U_M$ immagazzinata in un insieme di più circuiti, soggetti quindi
a mutua induzione.
\begin{example}
    Si consideri una regione di spazio uniformemente riempita con un materiale
    con $B/H=\mu$ costante
    in cui sia presente un campo $\vb{B}$.
    Si consideri un circuito $RL$ costituito da un solenoide con una lughezza molto maggiore del suo diametro in fase di carica,
    descritto dall'equazione \eqref{eqn:RL}.
    Il campo interno al solenoide è uniforme e coincide col campo medio. Tenuto conto della \eqref{eqn:induttanza}
    e della legge di Felici \ref{teo:felici} si ottiene quindi
    \[
        L\dv{I}{t}=NS\dv{B}{t}
    \]
    Sostituendo quindi nella \eqref{eqn:RL} e moltiplicando per $I\dd{t}$, si ricava
    \[
        fI\dd{t}=RI^2\dd{t}+INS\dd{B}
    \]
    Per le considerazioni fatte all'inizio del paragrafo, l'energia fornita all'induttanza è, indicando con $n=N/l$
    \[
        \dd{U}_L=INS\dd{B}=SlnI\dd{B}=nVI\dd{B}
    \]
    dove $V=Sl$è il volume del solenoide. Dividendo per $V$ in modo da ricavare la densità di energia
    e tenendo presente la \eqref{eqn:H_solenoide}, si ottiene
    \[
        \dd{u}_L=\frac{\dd{U}_L}{V}=nI\dd{B}=H\dd{B}
    \]
    Fin'ora, non si è fatto uso dell'ipotesi di $\mu$ costante, quindi i risultati ottenuti sono indipendenti dal fatto che il
    materiale sia paramagnetico, ferromagnetico o diamagnetico\footnote{
        si noti invece l'ipotesi fortemente restrittiva di avere a che fare con un solenoide, ovvero con un campo uniforme su un volume
        di geometria semplice}.
    Considerando anche questa ipotesi
    \[
        u_M=u_L(B)=\int_0^B \dd{u}_L=\int_0^B H\dd{B}=\int_0^B \rec{\mu}B\dd{B}=\rec{2\mu}B^2=\rec{2}BH=\rec{2}\mu H^2
    \]


    la densità di energia magnetica vale in conclusione
    \begin{equation}
        u_M=\rec{2}\mu H^2
    \end{equation}
\end{example}

\begin{thm}
    L'energia magnetica per un numero $N$ di circuiti è
    \begin{equation}
        U_M=\rec{2}\sum_{i,j=1}^N M_{ij}I_iI_j
    \end{equation}
    con $M_{ii}=L_i$, o equivalentemente
    \begin{equation}
        \label{eqn:UM_flussi}
        U_M=\rec{2}\sum_{i=1}^N I_i\Phi_i
    \end{equation}
    dove $\Phi_i$ è il flusso di tutti i campi di induzione magnetica presenti nel sistema attraverso il circuito i-esimo.
\end{thm}
\begin{proof}
    Si parte dal caso più semplice di due soli circuiti e si arriva ad una generalizzazione ad $N$ circuiti.
    Il sistema di due circuiti è descritto da
    \[
        \begin{cases}
            &  f_1=L_1\dv{I_1}{t}+M_{12}\dv{I_2}{t}+R_1I_1 \\
            &  f_2=L_2\dv{I_2}{t}+M_{21}\dv{I_1}{t}+R_2I_2 \\
        \end{cases}
    \]
    Moltiplicando la prima equazione per $\dd{Q_1}=I_1\dd{t}$ e la seconda per $\dd{Q_2}=I_2\dd{t}$,
    ricordando che $M_{21}=M_{12}$, poi sommando membro a membro si ottiene
    \[
        (f_1I_1\dd{t}+f_2I_2\dd{t})=(I_1^2R_1+I_2^2R_2)\dd{t}+[L_1I_1\dd{I_1}+L_2I_2\dd{I_2}+M_{12}(I_1\dd{I_2}+I_2\dd{I_1})]
    \]
    Per confronto, il membro fra parentesi quadre rappresenta il differenziale di energia magnetica, ovvero l'energia da fornire
    per incrementare di $\dd{I}$ le correnti $I_1$ e $I_2$.
    \[
        \dd{U_M}=L_1I_1\dd{I_1}+L_2I_2\dd{I_2}+M_{12}(I_1\dd{I_2}+I_2\dd{I_1})=\dd(\rec{2}L_1I_1^2+\rec{2}L_2I_2^2+M_{12}I_1I_2)
    \]
    Integrando da $0$ ai valori finali $I_1$, $I_2$
    \[
        U_M=\rec{2}L_1I_1^2+\rec{2}L_1I_1^2+M_{12}I_2I_2
    \]
    dove i primi due termini rappresentano l'energia dei signoli circuiti, mentre il terzo rappresenta l'energia di mututa induzione.
    Per l'uguaglianza fra i coefficenti di mutua induzione è possibile scrivere $M_{12}I_1I_2=1/2(M_{12}I_1I_2+M_{21}I_2I_1)$.
    Chiamando quindi $L_1=M_{11}$ ed $L_2=M_{22}$ la formula ottenuta può essere scritta come
    \[
        U_M=\rec{2}(M_{11}I_1^2+M_{12}I_1I_2+M_{21}I_2I_1+M_{22}I_2^2)=\rec{2}\sum_{i,j=1}^2M_{ij}I_iI_j
    \]

    Raccogliendo $I_1$ per i primi due termini del secondo membro e $I_2$ per gli altri due, si ha,
    dalle definizioni di induttanza e di coefficente di mutua induttanza:
    \[
        U_M=\rec{2}I_1(M_{11}I_1+M_{12}I_2)+\rec{2}I_2(M_{21}I_1+M_{22}I_2)=\rec{2}(I_1\Phi_1+I_2\Phi_2)=\rec{2}\sum_{i=1}^2I_i\Phi_i
    \]

    Entrambe le espressioni trovate per $U_M$ sono immediatamente generalizzabili al caso di $N$ circuiti.
\end{proof}


\begin{thm}
    Le forze agenti su un circuito percorso da corrente $I$ immerso in un campo magnetico esterno costante
    $\vb{B}_{ext}$ sono date dal gradiente (non cambiato di segno) dell'energia potenziale, che può essere scritta
    nel caso di un singolo circuito come
    \[
        U=I\Phi(\vb{B}_{ext})
    \]
\end{thm}
\begin{proof}
    Il sistema può essere schematizzato come costituio da
    $N$ circuiti, $N-1$ dei quali sono le sorgenti di $\vb{B}_{ext}$. Sia il circuito k-esimo percoro da corrente $I_k$ e abbia
    resistenza complessiva $R_k$. Si immagini di applicare una traslazione virtuale a questo circuito
    mediante una forza esterna $\vb{F}^{(k)}$ che imprima sul circuito una veloctià $\vb{v}^{(k)}$
    piccola, in modo tale che sia trascurabile l'energia cinetica associata a questo movimento.
    L'energia del circuito, ovvero l'energia magnetica\footnote{L'energia magnetica è l'energia potenziale totale
    immagazzinata nel circuito in quanto non sono presenti condensatori.}, varierà come conseguenza di tre contributi:
    $\vb{F}^{(k)}\vdot\dd{\vb{x}}$ lavoro compiuto
    dalla forza esterna; $\dd{L}_G$ lavoro compiuto dai generatori nel circuito; $\dd{L}_{R}$ lavoro
    dissipato (e quindi negativo) per effetto Joule. Per la conservazione dell'energia, rapportando tutto all'unità di tempo
    \[
        \dv{U_M}{t}=\vb{F}^{(k)}\vdot\dd{\vb{v}} + \dv{L_G}{t} - \dv{L_R}{t}
    \]
    Ricordando che
    \[
        \begin{split}
            &\dv{L_G}{t}=\sum I_j f_j=\sum I_j(I_j R_j + \dv{\Phi_j}{t} ) \\
            &\dv{L_R}{t}=\sum I^2_j R_j \\
            &\dv{U_M}{t}=\dv{t} \Biggl(\rec{2} \sum I_j \Phi_j \Biggr)=
            \rec{2} \sum I_j \dv{\Phi_j}{t}
        \end{split}
    \]
    Si ha
    \[
        \vb{F}^{(k)}\vdot\dd{\vb{v}}=-\rec{2}\sum I_j\dv{\Phi_j}{t}=-\dv{U_M}{t}
    \]
    Per garantire una traslazione virtuale a velocità trascurabile la forza esterna deve essere
    uguale in modulo e opposta alla forza $\vb{f}^{(k)}$ che il campo magentico esercita sul
    circuito k-esimo. Ne segue che $\vb{f}^{(k)}\vdot\dd{\vb{x}}=\dd{U_M}$, ovvero la forza è data
    dal gradiente di $U_M$.

    Svolgendo esplicitamente il gradiente dell'energia magnetica, se è costante la corrente ed il circuito è
    indeformabile (in modo che $L$ sia costante), le derivate dell'energia magnetica totale coincidono con le sole derivate dell'
    energia di accoppiamento, ovvero dell'energia dovuta all'interazione del circuito k-esimo con gli altri $N-1$ circuiti
    \[
        U_{acc}=\sum_{j\neq k} I_k \,M_{kj}I_j=I_k\Phi^{(k)}_{acc}(\vb{B})
    \]
    dove $\Phi^{(k)}_{acc}(\vb{B})$ è il flusso concatenato col k-esimo circuito, prodotto
    da tutti gli altri circuiti. Ma questo è proprio il flusso del campo esterno attraverso il
    circuito k-esimo, ovvero la tesi.
\end{proof}

Quelle ottenute sono espressioni di tipo integrale. È interessante cercare di trovare delle relazioni locali:
queste forme si riveleranno utili sia per generalizzare quanto visto fin'ora al caso non stazionario,
sia per trovare la densità di energia nella forma il più generale possibile.
\begin{thm}
    Dato sistema costituito da $N$ circuiti e un volume $\tau$ che racchiuda tutti i circuiti, l'energia magnetica del sistema è:
    \begin{equation}
        \label{eqn:UM_J_A}
        U_M= \rec{2}\int_\tau \Biggl(\vb{J}+\pdv{\vb{D}}{t}\Biggr)\vdot\vb{A}\dd{\tau}
    \end{equation}
\end{thm}
\begin{proof}
    In un sistema di $N$ circuiti si consideri nel dettaglio il circuito k-esimo.
    Questo sarà costituito da un conduttore di sezione non nulla $\sigma_k$ chiuso su se stesso
    e percorso da una corrente $I_k=\int_{\sigma_k}\vb{J}_k\vdot\dd{\vb{\sigma}_k}$.
    Si consideri la superficie $S_k$ che abbia $l_k$ come contorno. Per la \eqref{eqn:UM_flussi}
    \[
        \begin{split}
            U_M &=\rec{2}\sum_k I_k\Phi_k=\rec{2}\sum_k\Biggl[\int_{\sigma_k}\vb{J}_k\vdot\dd{\vb{\sigma}_k} \int_{S_k}\vb{B}\vdot\dd{\vb{S}_k} \Biggr]\\
            &=\rec{2}\sum_k\Biggl[\int_{\sigma_k}\vb{J}_k\vdot\dd{\vb{\sigma}_k} \int_{S_k}(\curl{\vb{A}})\vdot\dd{\vb{S}_k} \Biggr]=\\
            &=\rec{2}\sum_k\Biggl[\int_{\sigma_k}\vb{J}_k\vdot\dd{\vb{\sigma}_k} \oint_{l_k}\vb{A}\vdot\dd{\vb{l}_k} \Biggr]
        \end{split}
    \]
    Siccome $\dd{\vb{l}_k}$, $\vb{J}_k$, $\dd{\vb{\sigma}_k}$ sono paralleli, indicando con $\vu{l}_k$
    la direzioni di questi vettori, la relazione può essere riscritta come
    \[
        \begin{split}
            U_M&=\rec{2}\sum_k\Biggl[\int_{\sigma_k}J_k\dd{\sigma_k} \oint_{l_k}\vb{A}\vdot\vu{l}_k\dd{l_k} \Biggr]
            =\rec{2}\sum_k \int_{\sigma_k}\oint_{l_k}J_k\vu{l}_k \vdot \vb{A}\dd{\sigma_k}\dd{l_k} \\
            &=\rec{2}\sum_k \int_{\tau_k} \vb{J}_k \vdot \vb{A}\dd{\tau_k}
        \end{split}
    \]
    Siccome la densità di corrente è nulla al di fuori dei circuiti, si può elminiare la sommatoria
    trasformando l'integrale su $\tau_k$ in un integrale su un qualsiasi volume $\tau$ che racchiuda tutti i circuiti,
    ottenendo la tesi nel caso stazionario.

    Sostituendo la densità di corrente con la densità di corrente
    generalizzata, si ottiene la tesi nel caso generale.
\end{proof}

\begin{thm}
    Data una densità di corrente $J$ l'energia magnetica vale
    \begin{equation}
        \label{eqn:UM}
        U_M=\rec{2} \int_{S} (\vb{H}\cp\vb{A}) \vdot \dd{\vb{S}} +\rec{2} \int_{\tau}\vb{H}\vdot\vb{B}\dd{\tau}
    \end{equation}
\end{thm}
\begin{proof}
    Per la quarta equazione di Maxwell nel caso non stazionario, l'integrando nella \eqref{eqn:UM_J_A} può essere riscritto come
    \[
        \Biggl(\vb{J}+\pdv{\vb{D}}{t}\Biggr)\vdot\vb{A}=(\curl{\vb{H}})\vdot\vb{A}
    \]
    Tenuto conto dell'identità \eqref{app:eqn:div_cp}
    \[
        (\curl{\vb{H}})\vdot\vb{A}=\div(\vb{H}\cp\vb{A})+\vb{H}\vdot(\curl{\vb{A}})
    \]
    E indicando con $S$ la superficie che racchiude il volume $\tau$, $U_M$ può essere riscritta come
    \[
        \begin{split}
            U_M&=\rec{2} \int_{\tau} \div(\vb{H}\cp\vb{A})\dd{\tau} +\rec{2} \int_{\tau}\vb{H}\vdot(\curl{\vb{A}})\dd{\tau}\\
            &=\rec{2} \int_{S} (\vb{H}\cp\vb{A}) \vdot \dd{\vb{S}} +\rec{2} \int_{\tau}\vb{H}\vdot\vb{B}\dd{\tau}
        \end{split}
    \]
\end{proof}
L'equazione ottenuta è formalmente analoga alla \eqref{eqn:UE}: valgono tutte le considerazioni fatte in quel caso
e in particolare
\begin{cor}
    Se si prende in considerazione tutto lo spazio, allora l'energia vale
    \begin{equation}
        \label{eqn:uM_H_B}
        U_M=\int u\dd{\tau} \quad\quad\quad \text{ con } u=\frac{\vb{H}\vdot\vb{B}}{2}
    \end{equation}
\end{cor}
Un'osservazione su quanto ottenuto. Sia l'integranda della \eqref{eqn:UM_J_A} che la \eqref{eqn:uM_H_B}
godono della proprietà che il loro integrale esteso a tutto lo spazio sia pari all'energia magnetica.
Di fatto però, le due funzioni non sono necessariamente uguali punto per punto. Sono allora considerazioni di carattere puramente fisico
che portano ad interpretare la seconda come l'effettiva densità di energia.

Si conclude la sezione con un confronto fra l'energia magnetica e l'energia elettrostatica.
La prima analogia è fra le formule per l'energia di un'induttore \eqref{eqn:U_induttore} e quella per un condensatore \eqref{eqn:U_condensatore}.
In entrambi i casi la struttura della formula è quella di un prodotto fra l'integrale della densità della sorgente del campo
e la caratteristica del circuito. Inoltre, sia nel calcolo delle forze meccaniche dovute all'energia magnetica che di quelle dovute
all'energia elettrostatica si prendono le derivate positive dell'energia, contrariamente a quanto si fa nel caso meccanico.

