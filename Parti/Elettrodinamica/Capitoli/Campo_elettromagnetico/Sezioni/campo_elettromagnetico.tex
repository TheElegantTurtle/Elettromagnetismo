I campi $\vb{E}$ e $\vb{B}$, così come i campi $\vb{D}$ e $\vb{H}$, sono legati alle caratteristiche
delle sorgenti dalle equazioni di Maxwell
%lasciare questa riga vuota

\begin{minipage}[t]{0.5\textwidth}
    \[
        \begin{split}
            & \div{\vb{D}}=\rho                    \\
            & \curl{\vb{E}}=-\pdv{\vb{B}}{t}       \\
        \end{split}
    \]
\end{minipage}
\begin{minipage}[t]{0.5\textwidth}
    \[
        \begin{split}
            & \div{\vb{B}}=0                       \\
            & \curl{\vb{H}}=\vb{J}+\pdv{\vb{D}}{t} \\
        \end{split}
    \]
\end{minipage}
Alcune considerazioni sulle equazioni di Maxwell. Le sorgenti del campo elettrico ($\rho$) e del campo magnetico ($\vb{J}$)
non sono fra loro indipendenti ma sono legate dall'equazione di continuità \eqref{eqn:continuità} e ciò riflette
il fatto che fisicamente le stesse cariche elettriche, che danno luogo ad una densità di carica $\rho$, quando
sono in movimento danno luogo ad una densità di corrente. Il fatto che le cariche siano ferme o in
movimento dipende dal sistema di riferimento scelto per descrivere il fenomeno e quindi è relativo il fatto
di avere a che fare con un campo elettrico o con un campo magnetico.
Nelle quattro equazioni di Maxwell inoltre è evidente una certa asimmetria: al secondo membro della prima compare la densità di carica, mentre
il secondo membro della seconda è sempre nullo; allo stesso modo nel secondo membro della terza non compare un addendo
equivalente alla densità di corrente presente nella quarta. Questo riflette il fatto che non c'è evidenza sperimentale dell'esistenza
del monopolo magnetico, l'equivalente magnetico della carica elettrica.
Infine, le equazioni di Maxwell mostrano come la derivata temporale di $\vb{E}$ sia una delle sorgenti
di $\vb{B}$ e viceversa. Questa circostanza giustifica una trattazione unificata dei due fenomeni mediante un unico campo detto
\textit{campo elettromagnetico}. Bisogna osservare che un campo elettromagnetico può essere presente in regioni di spazio
prive di sorgenti e, più  avanti, si dimostrerà che esso possiede quantità di moto, energia e momento angolare. Si
tratta quindi di un'entità fisica e non di un mero artificio introdotto per trattare i due fenomeni in maniera unificata.
Nella trattazione del campo elettromagnetico si faranno due fondamentali approssimazioni: si considereranno le distribuzioni di
carica come continue, sebbene a livello microscopico la carica sia quantizzata e si considereranno i campi come funzioni
continue (approssimazione classica) nonostante la meccanica quantistica prescriva che il campo elettromagnetico sia quantizzato
in fotoni. Queste approssimazioni sono ragionevoli per la maggior parte dei fenomeni macroscopici.

Dalle considerazioni fatte emerge chiaramente l'importanza di studiare le proprietà delle equazioni di Maxwell.
Queste sono otto equazioni scalari in dodici incognite ($\vb{E},\vb{D}$,$\vb{B},\vb{H}$), per cui è necessario
fornire delle relazioni aggiuntive affinchè siano risolubili: queste sono le relazioni fenomenologiche che legano
$\vb{E}$ a $\vb{D}$ e $\vb{B}$ a $\vb{H}$. In questo in modo le incognite si riducono a sei. Le otto equazioni
non sono perciò indipendenti.
\begin{obs}
    La seconda equazione di Maxwell può essere dedotta dalla terza, analogamente la prima può esere dedotta dalla quarta\footnote{
        la dimostrazione viene presentata usando le equazioni nel vuoto. Nel caso dei materiali si procede in modo analogo.}.
\end{obs}
\begin{proof}
    Applicando l'operatore divergenza alla terza equazione di Maxwell si ha
    \[
        -\div{\pdv{\vb{B}}{t}}=-\pdv{t}\div{\vb{B}}=\div(\curl{\vb{E}})=0
    \]
    dove si è usato il teorema di Schwarz ed il fatto che la divergenza del rotore è nulla.
    Questo implica che $\div{\vb{B}}=cost$. Prima che venissero realizzate le sorgenti di $\vb{B}$, il campo era nullo
    in tutto lo spazio e anche la sua divergenza doveva quindi essere nulla. Siccome tutte le equazioni usate sono
    indipendenti da ciò che accade alle sorgenti, la costante non può che essere $0$.
    In maniera analoga, applicando la divergenza alla quarta equazione di Maxwell si ha
    \[
        0=\div{\vb{J}} + \div{\pdv{\vb{D}}{t}}=-\pdv{\rho}{t} + \pdv{t}\div{\vb{D}}=\pdv{t}(-\rho + \div{\vb{D}})
    \]
    Avendo sostituito $\div{\vb{J}}$ con l'espressione fornita dall'equazione di continuità \eqref{eqn:continuità}.
    Ne segue che $-\rho + \div{\vb{D}}=cost$ dalla quale si arriva alla prima equazione di Maxwell
    per considerazioni analoghe a quelle fatte nel caso precedente.
\end{proof}
Ne segue quindi che tutte le proprietà dei campi elettrici e magnetici sono contenute nella terza
e nella quarta equazione di Maxwell, ovvero un sistema di sei equazioni in sei incognite.
Specificata la configurazione spazio-temporale delle sorgenti $\rho$ e $\vb{J}$ (configurazione che dovrà rispettare
l'equazione di continuità), assegnate le condizioni iniziali e le condizioni al contorno, le equazioni di Maxwell
corredate dalle opportune relazioni fenomenologiche consentono in linea di principio di
calcolare i campi, tramite i quali è immediatamente nota l'azione subita da una carica campione (grazie alla relazione
$\vb{F}=q(\vb{E}+\vb{v}\cp\vb{B})$).


Un caso in cui è particolarmente semplice risolvere le equazioni di Maxwell è quello del
campo elettromagnetico in cui un solo materiale isotropo e omogeneo riempia lo spazio.
In questo caso è necessario aggiungere alle sorgenti macroscopiche dei campi
anche le correnti e le cariche microscopiche che i campi inducono nel materiale, che non sono note a priori proprio perchè
indotte dal campo. Nell'ipotesi però che il materiale sia omogeneo ed isotropo (e che non sia ferromagnetico, a meno che
$B(H)$ sia una curva univoca e lineare),
è sufficiente considerare le equazioni di Maxwell analoghe a quelle nel vuoto, con $\epsilon$ e $\mu$
al posto di $\epsilon_0$ e $\mu_0$, ovvero introducendo $\vb{D}=\epsilon \vb{E}$  e $\vb{H}=\vb{B}/\mu$.
I risultati ottenuti nel vuoto restano quindi validi validi.
Nel caso in cui diverse porzioni dello spazio siano rimepite con materiali diversi,
il problema viene trattato risolvendo le equazioni di Maxwell in ciascuna porzione di spazio e imponendo poi le condizioni di raccordo

\begin{minipage}[t]{0.5\textwidth}
    \[
        \begin{cases}
            & E_{t1}=E_{t2}\\
            & D_{n1}=D_{n2}
        \end{cases}
    \]
\end{minipage}
\begin{minipage}[t]{0.5\textwidth}
    \[
        \begin{cases}
            & H_{t1}=H_{t2}\\
            & B_{n1}=B_{n2}
        \end{cases}
    \]
\end{minipage}
che ovviamente, possono essere semplificate come visto sopra nel caso in cui tutti i materiali siano isotropi e omogenei.

È possibile dalle equazioni di Maxwell ricavare informazioni sulle caratteristiche delle sorgenti miscroscopiche
indotte dai campi, riscrivendo le equazioni usando i vettori polarizzazione elettrica $\vb{P}$ e magnetica $\vb{M}$
al posto di $\vb{D}$ ed $\vb{H}$, per semplice inversione delle definizioni
\[
    \begin{split}
        & \vb{D}=\epsilon_0 \vb{E} + \vb{P} \\
        & \vb{H}=\rec{\mu_0}\vb{B} - \vb{M} \\
    \end{split}
\]
Si ottiene in questo modo
\[
    \begin{split}
        & \epsilon_0\div{\vb{E}}=\rho-\div{\vb{P}}\\
        & \div{\vb{B}}=0 \\
        & \curl{\vb{E}}=-\pdv{\vb{B}}{t}\\
        & \rec{\mu_0}\curl{\vb{B}}=\vb{J}+\epsilon_0\pdv{\vb{E}}{t}+\pdv{\vb{P}}{t}+\curl{\vb{M}}
    \end{split}
\]
Queste 8 equazioni (ricoducibili a 6 fra loro indipendenti) non sono sufficienti a ricavare le
12 grandezze scalari $\vb{E}$, $\vb{B}$, $\vb{P}$, $\vb{M}$ a meno che i vettori di polarizzazione
siano noti a priori, ovvero siano note a priori le condizioni che li legano ai relativi
campi. Questo non avviene praticamente mai ma una volta risolte le equazioni con $\vb{D}$ e $\vb{H}$ è possibile
determinare le caratteristiche delle sorgenti microscopiche per confronto.
Infatti, rispetto alle equazioni di Maxwell nel vuoto, le equazioni appena ricavate differiscono per tre termini aggiuntivi:
\[
    \rho'_p=-\div{\vb{P}} \quad\quad \vb{J}'_M=\curl{\vb{M}} \quad\quad \vb{J}'_p=\pdv{\vb{P}}{t}
\]
che rappresentano rispettivamente la densità di carica di polarizzazione, la densità di corrente di
magnetizzazione e la densità di corrente di polarizzazione elettrica, ovvero la densità di corrente dovuta al fatto che quando
un dielettrico si polarizza necessariamente c'è un movimento ordinato di cariche all'interno del materiale. Quest'ultimo
è un termine caratteristico del caso non stazionario a differenza dei primi due che invece erano presenti già nel caso stazionario.

Da qui in avanti si farà riferimento solo ed esclusivamente a mezzi isotropi e omogenei.
