I fenomeni elettromagnetici sono soggetti al principio di conservazione dell'energia. Questo
significa che l'energia del campo magnetico più l'energia dei sistemi con cui il campo
interagisce deve essere costante nel tempo.
Si introduce la seguente definizione
\begin{defn}[vettore di Poynting]
    Si definisce vettore di Poynting
    \begin{equation}
        \label{eqn:poynting}
        \vb{I}=\vb{E}\cp\vb{H}=\frac{\vb{E}\cp\vb{B}}{\mu}
    \end{equation}
\end{defn}
Si ha che $[I]=J/(m^2s)=W/m^2 $.
\begin{thm}[teorema di Poynting]
    Detta $S$ una superficie chiusa di forma costante che racchiude il campo elettromagnetico e detto $\tau$ il volume interno a questa
    superficie, si ha che la variazione di energia del campo magnetico vale
    \begin{equation}
        \label{eqn:dU_em}
        -\pdv{U}{t}=\int_S \vb{I}\vdot\dd{\vb{S}}+\int_{\tau} \vb{E}\vdot\vb{J}\dd{\tau}
    \end{equation}
\end{thm}
\begin{proof}
    L'energia posseduta dal campo elettromagnetico contenuto in $S$ è
    \[
        U=\int_{\tau}\rec{2}\vb{E}\vdot\vb{D}\dd{\tau}+\int_{\tau}\rec{2}\vb{H}\vdot\vb{B}\dd{\tau}
    \]
    Derivando rispetto al tempo, tenendo conto che $\vb{D}=\epsilon \vb{E}$ e $\vb{B}=\mu\vb{H}$ si ha
    \[
        \begin{split}
            \pdv{U}{t}&=\rec{2}\int_{\tau}\Biggl[\pdv{t} (E^2\epsilon  ) + (H^2\mu  ) \Biggr] \dd{\tau}
            =\rec{2}\int_{\tau}\Biggl[2\Biggl( \pdv{\vb{E}}{t} \Biggr)\vdot \vb{E}\epsilon  +
            2\Biggl(\pdv{\vb{H}}{t} \Biggr)\vdot \vb{H}\mu \Biggr]\dd{\tau} \\
            &=\int_{\tau}\Biggl[\Biggl( \pdv{\vb{D}}{t} \vdot \vb{E} \Biggr) +
            \Biggl( \pdv{\vb{B}}{t} \vdot \vb{H} \Biggr)\Biggr] \dd{\tau}
        \end{split}
    \]
    Sostituendo le derivate sotto al segno di integrale con le espressioni fornite dalla terza e dalla quarta equazione di Maxwell
    \[
        \pdv{U}{t}=\int_{\tau}\Biggl[ (\curl{\vb{H}}) \vdot \vb{E} -\vb{E}\vdot\vb{J} - (\curl{\vb{E}}) \vdot \vb{H} \Biggr]\dd{\tau}
    \]
    Per l'identità \eqref{app:eqn:div_cp}
    \[
        \pdv{U}{t}=-\int_{\tau}\Biggl[ \div(\vb{E} \cp \vb{H}) + \vb{E}\vdot\vb{J} \Biggr]\dd{\tau}
    \]
    Separando ora i due integrali e usando il teorema della divergenza si ottiene quindi
    \[
        -\pdv{U}{t}=\int_{S} (\vb{E} \cp \vb{H})\vdot\dd{\vb{S}} + \int_{\tau} \vb{E}\vdot\vb{J} \dd{\tau}=
        \int_{S} \vb{I} \vdot\dd{\vb{S}}+ \int_{\tau} \vb{E}\vdot\vb{J} \dd{\tau}
    \]
\end{proof}

Si vuole ora dare un'interpretazione fisica a questo risultato.
Il flusso del vettore di Poynting, se positivo, ovvero se $\vb{I}$ è "uscente" dal volume,
comporta una diminuzione di energia nel tempo. $\vb{I}$ rappresenta lo spostamento di
energia dentro o fuori dal volume preso in considerazione. Per afferrare il concetto, può essere
utile scrivere il risultato del teorema in forma locale
$-\pdv{t}\int_\tau u \dd{\tau}=\int_\tau \div{\vb{I}} \dd{\tau} +\int \vb{E}\vdot\vb{J}\dd{\tau}$
da cui segue che $\-\pdv{t}u=\div{\vb{I}}+\vb{E}\vdot{J}$. Se non fosse presente il termine
$\vb{E}\vdot\vb{J}$ si avrebbe un'equazione di continuità per l'energia:
\[
    \div{\vb{U}}+\pdv{u}=0
\]
Questa equazione afferma che se si ha una variazione di energia si deve avere anche una "corrente di energia"
che la porti da qualche parte. Emerge a questo punto il significato del termine $\vb{E}\vdot\vb{J}$ che
rappresenta una variazione dell'energia elettromagnetica per trasformazione da (o in) altre forme di energia: è
in pratica l'energia ceduta o ricevuta dalla materia per azione del campo elettromagentico sulle cariche che
la costituiscono,
Si considerino a tal proposito le cariche contenute in una
porzione infinitesima $\dd{\tau}$ del volume $\tau$. Il numero di cariche per unità di volume
è $n=\dd{N}/\dd{\tau}$. La forza esercitata dal campo elettromagnetico su $\dd{\tau}$ è
\[
    \dd{\vb{F}}=\dd{N}q(\vb{E+\vb{v}_d\cp\vb{B}})=nq(\vb{E+\vb{v}_d\cp\vb{B}})\dd{\tau}
\]
La potenza trasferita dal campo magetico alle cariche libere presenti nell'elemento di volume è dunque
\begin{equation}
    \dd{P}=\dd{F}\vdot\vb{v}_d=nq\vb{v}_d\vdot (\vb{E}+\vb{v}_d\cp\vb{B})\dd{\tau}=nq\vb{v}_d\vdot \vb{E} \dd{\tau}=\vb{E}\vdot\vb{J}\dd{\tau}
    \label{eqn:potenza_em}
\end{equation}
Risulta subito evidente dai passaggi come la forza causata dal campo magnetico
non compiendo alcun lavoro non contribusca alla variazione di potenza.

Complessivamente allora la \eqref{eqn:dU_em} dice che la variazione di energia è dovuta sia all'interazione
del campo con la materia contenuta nel volume che al flusso del vettore di Poynting attraverso la superifice
che delimita il volume. Il vettore di Poynting è allora quel
vettore il cui flusso rappresenta l'energia del campo elettromagnetico che sfugge attraverso $S$.
