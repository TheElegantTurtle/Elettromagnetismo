Oltre che tramite le equazioni di Maxwell, tutte le caratteristiche del campo elettromagnetico possono essere ottenute fornendo i
quattro potenziali generalizzati $V,\vb{A}$. Il fatto che per ricavare i potenziali, da cui è deducibile il campo elettromagnetico,
servano quattro equazioni e che ne servano sei per dedurlo a partire dalle equazioni di Maxwell non è una contraddizione: le
equazioni di Maxwell infatti contengono al loro interno anche le condizioni affinchè il campo elettromagnetico ammetta potenziale
- condizioni che sono date per valide nel momento in cui si assume l'esistenza di un potenziale.

Il potenziale vettore $\vb{A}$ è definito da una relazione analoga a quella introdotta nel paragrafo \ref{par:potenziale_vettore}
in quanto la seconda equazione di Maxwell, condizione necessaria per la validità della definizione,
non varia fra caso stazionario e non stazionario. Per quanto riguarda invece il potenziale scalare
$V$, è necessaria una generalizzazione: nel caso stazionario l'esistenza del potenziale elettrico è garantita dal fatto che il
campo elettrico sia conservativo, ma nel caso non stazionario questo non è più vero.
Introducendo l'equazione definitoria del potenziale vettore \eqref{eqn:def_potenziale_vettore}
nella terza equazione di Maxwell
\[
    \curl{\vb{E}}=-\pdv{t}(\curl{\vb{A}})=-\curl(\pdv{\vb{A}}{t})
\]
Da cui
\[
    \curl(\vb{E}+\pdv{\vb{A}}{t})=0
\]
Il vettore fra parentesi è irrotazionale e ammette quindi potenziale
\begin{defn}[potenziale scalare]
    Si definisce potenziale scalare una funzione che soddisfi la condizione
    \begin{equation}
        \label{eqn:def_potenziale_scalare}
        -\grad{V}=\vb{E}+\pdv{\vb{A}}{t}
    \end{equation}
\end{defn}
Si osservi come nel caso stazionario questa definizione si riduca a quella fornita nel paragrafo \ref{par:potenziale_elettrico}.
Condizione necessaria all'introduzione dei potenziali è la validità della seconda e della terza equazione di
Maxwell, ovvero di quelle omogenee\footnote{In queste due equazioni infatti non compaiono i termini noti dovuti
alle sorgenti.}, che risultano quindi identicamente soddisfatte una volta sostituiti al loro interno i
campi con i potenziali:
\[
    \begin{split}
        &\div{\vb{B}}=\div(\curl{\vb{A}})=0\\
        &\curl{\vb{E}}+\pdv{\vb{B}}{t}=\curl(\vb{E}+\pdv{\vb{A}}{t})=-\curl{\grad{V}}=0
    \end{split}
\]
Per ricavare i potenziali è quindi necessario usare la prima e la quarta equazione di Maxwell. Sostituendo le equazioni
definitorie dei potenziali all'interno di queste ultime si ottiene
\[
    \begin{split}
        &\laplacian{V}+\pdv{t}(\div{\vb{A}})=-\frac{\rho}{\epsilon}\\
        &\laplacian{\vb{A}}-\epsilon\mu \pdv[2]{\vb{A}}{t}-\grad(\div{\vb{A}}+\epsilon\mu\pdv{V}{t})=-\mu\vb{J}
    \end{split}
\]
L'aver introdotto i potenziali non sembra aver semplificato di molto le equazioni da risolvere per trovare le caratteristiche del campo
elettromagnetico, dato che queste quattro equazioni non sono disaccoppiate. Si osservi che se i termini fra parentesi fossero
nulli, le equazioni risulterebbero disaccoppiate e assumerebbero una forma molto semplice.
Siccome i potenziali non sono univocamente definiti risulta allora
importante cercare la trasformazione più generale che lasci invariati i campi e vedere se tramite questa sia possibile
semplificare le equazioni.
\begin{thm}
    Dati due potenziali $\vb{A}$ e $V$ che soddisfano rispettivamente la \eqref{eqn:def_potenziale_vettore} e la
    \eqref{eqn:def_potenziale_scalare} si possono produrre dei potenziali $\vb{A}'$ e $V'$ tali che da entrambe le coppie di potenziali
    sia deducibile lo stesso campo elettromagnetico. Affinchè questo accada la trasformazione che lega le due coppie di potenziali deve
    essere del tipo
    \begin{equation}
        \label{eqn:trasformazioni_gauge}
        \begin{cases}
            & \vb{A}\to \vb{A}'=\vb{A}+\grad{\phi}\\
            & V\to V'=V-\pdv{\phi}{t}
        \end{cases}
    \end{equation}
    con $\phi$ una funzione scalare di classe $C^2$.
\end{thm}
\begin{proof}
    La prima delle due equazioni ha dimostrazione immediata, già affrontata a suo tempo nel paragrafo \ref{par:potenziale_vettore}.
    Per quanto riguarda la seconda invece
    \[
        \vb{E}'=-\grad{V'}-\pdv{\vb{A}'}{t}=-\grad{V}+\grad(\pdv{\phi}{t})-\pdv{\vb{A}}{t}-\pdv{t}(\grad{\phi})=-\grad{V}-\pdv{\vb{A}}{t}=\vb{E}
    \]
\end{proof}
La trasformazione \eqref{eqn:trasformazioni_gauge} è detta \textit{trasformazione di gauge} e la funzione $\phi$ è
detta \textit{funzione di gauge}. Un'opportuna trasformazione di gauge permette di trovare delle equazioni disaccoppiate
per i potenziali.

Se i potenziali elettrodinamici soddisfano la condizione, detta condizione di Lorentz,
    \begin{equation}
        \label{eqn:condizione_lorentz}
        \div{\vb{A}}+\epsilon\mu\pdv{V}{t}=0
    \end{equation}
il sistema di equazioni si riducono al sistema di quattro equazioni disaccoppiate
    \begin{equation}
        \label{eqn:potenziali_maxwell}
        \begin{split}
            &\laplacian{V}-\epsilon\mu \pdv[2]{V}{t}=-\frac{\rho}{\epsilon}\\
            &\laplacian{\vb{A}}-\epsilon\mu \pdv[2]{\vb{A}}{t}=-\mu\vb{J}
        \end{split}
   \end{equation}
Quando i potenziali soddisfano la condizione di Lorentz si dice che essi appartengono alla \textit{gauge di Lorentz}.
\begin{thm}
    Data una coppia di potenziali $\vb{A}$ e $V$ è sempre possibile trovare una funzione $\phi$ affinchè i potenziali $\vb
    {A}'$ e $V'$ generati a partire da questi due tramite trasformazione di gauge appartengano alla gauge di Lorentz.
\end{thm}
\begin{proof}
    Si vuole dimostrare che è possibile determinare $\phi$ attraverso un'equazione differenziale che ammette sempre soluzione.
    Bisogna avere
    \[
        0= \div{\vb{A}'}+\epsilon\mu\pdv{V'}{t}=\div{\vb{A}}+\epsilon\mu\pdv{V}{t}+\laplacian{\phi}-\epsilon\mu\pdv[2]{\phi}{t}
    \]
    Da cui segue
    \[
        \laplacian{\phi}-\epsilon\mu\pdv[2]{\phi}{t}= -\Biggl( \div{\vb{A}}+\epsilon\mu\pdv{V}{t} \Biggr)
    \]
    Per ipotesi il secondo membro è noto perchè sono noti i potenziali e quindi questa equazione ammette soluzione.
\end{proof}

La dimensione del prodotto $\epsilon\mu$ è quella dell'inverso di una velocità al quadrato ed è quindi possibile riscrivere
le equazioni per i potenziali usando l'operatore dalemebrtiano
\[
    \label{eqn:potenziali_maxwell}
    \begin{split}
        & \square V=-\frac{\rho}{\epsilon}\\
        & \square \vb{A}=-\mu\vb{J}
    \end{split}
\]
In questo modo diventa lampante il fatto che le equazioni per i potenziali nel caso non stazionario sono analoghe
a quelle del caso stazionario, con l'unica differenza di sostituire l'operatore laplaciano col dalembertiano: sono
equazioni di onde con sorgente. Questa gauge risulta particolarmente adatta a descrivere problemi radiativi.
In questa gauge una soluzione particolare e degna di nota,
in quanto immediata generalizzazione della soluzione per il caso stazionario, è quella dei \textit{potenziali ritardati}.
\begin{thm}[Potenziali ritardati]
    Se le sorgenti sono localizzate in una porzione di spazio finita le \eqref{eqn:potenziali_maxwell} ammettono come soluzione
    \begin{equation}
        \begin{split}
            &\vb{A}(\vb{r},t)=\frac{\mu}{4\pi}\int_{\tau}\frac{\vb{J}(\vb{r}', t-\Delta r/v)}{\Delta r}\dd{\tau'}\\
            &V(\vb{r},t)=\rec{4\pi\epsilon}\int_{\tau}\frac{\rho(\vb{r}', t-\Delta r/v)}{\Delta r}\dd{\tau'}
        \end{split}
        \label{eqn:potenziali_ritardati}
    \end{equation}
    avendo indicato $v=1/\sqrt{\epsilon\mu}$
\end{thm}
Il nome \textit{potenziali ritardati} assume senso nel momento in cui si osserva che la densità di corrente e la densità di
carica, in ogni posizione $\vb{r}'$ non vanno calcolate al tempo a cui si sta calcolando il potenziale ma ad un istante $t'=t-\Delta r/v$.
Si può intuire come questo tempo sia quello impegato dal segnale elettromagnetico per percorrere lo spazio
tra la posizione $\vb{r}'$ e la posizione $\vb{r}$ in cui si sta calcolando il potenziale.
La soluzione più generale possibile si ottiene sommando ai potenziali ritardati la soluzione generale delle equazioni omogenee corrispondenti
alle \eqref{eqn:potenziali_maxwell}. Nel prossimo capitolo si osserverà come queste soluzioni siano delle onde.
A grande distanza dalle sorgenti localizzate, i potenziali ritardati vanno a zero molto più velocemente delle soluzioni omogenee:
a grande distanza quindi permangono solo le onde elettromagnetiche.


%La scelta di $\phi$ non toglie arbitrarietà ai potenziali: infatti la funzione di gauge non è univocamente determinata
%dall'equazione che compare nella dimostrazione del teorema, ma a questa va sommata la soluzione generale dell'equazione
%omogenea associata $\phi_0$. Presi quindi due potenziali $\vb{A}'$ e $V'$ primo apparteneneti alla gauge di Lorentz,
%anche
%\[
%    \begin{cases}
%        & \vb{A}''=\vb{A}'+\grad{\phi_0}\\
%        & V''=V'-\pdv{\phi_0}{t}
%    \end{cases}
%\]
%appartengono alla gauge di Lorentz.


Per descrivere i problemi di interazione radiazione materia la gauge più utile è la \textit{gauge di Coulomb}.
Se i potenziali elettrodinamici soddisfano la condizione
    \begin{equation}
        \div{\vb{A}}=0
    \end{equation}
    allora le equazioni per i potenziali si riducono al sistema di quattro equazioni disaccoppiate
    \begin{equation}
        \begin{split}
            &\laplacian{V}=-\frac{\rho(\vb{r},t)}{\epsilon}\\
            &\laplacian{\vb{A}}-\epsilon\mu \pdv[2]{\vb{A}}{t}=\epsilon\mu \grad\pdv{V}{t}-\mu \vb{J}
        \end{split}
    \end{equation}
In questa gauge il potenziale scalare è quindi analogo al potenziale elettrico, con l'unica differenza che la densità di carica
dipende dal tempo
\[
    V(\vb{r},t)=\rec{4\pi\epsilon}\int\frac{\rho(\vb{r}',t)}{\abs{\vb{r}-\vb{r'}}}\dd{\tau'}
\]
La gauge di Coulomb è utile in assenza di sorgenti: il potenziale scalare è in questo caso nullo e l'equazione
per il potenziale vettore diventa $\square \vb{A}=0$; i campi diventano
\[
    \vb{B}=\curl{\vb{A}}\quad\quad \vb{E}=-\pdv{\vb{A}}{t}
\]


Ecco quindi che risulta giustificato il percorso seguito in questo paragrafo: per determinare i potenziali è sufficiente risolvere quattro
equazioni differenziali disaccoppiate. Una volta noti i potenziali è possibile risalire al campo elettromagnetico tramite semplici
operazioni di derivazione. Introducendo i potenziali l'unico ostacolo al calcolo del campo si trova nella struttura complicata delle
funzioni che entrano in gioco ma viene eliminato qualsiasi problema relativo alla struttura delle equazioni.
