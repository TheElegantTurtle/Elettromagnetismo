Si consideri un'onda elettromagnetica che incide su un conduttore. Gli elettroni liberi, sotto
l'effetto del campo elettromagnetico variabile
iniziano a muoversi con moto oscillatorio forzato, dissipando energia.
Ci si aspetta quindi che un'onda elettromagnetica in un conduttore si attenui e scaldi il conduttore.

Si consideri un conduttore ohmico. Nel caso stazionario si ha $\vb{J}=\sigma \vb{E}$ con $\sigma$ la
conducibilità elettrica. Siccome questa è una legge locale, ci si aspetta valga anche nel caso non
stazionario come effettivamente conferma l'esperienza (la conducibilità può però essere una funzione
della frequenza ed essere complessa).
La presenza di cariche in moto nel conduttore non permette di prendere $\vb{J}=0$
nella quarta equazione di Maxwell: questo è il motivo per cui le equazioni delle onde
nel conduttore assumono una forma diversa rispetto a quelle nel vuoto.
\begin{thm}
    In un conduttore ohmico, omogeneo e isotropo
    \[
        \begin{split}
            & \laplacian{\vb{H}}-\sigma\mu\pdv{\vb{H}}{t}-\epsilon\mu\pdv[2]{\vb{H}}{t}=0\\
            & \laplacian{\vb{E}}-\sigma\mu\pdv{\vb{E}}{t}-\epsilon\mu\pdv[2]{\vb{E}}{t}=0
        \end{split}
    \]
\end{thm}
\begin{proof}
    Tenuto conto della relazione fra $\vb{E}$ e $\vb{J}$ la quarta equazione di Maxwell assume la forma
    \[
        \curl{\vb{H}}=\sigma \vb{E} +\epsilon \pdv{\vb{E}}{t}
    \]
    Applicando l'operatore rotore ad ambo i membri
    \[
        -\laplacian{\vb{H}} +\grad(\div{\vb{H}})=\sigma(\curl{\vb{E}}) + \epsilon \pdv{t}(\curl{\vb{E}})
    \]
    Per l'ipotesi di omogeneità e isotropia $\vb{H}=\vb{B}/\mu$\footnote{Nel caso dei ferromagneti, questo
    è vero solo localmente.} e quindi per la seconda equazione di Maxwell
    $\div{\vb{H}}=\div{\vb{B}}/\mu=0$. Inoltre per la terza equazione di Maxwell
    $\curl{\vb{E}}=-\pdv*{\vb{B}}{t}=-\mu\pdv*{\vb{H}}{t}$. Sostituendo nella quarta equazione di Maxwell
    si ottiene
    \[
        \laplacian{\vb{H}}=\sigma\mu\pdv{\vb{H}}{t}+\epsilon\mu\pdv[2]{\vb{H}}{t}
    \]
    Applicando l'operatore rotore alla terza equazione di Maxwell e confrontandola con la quarta
    si ottiene un risultato analogo per il campo elettrico.
\end{proof}
La soluzione a queste equazioni viene proposta nel caso di un'onda piana che si propaga lungo l'asse $x$.
Le equazioni per il campo elettrico e magnetico assumono la forma
    \[
        \dv[2]{\phi}{x} - \sigma\mu\pdv{\phi}{t} - \epsilon\mu\pdv[2]{\phi}{t} =0
    \]
La derivata prima rappresenta il termine di smorzamento. Si noti come questo termine scompaia
quando $\sigma=0$, ovvero nel caso di un dielettrico perfetto.
\begin{thm}
L'onda piana che si propaga lungo l'asse $x$ in un conduttore omogeneo e isotropo ha forma
    \[
        \phi(x,t)=A e^{\gamma x} e^{j(\beta x + \omega t)}
    \]
    con
    \[
        \begin{split}
            &\beta=\omega\sqrt{\frac{\epsilon\mu}{2}\bigl[1\pm \sqrt{1+(\sigma/\omega\epsilon)^2}\bigr]}\\
            &\gamma= \frac{\omega\sigma\mu}{2\beta}
        \end{split}
    \]
\end{thm}
\begin{proof}
    Si pongano i campi nella forma
    \[
        \phi(x,t)=\phi{x}e^{-j\omega t}
    \]
    L'esponenziale complesso viene introdotto per comodità di calcolo. Siccome l'equazione differenziale
    è a coefficienti reali, il campo fisicamente significativo è costituito solo dalla parte reale
    della soluzione.
    Sostituendo nell'equazione delle onde si trova un'equazione differenziale per $\phi(x)$
    \[
        \dv[2]{\phi}{x}-j\omega\sigma\phi + \omega^2\epsilon\mu\phi=0
    \]
    È lecito prendere $\phi(x)$ nella forma $\phi(x)=A e^{j\alpha x}$ siccome l'equazione è lineare, ma siccome i coefficienti
    in questo caso sono complessi non si può più affermare che la parte fisicamente significativa sia solo
    quella reale. Sostituendo nell'equazione differneziale si trova un'equazione algebrica per $\alpha$:
    \[
        \alpha^2=\epsilon\mu\omega^2 + i\mu\sigma\omega
    \]
    $\alpha^2$ è un numero complesso e di conseguenza anche $\alpha$ è complesso che può essere preso
    nella forma $\alpha=\beta-j\gamma$. Inserendo nell'equazione precedente e uguagliando i vari termini
    si trova la tesi.
 \end{proof}
L'onda progressiva si ha scegliendo $\beta<0$ e di conseguenza
$\gamma<0$: $\abs{\gamma}$ svolge il ruolo di coefficiente di attenuazione.
Spesso $\sigma>>\epsilon\omega$\footnote{Ad esempio nel caso del rame $\sigma\simeq6\vdot10^7 \Omega^{-1}m^{-1}$.
Nel visibile $\omega\simeq10^{15}Hz$ e dunque $\epsilon\omega\simeq10^4<<10^7$.}
e i coefficienti $\beta$ e $\gamma$ assumono la semplice forma
\[
    \begin{split}
        &\beta=\sqrt{\frac{\omega\sigma\mu}{2}}\\
        &\gamma=-\sqrt{\frac{\omega\sigma\mu}{2}}\\
    \end{split}
\]

Il risultato appena trovato fornisce una giustificazione all'\textit{effetto pelle}: quando un conduttore è percorso
da una corrente alternata ad alta frequenza (e quindi $\omega$ grande) la corrente tende ad addensarsi sullo strato
superficiale del conduttore, riducendone la sezione utile ed aumentandone la resistenza.
Per valutare qualitativamente il fenomeno e capire quali entità fisiche entrano in gioco,
si consideri quindi un conduttore collegato ad un generatore di corrente alternata, ovvero un generatore nel quale gli
accumuli di carica sui morsetti varino nel tempo. Di conseguenza il campo elettrico prodotto varia nel tempo,
quindi varia la corrente circolante nel conduttore e anche il campo di induzione magnetica generato dal conduttore.
Si ha il fenomeno dell'autoinduzione, per cui si sviluppa un campo elettrico indotto $\vb{E}_i$ che si oppone alla variazione
temporale del campo elettrico che lo ha generato.
Concretamente, si prenda un conduttore cilindrico percorso da corrente variabile - ad esempio, crescente.
Per simmetria il campo $\vb{B}$ è costituito da cerchi centrati nell'asse del cilindo e cresce nel tempo
come la corrente dando luogo ad un campo indotto $\vb{E}_i$ orientato parallelamente all'asse del conduttore
con verso opposto a quello di $\vb{J}$. Si consideri ora una linea chiusa rettangolare $l$ di lati $h$ e $\dd{r}$,
che giaccia su un piano contenente l'asse del cilindro. Si prenda come verso positivo quello antiorario, in modo
che la normale alla superficie sia concorde in verso con $\vb{B}$ e che quindi il flusso sia positivo e crescente.
Per la legge di Faraday-Neumann
\[
    -\dv{t}\Phi_S(\vb{B}) =\oint_l \vb{E}_i\vdot\dd{\vb{l}}\simeq E_i(r+\dd{r})h - E_i(r)h
\]
Ma per come è stato impostato il problema $\vb{B}$ è crescente nel tempo, perciò la sua derivata è positiva.
Quindi $E_i(r)>E_i(r+\dd{r})$: $E_i$ aumenta all'aumentare della profondità del conduttore
e quindi ostacola maggiormente il passaggio di corrente negli strati profondi che negli strati superficiali.

%Una valutazione quantitativa del fenomeno può essere effettuata in modo semplice nel caso di un conduttore
%ohimico omogeneo infinito, delimitato da una superficie piana e immerso in un campo $\vb{E}(t)=\vb{E}_0e^{j\omega t}$.
%Si ponga il conduttore in un sistema di riferimento cartesiano in modo che occupi lo spazio $z<0$ e che il campo
%elettrico abbia solo la componente $y$ diversa da 0. La quarta equazione di Maxwell si scrive
%\[
%    \curl{\vb{H}}=\sigma\vb{E}_0e^{j\omega t} + \epsilon\pdv{\vb{E}_0e^{j\omega t}}{t}=
%    (\sigma+j\omega\epsilon)\vb{E}
%\]
%Applicando l'operatore divergenza ad entrambi i membri, nell'ipotesi che il conduttore sia omogeneo
%\[
%    0=\div(\curl{\vb{H}})=(\sigma+j\omega\epsilon)\div{\vb{E}}
%\]
%Questo implica che $\div{\vb{E}}=0$ e quindi, per la prima equazione di Maxwell nell'ipotesi di isotropia $\rho=0$.
%La densità di volume di cariche localizzate è quindi nulla.
%Il rapporto tra i moduli delle densità di corrente di spostamento e di conduzione vale
%\[
%    \frac{\epsilon\pdv{E}{t}}{J}=\frac{\abs{j\omega\epsilon E}}{\abs{\sigma e}}=\frac{\omega\epsilon}{\sigma}
%\]
%Sperimentalmente nei buoni conduttori $\omega\epsilon<<\sigma$ e quindi la corrente di spostamento può essere
%trascurata. Alla luce di quanto detto le equazioni di Maxwell utili al fine di caratterizzare il campo nel
%conduttore sono la terza e la quarta, che assumono la forma
%\[
%    \begin{split}
%        & \curl{\vb{E}}=-\pdv{\vb{B}}{t} \\
%        & \curl{\vb{B}}=-\mu\sigma \vb{E} \\
%    \end{split}
%\]
%Applicando il rotore alla prima di queste si ottiene: per il primo membro
%$\curl{\curl{\vb{E}}}=-\laplacian{\vb{E}}+\grad(\div{\vb{E}})=-\laplacian{\vb{E}}+\grad(\div{\vb{J}})/\sigma=-\laplacian{\vb{E}}$
%dove si è fatto uso del fatto che, essendo
%$\rho=0$, per l'equazione di continutità $\div{\vb{J}}=0$; per il secondo membro, usando la quarta equazione di Maxwell
%$-\pdv*{\curl{\vb{B}}}{t}=-\pdv*{\mu\sigma\vb{E}}{t}$.
%L'equazione per il campo elettrico diventa pertanto, ricordando che il campo elettrico ha solo componente $y$ dipendente da $z$(PERCHÈ??)
%\[
%    \begin{split}
%        & \pdv[2]{E_{0y}(z)e^{j\omega t}}{z}=\mu\sigma\pdv{E_{0y}(z)e^{j\omega t}}{t}\\
%        & \dv{E_{0y}(z)}{z}=j\omega\mu\sigma E_{0y}(z)\\
%    \end{split}
%\]
%che ha come soluzione
%\[
%    E_{0y}=Ae^{\alpha_1 z}+Be^{\alpha_2 z}
%\]
%con $\alpha_{1,2}$ soluzioni dell'equazione $\alpha^2-j\omega\mu\sigma=0$, ovvero $\alpha_{1,2}=\pm \beta(1+j)$
%(con $\beta=\sqrt{\omega\sigma\mu/s}$).
%La soluzione assume quindi la forma
%\[
%    E_{0y}=Ae^{\beta z}e^{j\beta z}+Be^{\beta z}e^{-j\beta z}
%\]
%Si osservi come per $z\to -\infty$ il secondo addendo diverga, portando ad una soluzione fisicamente inaccettabile:
%bisogna quindi imporre $B=0$.
%In conclusione si ottiene
%\[
%    E_y(z,t)=Ae^{\beta z}e^{j(\beta z +\omega t)}
%\]
%ovvero il campo elettrico (e quindi la densità di corrente, per la legge di Ohm locale) decresce esponenzialmente con
%la profondità ($-z$). La profondità caratteristica di penetrazione vale $1/\beta$.
