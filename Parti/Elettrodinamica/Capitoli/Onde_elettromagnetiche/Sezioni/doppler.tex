Nel caso delle onde sonore, se la sorgente, il mezzo materiale in cui si propaga l'onda
e l'osservatore sono in moto relativo si manifesta l'effetto Doppler: se l'osservatore si sta avvicindando alla sorgente
vede arrivare il suono a velocità maggiore e quindi incontra un numero maggiore di fronti d'onda a parità
di intervallo di tempo e quindi osserva una frequenza maggiore; viceversa se è in allontanamento.

Nel caso delle onde elettromagnetiche, a causa del principio di costanza della velocità della luce e del
fatto che non si propagano in un mezzo, ci si aspetta che l'effetto Doppler non sussista. In realtà,
a partire da considerazioni di carattere relativistico si può mostrare che effettivamente l'effetto
si manifesta anche per le onde elettromagnetiche. In particolari, per velocità molto minori della
velocità della luce la formula è analoga a quella dell'effetto Doppler acustico
\[
\nu'=\nu \Biggl(1\pm\abs{\frac{V}{c}}\Biggr)
\]
dove $\nu'$ e $\nu$ sono rispettivamente la frequenza osservata e la frequenza emessa dalla sorgente, $V$ è
la velocità relativa fra osservatore e sorgente e il segno deve essere scelto a seconda che
sorgente e osservatore siano in avvicinamento o in allontanamento.
