Una delle soluzioni più importanti delle equazioni di Maxwell è quella delle onde elettromagnetiche.
\begin{obs}
    Dato un dielettrico neutro, illimitato, isotropo, omogeneo e perfetto, si hanno le equazioni
    \begin{equation}
        \label{eqn:diff_onde}
        \begin{split}
            &\laplacian{\vb{E}}-\epsilon\mu \pdv[2]{\vb{E}}{t}=0\\
            &\laplacian{\vb{B}}-\epsilon\mu \pdv[2]{\vb{B}}{t}=0
        \end{split}
    \end{equation}
\end{obs}
\begin{proof}
    L'ipotesi di dielettrico illimitato, isotropo e omogeneo permette di usare le equazioni di Maxwell in forma analoga a quelle
    nel vuoto previa sostistuzione di $\epsilon_0$ e $\mu_0$ con $\epsilon$ e $\mu$.
    Per l'ipotesi di dielettrico perfetto si ha l'assenza di correnti macroscopiche e quindi $\vb{J}=0$, mentre per l'ipotesi di
    neutralità $\rho=0$. Perciò si ha
\begin{minipage}[t]{0.5\textwidth}
\[
    \begin{split}
        & \div{\vb{E}}=0                    \\
        & \curl{\vb{E}}=-\pdv{\vb{B}}{t}       \\
    \end{split}
\]
\end{minipage}
\begin{minipage}[t]{0.5\textwidth}
\[
    \begin{split}
        & \div{\vb{B}}=0                       \\
        & \curl{\vb{B}}=\mu\epsilon\pdv{\vb{E}}{t} \\
    \end{split}
\]
\end{minipage}
    Applicando l'operatore rotore alla terza equazione di Maxwell, per la \eqref{app:eqn:curl_curl} ricordando che per la prima
    equazione di Maxwell la divergenza del campo elettrico è nulla,
    \[
        \curl{\curl{\vb{E}}}=-\laplacian{\vb{E}}=-\curl{\pdv{\vb{B}}{t}}=-\pdv{t}(\curl{\vb{B}})
    \]
    Derivando rispetto al tempo la quarta equazione di Maxwell si ottiene
    \[
        \pdv{t}(\curl{\vb{B}})=\epsilon\mu\pdv[2]{\vb{E}}{t}
    \]
    Per sostituzione si ha l'equazione nella tesi relativa al campo elettrico. Applicando il rotore
    alla quarta equazione di Maxwell e derivando temporalmente la terza, si ottiene in maniera del tutto analoga l'
    equazione relativa al campo magnetico.
\end{proof}

L'aver applicato l'operatore rotore rende le equazioni ottenute non equivalenti alle equazioni di Maxwell. Se un campo
$\vb{E}$ soddisfa le equazioni di Maxwell infatti, le \eqref{eqn:diff_onde} sono soddisfatte anche da un campo
$\vb{E}+\vb{E}'$, con $\vb{E}' $ un qualunque campo irrotazionale. Le equazioni ottenute possono così avere per soluzioni anche campi
a divergenza non nulla a differenza delle equazioni di Maxwell: la solenoidalità delle soluzioni deve essere imposta
come condizione aggiuntiva.
Per le proprietà delle onde, il coefficente
$v=1/\sqrt{\epsilon\mu}$ rappresenta la velocità di propagazione dell'onda. In effetti dimensionalmente questo
coefficente è proprio una velocità.
Nel vuoto si
ha $v=c=1/\sqrt{\epsilon_0\mu_0}$ che risulta essere sperimentalmente uguale alla velocità della luce. Questo dimostra che la
luce è un'onda elettromagnetica.

-----------------------------------------------------

Alla luce di queste considerazioni si può
dare un significato fisico concreto a quanto visto in conclusione al precedente capitolo sui potenziali elettrodinamici in quanto
le equazioni qui ottenute hanno la stessa forma delle \eqref{eqn:potenziali_maxwell}.
I risultati ottenuti nella gauge di Coulomb, ovvero
\[
    \vb{B}=\curl{\vb{A}}\quad\quad \vb{E}-\pdv{\vb{A}}{t} \quad\quad \laplacian\vb{A}-\epsilon\mu \pdv[2]{\vb{A}}{t}=0
\]
danno informazioni sulle onde elettromagnetiche: il campo elettrico ed il campo magnetico sono ortogonali fra loro
e ortogonali alla direzione di propagazione dell'onda; il rapporto fra i loro moduli vale $v=(\epsilon\mu)^{-\rec{2}}$.
[PERCHè?]
