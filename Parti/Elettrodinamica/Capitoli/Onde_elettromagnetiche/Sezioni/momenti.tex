In questo paragrafo $n=\dv*{N}{\tau}$ torna ad indicare il numero di cariche per unità di volume e non
l'indice di rifrazione.

Si consideri un'onda elettromagnetica incidente su un materiale. Per quanto visto nel
paragrafo sul vettore di Poynting il campo elettromagnetico esercita una forza sulla
superficie. Dalla \eqref{eqn:potenza_em} si ha che la potenza assorbita dal materiale per
unità di volume vale $w=\dv*{P}{\tau}=\vb{E} \vdot \vb{J}$.

\begin{obs}
    \[
        \vb{J}=\sigma\vb{E}
    \]
    con $\sigma$ complesso. Quindi $\vb{J}$ ha lo stesso andamento di $\vb{E}$, ampiezza proporzionale
    all'ampiezza di $\vb{E}$ ed è sfasato rispeto ad $\vb{E}$ della fase del coefficiente $\sigma$.
\end{obs}
\begin{proof}
    Se il materiale è un conduttore la tesi è evidente sulla base della relazione che lega $\vb{E}$ e $\vb{J}$
    nei conduttori. In linea del tutto generale, sotto l'effetto del campo oscillante le cariche si comportano
    come oscillatori smorzati che compiono un moto oscillatorio forzato: per quanto visto nella dimostrazione dell'osservazione
    \ref{obs:r_dielettrici} $\dot{\vb{r}}$ è proporzionale al campo elettrico mediante un coefficiente complesso
    e dunque lo è anche la sua derivata -e di conseguenza, la velocità di deriva.
    Ma siccome $\vb{J}=nq\vb{v}_d$, si ha la tesi.
\end{proof}
Alla luce di questo risultato,
\[
w=\sigma E^2
\]
Il valor medio della potenza su un periodo, detto $E_0$ l'ampiezza del campo elettrico
e $\alpha$ la fase di $\sigma$, vale quindi
\[
    [w]=\frac{E_0^2}{2}\abs{\sigma}\cos{\alpha}
\]

\begin{thm}
    La quantità di moto che l'onda trasferisce nell'unità di tempo all'untià di volume del materiale è diretta
    come la velocità di propagazione dell'onda e vale
    \[
        [\vb{q}]=\frac{[w]}{v}\vu{v}
    \]
\end{thm}
\begin{proof}
    La quantità di moto trasferita nell'unità di tempo all'unità di volume è data dall'impulso trasferito nell'unità di tempo,
    cioè la media temporale della forza impressa
    per untià di volume. Dal paragrafo sul vettore di Poynting si ha che questa forza vale
    \[
        \vb{f}=\dv{\vb{F}}{\tau}=nq(\vb{E}+\vb{v}_d\cp\vb{B})
    \]
    Siccome la media temporale del campo elettrico è nulla
    \[
        [\vb{q}]\equiv [\vb{f}]=[nq\vb{v}_d\cp\vb{B}]=[\vb{J}\cp\vb{B}]
    \]
    Dato che campo elettrico e magnetico sono ortogonali e $\vb{J}$ è parallelo a $\vb{E}$
    \[
        [\vb{q}]=[JB]\vu{v}
    \]
    Tenendo conto della relazione fra i moduli di $B$ ed $E$
    \[
        [\vb{q}]=\frac{[\vb{J\vdot\vb{E}}]}{v}\vu{v}
    \]
    Ovvero la tesi.
\end{proof}

Qualora ci si trovi nel caso di assorbimento totale, ovvero nel caso in cui l'onda trasferisca al materiale tutta la sua energia,
conviene far riferimento all'energia incidente sull'unità di superficie nell'unità di tempo, ovvero al modulo del vettore
di Poynting.
In questo caso al posto della potenza per unità di volume  si avrà l'intensità, ovvero potenza per unità di superficie,
e al posto della quantita di moto per unità di volume e unità di tempo
si avrà la quantità di moto trasferità all'unità di superficie normale all'onda nell'unità di tempo $\vb{p}$. Perciò
\[
    \vb{p}=\frac{I}{v}\vu{v}=\frac{\vb{I}}{v}
\]
Quindi, dalla definizione di vettore di Poynting
\begin{equation}
    \vb{p}=\frac{\vb{E}\cp\vb{B}}{\mu v}
\end{equation}
$p$ ha le dimensioni di una forza diviso una superficie ed è quindi effettivamente una pressione. La stessa pressione, in verso
opposto la subisce una sorgente che emetta un'onda con intensità $I$. Una superficie perfettamente riflettente investita
ortogonalmente dall'onda subisce una pressione doppia.

Oltre alla quantità di moto le onde elettromagnetiche trasportando anche un momento angolare. Dato un polo $O$ è evidente
infatti che le onde trasportino il momento
\[
    \vb{L}=\int_S\vb{r}\cp\dd{\vb{p}}
\]
dove $\vb{r}$ è la distanza fra il polo ed il punto dell'onda con momento $\dd{\vb{p}}$.
Nel caso di un'onda opportunamente collimata questa espressione si semplifica a $\vb{L}=\vb{r}\cp\vb{p}$.
Inoltre, la radiazione elettromagnetica possiede un momento angolare intrinseco quando è
polarizzata circolarmente, ovvero quando il campo elettrico ruota attorno alla direzione di
propagazione. Il momento angolare intrinseco vale
\[
    \vb{S}=\pm\omega\vb{I}
\]
dove il segno dipende dal verso di polarizzazione, destrorsa o sinistrorsa. Il momento angolare intrinseco è
longitudinale, ovvero diretto come la velocità dell'onda.
Un'onda con polarizzazione lineare, in cui il campo elettrico oscilli su un piano fisso, non possiede momento
angolare intrinseco, in quanto può essere ottenuta come sovrapposizione di due onde identiche con polarizzazione
circolare in un caso destrorsa e nell'altro sinistrorsa.
