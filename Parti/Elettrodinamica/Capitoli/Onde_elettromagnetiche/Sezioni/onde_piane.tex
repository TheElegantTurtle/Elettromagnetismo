La forma più semplice della soluzione all'equazione delle onde è quella di onda piana, che corrisponde ad una configurazione piana
delle condizioni al contorno,
ovvero quella in cui $\vb{E}$ e $\vb{B}$ assumono lo stesso valore per tutti i punti di ogni piano ortogonale alla
direzione di propagazione\footnote{Fisicamente questa condizione non si verifica mai, tuttavia è un limite per molti
casi di interesse pratico come ad esempio lo studio di una piccola porzione di spazio molto lontana dalla sorgente
(approssimazione di sorgente puntiforme).}
che può essere presa, senza perdita di generalità, parallela all'asse $x$ - o,
in altri termini, ogni componente dei campi è indipendente da $y$ e $z$. In questo caso ciascuna delle
sei componenti del campo elettromagntico soddisfa un'equazione di D'Alembert, ovvero un'equazione nella forma
\[
    \pdv[2]{f}{x}-\epsilon\mu\pdv[2]{f}{t}=0
\]
La cui soluzione generale è quindi del tipo
\begin{equation}
    \psi(x,t)=f_1(x-vt)+f_2(x+vt)
\end{equation}

\begin{thm}
    I campi elettrico e magnetico di un'onda elettromagnetica piana che si propaga in un dielettrico illimitato, isotropo, omogeneo,
    perfetto e neutro sono legati dalla seguente relazione
    \[
        \vb{E}=\vb{B}\cp\vb{v}\quad\quad \frac{E}{B}=v
    \]
\end{thm}
Il teorema si dimostra banalmente grazie ai tre lemmi di seguito esposti, i quali derivano dal fatto che
per un dielettrico illimitato, omogeneo, isotropo, perfetto e neutro
le equazioni di Maxwell in forma scalare sono

\begin{minipage}[t]{0.5\textwidth}
    \[
        \begin{split}
            &\,\,\begin{cases}
                &\pdv{E_x}{x}=0 \\
            \end{cases} \\
            & \begin{cases}
                & \pdv{B_x}{t}=0 \\
                & \pdv{E_z}{x}=\pdv{B_y}{t} \\
                & \pdv{E_y}{x}=-\pdv{B_z}{t} \\
            \end{cases} \\
        \end{split}
    \]
\end{minipage}
\begin{minipage}[t]{0.5\textwidth}
    \[
        \begin{split}
            &\,\,\begin{cases}
                &\pdv{B_x}{x}=0 \\
            \end{cases} \\
            & \begin{cases}
                & \pdv{E_x}{t}=0 \\
                & \pdv{B_z}{x}=- \mu\epsilon \pdv{E_y}{t} \\
                & \pdv{B_y}{x}=\mu\epsilon \pdv{E_z}{t} \\
            \end{cases} \\
        \end{split}
    \]
\end{minipage}


\begin{lemma}
    In un dielettrico isotropo, illimitato, omogeneo e perfetto, le componenti dei campi parallele alla direzione di
    propagazione non contribuiscono alla propagazione del campo.
\end{lemma}
\begin{proof}
    Data l'equazione di D'Alembert, si vuole dimostrare che $E_x$, $B_x$ sono costanti nel tempo e uniformi nello spazio.
    Questo può essere visto facilmente osservando che nell'ipotesi di onda piana (campi indipendenti da $y$ e $z$) e di dielettrico neutro e perfetto
    le quattro equazioni di Maxwell danno, per la componente $x$ dei campi:
    \[
        \begin{split}
            I   &  \Longrightarrow \pdv{E_x}{x}=0 \\
            II  &  \Longrightarrow \pdv{B_x}{x}=0 \\
            III &  \Longrightarrow \pdv{B_x}{t}=\pdv{E_z}{y}-\pdv{E_y}{z}=0 \\
            IV  &  \Longrightarrow \epsilon\mu \pdv{E_x}{t}=\pdv{B_z}{y}-\pdv{B_y}{z}=0
        \end{split}
    \]
    Dato che non contribuiscono, le componenti dei campi in direzione $x$ possono essere considerate nulle.
\end{proof}
    Le onde elettromagnetiche sono fenomeni puramente trasversali.

\begin{lemma}
    In un'onda elettromagnetica piana campo elettrico e campo magnetico sono fra loro ortogonali.
\end{lemma}

\begin{proof}
    Dalle equazioni di Maxwell
    \[
        \begin{split}
            III & \Longrightarrow \pdv{E_z}{x}=\pdv{B_y}{t} \\
            III & \Longrightarrow \pdv{E_y}{x}=-\pdv{B_z}{t} \\
             IV & \Longrightarrow \pdv{B_z}{x}=- \mu\epsilon \pdv{E_y}{t} \\
             IV & \Longrightarrow \pdv{B_y}{x}=\mu\epsilon \pdv{E_z}{t} \\
        \end{split}
    \]
    Si osservi come se il campo elettromagnetico ha una componente $E_z$ deve avere anche una componente $B_y$
    e viceversa. Per il principio di sovrapposizione si può ottenere una soluzione generale
    sommando due soluzioni linearmente indipendenti, una con $\vb{E}$ diretto solo lungo $y$ e una con
    $\vb{E}$ diretto solo lungo $z$, opportunamente pesate.
    Si può allora portare avanti la discussione considerando il campo elettrico diretto solo in direzione
    una direzione, ad esempio $y$, ed il campo magnetico diretto di conseguenza solo in direzione $z$.
    \[
        E_z=0 \Longrightarrow \pdv{B_y}{x}=0,\pdv{B_y}{t}=0
    \]
    Ovvero, all'onda elettromagnetica non dà alcun contributo la componente del campo magnetico diretta lungo la direzione $y$
    e può quindi essere considerata nulla. Ma allora dato che per quanto dimostrato precedentemente era nulla la componente $x$
    dei campi, si ha che $\vb{E}=E_y$ e quindi $\vb{B}=B_z$. Siccome non si ha perdita di generalità nell'aver scelto una direzione
    fissa per i campi, si ha l'ortogonalità.
\end{proof}
Un'onda in cui i campi sono orientati in direzione fissa si dice avere polarizzazione lineare.

\begin{lemma}
    In un'onda elettromagntica piana si ha
    \[
        \frac{E_y}{B_z}=\frac{E}{B}=\pm v
    \]
\end{lemma}
\begin{proof}
    Per la dimostrazione di questo teorema sfrutta l'equazione di Maxwell
    \[
        III    \Longrightarrow \pdv{B_z}{x}= -\epsilon\mu \pdv{E_y}{t}
    \]
    Per il lemma precendente il campo elettrico è diretto solo in direzione $y$ mentre il campo magnetico è diretto solo in direzione $z$.
    L'equazione delle onde diventa perciò, per i due campi
    \[
        \begin{split}
            &\vb{E}(x\mp vt)=E_y(x\mp vt)=E_y(\xi)\\
            &\vb{B}(x\mp vt)=B_z(x\mp vt)=B_z(\xi)
        \end{split}
    \]
    Si osservi quindi come
    \[
        \pdv{E_y}{x}=\dv{E_y}{\xi}\quad \pdv{B_z}{t}=\dv{B_z}{\xi}\mp v
    \]
    L'equazione di Maxwell può essere scritta come
    \[
        \dv{E_y}{\xi}=\pm v \dv{B_z}{\xi}
    \]
    Che integrata, ponendo ragionevolmente a 0 la costante arbitraria dà $E_y=\pm v B_z$.
\end{proof}
Sostituendo $B=\mu H$ si ha che $E/H=\sqrt{\mu/\epsilon}=Z$. $Z$ ha le dimensioni di una resistenza e viene chiamata
\textit{impedenza caratteristica} del materiale. $Z_0=\sqrt{\mu_0/\epsilon_0}=377\Omega$.
Si osservi come la relazione che esprime il rapporto fra moduli del campo elettrico e del campo magnetico non costituisca una relazione
di confronto fisicamente significativa in quanto lega fra loro grandezze di dimensioni diverse.
Il seguente risultato, corollario del teorema appena dimostrato, fornisce un risultato relativo a questo confronto.
\begin{cor}
    In un'onda elettromagnetica piana, per ogni tempo e per ogni punto, le densità di energia associate a campo elettrico e
    campo magnetico sono uguali.
\end{cor}
\begin{proof}
    La dimostrazione è quasi immediata, grazie al teorema si ha infatti
    \[
        u_B=\rec{2}\frac{B^2}{\mu}=\rec{2}\frac{E^2}{v^2\mu}=\rec{2}\epsilon E^2=u_E
    \]
\end{proof}
Il teorema porta come conseguenza il fatto che campo elettrico e magnetico siano in fase.
Alla luce di questo risultato il vettore di Poynting può essere scritto come
\[
    \vb{I}=\rec{\mu} (\vb{B}\cp\vb{v})\cp\vb{B}=\frac{B^2}{\mu}\vb{v}=2u_M\vb{v}=(u_M+u_M)\vb{v}=(u_E+u_M)\vb{v}=u\vb{v}
\]
avendo indicato $u_E+u_M=u$: siccome $u$ è un'energia su un volume, questa è l'energia contenuta in un cilidro di
sezione unitaria perpendicolare alla direzione di propagazione dell'onda e con altezza pari alla velocità dell'onda,
il che conferma l'interpretazione fisica del vettore di Poynting fornita nel capitolo precedente:
il flusso del vettore di Poynting che come detto è l'energia del campo elettromagnetico che sfugge attraverso
una superficie $S$, è  di fatto l'energia che un'onda elettromagnetica trasporta nell'unità di tempo attraverso $S$.

%Un'altra forma significativa del vettore di Poynting è
%\[
%    \vb{I}=\epsilon E^2 \vb{v}=\epsilon E^2 \frac{\vu{v}}{\sqrt{\mu\epsilon}}=\frac{E^2}{Z}\vu{v}
%\]
%Da questa si ricava immediatamente il modulo
%\[
%    I(\vb{r},t)=E^2/Z=H^2Z
%\]
Si definisce \textit{intensità istantanea dell'onda} l'energia per unità di tempo e di superficie che fluisce
attraverso una superficie ortogonale alla direzione di propagazione della perturbazione.
\[
    \pdv{U}{t}=\pdv{u}{t}\dd{\tau}=\pdv{u}{t}\dd{l}\dd{S}=uv\dd{S}=I\dd{S}
\]
Ovvero, il modulo del vettore di Poynting è l'intensità istantanea.
Tipicamente, per la luce visibile si ha una frequenza di $10^{15}$Hz, cui corrisponde una lunghezza
d'onda di $10^3$[angstrom]: la maggioranza degli strumenti di misura (non ultimi i nostri occhi)
non hanno una risposta per tempi così piccoli. Di conseguenza è spesso più utile
parlare di intensità media $[I]\int_0^TI\dd{t}$.




Può essere utile descivere l'onda in un sistema di riferimento $sr$ in cui l'onda si propaga
in una direzione $\vu{n}$ qualunque, parallela a nessuno dei
vettori coordinati. Sia quindi $sr'$ un sistema di riferimento la cui origine coincide con l'origine di
$sr$ i cui assi sono però inclinati rispetto a quelli di $sr$ affinchè $\vu{n}$
sia parallelo ad $x'$ ($\vu{n}\equiv\vu{i}'$). La trasformazione fra i due sistemi di riferimento è data da una matrice di rotazione
\[
    \vb{r}'=\left(\begin{matrix}
                 \vu{i}'\vdot\vu{i} & \vu{i}'\vdot\vu{j} & \vu{i}'\vdot\vu{k} \\
                 \vu{j}'\vdot\vu{i} & \vu{j}'\vdot\vu{j} & \vu{j}'\vdot\vu{k} \\
                 \vu{k}'\vdot\vu{i} & \vu{k}'\vdot\vu{j} & \vu{k}'\vdot\vu{k} \\
            \end{matrix}\right)\vb{v}=
            R\vb{r}
\]
In $sr'$ l'onda piana ha forma $f(x'-vt)$. Esprimendo $x'$ in $sr$ si ha
\[f\bigl((\vu{i}'\vdot\vu{i})x + (\vu{i}'\vdot\vu{j})y + (\vu{i}'\vdot\vu{k})z \mp vt\bigr)=f(\vu{n}\vdot\vb{r}\mp vt)\]



Per quanto visto in appendice un'onda elettromagentica monocromatica che propaga in una direzione generica
è scritta nella forma
\[
    \begin{cases}
        & \vb{E}(\vb{r},t)=\vb{E}_0\cos(\vb{k}\vdot\vb{r}-\omega t)=\vb{E}_0e^{j(\vb{k}\vdot\vb{r}-\omega t)}\\
        & \vb{B}(\vb{r},t)=\vb{B}_0\cos(\vb{k}\vdot\vb{r}-\omega t)=\vb{B}_0e^{j(\vb{k}\vdot\vb{r}-\omega t)}
    \end{cases}
\]
 con $\omega=2\pi\nu$ e $\vb{k}=\vu{n}\omega/v$.
