Quanto visto porta ad un fenomeno analogo a quello delle onde stazionarie meccaniche.
Si consideri un'onda piana che si propaga con verso positivo rispetto all'asse $z$ e incide su un
conduttore perfetto (ovvero con conducibilità infinita) posto in $z=0$.
Verranno ora trattati campo elettrico e magnetico separatamente, cominciando col campo elettrico.
L'onda è polarizzata linearmente e
il campo elettrico può essere immaginato lungo la direzione $x$. Per l'ipotesi di conducibilità
infinita è trascurabile la parte di campo che penetra nel conduttore, ma siccome il campo elettrico
è tangente alla superficie di separazione per la condizione di continuità della componente tangente
$E=0$ immediatamente fuori dal conduttore. Nel semispazio vuoto sono presenti
sia presente l'onda incidente\footnote{Siccome il campo è diretto solo lungo la direzione $x$ l'equazione
può essere posta in forma scalare.} $E_i=E_{(+)}e^{i(kz-\omega t)}$
che l'onda riflessa $E_r=E_{(-)}e^{i(kz+\omega t)}$ che si sovrappongono.
Dato che il conduttore è perfetto le due onde hanno uguale ampiezza ma segno opposto.
Il campo elettrico nello spazio vuoto vale:
\[
    \begin{split}
        E_x=&\Re\bigl(E_i+E_r\bigr)=\Re\bigl(E_{(+)}e^{jkz}(e^{-j\omega t}-e^{j\omega t})\bigr)=\\
        =&\Re\bigl(-2jE_{(+)}\sin(\omega t)e^{jkz}\bigr)=2E_{(+)}\sin(kz)\sin(\omega t)
    \end{split}
\]
Questa non è un'onda perchè non ha $kz\pm\omega t$ come argomento: è un'oscillazione armonica di pulsazione
$\omega$ con ampiezza $2E_{(+)}\sin(kz)$
che presenta dei nodi fissi per i valori di $z$ tali da annullare il seno. La distanza fra due
nodi adiacenti vale
\[
    \Delta z=\frac{\pi}{k}=\frac{\pi}{2\pi}\lambda=\frac{\lambda}{2}
\]

Per quanto concerne il campo magnetico invece, per le condizioni di ortogonalità questo ha solo componente $y$.
Siccome $E/H=\pm Z$, dove si ha segno positivo per onde progressive e segno negativo per onde regressive,
\[
    \begin{split}
        H=&\Re(\frac{E_{+}}{Z}e^{i(kz-\omega t)}+\frac{-E_{(+)}}{-Z}e^{i(kz+\omega t)})=\Re(2\frac{E_{(+)}}{Z}e^{jkz}\cos(\omega t))\\
        =&2\frac{E_{(+)}}{Z}\cos(kz)\cos(\omega t)
    \end{split}
\]
Il campo magnetico è perciò sfasato di $\pi/2$ rispetto al campo elettrico. Inoltre, quando uno dei due campi è
nullo l'altro è massimo e vale il doppio del campo elettromagnetico totale. Questo risultato non è in contraddizione
col fatto che in un'onda elettromagnetica $E$ ed $H$ sono in fase, perchè un'onda stazionaria è il risultato
della sovrapposizione di due onde che viaggiano in versi opposti.

Sulla base di quanto detto si può capire cosa succede qualora venga lanciata un'onda elettromagnetica
fra due piani conduttori paralleli posti a distanza $L$, in modo tale che incida ortogonalmente sui due piani.
Siccome il campo elettrico si deve annullare sui due piani, ovvero deve avere dei nodi sui due piani, la
lunghezza $L$ deve contenere un numero intero di lunghezze d'onda
\[
    L=\frac{\lambda}{2}n
\]
