In questo paragrafo si vogliono studiare tre diverse sorgenti di onde elettromagnetiche.

%\subsection{Carica puntiforme in moto rettilineo uniforme}
%Si vuole calcolare il campo elettromagnetico generato da una carica in moto rettilineo uniforme. Per farlo
%è utile passare attraverso i potenziali ritardati. Ci si pone come primo obiettivo quindi quello di
%calcolare i potenziali ritardati per una particella carica puntiforme in qualsiasi stato di moto.
%\begin{thm}[Potenziali di Lienard-Wieckert]
%    I potenziali prodotti da una carica puntiforme $q$ in moto con velocità istantanea $\vb{v}$
%    in posizione istantanea $\vb{r}_s$ sono
%    \begin{equation}
%        \begin{split}
%            &V(\vb{r},t)=\rec{4\pi\epsilon_0}\frac{q}{\Delta r}\rec{1-\frac{v_r(t')}{c}}\\
%            &\vb{A}(\vb{r},t)=\frac{\mu_0}{4\pi}\frac{q\vb{v}(t')}{\Delta r}\rec{1-\frac{v_r(t')}{c}}=
%                V(\vb{r},t)\frac{\vb{v}(t')}{c^2}\\
%        \end{split}
%    \end{equation}
%    con $v_r$ la proiezione di $\vb{v}$ sulla direzione $\vb{r}-\vb{r}_s$,
%    $\Delta r=\abs{\vb{r}-\vb{r}_s}$ e $t'=t-\Delta r/c$.
%\end{thm}
%\begin{proof}
%Per la carica puntiforme le densità che compaiono nei potenziali ritardati assumono la forma
%    \[
%        \begin{split}
%            &\rho(\vb{r}',t')=q\delta^3 (\vb{r}'-\vb{r}_s(t'))\\
%            &\vb{J}(\vb{r}',t')=q\vb{v}_s(t')\delta^3 (\vb{r}'-\vb{r}_s(t'))\\
%        \end{split}
%    \]
%    dove $\delta^3$ è la delta di Dirac tridimensionale e $\vb{r}_s$, $\vb{v}_s$ sono posizione e velocità della particella.
%    I potenziali ritardati assumono quindi la forma
%    \[
%        \begin{split}
%            &V(\vb{r},t)=\frac{q}{4\pi\epsilon_0}\int_{\tau}\frac{\delta^3(\vb{r'}-\vb{r}_s(t'))}{\abs{\vb{r}-\vb{r}'}} \dd{\tau'}\\
%            &\vb{A}(\vb{r},t)=\frac{q}{4\pi\epsilon_0}\int_{\tau}\frac{\vb{v}_s(t')\delta^3(\vb{r'}-\vb{r}_s(t'))}{\abs{\vb{r}-\vb{r}'}} \dd{\tau'}\\
%        \end{split}
%    \]
%    Per poter calcolare gli integrali si usa il seguente artificio: si sostituisce $t'$ con $t^*$ e si integra tutto
%    sulla distribuzione $\delta(t^*-t')$
%    \[
%        \begin{split}
%            &V(\vb{r},t)=\frac{q}{4\pi\epsilon_0}\iint\frac{\delta^3(\vb{r'}-\vb{r}_s(t^*))}{\abs{\vb{r}-\vb{r}'}} \delta(t^*-t')\dd{t*}\dd{\tau}\\
%            &\vb{A}(\vb{r},t)=\frac{q}{4\pi\epsilon_0}\iint\frac{\vb{v}_s(t^*)\delta^3(\vb{r'}-\vb{r}_s(t^*))}{\abs{\vb{r}-\vb{r}'}} \delta(t^*-t')\dd{t*}\dd{\tau}\\
%        \end{split}
%    \]
%    Scambiando l'ordine di integrazione, l'integrale sul volume della delta di Dirac tridimensionale seleziona gli $\vb{r}'=\vb{r}_s(t*)$
%    \[
%        \begin{split}
%            &V(\vb{r},t)=\frac{q}{4\pi\epsilon_0}\int\rec{\abs{\vb{r}-\vb{r}_s(t*)}} \delta(t^*-t')\dd{t'}\\
%            &\vb{A}(\vb{r},t)=\frac{q}{4\pi\epsilon_0}\int\frac{\vb{v}_s(t^*)}{\abs{\vb{r}-\vb{r}_s(t^*)}} \delta(t^*-t')\dd{t'}\\
%        \end{split}
%    \]
%    Questo integrale non può essere ancora eseguito, perchè $t'$ è una funzione della distanza fra $\vb{r}$ e $\vb{r}_s(t^*)$
%    e quindi dipende da $t^*$.
%    È noto che
%    \[
%        \delta (f(t^*))=\sum_i\frac{\delta(t^*-t_i)}{\abs{f'(t_i)}}
%    \]
%    dove i $t_i$ sono zeri della funzione $f$. Fissato il sistema di coordinate e la traiettoria della sorgente
%    c'è un solo $t'$ possibile e quindi l'espressione per la delta si riduce a
%    \[
%        \begin{split}
%            &\delta(t^*-t')=\delta(t*-t')(\pdv{t*}(t*-t'))^{-1}=\delta(t*-t')[\pdv{t*}(t*-(t-\rec{c}\abs{\vb{r}-\vb{s}_s(t*)}))]^{-1}=\\
%            &=\delta(t*-t')(1+\rec{c}\frac{\vb{r}-\vb{r}_s(t')}{\abs{\vb{r}-\vb{r}_s(t')}}\vdot (-\vb{v}(t')))^{-1}\\
%            &=\delta(t*-t')(1-frac{\vb{v}}{c}\frac{\vb{r}-\vb{r}_s}{\abs{\vb{r}-\vb{r}_s}})^{-1}\\
%        \end{split}
%    \]
%
%
%    Sostituendo nell'integrale, si ha la tesi.
%\end{proof}
%
%Si immagini ora una carica puntiforme $q$ che si muove con velocità $\vb{v}=v\vu{x}$ costante
%($\vb{r'}=x'\vu{x}$).
%Si vogliono calcolare i potenziali generati dalla carica nel punto $\vb{r}$ al tempo $t$.
%Ponendo a $t=0$ il momento in cui la carica si trova in $x'=0$, al tempo $t$ la carica si trova in
%$x'=v=vt$. Per calcolare i potenziali è però necessario conoscere la distanza fra il punto in cui si vuole calcolare
%il potenziale e la posizione  in cui si trovava la carica al tempo $t'=t-\Delta r/c$. Chiaramente $x'=vt'$,
%da cui segue che $\Delta r=\sqrt{(x-vt')^2+y^2+z^2}$. Invertendo l'equazione per $t'$ si trova
%$\Delta r=c(t-t')$. Sostituendo ed elevando al quadrato si ottiene $c^2(t-t')^2=(x-v't)^2+x^2+y^2$.
%Svolgendo i quadrati e risolvendo per $t'$
%\[
%    \Biggl(1-\frac{v^2}{c^2}\Biggr)t'=t-\frac{vx}{c^2}-\rec{c}\sqrt{(x-vt)^2+\Biggl(1-\frac{v^2}{c^2}\Biggr)(y^2+z^2)}
%\]
%Che può essere scritta in forma più compatta chiamando $\gamma^{-2}=1-v^2/c^2$
%\[
%    \frac{t'}{\gamma^2}=t-\frac{vx}{c^2}-\rec{c}\sqrt{(x-vt)^2+\frac{y^2+z^2}{\gamma^2}}
%\]
%Si trova agevolmente $\Delta r$ sostituendo il valore ottenuto per $t'$ nell'espressione $\Delta r=c(t-t')$.
%Inserendo questo risultato nei potenziali di Lienard-Wieckert e raccogliendo un $\gamma^{-2}$ al denominatore
%\[
%    \begin{split}
%        &V(\vb{r},t)=\rec{4\pi\epsilon_0}\frac{q\gamma}{\sqrt{\gamma^2(x-vt)^2+y^2+z^2}}\\
%        &\vb{A}(\vb{r},t)=\frac{v}{c^2}V(\vb{r},t)\vu{x} \\
%    \end{split}
%\]
%Ora è possibile ricavare i campi (RIVEDO CONTI)
%\[
%    \begin{split}
%        &\vb{E}=-\grad{V}-\pdv{\vb{A}}{t}= \frac{q}{4\pi\epsilon_0}\frac{x\vu{x}+\gamma y\vu{y}+\gamma z \vu{z}}{(\Delta r)^3}\\
%        &\vb{B}=\curl{\vb{A}}=\pdv{A}{z}\vu{y} - \pdv{A}{y}\vu{z}=\frac{q\gamma v}{4\pi c^2}\rec{(\Delta r)^3}(z\vu{y}-y\vu{z})\\
%    \end{split}
%\]
%
%Per comprendere l'andamento qualitativo del campo elettromagnetico ci si limiti a considerare il piano $xy$. Qui
%$E_z=0$ e si ritrova facilmente la relazione $\vb{b}=(\vb{v}\cp\vb{E})/c^2$. Se si pone l'osservatore sull'asse $x$
%anche la componente $y$ del campo sarà nulla e l'espressione per il campo elettrico si riduce a
%\[
%    \vb{E}=\rec{\gamma}\frac{q}{4\pi\epsilon_0}\frac{x}{\abs{x-vt}}\vu{x}
%\]
%Siccome $\gamma$ cresce al crescere della velocità, più la particella va veloce più il campo elettrico viene soppresso. Viceversa
%ponendo l'osservatore perpendicolarmente al moto della particella, alla stessa altezza della particella, il campo si riduce a
%\[
%    \vb{E}= \gamma\frac{q}{4\pi\epsilon_0}\frac{x}{\abs{x-vt}}\vu{x}
%\]

\subsection{Dipolo oscillante}
Ci si ponga nel vuoto.
Un dipolo oscillante è costituito da due cariche uguali e opposte poste a distanza variabile nel tempo.
Equivalentemente però, si può consdierare la distanza fissa e le cariche variabili. Il dipolo può essere
quindi schematizzato con un segmento rettilineo di conduttore
percorso da corrente alternata. Fisicamente questa situaione può essere realizzata prendendo un conduttore rettilineo
con alle estremità due sfere costituenti le armature di un condensatore. Il circuito equivalente a tale dispositivo
è costituito da un generatore di corrente alternata, una resistenza ed un condensatore tutti disposti in serie.
Nell'ipotesi in cui
$\lambda=c\,2\pi/\omega$ (ovvero la lunghezza d'onda della radiazione emessa) sia molto maggiore della
lunghezza del conduttore, la corrente è indipendente dalla posizione sul conduttore. Si consideri inizialmente
per semplicità che la corrente vari in modo armonico $I(\vb{r},t)=I(t)=I_0\cos{\omega t}$. La carica presente sulle armature del
condensatore vale
\[
    q(t)=\int I\dd{t}=\frac{I_0}{\omega}\sin{\omega t}
\]
Sia $d$ la distanza fra le maglie del condensatore (ovvero la lunghezza del conduttore rettilineo),
orientato ortogonamente all'asse $z$ con il centro nell'origine del sistema di coordinate.
Le due sfere dotate di carica $+q$ e $-q$ costituiscono un dipolo con momento
\[
    \vb{p}=qd\vu{k}=\frac{I_0\,d}{\omega}\sin{\omega t}\vu{k}
\]
Ci si ponga nella gauge di Lorentz. Il potenziale vettore vale,
per le equazioni dei potenziali ritardati \eqref{eqn:potenziali_ritardati}
\[
    \vb{A}(\vb{r},t)=\frac{\mu_0}{4\pi}\int_\tau
    \frac{\vb{J}(\vb{r}',t-\abs{\vb{r}-\vb{r}'}/c)}{\abs{\vb{r}-\vb{r}'}}\dd{\tau'}
\]
Se $S$ è la sezione del conduttore, $\vb{J}(\vb{r},t)\vdot\vb{S}=I(t)\vu{k}$ dove per le ipotesi fatte
$I$ non dipende da $t$.
Inoltre siccome le coordinate con l'apice si riferiscono all'interno del conduttore $\dd{\tau'}=S\dd{z'}$.
Nell'ipotesi di essere molto lontani dal conduttore $\abs{\vb{r}-\vb{r}'}\simeq r$ si ha
\[
    \vb{A}(\vb{r},t)=\frac{\mu_0}{4\pi}\int\frac{I(t-r/c)}{r}\dd{z'}\vu{k}=\frac{\mu_0 I_0 d}{4\pi}
    \frac{\cos[\omega(t-r/c)]}{r}\vu{k}=
    \frac{\mu_0}{4\pi}\frac{\dot{\vb{p}}(t-r/c)}{r}
\]
Dalla condizione di Lorentz si può ricavare il potenziale scalare. Tenendo conto che $\vb{A}=A\vu{k}$ si ha
\[
    \pdv{V}{t}=-\rec{\epsilon_0\mu_0}\pdv{\vb{A}}{z}=\rec{4\pi\epsilon_0}\Biggl(\frac{\ddot{p}(t-r/c)}{cr}+
    \frac{\dot{p}(t-r/c)}{r^2} \Biggr)\frac{z}{r}
\]
Da cui, facendo un integrale indefinito rispetto al tempo e ponendo a zero la costante arbitraria
\[
    V(\vb{r},t)=\rec{4\pi\epsilon_0}\Biggl(\frac{\dot{p}(t-r/c)}{cr}+ \frac{p(t-r/c)}{r^2} \Biggr)\frac{z}{r}
\]

Si possono ora ricavare i campi a partire dalle \eqref{eqn:def_potenziale_scalare} e dalla
\eqref{eqn:def_potenziale_vettore}. Il calcolo è semplice se effettuato in coordinate sferiche, basta tenere
conto che $z/r=\cos{\theta}$ e che il versore $\vu{k}$ si scrive $\vu{k}=(\cos{\theta},-\sin{\theta},0)$.
Si ottiene infine
\[
    \begin{split}
        & B_r=0 \\
        & B_\theta=0 \\
        & B_\phi=\frac{\mu_0}{4\pi}\frac{\sin{\theta}}{r} \Biggl(\frac{\ddot{p}(t-r/c)}{c}+\frac{\dot{p}(t-r/c)}{r}\Biggr) \\
        & E_r=\rec{4\pi\epsilon_0}\frac{2\cos{\theta}}{r} \Biggl(\frac{\ddot{p}(t-r/c)}{c}+\frac{\dot{p}(t-r/c)}{r}\Biggr) \\
        & E_\theta=\rec{4\pi\epsilon_0}\frac{\sin{\theta}}{r} \Biggl(\frac{\ddot{p}(t-r/c)}{c^2}+\frac{\dot{p}(t-r/c)}{cr})+
        \frac{p(t-r/c)}{r^2}\Biggr) \\
        & E_\phi=0
    \end{split}
\]
Chiaramente $\vb{B}\vdot\vb{E}=0$, a conferma che i campi sono ortogonali. Le linee di forza del campo magnetico
sono circonferenze centrate sull'asse $z$, mentre il campo elettrico si trova nel piano $zr$. Tutti i termini che
contribuiscono ai campo hanno una dipendenza spazio-temporale del tipo $t-r/c$ e sono divisi per potenze di $r$:
i fronti d'onda sono dunque sfere che si propagano con velocità $c$ sulle quali l'ampiezza dipende da $\theta$
e si attenua all'aumentare di $r$.

Il vettore di Poynting vale, lasciando sottointesa la dipendenza di $p$ da $t-r/c$
\[
    \begin{split}
        \vb{I}=&\frac{\vu{r}}{16\pi^2\epsilon_0}\Biggl[ \Biggl(\frac{\ddot{p}}{c^6r}\Biggr)^2 \sin^2{\theta}-
        \Biggl(2\frac{\ddot{p}\dot{p}}{c^2r}+\frac{\dot{p}p}{r^3}+\frac{\ddot{p}p+\dot{p}^2}{cr^2}\Biggr)\frac{\sin^2{\theta}}{r^2}\Biggr]+ \\
        +&\frac{\vu{\theta}}{16\pi^2\epsilon_0}\Biggl[ \Biggl(\frac{\ddot{p}\dot{p}}{c^2r}+\frac{\dot{p}p}{r^3}+
        \frac{\ddot{p}p+\dot{p}^2}{cr^2}\Biggr)\frac{2\sin{\theta}\cos{\theta}}{r^2} \Biggr]
    \end{split}
\]
Si considerino ora le espressioni per $\vb{p}$ e le sue derivate. Chiamando $p_0=I_0d/\omega$, si ha:
$p=p_0\sin(\omega(t-r/c))$, $\dot{p}=\omega p_0\cos(\omega(t-r/c))$, $p=-\omega^2p_0\sin(\omega(t-r/c))$.
Chiaramente i termini $\ddot{p}\dot{p}$ e $\dot{p}p$ hanno media temporale nulla e inoltre, mediando su un
periodo $\ddot{p}p=-\dot{p}^2$, per cui $\ddot{p}p+\dot{p^2}=0$.
Mediando temporalmente il vettore di Poynting quindi si ottiene
\begin{equation}
    \vb{I}=\frac{\sin^2{\theta}}{16\pi^2\epsilon_0 c^3}\frac{\ddot{p}}{r^2}\vb{r}\Rightarrow
    [\vb{I}]\frac{\omega^4 p_0^2 \sin^2{\theta}}{32\pi^2\epsilon_0 c^3}\rec{r^2}\vb{r}
    \label{eqn:valormedio_poynting}
\end{equation}
ovvero restano solo i termini che decrescono come $r^{-2}$. Ne segue che il flusso attraverso una sfera centrata
nella sorgente non dipende da $r$. Questo flusso fornisce la potenza media irraggiata dal dipolo.
\[
    [P]=\int_S\bar{\vb{I}}\vdot\dd{\vb{S}}= \int_S \bar{\vb{I}}r^2\sin{\theta}\dd{\theta}\dd{\phi}=\frac{\omega^4 p_0^2}{2}\rec{6\pi\epsilon_0 c^3}=
\]
ovvero
\begin{equation}
    [P]=\frac{\omega^4 p_0^2}{2}\frac{\mu_0}{6\pi c}=\frac{\mu_0}{6\pi c}[p]
    \label{eqn:potenza_dipolo}
\end{equation}
Per il principio di sovrapposizione, la formula trovata per la potenza è applicabile ad un dipolo oscillante con legge
qualunque sviluppando la legge oraria in serie di Fourier o ad un insieme di dipoli oscillanti.
Per quanto riguarda i termini inversamente proporzionali a potenze di $r$ maggiori di due, questi sono nulli quando si
considera il flusso medio ma sono anche trascurabili quando si è lontani dal dipolo ($r>>\lambda$). Nel caso in cui
$r<<\lambda$ questi costituiscono il \textit{campo vicino}, un campo variabile nel tempo ma localizzato intorno alla
sorgente, al quale non è associato alcun trasporto di energia.

\subsection{Carica puntiforme in moto accelerato}
Si consideri una carica $q$ dotata di accelerazione $a$. È possibile usare gli stessi risultati ottenuti nel caso del
dipolo oscillante a patto di sostituire $\ddot{p}^2$ con $(qa)^2$ nella \eqref{eqn:valormedio_poynting}.
Con passaggi analoghi a quelli visti nel caso del dipolo, per velocità non relativistiche si ottiene la \textit{formula di Larmor}
\begin{equation}
    \bar{P}=\frac{\mu_0}{6\pi c}(qa)^2
\end{equation}
