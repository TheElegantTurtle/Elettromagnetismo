A seconda della loro frequenza, le onde elettromagnetiche sono prodotte da tipi diversi di sorgente,
hanno proprietà diverse e diversi modi di interagire con la materia.

$10^3-10^9 Hz$ - \textit{Onde a radiofrequenza}.
Usate per le comunicazioni, sono prodotte da circuiti oscillanti accoppiati ad antenne.

$10^9-10^{11} Hz$ - \textit{Microonde}.
Usate per lo studio di strutture atomiche o molecolari e nelle comunicazioni, sono prodotte da circuiti oscillanti associati
a dispositivi meccanici come cavità risonanti  o guide d'onda.

$5\vdot 10^{11}-4\vdot 10^{14} Hz$ - \textit{Infrarossi}.
Viene ulteriormente divisa in \textit{lontano, medio, vicino infrarosso}. È spontaneamente emessa
dai corpi caldi.

$4\vdot 10^{14}-8\vdot 10^{14} Hz$ - \textit{Radiazione visibile}.
Viene emessa da atomi e molecole quando i relativi elettroni compiono una transizione da uno
stato eccitato allo stato fondamentale.

$8\vdot 10^{14}-3\vdot 10^{17} Hz$ - \textit{Radiazione ultravioletta}.
Anche questa è emessa da atomi e molecole, in particolare nei gas sottoposti ad una scarica elettrica.

$3\vdot 10^{17}-5\vdot 10^{19} Hz$ - \textit{Raggi X}.
Sono generati da cariche che subiscono una forte accelerazione, come un raggio catodico che viene
bruscamente fermato dall'impatto con un materiale soldio. La radiazione emessa (costituita non solo da
raggi X) prende il nome di \textit{bremsstrahlung}. Molte stelle o ammassi stellari sono sorgenti di
questo tipo di radiazione. Trovano applicazione in radiochimica e medicina grazie al diverso assorbimento
ad opera di materiali con diversa densità e consistenza.

$- 10^{18} Hz$ - \textit{Raggi gamma}.
La loro emissione si accompagna a molti processi nucleari. A queste frequenze la descrizione
dell'interazione fra campo elettromagnetico e materia non può prescindere dalla meccanica
quantistica.
