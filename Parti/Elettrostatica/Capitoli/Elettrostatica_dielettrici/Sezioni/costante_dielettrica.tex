\label{par:costante_dielettrica}
\begin{obses}
Preso un condensatore a geometria piana e riempito lo spazio fra le armature con un materiale dielettrico omogeneo e isotropo,
lasciando inalterata la geometria del consensatore, si osserva che a parità di carica la differenza di potenziale diminuisce.
\end{obses}
Dalla definizione di capacità, questo significa che la capacità è aumentata.
Ma dalla \eqref{eqn:capacità_piano} si osserva che ciò può essere dovuto esclusivamente
ad una variazione della costante dielettrica\footnote{Lo stesso discorso si può condurre per condensatori sferici e cilindici,
infatti per quanto ben più complicate, le espressioni della capacità dipendono anche in quel caso esclusivamente
dalla costante dielettrica e dalla geometria del condensatore.}.
\begin{defn}[Costante dielettrica relativa]
  Si definisce costante dielettrica relativa $\epsilon_r=C/C_0>1$,
  dove $C$ è la capacità del condensatore immerso nel dielettrico e $C_0$ la capacità del condensatore nel vuoto.
\end{defn}

\begin{defn}[Costante dielettrica assoluta]
  Si definisce costante dielettrica assoluta $\epsilon=\epsilon_0\epsilon_r>\epsilon_0$.
\end{defn}
In questo modo si ha che
\[
C=\epsilon_r C_0=\epsilon_r\,\epsilon_0\frac{S}{d}=\epsilon\frac{S}{d}
\]
Pragmaticamente quindi, l'effetto della presenza di un dielettrico in tutto lo spazio interessato da un potenziale
generato da un sistema di cariche è quello di diminuire il potenziale secondo un fattore $\epsilon_r$.
