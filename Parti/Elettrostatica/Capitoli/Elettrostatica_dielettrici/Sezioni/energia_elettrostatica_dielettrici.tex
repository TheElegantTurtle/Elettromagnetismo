L'energia elettrostatica nei dielettrici si ottiene in modo analgo a come si è ottenuta la \eqref{eqn:energia_distribuzione}
considerando però che la disposizione nello spazio delle cariche libere a partire dall'infinito
induce una ridistribuzione delle cariche di polarizzazione che modifica il potenziale.
Bisognerebbe calcolare il lavoro necessario sia a costruire la distribuzione di cariche libere
che il lavoro necessario a costruire la configurazione delle cariche di polarizzazione,
ma siccome il potenziale contiene sia l'informazione relativa all'interazione fra le cariche libere
che l'informazione relativa all'interazione fra cariche libere e di polarizzazione
la \eqref{eqn:energia_distribuzione} continua a definire l'energia elettrostatica
con l'unica differenza che $\rho$ soddisfa la prima equazione di Maxwell in presenza di dielettrici.
Con passaggi analoghi a quelli già visti nel caso del vuoto si può esprimere
\[
U=\int u\dd{\tau}
\]
con
\[
u=\frac{\vb{D}\vdot\vb{E}}{2}
\]

Nel caso in cui il dielettrico sia isotropo si ha
\[
u=\rec{2}\epsilon E^2=\rec{2}\frac{D^2}{\epsilon}
\]
