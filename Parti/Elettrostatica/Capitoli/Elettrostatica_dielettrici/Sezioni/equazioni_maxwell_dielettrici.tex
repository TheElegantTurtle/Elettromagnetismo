\begin{defn}[Spostamento elettrico]
    Si definisce vettore di spostamento elettrico
    \[
        \vb{D}=\epsilon_0\vb{E}+\vb{P}
    \]
\end{defn}

\begin{obs}
    In dielettrici perfetti e isotropi, il vettore di spostamento elettrico ed il campo elettrico sono legati dalla relazione
    \begin{equation}
        \vb{D}=\epsilon\vb{E}
        \label{eqn:rel_D_E}
    \end{equation}
\end{obs}
\begin{proof}
    Nel caso di dielettrici perfetti isotropi dalla definizione si suscettibilità elettrica,
    coerentemente con l'uguaglianza simbolica introdotta prima
    $\vb{D}=\epsilon_0\vb{E}+\vb{P}=(\epsilon_0+\epsilon_0\chi)\vb{E}=\epsilon_0(1+\chi)\vb{E}=\epsilon_0\epsilon_r\vb{E=\epsilon\vb{E}}$.
\end{proof}

\begin{thm}
    In presenza di dielettrici la prima e la terza equazione di Maxwell assumono la seguente forma:
    \begin{equation}
        \div{\vb{D}}=\rho
    \end{equation}
    \begin{equation}
        \curl{\vb{E}}=0
    \end{equation}
\end{thm}
\begin{proof}
    In presenza di dielettrico la struttura del campo elettrico non cambia, per cui la sua circuitazione
    deve continuare ed essere nulla. Per quanto riguarda la prima equazione di Maxwell,
    introducendo la densità delle cariche di polarizzazione $\rho_p$, questa diventa
    \[
        \div{\vb{E}}=\frac{\rho+\rho_p}{\epsilon_0}
    \]
    Siccome la densità di carica di polarizzazione non è nota a priori, dalla \eqref{eqn:relazione_rhop_P},
    tenuto conto che $\epsilon_0$ è costante, si ottiene
    \[
        \begin{split}
            & \div{\epsilon_0 \vb{E}}=\rho-\div{\vb{P}}\\
            & \div{\epsilon_0 \vb{E}+\vb{P}=\rho}
        \end{split}
    \]
    Dalla definizione di vettore di spostamento elettrico, la tesi.
\end{proof}
