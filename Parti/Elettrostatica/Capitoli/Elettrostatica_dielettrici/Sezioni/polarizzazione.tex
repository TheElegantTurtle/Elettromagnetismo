Evidetentemente i dielettrici si comportano in modo molto diverso dai conduttori. Questa differenza di comportamento
a livello macroscopico è ricoducibile a differenze di comportamento microscopico. I conduttori tipicamente sono metalli,
caratterizzati da una struttura cristallina in cui ogni atomo si trova al vertice di un poliedro.
Questa struttura fa si che gli elettroni più esterni di ciascun atomo siano liberi
- ovvero l'energia di interazione di questi elettroni col nucleo sia inferiore all'energia di agitazione termica.
L'applicazione di un campo elettrico induce su questi elettroni un movimento ordinato. Nei dielettrici
gli elettroni sono invece fortemente legati al nucleo e vengono strappati dalla loro posizione
solo in seguito a forze localizzate molto intense (come lo strofinio). Complessivamente quindi il dielettrico
può considerarsi neutro. Nonostante questo però esso produce un campo elettrico quando possiede un momento di dipolo diverso da zero,
che viene indotto da un campo elettrico esterno. Questo fenomeno è chiamato
\textit{polarizzazione elettrica} e può essere di due tipi: per deformazione e per orientamento.
Nella descrizione del fenomeno di polarizzazione si fa riferimento al campo elettrico locale $\vb{E}_l$,
ovvero il campo agente sui singoli atomi (o sulle singole molecole) dovuto sia alle cariche libere localizzate
che generano il campo elettrico esterno che al campo generato dagli atomi (molecole) del dielettrico meno quello considerato.
\begin{description}
    \item[Polarizzazione per deformazione]
        L'atomo è schematizzabile come un sistema elettricamente neutro costituito da un nucleo puntiforme con carica $Q_+=Ze$
        e da una distribuzione di carica a simmetria sferica variabile con carica totale $Q_-=-Q_+$ dovuta alla nube elettronica.
        A causa della simmetria sferica
        il campo elettrico generato dall'atomo è nullo. Se però interviene un campo elettrico esterno
        il nucleo risente di una forza $\vb{f}_p=Ze\vb{E}_{l}$ e il baricentro della nube elettronica risente di una forza $\vb{f}_e=-Ze\vb{E}_{l}$.
        Ne consegue che il nucleo ed il baricentro della nube elettronica
        si allontanano di una sistanza $\vb{r}$ e si attraggono quindi con una forza $\vb{f'}$.
        Si raggiunge una situazione di equilibrio quando $\abs{\vb{f'}}=\abs{\vb{f}_p}\bigl(=\abs{\vb{f}_e}\bigr)$: in questa situazione si hanno due
        cariche uguali e opposte ad una distanza $\vb{r}$, ovvero un dipolo.
        Immaginando un modello in cui gli elettroni sono legati elasticamente ai nuclei è naturale aspettarsi che,
        per campi elettrici esterni non troppo intensi\footnote{Questa condizione è largamente rispettata
        per tutti i campi elettrici realizzabili nella pratica},
        $\vb{r}$ sia proporzionale all'intensità del campo elettrico.
        Ricordando la \eqref{eqn:momento_di_dipolo} il momento di dipolo dovuto alla deformazione può quindi essere scritto come
        \[
            \vb{p}=\alpha_d\vb{E}_{l}
        \]
        $\alpha_d$ viene detta polarizzabilità elettronica.
        Un fenomeno analogo lo si può osservare anche nelle molecole poliatomiche:
        oltre alla polarizzazione elettronica è presente quella che viene detta \textit{polarizzazione atomica}.

    \item[Polarizzazione per orientamento]
        Molte molecole non sono simmetriche e perciò possiedono un momento di dipolo. Poichè solitamente le molecole sono orientate casualmente,
        il momento di dipolo medio è nullo e perciò non si osservano effetti a livello macroscopico.
        Quando però viene applicato un campo elettrico esterno i momenti di dipolo
        tendono ad orientarsi parallelamente a questo, e quindi il valor medio risulta diverso da zero.
        \begin{lemma}
            Dato un insieme di dipoli praticamente liberi, per i quali siano trascurabili le mutue interazioni, il momento di dipolo medio vale
            \begin{equation}
                \expval{\vb{p}}=\alpha_0\vb{E}_l \quad\quad\quad \alpha_0=\frac{p_0^2}{3KT}
            \end{equation}
        \end{lemma}
        \begin{proof}
            In presenza del campo elettrico locale ogni dipolo è sottoposto ad un momento meccanico $\vb{M}=\vb{p}\cp\vb{E}_l$
            che tende ad orientarlo come $\vb{E}_l$; l'agitazione termica invece favorisce l'orientamento casuale.
            L'equilibrio statistico fra queste due tendenze è descritto dalla funzione di Boltzmann
            \[
                P(U) = A e^{-\frac{U}{k_bT}}
            \]
            dove $A$ è una costante di normalizzazione, $U$ è l'energia del dipolo, $k_b$
            è la costante di Boltzmann e $T$ è la temperatura in gradi kelvin.
            Fissato il momento di dipolo ed il modulo del campo elettrico locale, si ha per la \ref{eqn:U_dipolo}
            $U=-p_0E_l\cos\theta=U(\theta)$, con $\theta$ l'angolo fra i due vettori.
            Ambientando il problema in un sistema di riferimento con l'asse $z$ rivolto nello stesso verso del campo $\vb{E}_l$,
            la probabilità che il dipolo sia orientato entro un angolo solido $\dd{\Omega}=\dd{\theta}\dd{\phi}$ è
            \[
                \dd{P}=P(U(\theta))\sin\theta \dd{\theta}\dd{\phi}=A e^{\frac{pE_l\cos\theta}{KT}}\sin\theta \dd{\theta}\dd{\phi}
            \]
            Nell'ipotesi di essere lontani dallo zero assoluto, e che l'intensità del campo locale non sia
            troppo alta, l'esponenziale può essere sviluppato al primo ordine ottenendo:
            \[
                \dd{P}=A \Biggl(1 + \frac{pE_l\cos\theta}{KT}\Biggr) \sin\theta \dd{\theta}\dd{\phi}
            \]
            È conveniente ora cambiare variabile: $x=\cos\theta$, $\dd{x}=-\sin\theta \dd{\theta}$.
             Si ottiene quindi
            \[
                \dd{P}=A \Biggl(1+\frac{pE_l}{KT}x\Biggr)\dd{x}\dd{\phi}
            \]
            La costante di normalizzazione si ottiene imponendo che la probabilità su tutto l'angolo
            solido\footnote{$\phi$ va da $0$ a $2\pi$, $x$ da $1$ a $-1$.} sia 1
            da cui segue, svolgendo l'integrale, $A=1/(4\pi)$.
            Si osservi ora come $\vb{p}$ abbia simmetria cilindrica attorno a $\vb{E}_l$ e la componente del momento di dipolo ortogonale
            al campo elettrico sia dunque, in media, nulla: $\expval{\vb{p}}$ è orientato come $\vb{E}_l$.
            In modulo il valor medio del momento di dipolo vale allora
            \[
                \abs{\expval{\vb{p}}}=\abs{\expval{\vb{p}_{//}}}=\int_0^{2\pi}\int_0^{\pi} p \cos{\theta}\dd{P}=
                \int_{+1}^{-1}-\rec{2}\Biggl(px + \frac{p^2E_l }{KT}x^2 \Biggr)\dd{x}=\frac{p^2}{3k_b T}E_l
            \]
            ovvero la tesi.
        \end{proof}
        $\alpha_0$ viene detta polarizzabilità molecolare.
        Il risultato ottenuto ha evidenti analogie con quello dedotto nel caso della polarizzazione per deformazione, con la sola differenza che
        in questo caso non è direttamente il momento di dipolo ma il suo valor medio ad essere proporzionale al campo elettrico.

\end{description}
Sulla base di queste considerazioni ci si aspetta che un dielettrico immerso in un campo elettrico
possegga un momento di dipolo medio $\expval{\vb{p}}$ non nullo, orientato come il campo elettrico
esterno.

L'effetto macroscopico dei fenomeni appena elencati può essere descritto introducedo il vettore polarizzazione elettrica.
\begin{defn}[Vettore polarizzazione elettrica]
    Si definisce il vettore polarizzazione elettrica $\vb{P}$ come il momento di dipolo elettrico per unità di volume posseduto dal dielettrico
    \[
        \vb{P}=\lim_{\tau\to 0}\frac{\sum \vb{p}_i}{\tau}=\frac{\dd{N}  \expval{\vb{p}}}{\dd{\tau}}
    \]
\end{defn}
Con $\dd{N}$ il numero di molecole contenute nell'elemento di volume $\dd{\tau}$.
Dalla definizione si ottiene che l'elemento di volume possiede un momento di dipolo $\dd{\vb{p}}=\vb{P}\dd{\tau}$.
Sebbene quindi le cariche in un dielettrico non siano libere di muoversi, il fatto che un dielettrico sia polarizzato
può essere schematizzato con la presenza sulla sua superficie di cariche aggiuntive con una distribuzione $\sigma_p$
e nel suo volume di cariche aggiuntive con distribuzione $\rho_p$.
\begin{thm}
    In un dielettrico polarizzato si ha che
    \begin{equation}
        \sigma_p=\vb{P}\vdot\vb{n}
    \end{equation}
    \begin{equation}
        \label{eqn:relazione_rhop_P}
        \rho_p=-\div{\vb{P}}
    \end{equation}
\end{thm}
\begin{proof}
    Si vuole calcoalre il campo elettrico generato da un dielettrico che occupi un volume $\tau$ e dotato di polarizzazione $\vb{P}(x',y',z')$. Dalla \eqref{eqn:V_dipolo} si ha che l'elemento di volume $\dd{\tau}$ in posizione $\vb{r'}$ porta al potenziale $V(x,y,z)=V(\vb{r})$ il contributo infinitesimo
    \[
        \dd{V(\vb{r})}=\rec{4\pi\epsilon_0}\frac{\vb{P}\vdot(\vb{r}-\vb{r'})}{\abs{\vb{r}-\vb{r'}}^3}\dd{\tau}
    \]
    Di conseguenza il potenziale su tutto lo spazio vale
    \[
        V(\vb{r})=\rec{4\pi\epsilon_0}\int_{\tau} \frac{\vb{P}(\vb{r'})\vdot(\vb{r}-\vb{r'})}{\abs{\vb{r}-\vb{r'}}^3}\dd{\tau'}
    \]
    Si può scrivere per la \eqref{app:eqn:grad_r}
    \[
        V(\vb{r})=\rec{4\pi\epsilon_0}\int_{\tau} \vb{P}\vdot \grad'{\rec{\abs{\vb{r}-\vb{r'}}}}\dd{\tau'}
    \]
    e, ricordando la \eqref{app:eqn:div_scalare_vettore}
    \[
        V(\vb{r})=\rec{4\pi\epsilon_0}\int_{\tau} \grad'\vdot{\Biggl(\frac{\vb{P}}{\abs{\vb{r}-\vb{r'}}}\Biggr)}\dd{\tau'}-\rec{4\pi\epsilon_0}\int_{\tau}{\frac{ \grad'\vdot\vb{P}}{\abs{\vb{r}-\vb{r'}}}}\dd{\tau'}
    \]
    Grazie al teorema della divergenza il primo integrale può essere riscritto, indicando con $S$ la superficie che racchiude il volume
    \[
        \int_{\tau} \grad'\vdot{\Biggl(\frac{\vb{P}}{\abs{\vb{r}-\vb{r'}}}\Biggr)}\dd{\tau'}=\int_{S} \frac{\vb{P}\vdot\vb{n}}{\abs{\vb{r}-\vb{r'}}}\dd{S}
    \]
    Siccome la soluzione al problema dell'elettrostatica deve essere unica e il potenziale deve anche essere uguale a
    \[
        V(\vb{r})=\rec{4\pi\epsilon_0}\int_{S} \frac{\sigma(\vb{r}')}{\abs{\vb{r}-\vb{r'}}}\dd{S'}+\rec{4\pi\epsilon_0}\int_{\tau}\frac{\rho(\vb{r}')}{\abs{\vb{r}-\vb{r'}}}\dd{\tau'}
    \]
    Per confronto fra le integrande si ottiene la tesi.
\end{proof}
Come è intuitivo, se il vettore polarizzazione è uniforme (da leggersi "è indipendente dalla posizione") il volume del dielettrico
è complessivamente neutro e le cariche di polarizzazione si manifestano solo in superficie.

Si vuole ora trovare una relazione fra il vettore polarizzazione e il campo elettrco macroscopico che agisce internamente al dielettrico.
Per farlo, è necessario prima puntualizzare alcuni aspetti del campo elettrico locale.
Il campo agente sulla singola molecola è generato sia dalle cariche libere che dalle molecole circostanti quella considerata:
quest'ultimo contributo dipende fortemente sia dalla posizione che dal tempo.
Si tratta però di posizioni e tempi molto piccoli rispetto al sistema macroscopico
e perciò si sceglie di considerare regioni di spazio piccole rispetto al sistema macroscopico
ma comunque grandi rispetto a lunghezze e tempi atomici: in questo modo la media dei campi elettrici
su queste porzioni di spazio dipende in maniera regolare dalla posizione.
Questa media viene indicata con $\vb{E}_l$ e ci si riferirà sempre a quest'ultima
quando si parlerà di campo elettrico locale.


Per quanto visto, ci si aspetta che il vettore polarizzazione sia proporzionale al campo elettrico locale, ovvero che abbia una forma
\[
    \vb{P}=n\alpha\vb{E}_l \quad\quad\quad n=\dv{N}{\tau},\,\alpha=\alpha_0+\alpha_d
\]
In generale la dipendenza del vettore polarizzazione dal campo elettrico può essere espressa nella forma
\[
    \begin{cases}
        &  P_x=\alpha_{11}E_x+\alpha_{12}E_y+\alpha_{13}E_z \\
        &  P_y=\alpha_{21}E_x+\alpha_{22}E_y+\alpha_{23}E_z \\
        &  P_z=\alpha_{31}E_x+\alpha_{32}E_y+\alpha_{33}E_z
    \end{cases}
\]
la matrice dei coefficenti $\alpha$, non necessariamente costanti, si chiama \textit{tensore di polarizzazione}.
Si dice \textit{perfetto} un dielettrico con matrice di polarizzazione costante.
Questa struttura del vettore polarizzazione diventa particolarmenete utile
nel momento in cui si trattano \textit{solidi cristallini anisotropi}. In alcuni casi questi dielettrici,
che prendono quando ciò si verifica il nome di \textit{ferroelettrici}, possono presentare una polarizzazione elettrica permanente
che segue una \textit{curva di isteresi} in funzione di $\vb{E}$.
In questi materiali si parla di \textit{piezoelettricità}, ovvero la polarizzazione
dipende dalle sollecitazioni meccaniche a cui il cristallo è sottopsto.

Si introduce ora
\begin{defn}[Suscettibilità elettrica]
    Si definisce suscettibilità elettrica
    \[
        \chi=\frac {P}{\epsilon_0E}
    \]
\end{defn}

\begin{obs}
    Per sostanze a bassa densità
    \[
        \chi=\frac{n}{\epsilon_0}\alpha=\frac{n}{\epsilon_0}\Biggl(\alpha_d+\frac{p^2}{3KT}\Biggr)
    \]
\end{obs}
\begin{proof}
    L'ipotesi di bassa densità implica che le interazioni reciproche fra le molecole siano trascurabili, per cui $\vb{E}_l\simeq\vb{E}$. Da questo segue
    \[
        \epsilon_0\chi\vb{E}=\vb{P}=n\Biggl(\alpha_d+\frac{p^2}{3KT}\Biggr)\vb{E}_l \simeq n\Biggl(\alpha_d+\frac{p^2}{3KT}\Biggr)\vb{E}
    \]
    Si ha quindi la tesi.
\end{proof}

Per i liquidi densi invece l'interazione fra molecole non è trascurabile. Si da il seguente lemma di cui viene omessa la dimostrazione.
\begin{lemma}[relazione di Lorentz]
    Nell'ipotesi in cui il campo generato dalle molecole sia puramente dipolare, la distribuzione dei dipoli sia uniforme,
    il momento dei dipoli sia parallelo al campo esterno e che i momenti di dipolo siano tutti uguali fra loro si ha:
    \[
        \vb{E}_l=\vb{E}+\frac{\vb{P}}{3\epsilon_0}
    \]
\end{lemma}

\begin{thm}[relazione di Clausius-Mossotti]
    Per un dielettrico perfetto, nelle ipotesi del lemma
    \[
        \alpha=\frac{3\epsilon_0}{n}\frac{\epsilon_r-1}{\epsilon_r-2}
    \]
\end{thm}
\begin{proof}
    Il vettore polarizzazione vale
    \[
        \vb{P}=n\alpha\Biggl(\vb{E}+\frac{\vb{P}}{3\epsilon_0}\Biggr)
    \]
    da cui
    \[
        \vb{P}=\Biggl(\frac{n\alpha}{1-\frac{n\alpha}{3\epsilon_0}}\Biggr)\vb{E}
    \]
    Per confronto con la definizione di suscettibilità elettrica si ha
    \[
        \chi=\frac{n\alpha}{1-\frac{n\alpha}{3\epsilon_0}}\rec{\epsilon_0}=\frac{3n\alpha}{3\epsilon_0-n\alpha}
    \]
    Si introduce per i dielettrici perfetti l'uguaglianza simbolica $\chi=\epsilon_r-1$,
    che più avanti nel capitolo troverà una giustificazione\footnote{Si vedano le considerazioni in calce
    al teorema \ref{thm:problema-elettrostatica-dielettrici}}.
    \[
        \epsilon_r-1=\frac{3n\alpha}{3\epsilon_0-n\alpha}
    \]
    Risolvendo l'equazione per $\alpha$ si trova la tesi.

\end{proof}
