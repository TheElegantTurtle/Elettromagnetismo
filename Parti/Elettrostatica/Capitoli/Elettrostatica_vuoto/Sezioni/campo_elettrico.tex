Prese due cariche $q$ e $Q$ la quantità $\vb{f}/q$ è indipendente da $q$.
\begin{defn}
    Si definisce campo elettrico $\vb{E}$ generato dalla carica Q il rapporto $\vb{f}/q$. Q viene detta sorgente del
    campo; q viene detta carica di prova.
\end{defn}
Esplicitamente si ha
\begin{equation}
    \vb{E}(\vb{r})=\frac{1}{4\pi\epsilon_0}\frac{q}{r^2}\vu{r}
    \label{eqn:E}
\end{equation}
Il modulo di $\vb{E}$ viene detto \textit{intensità} del campo elettrico.
È importante osservare che se le sorgenti sono disposte su un corpo esteso, la presenza di una carica di prova può
modificarne la distribuzione. Convenzionalmente allora si suole definire il campo elettrico come
\[
    \vb{E}=\lim_{q\to 0}\frac{\vb{f}}{q}
\]
Questa definizione è solo formale: essendo la carica elettrica quantizzata infatti, non si può parlare di passagio
al limite in senso matematico.
\begin{obs}
    Date $n$ sorgenti, il principio di sovrapposizione per la forza di Coulomb si estende banalmente al campo elettrico:
    \[
        \vb{E}(\vb{r})=\rec{4\pi\epsilon_0}\sum_{i=1}^n \frac{Q_i}{\abs{\vb{r}-\vb{r}_i}^3}(\vb{r}-\vb{r}_i)
    \]
\end{obs}

Quando si ha a che fare con un elevato numero di cariche puntiformi risulta utile introdurre il concetto di \textit
{distribuzione continua di carica} descritta da una \textit{densità di carica}:
\begin{defn}
    Si definisce \textit{densità di carica} una funzione $\omega(x_1,x_2\ldots x_n)$ tale che $\dd{q}=\omega\,\dd{\mu}$ con
    $\dd{\mu}=\dd{x_1}\,\dd{x_2}\,\ldots\,\dd{x_n}$
\end{defn}
Si ha allora
\begin{equation}
    \label{eqn:E_distribuzione}
    \vb{E}(\vb{r})=\frac{1}{4\pi\epsilon_0}\,\int\frac{\omega(\vb{r'})}{\abs{\vb{r}-\vb{r'}}^3}(\vb{r}-\vb{r'})d\mu'
\end{equation}
Nella pratica si parla di densità spaziale $\rho$, densità superficiale $\sigma$, densità lineare $\lambda$
(rispettivamente $n=3$, $n=2$, $n=1$).
