Dipendendo il potenziale linearmente dalla carica, il rapporto $Q/V$ è costante ed è allora giustificata la seguente definizione
\begin{defn}[Capacità]
    Si definisce capacità C il rapporto
    \[
        C=\frac{Q}{V}
    \]
\end{defn}

Si considerino due conduttori $S_1$ ed $S_2$ sufficientemente vicini, affinchè interagiscano per induzione. Fissata la configurazione
geometrica, venga fornita a $S_1$ una carica $Q_1$, mantenendo pari a zero la carica di $S_2$. Moltiplicando per un fattore
$a$ la carica $Q_1$ si dimostra che anche i potenziali generati dai due conduttori, $V'_1$ e $V'_2$, vengono motliplicati per lo
stesso fattore: è quindi presente una relazione di proporzionalità diretta esprimibile dal sistema
\[
    \begin{cases}
        & V'_1=p_{11}Q_1 \\
        & V'_2=p_{21}Q_1
    \end{cases}
\]
Ripetendo lo stesso ragionamento, fornendo questa volta una carica $Q_2$ ad $S_2$ e mantenendo nulla la carica su
$S_1$ si ottiene
\[
    \begin{cases}
        & V''_1=p_{12}Q_2 \\
        & V''_2=p_{22}Q_2
    \end{cases}
\]
Per il principio di sovrapposizione, se ad ambedue i conduttori viene fornita una carica si ottiene il sistema
\begin{equation}
    \label{eqn:sist_p}
    \begin{cases}
        & V_1=p_{11}Q_1 + p_{12}Q_2\\
        & V_2=p_{21}Q_1 + p_{22}Q_2
    \end{cases}
\end{equation}

I coefficenti $p_{ij}$ si chiamano \textit{coefficenti di potenziale} e dipendono solo dalla geometria del sistema. La
situazione può essere generalizzata al caso con N conduttori ottenendo
\[
    V_i=\sum_{j=1}^N p_{ij}\,Q_j
\]
Fissate le cariche, i potenziali sono univocamente determinati. Da questo segue che la matrice rappresentativa del sistema
deve essere non singolare, ovvero $\det{p_{ij}}\neq 0$. In virtù di questo fatto, il sistema può essere invertito ottenendo che
\[
    Q_i=\sum_{j=1}^N c_{ij}\,V_j
\]
I coefficenti $c_{ij}$ si chiamano \textit{coefficenti di induzione}. Le matrici dei coefficenti sono ovviamente legate dalla relazione:
\[
    \Bqty{c_{ij}}=\Bqty{p_{ij}}^{-1}
\]
Le matrici di induzione e di potenziale sono simmetriche.

Si immagini ora un sistema di due conduttori disposti in una configurazione tale che il fenomeno di induzione che un condensatore
esercita sull'altro sia completo, ovvero fornita una carica $Q$ al primo conduttore il secondo si carica per induzione con una
carica $-Q$. Quando l'induzione è completa tutte le linee di campo uscenti da un conduttore terminano sull'altro conduttore.
\begin{defn}[Condensatore]
    Si definisce condensatore un sistema di due conduttori che goda della proprietà di cui sopra.
\end{defn}
Esistono tre tipi di condensatori:
\begin{description}
    \item[Condensatori sferici] Un conduttore è contenuto nella cavità di un altro conduttore;
    \item[Condensatori cilindrici] Un conduttore si sviluppa lungo una linea ed è avvolto da un secondo conduttore con struttura
        tubolare - l'induzione completa si realizza quando la lunghezza del condensatore è molto maggiore rispetto alle dimensioni trasversali;
    \item[Condensatori piani] Due conduttori si affacciano l'un l'altro in modo tale che le dimensioni lineari della superficie siano
        molto maggiori rispetto alla distanza - l'induzione completa si realizza nel limite di superficie infinita.
\end{description}
I due conduttori che formano il condensatore sono detti \textit{armature}. Un condensatore viene caricato quando si stabilisce
una differenza di potenziale fra le armature e su di esse si distribuiscono carice uguali in modulo e di segno opposto
(la carica totale di un condensatore carico è quindi nulla).

Fornendo una carica $Q_1$ all'armatura interna di un condensatore e lasciando la seconda elettricamente isolata,
su quest'ultima le cariche si ridistribuiscono per induzione completa in modo che la faccia interna sia dotata di carica $-Q_1$
e quella esterna $+Q_1$. Collegandola ora a terra, l'armatura esterna resta dotata di carica $-Q_1$.
Con un ragionamento analogo ci si rende conto che lo stesso avviene nel caso in cui quella collegata a terra fosse l'armatura interna.
\begin{thm}
    In un condensatore, il modulo della carica sulle maglie e la differenza di potenziale sono direttamente proporzionali.
    \[
        Q=C\Delta V \quad\quad C=\frac{1}{p_{11}+p_{22}-2p_{12}}
    \]
\end{thm}
\begin{proof}
    Si consideri il sistema \ref{eqn:sist_p} per il caso considerato, ricordando la proprietà di simmetria:
    \[
        \begin{split}
            & V_1=p_{11}Q-p_{12}Q=(p_{11}-p_{12})Q\\
            & V_2=p_{21}Q-p_{22}Q=(p_{21}-p_{22})Q
        \end{split}
    \]
    Sottraendo membro a memebro, la tesi. La seconda espressione per $C$ si ottiene facilmente invertendo la matrice $\Bqty{p_{ij}}$.
\end{proof}

È immediato da quanto visto ricavare l'energia immagazzianata da un condensatore. Ulteriori considerazioni verrano svolte
nel capitolo sulla corrente stazionaria, dopo aver introdotto il concetto di forza elettromotrice.
\begin{thm}
    Per un condensatore con capacità $C$, caricato fino ad avere una differenza di potenziale $\Delta V$ fra le armature
    l'energia elettrostatica è
    \begin{equation}
        U=\rec{2}Q\Delta V=\rec{2}C(\Delta V)^2=\rec{2}\frac{Q^2}{C}
        \label{eqn:U_condensatore}
    \end{equation}
\end{thm}
\begin{proof}
    La distribuzione di carica è una distribuzione superficiale. Si indichino con $S_a$ ed $S_b$ le superfici delle due armature.
    Usando l'espressione per l'energia elettrostatica \ref{eqn:energia_distribuzione}, si ha
    tenendo a mente
    \[
        U=\rec{2} \int_{S_a \cup S_b} \sigma V \dd{S}=\frac{V_a}{2}\int_{S_a} \sigma\dd{S}+\frac{V_b}{2}\int_{S_b} \sigma\dd{S}=
        \frac{V_a Q + V_b (-Q)}{2}=\rec{2} Q\Delta V
    \]
    Dalla definizione di capacità si ottengono le altre forme dell'asserto.
\end{proof}

Si vogliono ora ricavare delle formule specifiche per le capacità delle tre tipologie di condensatori.
\begin{obs}[Capacità del condensatore piano]
    Per un condensatore piano con maglie di superficie $S$ poste a distanza $d$, con le dimesioni lineari di $S$ molto maggiori di $d$ vale la formula
    \begin{equation}
        \label{eqn:capacità_piano}
        C=\epsilon_0\frac{S}{d}
    \end{equation}
\end{obs}
\begin{proof}
    Quella presente sulla superficie delle maglie è una distribuzione di carica superficiale che nelle ipotesi può considerarsi infinita,
    per cui il campo elettrico generato è
    \[
        \vb{E}=\frac{\sigma}{\epsilon_0}\vu{n}=\frac{Q}{S\epsilon_0}\vu{n}
    \]
    Da questo segue che la differenza di potenziale vale
    \[
        \Delta V=\int_1^2 \vb{E}\vdot d\vb{l}=\int_1^2\frac{Q}{S\epsilon_0}\vu{n}\vdot d\vb{l}=\frac{d}{S\epsilon_0}Q
    \]
    dato che la somma della proiezione di tutti gli elementi di linea sulla normale equivale alla distanza $d$ fra le superfici.
    Dalla definizione di capacità come rapporto fra differenza di potenziale e carica, la tesi.
\end{proof}

\begin{obs}[Capacità del condensatore sferico]
    Per un condensatore sferico con maglie di raggio $R_1$ ed $R_2$  poste a distanza $d=R_2-R_1$ vale la formula
    \begin{equation}
        \label{eqn:capacità_sferico}
        C=4\pi\epsilon_0\frac{R_1R_2}{d}
    \end{equation}
    Se vale la condizione $R_1\cong R_2$ allora la formula si riduce a:
    \[
        C=\epsilon_0\frac{S}{d}
    \]
\end{obs}
\begin{proof}
    Nella zona tra i due conduttori il campo elettrico vale
    \[
        \vb{E}(r)=\frac{Q}{4\pi\epsilon_0}\frac{\vu{r}}{r^2} \quad\quad\quad R_1<r<R_2
    \]
    Da questo segue che la differenza di potenziale vale
    \[
        \Delta V=\int_{R_1}^{R_2} \frac{Q}{4\pi\epsilon_0}\frac{\vu{r}}{r^2} \vdot d\vb{l}=\frac{Q}{4\pi\epsilon_0}\frac{R_2-R_1}{R_2R_1}
    \]
    Scegliendo come percorso, ad esempio, il segmento giacente sul raggio delle sfere. Dalla definizione di capacità, si
    ha la prima parte dell'asserto.
    La seconda parte dell'asserto si ottiene considerando che la condizione $R_2\cong R_1$ implica $R_1R_2\cong R^2$. Si ha quindi
    \[
        C=4\pi\epsilon_0\frac{R^2}{d}=\epsilon_0\frac{S}{d}
    \]
\end{proof}

\begin{obs}[Capacità del condensatore cilindrico]
    Per un condensatore cilindrico con lunghezza $l$ molto maggiore dei raggi $R_1$ ed $R_2$ delle maglie ($R_1<R_2$) vale la formula
    \begin{equation}
        \label{eqn:capacità_cilindirco}
        C=2\pi\epsilon_0\frac{l}{\ln(R_2/R_1)}
    \end{equation}
    Se vale la condizione $R_1\cong R_2$ allora la formula si riduce a:
    \[
        C=\epsilon_0\frac{S}{d}
    \]
\end{obs}
\begin{proof}
    Il ragionamento è analogo a quello precedente, ricordando che nella zona tra i due conduttori il campo elettrico vale
    \[
        \vb{E}(r)=\frac{\lambda}{2\pi\epsilon_0}\frac{\vu{r}}{r} \quad\quad\quad R_1<r<R_2
    \]

    La seconda parte dell'asserto si ottiene nuovamente considerando che la condizione $R_2\cong R_1$ implica $R_2-R_1=d<<R_2,R_1$.
    Si ha quindi, sviluppando
    \[
        \ln\frac{R_2}{R_1}=\ln\frac{R_1+d}{R_1}=\ln\Biggl(1+\frac{d}{R_1} \Biggr)\cong \frac{d}{R_1}
    \]
    Da cui segue
    \[
        C=2\pi\epsilon_0\frac{l}{\ln(R_2/R_1)}=\epsilon_0\frac{2\pi R_1 l}{d}=\epsilon_0\frac{S}{d}
    \]
\end{proof}



Si volgiono ora studiare invece i sistemi di condensatori. Si danno le seguenti definizioni:
\begin{defn}[Condensatori in parallelo]
    Si dicono in parallelo due consensatori collegati in modo tale che la prima armatura del primo sia collegata alla prima armatura
    del secondo e che la seconda armatura del primo sia collegata alla seconda armatura del secondo.
\end{defn}

\begin{defn}[Condensatori in serie]
    Si dicono in serie due consensatori collegati in modo tale che la seconda armatura del primo sia collegata alla prima armatura del secondo.
\end{defn}

Si osserva che i condensatori collegati in parallelo costituiscono di fatto un sistema di due conduttori. I condensatori collegati in
serie invece, costituiscono un sistema di tre conduttori%aggiungo immagine ??
\begin{thm}[Capacità dei condensatori in parallelo]
    Dati $n$ condensatori collegati in parallelo la capacità del sistema è equivalente a quella di un unico condensatore con capacità
    \[
        C=\sum_{i=1}^n C_i
    \]
\end{thm}
\begin{proof}
    Si consideri per semplicità un sistema di due condensatori - le generalizzazione è banale. Considerando separatamente i due condensatori si ha
    \[
        \begin{cases}
            & Q_1=C_1\Delta V_1 \\
            & Q_2=C_2\Delta V_2
        \end{cases}
    \]
    In virtù del collegamento in parallelo, le differenze di potenziale ai capi delle armature dei condensatori sono uguali. Sommando membro a membro
    \[
        Q_1+Q_2=(C_1+C_2)\Delta V
    \]
    Siccome $Q_1+Q_2$ rappresenta la carica totale del sistema, dalla definizione di capacità si  ha l'asserto.
\end{proof}

\begin{thm}[Capacità dei condensatori in serie]
    Dati $n$ condensatori collegati in serie la capacità del sistema è equivalente a quella di un unico condensatore con capacità
    \[
        \frac{1}{C}=\sum_{i=1}^n\frac{1}{C_i}
    \]
\end{thm}
\begin{proof}
    Si consideri nuovamente un sistema di due condensatori. Considerando separatamente i due condensatori si ha
    \[
        \begin{cases}
            & \Delta V_1=\frac{ Q_1 }{C_1}\\
            & \Delta V_2=\frac{ Q_2 }{C_2}\\
        \end{cases}
    \]
    In virtù del collegamento in serie, per l'induzione completa la carica sulla seconda armatura del primo condensatore sarà uguale
    in modulo e opposta alla carica sulla prima armatura. Un ragionamento analogo vale per il secondo condensatore, tenendo
    però conto anche del fatto che la carica sulla prima armatura del secondo condensatore sarà uguale alla carica sulla seconda
    armatura del primo. A questo punto quindi, sottraendo membro a membro le equazioni del sistema si ottiene:
    \[
        \Delta V_1 - \Delta V_2 = \Biggl(\frac{1}{C_1} + \frac{1}{C_2} \Biggr) Q
    \]

    Siccome $\Delta V_1 - \Delta V_2 $ rappresenta la caduta di potenziale ai capi del sistema di condensatori, dalla definizione di
    capacità si  ha l'asserto.
\end{proof}
