\begin{defn}[Conduttore]
    Si definisce conduttore un oggetto indeformabile all'interno del quale vi sono elettroni liberi di muoversi
\end{defn}
Un'ipotesi fondamentale dell'elettrostatica è l'invarianza temporale delle grandezze in gioco.
In particolare questo vale per le distribuzioni di carica: è necessario supporre allora che le cariche libere nei conduttori
non si muovano, ovvero che \textit{all'interno dei conduttori il campo elettrico sia nullo}.
Quando un conduttore viene immerso in un campo elettrico le cariche libere si muovono
alla ricerca di una configurazione di equilibrio, che viene raggiunta quando la loro disposizione
è tale da annullare il campo elettrico che si è formato internamente al conduttore.
\begin{thm}
    Passando da un mezzo materiale ad un altro, la componente tangenziale del campo elettrico non può subire discontinuità.
    \label{lemma:discontinuità_E}
\end{thm}
\begin{proof}
    La dimostrazione segue dal fatto che il campo elettrico è conservativo. Si ha infatti che l'integrale di linea
    lungo un qualsiasi percorso chiuso deve essere nullo. Chiamando $\vb{E}_1$ il campo elettrico in un materiale
    e $\vb{E}_2$ il campo elettrico nell'altro, con a priori $\vb{E}_1\neq\vb{E}_2$, si consideri un percorso rettangolare
    prossimo alla superficie di separazione dei due materiali. Si chiamino $dl$ i segmenti paralleli alla superficie
    -uno in un mezzo ed uno nell'altro- e $dn$ i segmenti ortogonali alla superficie. Al tendere di $dn$ a 0,
    il contributo alla circuitazione dato dai tratti ortogonali è nullo. Si ha quindi che
    \[
        \vb{E}_1\vdot d\vb{l} - \vb{E}_2\vdot d\vb{l}=E_1^t\,dl-E_2^t\,dl=0
    \]
    Dove con $E^t$ si indica la componente del campo elettrico tangente alla superficie.
    Segue immediatamente la tesi.
\end{proof}

\begin{cor}
    \label{cor:ortogonalità}
    In prossimità di un conduttore il campo elettrico è ortogonale alla superficie del conduttore.
\end{cor}
\begin{proof}
    Nel lemma, si consideri uno dei due mezzi come un conduttore, sia questo il secondo dei due mezzi.
    Si ha quindi $\vb{E}_2=0$ e in particolare $E_2^t=0$.
    Segue allora che la componente tangenziale del campo esterno al conduttore è nulla.
\end{proof}

Dalle proprietà appena discusse del campo elettrico, seguono immediatamente quelle del potenziale
\begin{cor}
    Il volume interno e la superficie dei conduttori sono equipotenziali.
\end{cor}
\begin{proof}
    Ricordando che $\vb{E}=-\grad{V}$ il fatto che il volume sia equipotenziale segue immediatamente dal fatto che
    il campo elettrico interno è nullo.
    La seconda parte dell'asserto si ottiene considerando che il gradiente è ortogonale alle superfici equipotenziali
    e che il campo elettrico è ortogonale alla superficie del conduttore.
\end{proof}

Si vuole trovare ora una relazione fra il potenziale interno e quello esterno al conduttore. Si introduce la definizione:
\begin{defn}[lavoro di estrazione]
    Si definisce lavoro di estrazione $L_e$ il lavoro necessario per portare una carica di prova $q$ dall'interno
    all'esterno del conduttore, in prossimità della superficie.
    \begin{equation}
        \Delta V=V_i-V_e=-\frac{L_e}{q}
    \end{equation}
\end{defn}

\begin{obses}
    Si osserva che $V_i-V_e$ è una quantità positiva dipendente esclusivamente dal materiale che cotituisce il conduttore.
    Viene detta \textit{funione lavoro}.
\end{obses}

Si ha infine questo importante corollario:
\begin{cor}
    Le cariche presenti su un conduttore si dispongono tutte sulla superficie del conduttore.
\end{cor}
\begin{proof}
    Si consideri una superficie chiusa arbitraria $\Sigma$ interna al coduttore. Essendo il campo elettrico nullo,
    il flusso attraverso questa superficie è nullo. Per il teorema di Gauss la carica totale interna a $\Sigma$ è nulla.
    Essendo la superficie arbitraria, si può fare il limite per $\Sigma$ tendente ad un qualsiasi punto interno al conduttore.
    L'unica possibilità per una carica sul conduttore è quindi quella di disporsi sulla superficie esterna di quest'ultimo.
\end{proof}
Segue quindi che la distribuzione delle cariche di un conduttore viene descritta da una distribuzione di superficie e non di volume.

Un importante risultato relativo ai conduttori è il \textit{teorema di Coulomb}.
\begin{thm}[teorema di Coulomb]
    In un punto vicino ad un conduttore il campo elettrico vale
    \[
        E=\frac{\sigma}{\epsilon_0}
    \]
    diretto secondo la normale uscente se $\sigma>0$, entrante se $\sigma<0$
\end{thm}
\begin{proof}
    Si consideri un conduttore C ed un cilindretto di base $dS$ e altezza $dh$, in modo tale che questa sia un infinitesimo
    di ordine superiore rispetto alla base. L'altezza del cilindro è ortogonale alla superficie di C. Si vuole applicare
    al cilindretto il teorema di Gauss. Il flusso uscente, per il corollario \ref{cor:ortogonalità}, si riduce al
    flusso attraverso la base superiore. La carica interna è $dQ=\sigma dS$.
    Si ha quindi:
    \[
        \vb{E}\vdot d\vb{S}=\frac{\sigma dS}{\epsilon_0}
    \]
    Per il corollario a cui si è già fatto riferimento $\vb{E}=E\vu{n}$. D'altro canto $d\vb{S}=dS\vb{n}$, da cui la tesi.
\end{proof}

Un altro importante risultato emerge a partire dalla seguente considerazione intuitiva: le cariche libere si dispongono
lungo la superficie del conduttore formando uno strato con uno spessore nell'ordine dell'[Angstrom]. Queste cariche
esercitano una forza di mutua repulsione che si deve tradurre in una pressione elettrostatica. Vale infatti il seguente teorema:
\begin{thm}%RIVED QUAND USARE MODULI O VETTORI
    In un punto della superficie del conduttore la pressione elettrostatica è pari alla densità di energia del campo
    elettrostatico in vicinanza del punto.
\end{thm}
\begin{proof}
    Si consideri un campo elettrico attorno al quale sia presente un campo elettrostatico $\vb{E}$ generato dalla distribuzione
    di cariche sul conduttore e dalle cariche nello spazio circostante. Si supponga di modificare per una quantità
    infinitesima e in maniera quasistatica (in modo da trascurare l'energia cinetica dovuta al movimento) la geometria
    del conduttore, facendolo espandere verso l'esterno di un tratto $\delta \vb{x}$ ogni elemento di superficie
    $dS$ ortogonalmente a se stesso. Per farlo è necessario applicare una forza esterna $\delta \vb{F}^{(e)}$ che
    compirà un lavoro $\delta L^{(e)}=\delta\vb{F}^{(e)}\vdot\delta\vb{x}$. Gli spostamenti sono chiamati \textit{spostamenti virtuali};
    i lavori \textit{lavori virtuali}. In assenza di energia cinetica il lavoro corrisponde alla variazione di energia elettrostatica del sistema
    \[
        \delta U=\delta L^{(e)}=\delta\vb{F}^{(e)}\vdot\delta\vb{x}
    \]
    Considerando che essendo la trasformazione quasistatica la forza esterna deve essere uguale e opposta alla forza elettrostatica si ha, in modulo:
    \[
        \delta\vb{F}^{(e)}=-\delta\vb{F}=\frac{\delta U}{\delta x}
    \]
    L'elemento di volume $dS\delta x$ aveva prima dello spostamento energia $\delta U_i=u\,ds\delta x$ e dopo lo spostamento
    ha energia $\delta U_f=0$, in quanto tutto l'elemento di volume si trova dentro al conduttore dove il campo elettrico
    è nullo. Si ha quindi $\delta U=\delta U_f -\delta U_i=-u\,dS\delta x$. Da questo segue che
    \[
        \delta F=-\frac{\delta U}{\delta x}=u\,dS
    \]
    Dividendo per l'elemento di superficie, dalla definizione di pressione, si ottiene la tesi.
\end{proof}
Il metodo dei lavori virtuali può essere usato per calcolare qualcunque sollecitazione meccanica agente sul
conduttore. A titolo di esempio, si consideri un conduttore spostato rigidamente di un tratto $\delta \vb{x}$.
La risultante delle forze elettrostatiche $R_x$ compie un lavoro $R_x \delta x=-\delta U$ con $U$ l'energia elettrostatica,
da cui si ha che $R_x=-\delta U/\delta x$. Una trattazione più approfondita delle problematiche relative a questo
metodo verrà svolta nel capitolo sulla corrente stazionaria.

Con quanto appena visto è possibile osservare due effetti non intuitivi caratteristici dei conduttori.

\subsubsection{Schermo elettrostatico}
Si consideri un conduttore cavo, carico con una carica $q$. La carica, per quanto visto, si dispone sulla superficie
del conduttore. Sia per assurdo $\vb{E}$ il campo interno alla cavità. Si consideri un percorso chiuso in parte interno
alla cavità ed in parte interno al conduttore: lungo questo percorso l'integrale di linea del campo elettrico non
sarebbe nullo, infatti la parte di percorso interna al conduttore darebbe contributo nullo, ma non si potrebbe dire
lo stesso per la parte interna alla cavità. Si viola quindi la conservatività del campo elettrico. Questo ragionamento
può essere ripetuto identicamente considerando anche una distribuzione di carica esterna al conduttore. Il conduttore
cavo funziona quindi da schermo elettrostatico.

\subsubsection{Potere delle punte}
Si immaginino due sfere con diverso raggio in cui tutte le grandezze fisiche caratteristiche abbiano simmetria sferica
(e quindi, sufficientemente lontane affinchè la distribuzione di carica di una non perturbi la distribuzioe di carica
dell'altra). I due potenziali allora valgono:
\[
    \begin{split}
        & V_1=\frac{1}{4\pi\epsilon_0}\frac{Q_1}{R_1}\\
        & V_2=\frac{1}{4\pi\epsilon_0}\frac{Q_2}{R_2}\\
    \end{split}
\]
Se ora queste due sfere sono collegate da un filo in modo che costituiscano un unico condutore, si deve avere $V_1=V_2$, ovvero
\[
    \frac{Q_1}{Q_2}=\frac{R_1}{R_2}
\]
La carica si distribuscie sulle due sfere in maniera proporzionale ai raggi.
Considerando ora invece la distribuzione superficiale di carica
\[
    \sigma=\frac{Q}{4\pi R^2}
\]
Si ottiene
\[
    \frac{\sigma_1}{\sigma_2}=\frac{Q_1}{R_1^2}\frac{R_2^2}{Q_2}=\frac{R_2}{R_1}
\]
Ovvero la distribuzione di carica è inversamente proporzionale ai raggi e quindi per il teorema di Coulomb il campo
elettrico è più intenso in prossimità della sfera più piccola. Generalizzando si può affermare che \textit{in vicinanza
di un conduttore il campo elettrico è tanto più intenso tanto più è piccolo il raggio di curvatura}.
