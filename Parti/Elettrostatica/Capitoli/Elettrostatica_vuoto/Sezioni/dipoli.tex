\begin{defn}[dipolo elettrico]
    Si definisce dipolo elettrico un sistema costituito da due cariche q uguali ed opposte poste
    ad una distanza $\delta$ fissa. Il vettore che collega le due cariche è chiamato $\vb*{\delta}$,
    ha intensità $\abs{\vb*{\delta}}=\delta$ ed è orientato dalla carica negativa alla positiva.
\end{defn}

Il dipolo è caratterizzato dal \textit{momento di dipolo}
\begin{equation}
    \label{eqn:momento_di_dipolo}
    \vb{p}=q\vb*{\delta}
\end{equation}
Nel caso di distribuzioni di carica, la definizione può essre estesa
\begin{equation}
    \label{defn:p_rho}
    \vb{p}=\int_\tau \rho\vb{r'} \dd{\tau}
\end{equation}

\begin{thm}
    A distanze molto maggiori delle dimensioni lineari del dipolo ($r>>\delta$) il potenziale generato dal dipolo vale
    \begin{equation}
        \label{eqn:V_dipolo}
        V(\vb{r})=\frac{1}{4\pi\epsilon_0}\frac{\vb{p}\vdot\vb{r}}{r^3}
    \end{equation}
\end{thm}
\begin{proof}
    Definiti:\\
    $r^+$ la distanza della carica positiva dall'osservatore; \\
    $r^-$ la distanza della carica negativa dall'osservatore\\
    Si ha che
    \[
        V(\vb{r})=\frac{q}{4\pi\epsilon_0}\Bigl(\frac{1}{r^+}-\frac{1}{r^-}\Bigr)=\frac{q}{4\pi\epsilon_0}\frac{r^--r^+}{r^+\,r^-}
    \]
    Nell'ipotesi in cui $\delta<<r$ valgono le segueti approssimazioni:
    \[
        \begin{split}
            & r^+\,r^- \cong r^2\\
            & r^+-r^- \cong \delta\cos\alpha
        \end{split}
    \]
    con $\alpha$ l'angolo che il raggio vettore $\vb{r}$ forma con $\vb{p}$.
    Segue che il potenziale può essere scritto come
    \[
        V(\vb{r})=\frac{q\delta\cos\alpha}{4\pi\epsilon_0\, r^2}=\frac{p\cos\alpha}{4\pi\epsilon_0\, r^2}=
        \frac{p\,r\,\cos\alpha}{4\pi\epsilon_0\, r^3}=\frac{\vb{p}\vdot \vb{r}}{4\pi\epsilon_0\, r^3}
    \]
\end{proof}
Si osservi come il potenziale del dipolo decresca come $1/r^2$
\begin{cor}
    Il campo elettrico generato dal dipolo giace esclusivamente nel piano pr ed è dotato di una componente radiale
    \[
        E_r=\frac{1}{4\pi\epsilon_0}\frac{2p\cos\theta}{r^3}
    \]
    e di una componente angolare
    \[
        E_\theta=\frac{1}{4\pi\epsilon_0}\frac{p\sin\theta}{r^3}
    \]
\end{cor}
\begin{proof}
    Noto il potenziale è possibile ricavare il campo elettrico semplicemente facendone il gradiente.
    In questo caso la soluzione più immediata consiste nell'applicare il gradiente in coordinate polari.
    Si sceglie l'asse $z$ coincidente con la direzione di $\vb{p}$ e l'angolo $\theta$ coincidente con l'angolo $\alpha$ definito sopra.
    In questo modo il potenziale può essere scritto come
    \[
        V(r,\theta,\phi)=\frac{1}{4\pi\epsilon_0}\frac{p\cos\theta}{r^2}
    \]
    Applicando il gradiente in coordinate polari, si ottiene la tesi.
\end{proof}
Le linee di forza partono dalla carica positiva e si chiudono sulla carica negativa.

Ci si pone ora come obiettivo quello di descrivere la dinamica di un dipolo immerso in un campo elettrico.
Le forze in gioco sono conservative, quindi possono essere dedotte da una funzione potenziale $U$.
Il problema di descrivere la dinamica del dipolo si traduce nel problema di determinare il potenziale delle forze.
\begin{lemma}
    Si consideri un dipolo con momento $\vb{p}$ immerso in un campo $\vb{E}$ uniforme, l'energia potenziale del dipolo vale
    \begin{equation}
        \label{eqn:U_dipolo}
        U=-\vb{E}\vdot\vb{p}
    \end{equation}
\end{lemma}
\begin{proof}
    Chiamando $U_+$ l'energia potenziale dovuta alla carica positiva del dipolo
    e $U_-$ l'energia potenziale dovuta alla carica negativa si ottiene
    \[
        U=U_++U_-=qV(\vb{r}+\vb{\delta})-qV(\vb{r})
    \]
    Ponendo ora $V(\vb{r}+\vb{\delta})=V(\vb{r})+\dd{V}$ si ottiene

    \[
        U=qV(\vb{r})+q\dd{V}-qV(\vb{r})=q\dd{V}
    \]
    Il differenziale del potenziale può essere scritto, in quanto forma differenziale,
    come $\grad{V}\vdot\dd{\vb{l}}$ e da ciò segue, osservando che $\dd{\vb{l}}=\vb*{\delta}$
    \[
        U=q\grad{V}\vdot\vb*{\delta}=\grad{V}\vdot q\vb*{\delta}
    \]
    ottenendo la tesi.
\end{proof}

\begin{thm}
    Per un dipolo si ha che
    \begin{equation}
        \vb{F}=-(\vb{p}\vdot\grad)\vb{E}
    \end{equation}
    \begin{equation}
        \vb{M}=\vb{p}\cp\vb{E}
    \end{equation}
\end{thm}
\begin{proof}
    Dalla dinamica si ha che il lavoro elementare vale
    \[
        \dd{L}=\vb{F}\vdot\dd{\vb{l}}+\vb{M}\vdot\dd{\vb*{\theta}}
    \]
    dove $\dd{\vb*{\theta}}=\vu{n}\dd{\theta}$, con $\vu{n}$ il versore dell'asse di rotazione.
    D'altra parte è vero che
    \[
        -\dd{L}=\dd{U}=\pdv{U}{l}\dd{l}+\pdv{U}{\theta}\dd{\theta}
    \]
    Per confronto si ha quindi che
    \[
        \begin{cases}
            & \vb{F}\vdot\dd{\vb{l}}=-\pdv{U}{l}\dd{l} \\
            & \vb{M}\vdot\dd{\vb*{\theta}}=-\pdv{U}{\theta}\dd{\theta}
        \end{cases}
    \]
    Per il lemma $U=-\vb{E}\vdot\vb{p}=-Ep\cos\theta$, dove la dipendenza dalla posizione la si ha
    solo in $E$ mentre la dipendenza dall'angolo la si ha solo nel coseno.
    Si ha quindi che, ricordando che $\vb{p}$ non dipende dalla posizione
    \[
        \begin{cases}
            & \vb{F}\vdot\dd{\vb{l}}=-\pdv{U}{l}\dd{l}=\dd{U}|_{\theta=cost}=\grad{U}\vdot\dd{\vb{l}}=
            \grad{(\vb{E}\vdot\vb{p})}\vdot\dd{\vb{l}}=(\vb{p}\vdot\grad)\vb{E}\vdot \dd{\vb{l}}\\
            & \vb{M}\dd{\vb{\theta}}=-\pdv{U}{\theta}\dd{\theta}=-Ep\sin\theta\dd{\theta}=
            -(\vb{E}\cp\vb{p})\dd{\vb{\theta}}=(\vb{p}\cp\vb{E})\dd{\vb{\theta}}
        \end{cases}
    \]
    Per confronto si ottiene la tesi.
\end{proof}
