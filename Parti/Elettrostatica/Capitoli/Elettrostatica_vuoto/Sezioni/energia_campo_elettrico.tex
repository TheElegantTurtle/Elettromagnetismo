Un sistema di cariche che interagiscono reciprocamente possiede una certa \textit{energia elettrostatica di interazione}
dovuta al lavoro necessario per portare le cariche nella configurazione
desiderata partendo da una condizione di assenza di interazione fra le cariche.\\
\begin{thm}[Energia elettrostatica per un sistema discreto di cariche puntiformi]
    L'energia elettrostatica di interazione per un sistema discreto di $N$ cariche puntiformi vale
    \begin{equation}
        \label{eqn:UE_puntiformi}
        U=\rec{2}\sum_{j\neq i}^N\frac{q_iq_j}{4\pi\epsilon_0 r_{ij}}
    \end{equation}
\end{thm}
\begin{proof}
    Immaginiamo che inizialmente tutte le cariche siano all'infinito, in modo che non sentano
    l'interazione reciproca. Il posizionamento nello spazio della prima carica viene effettuato compiendo lavoro nullo,
    perchè inizialmente non è presente nessun campo elettrico.
    Ora è presente nello spazio il campo $\vb{E_1}$ prodotto dalla prima carica $q_1$.
    Per portare quindi la seconda carica $q_2$ dall'infinito fino ad una distanza $r_{12}$ dalla prima carica, è necessario compiere in lavoro
    \[
        L_2=\int_{-\infty}^{r_{12}}-q_2\vb{E_1}\vdot d\vb{l}=-\frac{q_1q_2}{4\pi\epsilon_0}\int_{-\infty}^{r_{12}}\frac{dr}{r^2}
        =\frac{q_1q_2}{4\pi\epsilon_0}\rec{r_{12}}
    \]
    Dove il segno meno deriva dal fatto che il lavoro è quello associato alla forza esterna che
    sposta la carica opponendosi alla forza di coulomb.
    Il lavoro necessario per spostare la terza carica $q_3$ ad una distanza $r_{13}$ da $q_1$
    e ad una distanza $r_{23}$ da $q_2$,  per il principio di sovrapposizione è uguale al lavoro necessario
    a spostare $q_3$ nel campo generato dalla prima carica e nel campo generato dalla seconda. Si ha quindi
    \[
        L_3=\frac{q_1q_3}{4\pi\epsilon_0}\rec{r_{13}}+\frac{q_2q_3}{4\pi\epsilon_0}\rec{r_{23}}
    \]
    L'energia posseduta dal sistema di tre cariche è quindi
    \[
        U=L_1+L_2+L_3=0+\frac{q_1q_2}{4\pi\epsilon_0}\rec{r_{12}}+\Biggl(\frac{q_1q_3}{4\pi\epsilon_0}\rec{r_{13}}
        +\frac{q_2q_3}{4\pi\epsilon_0}\rec{r_{23}}\Biggr)=\rec{2}\sum_{j\neq i}^3\frac{q_iq_j}{4\pi\epsilon_0 r_{ij}}
    \]
    dove il termine $1/2$ è necessario in quanto la sommatoria comprende sia l'interazione di $q_i$ con $q_j$
    che l'interazione di $q_j$ con $q_i$. Proseguendo col ragionamento, si ottiene la formula per $N$ cariche.
\end{proof}
Una comoda riscrittura della formula appena dimostrata si ottiene considerando
\[
    U=\rec{2}\sum_{i=1}^Nq_i\sum_{i\neq j}^N\frac{q_j}{4\pi\epsilon_0}\rec{r_{ij}}
\]
E osservando che la seconda sommatoria corrisponde al potenziale $V_i$ sentito dalla carica
i-esima dovuto a tutte le altre $N-1$ cariche e quindi si ottiene
\begin{equation}
    U=\rec{2}\sum_{i=1}^N q_i V_i
\end{equation}
L'utilità di questa riscrittura è dovuta al fatto che consente di generalizzare quanto visto
al caso di una distribuzione continua di carica contenuta in un insieme $A$
\begin{equation}
    \label{eqn:energia_distribuzione}
    U=\rec{2}\int_{A} \omega V \dd{\mu}
\end{equation}
Le equazioni trovate esprimono l'energia potenziale elettrica in termini di posizione reciproca delle cariche, mettendo
in evidenza quindi l'interazione fra queste mediante forza di Coulomb. Un altro approccio consiste invece nell'
enfatizzare il ruolo del campo elettrico interpretando l'energia potenziale elettrica come quell'energia
"immagazzinata" dal campo elettrico -che è un modo altisonante per dire: l'energia necessaria a generare
il campo. Per farlo, si sfrutta la prima equazione di Maxwell.
\begin{thm}
    Data una distribuzione di carica $\rho$ l'energia elettrostatica vale
    \begin{equation}
        \label{eqn:UE}
        U=\frac{\epsilon_0}{2}\int_S V\vb{E}\vdot d\vb{S} + \frac{\epsilon_0}{2}\int_V E^2 d\tau
    \end{equation}
\end{thm}
\begin{proof}
    Tenuto conto della prima equazione di Maxwell \eqref{eqn:prima_maxwell} e della \eqref{eqn:energia_distribuzione} si ha che
    \[
        U=\rec{2}\int_{V} \rho V d\tau=\frac{\epsilon_0}{2}\int_V \div{\vb{E}} V d\tau
    \]
    Applicando la \eqref{app:eqn:div_scalare_vettore}
    ricordando che il gradiente del potenziale è il campo elettrico cambiato di segno, si ottiene
    \[
        U=\frac{\epsilon_0}{2}\int_V \div{(V\vb{E})}d\tau + \frac{\epsilon_0}{2}\int_V E^2 d\tau
    \]
    Per il teorema della divergenza infine si ha
    \[
        \int_V \div{(V\vb{E})}d\tau=\int_S V\vb{E}\vdot d\vb{S}
    \]
    ottenendo la tesi.
\end{proof}

\begin{cor}
    Se si prende in considerazione tutto lo spazio, allora l'energia vale
    \[
        U=\int ud\tau \quad\quad\quad \text{ con } u=\frac{\epsilon_0 E^2}{2}
    \]
\end{cor}
\begin{proof}
    Si considerino i risultati del teorema precedente: fissata la distribuzione di carica in una regione finita di spazio,
    all'allargarsi del volume di integrazione l'integrale di volume aumenta e di paripasso l'intergale di superficie
    diventa trascurabile. Si ha quindi la tesi.
\end{proof}
La $u$ che compare nel risultato del corollario prende il nome di \textit{densità di energia del campo elettrostatico}.\\
Il risultato a cui si giunge sembra in apparenza contraddittorio: $u$ è chiaramente positiva e di conseguenza lo è il
suo integrale, mentre non è difficile immaginare un caso in cui \eqref{eqn:UE_puntiformi} sia negativa -basta
prendere un sistema costituito da due cariche di segno opposto. La differenza risiede nel fatto che la
\eqref{eqn:energia_distribuzione} e tutte le equazioni che ne conseguono contengono un termine di auto-energia,
ovvero l'energia necessaria alla costruzione della distribuzione di carica. Si consideri infatti
il sistema già citato costituito da due cariche puntiformi elementari. L'energia fornita da \eqref{eqn:UE_puntiformi}
è
\[
    U=\rec{4\pi\epsilon_0} \frac{q_1 q_2}{r_{12}}
\]
Si vuole ora calcolare l'energia passando per la densità di energia.
\[
    u=\rec{32\pi^2\epsilon_0}\Biggl[\frac{q_1^2}{\abs{\vb{r}-\vb{r}_1}} + \frac{q_2^2}{\abs{\vb{r}-\vb{r}_2}}+
    2q_1 q_2 \frac{(\vb{r}-\vb{r}_1)\vdot(\vb{r}-\vb{r}_2)}{\abs{\vb{r}-\vb{r}_1}^3\abs{\vb{r}-\vb{r}_2}^3} \Biggr]
\]
Si integri quanto ottenuto. In particolare, ci si focalizzi sul terzo termine e si introduca il cambiamento di variabile
$\vb*{\rho}=(\vb{r}-\vb{r}_1)/\abs{\vb{r}_1-\vb{r}_2}$, $\dd{\vb{\tau}}=\abs{\vb{r}-\vb{r}_1}^3\dd{\vb*{\rho}}$. Chiamando
$\vu{r}=(\vb{r}_1-\vb{r}_2)/\abs{\vb{r}_1-\vb{r}_2}$ si ottiene
\[
    \frac{ q_1 q_2}{16\pi^2\epsilon_0}
    \int\frac{(\vb{r}-\vb{r}_1)\vdot(\vb{r}-\vb{r}_2)}{\abs{\vb{r}-\vb{r}_1}^3\abs{\vb{r}-\vb{r}_2}^3}\dd{\tau}=
    \frac{ q_1 q_2}{16\pi^2\epsilon_0}
    \int\rec{\abs{\vb{r}_1-\vb{r}_2}^3}\frac{\vb*{\rho}\vdot(\vb*{\rho}+\vu{r})}{\rho^3\abs{\vb*{\rho}+\vu{r}}^3}\dd{\vb*{\rho}}
\]
Usando la \eqref{app:eqn:grad_r} e di seguito la \eqref{app:eqn:div_scalare_vettore} l'integrale assume la forma
\[
    \frac{ q_1 q_2}{16\pi^2\epsilon_0}\rec{\abs{\vb{r}_1-\vb{r}_2}^3}
    \Biggl[-\int\div{\frac{\vb*{\rho}}{\rho^3\abs{\vb*{\rho}+\vu{r}}^3}\dd{\vb*{\rho}}}
    -\int\rec{\abs{\vb*{\rho}+\vu{r}}^3}{\div{\frac{\vb*{\rho}}{\rho^3}}\dd{\vb*{\rho}}}\Biggr]
\]
Per il teorema della divergenza, il primo è un integrale sulla frontiera del volume. Siccome il volume di integrazione
è tutto lo spazio, l'integrale sulla superficie va a $0$. Per quanto concerne il secondo integrale, usando nuovamente
la \eqref{app:eqn:grad_r}
\[
    \int\rec{\abs{\vb*{\rho}+\vu{r}}^3}\laplacian{\rec{\rho}}\dd{\vb*{\rho}}=
    \int\rec{\abs{\vb*{\rho}+\vu{r}}^3}4\pi\delta(\vb*{\rho})\dd{\vb*{\rho}}=
    \frac{4\pi}{\abs{\vu{r}}^3}=4\pi
\]
dove è stato fatto uso anche della \eqref{app:eqn:laplacian_r}. In conclusione, mettendo tutto insieme si ottiene che
questo terzo termine rappresenta l'energia di interazione mentre gli altri due termini rappresentano l'auto-energia.
Per quanto visto, siccome l'energia di interazione può essere sia positiva che negativa e l'energia
totale è positiva, l'auto-energia deve essere necessariamente positiva.
