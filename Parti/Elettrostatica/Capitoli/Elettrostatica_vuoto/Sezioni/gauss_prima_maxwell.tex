\begin{thm}
    Data una qualunque superficie chiusa $S$ nel vuoto, il flusso del campo elettrostatico $\Phi_S(\vb{E})$
    dipende esclusivamente dalle cariche interne alla superficie secondo la legge
    \[
        \Phi_S(\vb{E})=\frac{Q_{TOT}^{int}}{\epsilon_0}
    \]
\end{thm}
\begin{proof}
    Si consideri il caso in cui all'interno della superficie S sia presente solo una carica puntiforme Q. Il flusso elementare è
    allora dato da
    \[
        d\Phi_S(\vb{E})=\vb{E}\vdot \dd{\vb{S}}=\frac{1}{4\pi\epsilon_0}\frac{Q}{r^2}\vu{r}\vdot\vu{n}\,\dd{S}
    \]
    Il prodotto scalare rappresenta la proiezione di $\vu{n}$ su $\vu{r}$, ovvero la proiezione di $d\vb{S}$ sull'elemento di
    superficie di una sfera di raggio r e centro in Q, ma allora il rapporto fra questa proiezione ed $r^2$ è per definizione
    l'angolo solido sotteso alla superficie infinitesima $d\Omega$. In conclusione si ha
    \[
        d\Phi_S(\vb{E})=\frac{Q}{4\pi\epsilon_0}\dd{\Omega}
    \]
    Il flusso totale è allora
    \[
        \Phi_S(\vb{E})=\int \dd{\Phi_S(\vb{E})}=\frac{Q}{4\pi\epsilon_0}\int\dd{\Omega}=\frac{Q}{4\pi\epsilon_0}4\pi=\frac{Q}{\epsilon_0}
    \]
    Per il principio di sovrapposizione e la linearità dell'integrale si arriva facilmente alla tesi.

    Si consideri ora il caso in cui sia presente una carica puntiforme $Q_e$ esterna alla superficie.
    Si considerino i coni che danno origine agli angoli solidi $d\Omega$ in ogni direzione dello spazio:
    di questi si è interessati solo a quelli che intercettano S. Il flusso di campo elettrico attraverso le due superfici
    originate dall'intersezione fra i coni e S vale in modulo $Q_e\,d\Omega/4\pi\epsilon_0 $per entrambe le superfici, con
    segno però opposto a causa del fatto che la proiezione avviene in un caso con un angolo inferiore a $\pi/2$ e nell'altro
    caso con un angolo superiore a $\pi/2$. Il contributo complessivo è quindi nullo.
\end{proof}

\begin{cor}
    Nel caso in cui si abbia una distribuzione continua di carica, invece che un insieme di cariche puntiformi,
    vale ancora la tesi del teorema di Gauss e il flusso è dato da
    \[
        \Phi_S(\vb{E})=\rec{\epsilon_0}\int \omega\,\dd{\mu}
    \]
\end{cor}

Come è evidente dalla dimostrazione, il teorema permette di calcolare il flusso indipendentemente dalla forma della superficie o
dalla posizione occupata dalle singole cariche interne. Si rivela dunque uno strumento estremamente utile per calcolare, noto
il campo elettrico, le cariche presenti in una porzione di spazio o viceversa, nota la distribuzione di carica, di
ricavare il campo elettrico. Si presenta tuttavia come una diretta conseguenza della legge di Coulomb\footnote{Si
osservi infatti come il punto chiave della dimostrazione, ovvero l'identificazione dell'angolo solido, dipenda direttamente dal
fatto che la leggge di Coulomb vada come $1/r^2$.} e non aggiunge quindi informazioni rispetto a quelle già note.
Il teorema di Gauss fornisce inoltre alcune indicazioni utili alla rappresentazione del campo elettrico.
\begin{defn}[Linea di forza del campo]
    Si definisce \textit{linea di forza del campo} una linea che è in ogni suo punto tangente alla direzione del campo.
\end{defn}
\begin{defn}[Tubo di flusso]
    Presa una linea chiusa, in ogni suo punto passa una linea di forza del campo. La superficie tubolare definita da queste
    linee di forza è detta tubo di flusso.
\end{defn}
Nel caso del campo elettrico $\vb{E}$ si consideri una porzione di tubo di flusso compresa fra due superfici $S_1$
ed $S_2$: assieme alla superficie laterale queste due superfici definiscono una superficie chiusa $S$. Per il teorema di Gauss,
se non sono presenti cariche interne ad S, il flusso totale $\Phi_S^{tot}(\vb{E})=0$. Inoltre, per definizione di linea
di forza il flusso attraverso la superficie laterale del tubo è nullo.
Si ha quindi, tenendo conto della diversa orientazione di $S_1$ ed $S_2$ rispetto a $S$, $\Phi_{S_1}(\vb{E})=\Phi_{S_2}(\vb{E})$.
Siccome quanto detto è indipendente dalla porzione di tubo scelta si ha che il flusso del campo elettrico è una caratteristica del
tubo di flusso. Da questo segue che la sezione del tubo di flusso si riduce nelle zone in cui il campo elettrico è maggiore
, ovvero le linee di forza si addensano, e viceversa.

Risulta conveniente espriere in forma locale il teorema di Gauss.
\begin{cor}[Prima equazione di Maxwell]
    Nell'ipotesi che sia possibile applicare il teorema della divergenza al campo elettrico
    \begin{equation}
        \label{eqn:prima_maxwell}
        \div{\vb{E}}=\frac{1}{\epsilon_0}\rho
    \end{equation}
\end{cor}
\begin{proof}
Applicando il teorema della divergenza alla legge di Gauss si ricava che
\[
    \Phi_S(\vb{E})=\int_\tau \div{\vb{E}} \dd{\tau}=\frac{1}{\epsilon_0}\int_\tau \rho\, \dd{\tau}
\]
Siccome il teorema di Gauss vale qualunque sia la superficie di integrazione, la formula appena ottenuta deve valere qualunque sia
il volume di intergazione, il che implica l'uguaglianza delle integrande, cioè la tesi.
\end{proof}

La limitazione di questa equazione rispetto a quella fornita dal teorema di Gauss è situata nel fatto che viene richiesta la
validità del teorema della divergenza, ovvero che il campo elettrico sia derivabile in ogni punto del dominio considerato. Il
vantaggio risiede però nel fatto che il teorema di Gauss lega la distribuzione di carica interna alla superficie col campo
sulla superficie (e quindi nel caso non stazionario, una variazione della distribuzione di carica non si traduce in una
simultanea variazione del campo) mentre l'equazione di Maxwell lega grandezze calcolate nella stessa posizione e può
dunque essere immediatamente generalizzata al caso non stazionario.
