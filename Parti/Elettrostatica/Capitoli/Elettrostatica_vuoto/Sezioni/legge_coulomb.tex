\begin{obses}[Legge di Coulomb]
  Prese due cariche $q_1$ e $q_2$ nel vuoto poste a distanza $\vb{r}$, la forza $\vb{f_{21}}$ che $q_2$ subisce ad opera di $q_1$ è
  \begin{equation}
      \vb{f}_{21}=\frac{1}{4\pi\epsilon_0}\frac{q_1\,q_2}{r^2}\vu{r}_{21}
  \end{equation}
\end{obses}
  Detta $\vb{r}_1$ la posizione di $q_1$ e $\vb{r}_2$ la posizione di $q_2$ si ha $\vb{r}_{21}=\vb{r}_2-\vb{r}_1$.
\begin{obses}[Principio di sovrapposizione]
    Sperimentalmente la forza di Coulomb subita dalla carica $q$ è pari alla somma vettoriale delle forze eserciate dalle $n$ cariche $Q_i \quad i=1 \text{...} n$.
\end{obses}

Ovviamente vale il terzo principio di Newton per cui $\vb{f}_{21}=-\vb{f}_{12}$.
