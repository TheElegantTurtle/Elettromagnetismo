Si consideri una distribuzione di carica in una regione limitata di spazio complessivamente neutra, ovvero tale che
\[
    q=\int\rho\,d\tau=0
\]
Considerata una superficie chiusa che racchiude la distribuzione di carica, per il teorema di Gauss il flusso del campo elettrico
uscente è nullo. In particolare, se la distribuzione ha simmetria sferica, il campo elettrico esterno alla distribuzione di carica
è ovunque nullo. Se la simmetria non è sferica invece non è detto che il campo su una superficie chiusa $S$ che racchiude la
distribuzione sia nullo, come in effetti dimostra l 'esempio del dipolo.
Risulta conveniente caratterizzare la distribuzione di carica con alcune proprietà globali che consentano di calcolare le caratteristiche
del campo generato. Per farlo si parte dal potenziale definito in \ref{eqn:V_rho}. Essendo la distribuzione concentrata in una
regione limitata di spazio si può scegliere il potenziale nullo all'infinito ($r_0\rightarrow +\infty$). Il termine al denominatore
in coordinate cartesiane è
$f(x',y',z')=\bigl[(x-x')^2+(y-y')^2+(z-z')^2\bigr]^{\frac{1}{2}}$. Si vuole sviluppare la funzione al primo ordine
\[
    f(x',y',z')=f(0)+\pdv{f}{x'}(0)+\pdv{f}{y'}(0)+\pdv{f}{z'}(0)
\]
e tenuto conto della definizione di $f$ si ha che
\[
    \begin{split}
        & f(0)=\bigl[(x)^2+(y)^2+(z)^2\bigr]^{-\frac{1}{2}} \\
        & \pdv{f}{x'}(0)=\eval{(x-x')\bigl[(x-x')^2+(y-y')^2+(z-z')^2\bigr]^{-\frac{3}{2}}}_{x',y',z'=0}=\frac{x}{r^3} \\
        & \pdv{f}{y'}(0)=\eval{(y-y')\bigl[(x-x')^2+(y-y')^2+(z-z')^2\bigr]^{-\frac{3}{2}}}_{x',y',z'=0}=\frac{y}{r^3} \\
        & \pdv{f}{z'}(0)=\eval{(z-z')\bigl[(x-x')^2+(y-y')^2+(z-z')^2\bigr]^{-\frac{3}{2}}}_{x',y',z'=0}=\frac{z}{r^3}
    \end{split}
\]
Sostituendo quindi si ottene
\[
    f(x',y',z')=\frac{1}{r}+\frac{1}{r^3}(xx'+yy'+zz')=\frac{1}{r}+\frac{\vb{r}\vdot\vb{r'}}{r^3}
\]
Da cui, ricordando che il potenziale all'infinito è nullo e la definizione \ref{defn:p_rho}, si ottiene
\begin{equation}
    V(\vb{r})=\rec{4\pi\epsilon_0}\int_{V}f\rho \dd{\tau'} =\frac{1}{4\pi\epsilon_0}\frac{q}{r}+\frac{1}{4\pi\epsilon_0}\frac{\vb{p}\vdot\vb{r}}{r^3}
\end{equation}
Si parla di \textit{primo e secondo termine di dipolo}. Il secondo termine di dipolo decresce più rapidamente del primo,
rappresentando una correzione spesso trascurabile. Quando però il sistema è elettricamente neutro, il secondo termine diventa quello
dominante.
