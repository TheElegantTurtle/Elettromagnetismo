Il problema dell'elettrostatica consiste nel calcolare, note le cariche totali possedute da dei conduttori,
il campo elettrico generato e la distribuzione delle cariche sui conduttori stessi.
Con distribuzioni di carica note, la soluzione è concettualmente semplice: basta applicare la \eqref{eqn:E_distribuzione}
per ottenere il campo elettrico e di seguito tutte le grandezze cercate. In presenza di conduttori
la distribuzione di cariche libere è però influenzata dalla reciproca interazione. Per riuscire a risolvere il problema
risultano allora di fondamentale importanza le due equazioni di Maxwell trovate.
\begin{thm}
Il potenziale soddisfa la seguente equazione
\begin{equation}
\laplacian V=-\frac{\rho}{\epsilon_0}
\end{equation}
equivalente alla prima equazione di Maxwell \eqref{eqn:prima_maxwell} ed alla terza \eqref{eqn:terza_maxwell}.
\end{thm}
L'equazione presentata nel teorema si chiama \textit{equazione di poisson}
\begin{proof}
  Per definizione $\vb{E}=-\grad{V}$. Dalla prima equazione di Maxwell si ha che:
  \[
  \frac{\rho}{\epsilon_0}=\div{\vb{E}}=\div{(-\grad{V})}=-\laplacian{V}
  \]
  dimostrando in questo modo sia l'equazione di Poisson che l'equivalenza fra quest'ultima e la prima equazione di Maxwell.
    L'ultima parte della tesi si ottiene facilmente: essendo $\vb{E}$ espresso come gradiente di $V$ allora
    la terza equazione di Maxwell è verificata e viceversa, se la terza equazione di Maxwell è verificata
    allora il campo elettrico è conservativo e può quindi essere espresso in termini di gradiente di una funzione scalare.
\end{proof}

\begin{thm}[Esistenza e unicità della soluzione all'equazione di Poisson]
Fissata la funzione $\rho$ localizzata in una porzione finita di spazio, l'equazione di Poisson ammette
    una ed una sola soluzione che soddisfi le condizioni al contorno del dominio di definizione.
\end{thm}
Questo risultato è importante nel caso in cui siano presenti dei conduttori: in caso contrario infatti
il potenziale e di conseguenza il campo elettrico sono forniti direttamente dalla \ref{eqn:V_rho}.
Segue direttamente l'importante corollario:
\begin{cor}
  L'equazione di Poisson caratterizza completamente il potenziale.
\end{cor}
Si può allora dire che il problema generale dell'elettrostatica consista nel risolvere l'equazione di Poisson
con determinate condizioni al contorno.

Si elencano ora le tre situazioni che si presentano più di frequente.

\textit{Problema di Dirichlet - Non sono presenti cariche localizzate; il campo è generato
da un sistema di conduttori di geometria nota, di cui si conoscono i potenziali.}
L'equazione di Poisson si riduce in questo caso all'equazione di Laplace
\[
    \laplacian{V}=0
\]
Il problema è definito dal punto di vista matematico, essendo note le condizioni al contorno ($V=0$ all'infinito,
$V=V_i$ sulla superficie dell'i-esimo conduttore). I passi risolutivi sono i seguenti:
\begin{enumerate}
    \item si risolve l'equazione di Poisson ottenendo il potenziale nello spazio circostante i conduttori;
    \item dal potenziale si può ricavare il campo;
    \item il valore del campo in prossimità dei conduttori consente di calcolare la densità di carica superficiale;
    \item integrando la densità di ogni conduttore sulla sua superficie si ottine la carica disposta sui conduttori;
    \item avendo le cariche ed essendo i potenziali noti è possibile ricavare i coefficienti di capacità.
\end{enumerate}

\textit{Non sono presenti cariche localizzate; il campo è generato
da un sistema di conduttori di geometria nota, di cui si conoscono le cariche.}
Questo problema è l'inverso del problema di Dirichlet
I passi risolutivi sono i seguenti:
\begin{enumerate}
    \item si scelgono in modo arbitrario i potenziali dei conduttori;
    \item si risolve il problema di Dirichlet relativo;
    \item ricavati i coefficenti di capacità, si ottenogno i potenziali veri a partire dalle cariche note;
    \item si risolve il problema di Dirichlet relativo ai potenziali trovati.
\end{enumerate}

\textit{Sono presenti cariche localizzate con distribuzione $\rho$ nota; è inoltre presente
un sistema di conduttori di geometria nota, di cui si conoscono le cariche.}
Il potenziale dell'i-esimo conduttore, per il principio di sovrapposizione, è dato da
\[
    V_i=V_i(\rho)+\sum_{j=1}^N p_{ij}Q_j
\]
Di questa equazione sono noti i coefficienti di potenziale determinati dalla geometria del
sistema, mentre non si conoscono i potenziali generati dalla distribuzione di carica.
I passi risolutivi sono i seguenti:
\begin{enumerate}
    \item si scelgono in modo arbitrario i potenziali dei conduttori;
    \item si risolve il problema di Dirichlet relativo;
    \item ricavate le cariche relative, si può invertire l'equazione per i potenziali dei conduttori
        in modo da ricavare i potenziali dovuti alla distribuzione di carica;
    \item si risolve l'equazione per i potenziali dei conduttori usando le cariche note.
\end{enumerate}

Si elencano ora tre casi notevoli in cui il problema generale dell'elettrostatica può
essere risolto senza ricorrere all'equazione di Poisson.

\subsubsection{Metodo delle cariche immagine}
Si consideri un conduttore $S$ collegato a terra, in modo che il suo potenziale sia 0, e che siano presenti
N cariche puntiformi $Q_i$. Se la geometria del conduttore è semplice, si può immaginare di eliminare il
conduttore posizionando dalla parte opposta delle $Q_i$ rispetto al conduttore, M cariche $Q'_i$, dette cariche immagie, in modo tale
che il potenziale complessivo su $S$ sia
comunque nullo. Nella porzione di spazio $\tau$ che contiene le $Q_i$ la distribuzione di carica e le condizioni
al contorno sono le stesse sia in presenza delle cariche immagine che della distribuzione di carica sul conduttore
indotta dalle $Q_i$, perciò per il
teorema di unicità la configurazione di potenziale in $\tau$ è la stessa in entrambi i casi.

Il grande vantaggio di questo metodo è che introducendo le cariche immagine il potenziale è calcolabile usando
l'espressione per il potenziale generato da un numero discreto di cariche.

\subsubsection{Equazione di Laplace undimensionale}
Se il sistema è costituito da piani omogeneamente carichi, infiniti e paralleli fra loro, il potenziale
dipende esclusivamente dalla coordinata $x$ ortogonale ai piani. L'equazione di Laplace si riduce allora a
\[
    \dv[2]{V}{x}=0
\]
Di conseguenza $V(x)=ax+b$, con le costanti determinate dalle condizioni al contorno.

\subsubsection{Separazione di variabili}
L'equazione di Laplace nel caso tridimensionale può essere risolta ipotizzando che il potenziale sia nella forma
\[
    V(x,y,z)=X(x)Y(y)Z(z)
\]
Questo è possibile solo se la decomposizione è compatibile con le equazioni al contorno assegnate.
Sostituendo nell'equazione di Laplace e dividendo per $XYZ$ si ottiene:
\[
    \rec{X}\dv[2]{X}{x}+\rec{Y}\dv[2]{Y}{y}+\rec{Z}\dv[2]{Z}{z}=0
\]
Affinchè la somma sia nulla per ogni valore di $X$, $Y$, $Z$ è necessario chee ciascuno dei
termini della somma sia costante, ovvero
\[
    \begin{split}
        & \rec{X}\dv[2]{X}{x} = K^2_1 \\
        & \rec{Y}\dv[2]{Y}{y} = K^2_2 \\
        & \rec{Z}\dv[2]{Z}{z} = -K^2_3
    \end{split}
\]
con $K^2_1+K^2_2-K^2_3=0$.
La soluzione delle tre equazioni è
\[
    \begin{split}
        & X =A_1\cos(K_1x+B_1) \\
        & Y =A_2\cos(K_2y+B_2) \\
        & Z =A_3\cos((K_1+K_2)^{\rec{2}}z+B_3) \\
    \end{split}
\]
I vaori delle costanti si trovano imponendo le condizioni al contorno.
