Si vuole preliminarmente studiare il comportamento di un atomo immerso in un campo di induzione
magnetica esterno. Per questo studio ci si limiterà all'approssimazione del modello planetario
di Rutherford, che consente di ottenere comunque informazioni interssanti.

L'idea di fondo è che un elettrone in orbita attorno al nucleo può essere visto come una spira
percorsa da corrente. Vogliamo quindi ricavare il momento magnetico, ovvero ricavare la superficie della spira
$S=\pi r^2$ e la corrente $I=q_e/T$, dove $T$ è il periodo.
Detto $r$ il raggio dell'orbita e $T$ il periodo di rivoluzione, si ha dalla legge di Coulomb
e dal secondo postulato della dinamica di Newton
\[
\rec{4\pi\epsilon_0}\frac{e^2}{r^2}=m_e\omega^2 r=\frac{(2\pi)^2}{T^2}m_e r
\]
da cui
\[
T=\frac{4\pi}{e}\sqrt{\pi\epsilon_0 m_e r^3}
\]
Il raggio dell'orbita può essere valutato a partire dal lavoro di prima ionizzazione,
ovvero dal lavoro necessario per strappare l'elettrone dalla sua orbita e portarlo
all'infinito. All'infinito infatti l'elettrone può essere considerato fermo e ha
dunque energia nulla e dunque il lavoro di ionizzazione equivale all'energia posseduta
dall'elettrone in orbita, cambiata di segno.
\[
L=-\Biggl(\rec{2}m_e v^2 - \rec{4\pi\epsilon_0}\frac{e^2}{r} \Biggr)=-\Biggl(\rec{2}m_e (\omega r)^2 - \rec{4\pi\epsilon_0}\frac{e^2}{r} \Biggr)=
-\Biggl(\rec{8\pi\epsilon_0} \frac{e^2}{r} - \rec{4\pi\epsilon_0}\frac{e^2}{r} \Biggr)
\]
E quindi
\[
L=\rec{8\pi\epsilon_0} \frac{e^2}{r} \Longrightarrow  r=\frac{e^2}{8\pi\epsilon_0}\rec{L}
\]

Inserendo i dati per l'atomo di idrogeno ($L=13.5 eV$) si ottiene una corrente $I=1mA$ e
un momento magnetico $m=9.35\vdot 10^{-24} Am^2$, in buon accordo coi dati sperimentali.

Il momento magnetico è proporzionale al momento angolare orbitale dell'elettrone rispetto al
nucleo, in quanto entrambi i vettori giacciono sull'asse perpendicolare alla spira. Siccome la carica dell'elettrone
è negativa, i due vettori sono antiparalleli. Il loro rapporto dipende solo da caratteristiche
intrinseche dell'elettrone e vale
\[
\frac{m}{L}=\frac{e^-}{2m_e}=g_0
\]
Viene chiamato \textit{fattore giromagnetico orbitale g}.
Oltre al momento orbitale è necessario considerare anche il momento orbitale di spin
$s=\hbar/2$, uguale per protone, neutrone ed elettrone, che da luogo al fattore giromagnetico
intrinseco
\[
g=C\frac{e^-}{2m}
\]
dove $C$ è una costante che vale $2$ per l'elettrone, $2.79$ per il protone e $1.91$ per il neutrone.
Il momento magnetico totale si ottiene come somma vettoriale del momento magnetico orbitale e di qiello di
spin\footnote{Bisogna in realtà tenere conto anche del principio di esclusione di Pauli e della quantizzazione
del momento angolare}, ma siccome la massa di neutrone e protone è di tre ordini di grandezza più grande rispetto a quella dell'elettrone
il loro contributo al momento magnetico dell'atomo può spesso essere trascurato.

Atomi con simmetria sferica mostrano avere momento di dipolo magnetico nullo. In generale però
anche quando il momento di dipolo di ogni singolo atomo non è nullo, è nullo il momento complessivo
perchè i momenti di dipolo dei singoli atomi sono disposti casualmente. Un campo magnetico esterno
ha l'effetto di indurre un momento magnetico non nullo sul materiale, in maniera che però è diversa a seconda
della tipologia del materiale come verrà discusso in un paragrafo dedicato.
