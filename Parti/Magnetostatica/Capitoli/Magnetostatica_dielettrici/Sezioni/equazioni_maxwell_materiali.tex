\begin{defn}[Campo Magnetico]
    Si definisce vettore campo magnetico
    \[
        \vb{H}=\rec{\mu_0}\vb{B}-\vb{M}
    \]
\end{defn}

\begin{thm}
    In presenza di materiali la seconda e la quarta equazione di Maxwell assumono la seguente forma:
    \begin{equation}
        \div{\vb{B}}=0
    \end{equation}
    \begin{equation}
        \curl{\vb{H}}=\vb{J}
    \end{equation}
\end{thm}
\begin{proof}
    L'osservazione di partenza è che la presenza dei materiali impone una semplice modifica sulle
    equazioni di Maxwell per il campo di induzione magnetica, ovvero nella quarta equazione di Maxwell
    oltre alla densità di corrente macroscopica va considerata anche la densità delle correnti atomiche
    presenti nel materiale. Formalmente
    \[
        \curl{\vb{B}}=\mu_0(\vb{J}+\vb{J}_m)
    \]
    In generale, sulle superfici di separazione fra materiali, le equazioni di Maxwell non sono definite
    in quanto $\vb{B}$ subisce una discontinuità. Ha senso limitarsi a considerare quindi sono l'interno
    di un materiale dove $\vb{J}_m=\vb{J}_{mV}$. Di conseguenza, per la \ref{eqn:M_JmV}
    \[
        \curl{\vb{B}}=\mu_0 \vb{J}+ \mu_0\curl{\vb{M}}
    \]
    Da cui segue che
    \[
        \curl(\frac{\vb{B}-\mu_0\vb{M}}{\mu_0})=\vb{J}
    \]
    Ovvero dalla definizione di $\vb{H}$, la tesi.
\end{proof}
All'interno delle due equazioni non compare lo stesso campo e quindi queste non ammettono soluzione univoca a meno di
conoscere la relazione fra $\vb{H}$ e $\vb{B}$ -ovvero, dalla definizione di campo magnetico, fra $\vb{B}$ ed $\vb{M}$.
Questo aspetto sarà approfondito nel paragrafo sui tipi di materiali magnetici.

La forma integrale della quarta equazione di Maxwell fornisce una versione per il teorema della circuitazione di
Ampère valida nei materiali.
\begin{cor}[Teorema della circuitazione di Ampère]
    Prese delle correnti
    concatenate $I_i$ con segno positivo quando vedono girare il contorno orientato $l$
    in senso antiorario
    \begin{equation}
        \oint_l\vb{H}\vdot\dd{\vb{l}}=\sum \vb{I}_i
    \end{equation}
\end{cor}
\begin{proof}
    La dimostrazione è analoga a quella del teorema di Ampère nel vuoto corollario \ref{teo:Ampère}
\end{proof}


Come accennato, le equazioni di Maxwell per il magnetismo nella materia valgono solo
all'interno di un materiale. Per caratterizzare i fenomeni magnetici in tutto lo spazio
è quindi necessario risolvere le equazioni all'interno di ogni materiale che riempie
lo spazio e poi usare delle condizioni di raccordo sulla superficie che separa un
materiale dall'altro.
\begin{thm}
    Passando da un mezzo materiale all'altro, la componente normale di $\vb{B}$ non subisce
    alcuna discontinuità, così come la componente tangenziale di $\vb{H}$. In formule
    \[
        \vb{B}_{n1}=\vb{B}_{n2}
    \]
    \[
        \vb{H}_{t1}=\vb{H}_{t2}
    \]
\label{teo:condizioni_raccordo_campomagnetico}
\end{thm}
\begin{proof}
    Per dimostrare la prima equazione si consideri un cilindretto $C$ con le basi parallele alla
    superficie di separazione fra due materiali; Per la seconda si consideri invece un piccolo
    percorso rettangolare $l$ con due lati paralleli e due normali alla superficie di separazione.
    Il flusso di $\vb{B}$ uscente dalla superficie di $C$ deve essere nullo, così come la circuitazione
    di $\vb{H}$ su $l$ -in quanto non sono presenti correnti macroscopiche concatenate al percorso, ma solo correnti
    microscopiche. Analogamente a quanto fatto per [REF] si ha la tesi.
\end{proof}
Per misurare quindi $B$ e $H$ all'interno di un materiale è sufficiente misurare $B$ e $H$ in un taglio
sulla superficie del materiale rispettivamente ortogonale o parallelo alle linee di forza del campo.
Le equazioni di Maxwell ricavate in questo paragrafo  non dipendono dalle correnti microscopiche. Data la
facililità  di misurazione del campo magnetico $\vb{H}$ questa riscrittura è estremamente vantaggiosa.
