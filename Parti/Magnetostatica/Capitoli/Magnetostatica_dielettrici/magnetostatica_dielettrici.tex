Si vogliono ora studiare i fenomeni magnetici in uno spazio non più vuoto ma riempito con dei materiali.

\section{Aspetti atomici del magnetismo}
Si vuole preliminarmente studiare il comportamento di un atomo immerso in un campo di induzione
magnetica esterno. Per questo studio ci si limiterà all'approssimazione del modello planetario
di Rutherford, che consente di ottenere comunque informazioni interssanti.

L'idea di fondo è che un elettrone in orbita attorno al nucleo può essere visto come una spira
percorsa da corrente. Vogliamo quindi ricavare il momento magnetico, ovvero ricavare la superficie della spira
$S=\pi r^2$ e la corrente $I=q_e/T$, dove $T$ è il periodo.
Detto $r$ il raggio dell'orbita e $T$ il periodo di rivoluzione, si ha dalla legge di Coulomb
e dal secondo postulato della dinamica di Newton
\[
\rec{4\pi\epsilon_0}\frac{e^2}{r^2}=m_e\omega^2 r=\frac{(2\pi)^2}{T^2}m_e r
\]
da cui
\[
T=\frac{4\pi}{e}\sqrt{\pi\epsilon_0 m_e r^3}
\]
Il raggio dell'orbita può essere valutato a partire dal lavoro di prima ionizzazione,
ovvero dal lavoro necessario per strappare l'elettrone dalla sua orbita e portarlo
all'infinito. All'infinito infatti l'elettrone può essere considerato fermo e ha
dunque energia nulla e dunque il lavoro di ionizzazione equivale all'energia posseduta
dall'elettrone in orbita, cambiata di segno.
\[
L=-\Biggl(\rec{2}m_e v^2 - \rec{4\pi\epsilon_0}\frac{e^2}{r} \Biggr)=-\Biggl(\rec{2}m_e (\omega r)^2 - \rec{4\pi\epsilon_0}\frac{e^2}{r} \Biggr)=
-\Biggl(\rec{8\pi\epsilon_0} \frac{e^2}{r} - \rec{4\pi\epsilon_0}\frac{e^2}{r} \Biggr)
\]
E quindi
\[
L=\rec{8\pi\epsilon_0} \frac{e^2}{r} \Longrightarrow  r=\frac{e^2}{8\pi\epsilon_0}\rec{L}
\]

Inserendo i dati per l'atomo di idrogeno ($L=13.5 eV$) si ottiene una corrente $I=1mA$ e
un momento magnetico $m=9.35\vdot 10^{-24} Am^2$, in buon accordo coi dati sperimentali.

Il momento magnetico è proporzionale al momento angolare orbitale dell'elettrone rispetto al
nucleo, in quanto entrambi i vettori giacciono sull'asse perpendicolare alla spira. Siccome la carica dell'elettrone
è negativa, i due vettori sono antiparalleli. Il loro rapporto dipende solo da caratteristiche
intrinseche dell'elettrone e vale
\[
\frac{m}{L}=\frac{e^-}{2m_e}=g_0
\]
Viene chiamato \textit{fattore giromagnetico orbitale g}.
Oltre al momento orbitale è necessario considerare anche il momento orbitale di spin
$s=\hbar/2$, uguale per protone, neutrone ed elettrone, che da luogo al fattore giromagnetico
intrinseco
\[
g=C\frac{e^-}{2m}
\]
dove $C$ è una costante che vale $2$ per l'elettrone, $2.79$ per il protone e $1.91$ per il neutrone.
Il momento magnetico totale si ottiene come somma vettoriale del momento magnetico orbitale e di qiello di
spin\footnote{Bisogna in realtà tenere conto anche del principio di esclusione di Pauli e della quantizzazione
del momento angolare}, ma siccome la massa di neutrone e protone è di tre ordini di grandezza più grande rispetto a quella dell'elettrone
il loro contributo al momento magnetico dell'atomo può spesso essere trascurato.

Atomi con simmetria sferica mostrano avere momento di dipolo magnetico nullo. In generale però
anche quando il momento di dipolo di ogni singolo atomo non è nullo, è nullo il momento complessivo
perchè i momenti di dipolo dei singoli atomi sono disposti casualmente. Un campo magnetico esterno
ha l'effetto di indurre un momento magnetico non nullo sul materiale, in maniera che però è diversa a seconda
della tipologia del materiale come verrà discusso in un paragrafo dedicato.


\section{Vettore polarizzazione}
I fenomeni di polarizzazione magnetica possono essere descritti macroscopicamente intoducendo il vettore
polarizzazione magnetica.
\begin{defn}[Vettore polarizzazione magnetica]
    Si definisce il vettore polarizzazione magnetica $\vb{M}$ come il momento
    di dipolo magnetico per unità di volume posseduto dal materiale
    \[
        \vb{M}=\lim_{\tau\to 0}\frac{\sum \vb{m}_i}{\tau}=\frac{\dd{N}  \expval{\vb{m}}}{\dd{\tau}}
    \]
\end{defn}
Con $\dd{N}$ il numero di dipoli magentici contenute nell'elemento di volume $\dd{\tau}$.
Dalla definizione si ottiene che l'elemento di volume possiede un momento di dipolo $\dd{\vb{m}}=\vb{M}\dd{\tau}$.
Dei ragionamenti qualitativi preliminari ad una dimostrazione rigorosa permettono di comprendere in che modo
questo vettore sia legato alle correnti amperiane, cioè le correnti atomiche microscopiche.
Si consideri a tal fine un cilindro di materiale polarizzato lungo l'asse del cilindro stesso (ovvero tale che
$\vb{M}$ giaccia sull'asse del cilindro).
Dato che gli atomi sono sempre circondati da elettroni in movimento
l'unica possibilità per avere $M=0$ è che le spire microscopiche siano disposte casualmente
e che quindi il momento magnetico medio sia $0$. Quando $M\neq0$ significa che le spire microscopiche
sono prevalentemente orientate nella stessa direzione, ovvero giacciono sul piano ortogonale a $\vb{M}$.
Se $\vb{M}$ fosse indipendente dalla posizione, allora internamente al
materiale il valor medio delle correnti micorscopiche sarebbe nullo: infatti, prendendo un piano ortogonale
all'asse del cilindo, prendendo un punto $P$ su questo piano, si avrebbe in media perfetta compensazione fra
le correnti dei dipoli disposti simmetricamente a $P$. Non sarebbe nulla però la corrente sulla superficie
laterale del cilindro, descritta da un vettore densità $\vb{J}_{mS}$ che può essere definito come
\[
    \dd{I_{mS}}=\vb{J}_{mS}\vdot \dd{h}\vu{n}
\]
dove $Q_m$ è la carica microscopica media che flusce attraverso un segmento di lunghezza $\dd{h}$ ($\vu{n}$ è il
versore normale al segmento).
Se Se $\vb{M}$ non fosse indipendente dalla posizione, allora alla corrente microscopica superficiale andrebbe sommato
il contributo di una corrente microscopica interna al materiale descritta da una densità $\vb{J}_{mV}$
\[
    \dd{I_{mV}}=\vb{J}_{mV}\vdot \dd{\vb{S}}
\]

\begin{thm}
    Le relazioni fondamentali tra il vettore di polarizzazione e le densità di corrente
    di superficie e di volume in un materiale sono
    \begin{equation}
        \vb{J}_{mS}=\vb{M}\cp\vu{n}
        \label{eqn:M_JmS}
    \end{equation}
    \begin{equation}
        \vb{J}_{mV}=\curl{\vb{M}}
        \label{eqn:M_JmV}
    \end{equation}
    con $\vu{n}$ il versore normale alla superficie del materiale.
\end{thm}
\begin{proof}
    Si consideri un certo materiale di forma qualunque che occupi un volume $\tau$ delimitato da una superficie $S$.
    Sia $\vb{M}=\vb{M}(\vb{r}')$ il vettore di polarizzazione magnetica. Allora si ha, dalla definizione di $\vb{M}$
    \[
        \vb{A}(\vb{r})=\frac{\mu_0}{4\pi} \int_{\tau}\frac{\dd{\vb{m}}\cp(\vb{r}-\vb{r'})}{\abs{\vb{r}-\vb{r'}}^3}=
        \frac{\mu_0}{4\pi} \int_{\tau}\frac{\vb{M}(\vb{r'})\cp(\vb{r}-\vb{r'})}{\abs{\vb{r}-\vb{r'}}^3}\dd{\tau'}
    \]
    Ricordando la dimostrazione del teorema [REF] e usando l'equazione \eqref{app:eqn:curl_scalare_vettore}
    \[
        \vb{A}(\vb{r})=\frac{\mu_0}{4\pi} \int_{\tau}\frac{\curl'\vb{M}(\vb{r'})}{\abs{\vb{r}-\vb{r'}}}-
        \frac{\mu_0}{4\pi} \int_{\tau}\grad'\Biggl[\frac{\vb{M}(\vb{r'})}{\abs{\vb{r}-\vb{r'}}}\Biggr]\dd{\tau}
    \]
    Grazie alla seconda identità di green
    \[
        \vb{A}(\vb{r})=\frac{\mu_0}{4\pi} \int_{\tau}\frac{\curl'\vb{M}(\vb{r'})}{\abs{\vb{r}-\vb{r'}}}+
        \frac{\mu_0}{4\pi} \int_{S}\frac{\vb{M}(\vb{r'})\cp\vu{n}'\dd{S'}}{\abs{\vb{r}-\vb{r'}}}
    \]
    Ma per la \ref{eqn:A_I} il potenziale vettore può anche essere scritto come
    \[
        \vb{A}(\vb{r})=\frac{\mu_0}{4\pi}\int\frac{\vb{J}_{mV}(\vb{r}')}{\abs{\vb{r}-\vb{r}'}}\dd{\tau'}+
        \frac{\mu_0}{4\pi}\int\frac{\vb{J}_{mS}(\vb{r}')}{\abs{\vb{r}-\vb{r}'}}\dd{S'}
    \]
    Per confronto fra le integrande si ottiene la tesi.
\end{proof}


\section{Equazioni di Maxwell}
\begin{defn}[Campo Magnetico]
    Si definisce vettore campo magnetico
    \[
        \vb{H}=\rec{\mu_0}\vb{B}-\vb{M}
    \]
\end{defn}

\begin{thm}
    In presenza di materiali la seconda e la quarta equazione di Maxwell assumono la seguente forma:
    \begin{equation}
        \div{\vb{B}}=0
    \end{equation}
    \begin{equation}
        \curl{\vb{H}}=\vb{J}
    \end{equation}
\end{thm}
\begin{proof}
    L'osservazione di partenza è che la presenza dei materiali impone una semplice modifica sulle
    equazioni di Maxwell per il campo di induzione magnetica, ovvero nella quarta equazione di Maxwell
    oltre alla densità di corrente macroscopica va considerata anche la densità delle correnti atomiche
    presenti nel materiale. Formalmente
    \[
        \curl{\vb{B}}=\mu_0(\vb{J}+\vb{J}_m)
    \]
    In generale, sulle superfici di separazione fra materiali, le equazioni di Maxwell non sono definite
    in quanto $\vb{B}$ subisce una discontinuità. Ha senso limitarsi a considerare quindi sono l'interno
    di un materiale dove $\vb{J}_m=\vb{J}_{mV}$. Di conseguenza, per la \ref{eqn:M_JmV}
    \[
        \curl{\vb{B}}=\mu_0 \vb{J}+ \mu_0\curl{\vb{M}}
    \]
    Da cui segue che
    \[
        \curl(\frac{\vb{B}-\mu_0\vb{M}}{\mu_0})=\vb{J}
    \]
    Ovvero dalla definizione di $\vb{H}$, la tesi.
\end{proof}
All'interno delle due equazioni non compare lo stesso campo e quindi queste non ammettono soluzione univoca a meno di
conoscere la relazione fra $\vb{H}$ e $\vb{B}$ -ovvero, dalla definizione di campo magnetico, fra $\vb{B}$ ed $\vb{M}$.
Questo aspetto sarà approfondito nel paragrafo sui tipi di materiali magnetici.

La forma integrale della quarta equazione di Maxwell fornisce una versione per il teorema della circuitazione di
Ampère valida nei materiali.
\begin{cor}[Teorema della circuitazione di Ampère]
    Prese delle correnti
    concatenate $I_i$ con segno positivo quando vedono girare il contorno orientato $l$
    in senso antiorario
    \begin{equation}
        \oint_l\vb{H}\vdot\dd{\vb{l}}=\sum \vb{I}_i
    \end{equation}
\end{cor}
\begin{proof}
    La dimostrazione è analoga a quella del teorema di Ampère nel vuoto corollario \ref{teo:Ampère}
\end{proof}


Come accennato, le equazioni di Maxwell per il magnetismo nella materia valgono solo
all'interno di un materiale. Per caratterizzare i fenomeni magnetici in tutto lo spazio
è quindi necessario risolvere le equazioni all'interno di ogni materiale che riempie
lo spazio e poi usare delle condizioni di raccordo sulla superficie che separa un
materiale dall'altro.
\begin{thm}
    Passando da un mezzo materiale all'altro, la componente normale di $\vb{B}$ non subisce
    alcuna discontinuità, così come la componente tangenziale di $\vb{H}$. In formule
    \[
        \vb{B}_{n1}=\vb{B}_{n2}
    \]
    \[
        \vb{H}_{t1}=\vb{H}_{t2}
    \]
\label{teo:condizioni_raccordo_campomagnetico}
\end{thm}
\begin{proof}
    Per dimostrare la prima equazione si consideri un cilindretto $C$ con le basi parallele alla
    superficie di separazione fra due materiali; Per la seconda si consideri invece un piccolo
    percorso rettangolare $l$ con due lati paralleli e due normali alla superficie di separazione.
    Il flusso di $\vb{B}$ uscente dalla superficie di $C$ deve essere nullo, così come la circuitazione
    di $\vb{H}$ su $l$ -in quanto non sono presenti correnti macroscopiche concatenate al percorso, ma solo correnti
    microscopiche. Analogamente a quanto fatto per [REF] si ha la tesi.
\end{proof}
Per misurare quindi $B$ e $H$ all'interno di un materiale è sufficiente misurare $B$ e $H$ in un taglio
sulla superficie del materiale rispettivamente ortogonale o parallelo alle linee di forza del campo.
Le equazioni di Maxwell ricavate in questo paragrafo  non dipendono dalle correnti microscopiche. Data la
facililità  di misurazione del campo magnetico $\vb{H}$ questa riscrittura è estremamente vantaggiosa.


\section{Materiali dia-, para-, ferromagnetici}
Si vuole ora cercare di capire quale sia la relazione fra $\vb{B}$ e $\vb{H}$ in un materiale.
Nel vuoto evidentemente, essendo $\vb{M}=0$ si ha
\[
    \vb{H}=\rec{\mu_0}\vb{B}
\]

È allora sensato, considerando la linearità della teoria finora sviluppata pensare di scrivere
\[
    \vb{B}=\norm{\mu}\vb{H}
\]
dove $\norm{\mu}$ è un tensore.
\begin{obses}
    Nei mezzi materiali isotropi e omogenei $\vb{B}$ e $\vb{H}$ risultano sperimentalmente paralleli fra loro.
\end{obses}
Questo significa in particolare che $\vb{M}$ risulta essere parallelo o antiparallelo a $\vb{B}$ e che quindi
si può scrivere
\begin{equation}
    \vb{H}=\rec{\mu_0\mu_r}\vb{B}=\rec{\mu}\vb{B}
    \label{eqn:H_B}
\end{equation}
dove $\mu$ e $\mu_r$ sono delle costanti dette rispettivamente \textit{permeabilità megnetica} e
\textit{permeabilità megnetica relativa} del materiale in esame. Ovvero, nei materiali omogenei e isotropi
il tensore si riduce ad una costante.
\begin{thm}[Legge di rifrazione delle linee di forza]
    Presa la superficie di separazione fra due materiali ($1$ e $2$) isotropi e omogenei
    con permeabilità magnetica $\mu_1$ e $\mu_2$, detti $\vb{H}_1$ il campo nel materiale $1$ e
    $\vb{H}_2$ il campo nel materiale $2$ (analogamente per $\vb{B}$), detti $\theta_1$ e $\theta_2$ gli angoli che una
    linea di forza del campo forma con la normale alla superficie nei due materiali, si ha
    \[
        \frac{H_{t1}/{H_{n1}}}{H_{t2}/{H_{n2}}}=\frac{B_{t1}/{B_{n1}}}{B_{t2}/{B_{n2}}}=\frac{\tan{\theta_1}}{\tan{\theta_2}}=\frac{\mu_1}{\mu_2}
    \]
\end{thm}
\begin{proof}
    Riscrivendo la \eqref{eqn:H_B} per le componenti normali, ricordando i risultati del
    teorema \ref{teo:condizioni_raccordo_campomagnetico}, si ha
    \[
        \begin{cases}
            & H_{t1}=H_{t_2} \\
            & \mu_1 H_{n1}=\mu_2 H_{n2} \\
        \end{cases}
    \]
    \[
        \begin{cases}
            & B_{n1}=B_{n_2} \\
            & \mu_1 B_{t1}=\mu_2 B_{t2} \\
        \end{cases}
    \]
    Facendo i rapporti membro a membro si ottiene la tesi.
\end{proof}

Sono le caratteristiche di $\mu_r$ che permettono una classificazione dei materiali in:
diamagneti, paramagneti e ferromagneti.
Prima di addentrarsi nella descrizione di questi materiali è necessario introdurre la seguente quantità
\begin{defn}[Suscettività magnetica]
    Si definisce suscettività magnetica di un materiale
    \[
        \chi_m=\mu_r-1
    \]
\end{defn}
Per comprendere l'utilità di questa definizione bisogna osservare che, dalla definizione di campo magnetico,
per i materiali isotropi
\begin{equation}
    \vb{M}=\chi_m\vb{H}
    \label{eqn:M_H}
\end{equation}
Ovvero la suscettibilità magnetica rappresenta il fattore di proporzionalità fra il campo magnetico e
il vettore polarizzazione magnetica.


\subsubsection{Diamagneti}
Nelle sostanze diamagnetiche la permeabilità magnetica relativa è costante ed è prossima a $1$ in maniera tale che
$\chi_m<0$: il momento magnetico indotto nel materiale è di verso opposto rispetto al campo che lo induce. Generalmente
per queste sostanze si ha $\chi_m \in [-10^{-6},-10^{-9}]$.Per questi materiali
la suscettività magnetica è indipendente dalla temperatura. Inoltre, queste sostanze non presentano alcun tipo di
saturazione: le relazioni \eqref{eqn:H_B} e \eqref{eqn:M_H} valgono a prescindere dai valori assunti dai campi in esame.
Per quanto detto, riscrivendo la \eqref{eqn:H_B} come $\vb{B}=\mu_0(1+\chi_m)\vb{H}$, è chiaro come la perturbazione
causata dalla presenza di materiali diamagnetici sia di norma trascurabile.

\subsubsection{Paramagneti}
Anche nelle sostanze paramagnetiche la permebalitià magnetica relativa è costante e prossima a $1$, ma in questo caso
si ha $\chi_m>0$, ovvero il momento magnetico indotto nel materiale ha lo stessso verso del campo che lo induce.
Di norma per queste sostanze si ha $\chi_m \in [10^{-6},10^{-2}]$. La suscettività magnetica varia in funzione della
temperatura secondo la legge di curie
\begin{equation}
    \chi_m=C\frac{\rho}{T}
\end{equation}
dove $\rho$ è la densità del materiale e $C$ è una costante scritta in funzione delle grandezze atomiche.
All'avvicinarsi della temperatura allo 0 assoluto i campi in questi materiali sono molto diversi a quelli che
si hanno nel vuoto, ma in condizioni standard anche in questo caso la perturbazione è trascurabile.

\subsubsection{Ferromagneti}
Il comportamento dei ferromagneti è il più complesso: le relazioni fra campi e momenti di magentizzazione
non sono nè lineari nè univoche. Inoltre, le proprietà di questi materiali sono fortemente dipendenti dalle
loro caratteristiche chimico-fisiche.
Effettuando misure su un materiale isotropo si osserva che $\vb{B}(\vb{H})$ e $\vb{M}(\vb{H})$ assumono
sempre lo stesso verso di $\vb{H}$, per cui ci si può limitare a considerarle come relazioni scalari.
In particolare è istruttivo descrivere l'andamento di $B$ in funzione di $H$.

Partendo da $(H,B)=(0,0)$ all'aumentare di $H$ aumenta $B$ seguendo una curva $I$ detta
\textit{curva di prima magnetizzazione}. Raggiunto un valore $H_m\simeq 10^5 As/m$, da lì in poi $B$ aumenta
solo proporzionalmente ad $H$: dalla definizione di campo magnetico questo significa che $M$
ha raggiunto il valore di saturazione, oltre al quale non aumenta più.
Facendo diminuire $H$ a partire dal valore $H_m$ si ha che per un primo tratto $B$ segue la curva $I$, ma una
volta superato un valore $H_s$ si ha che $B$ scende mantenendosi sempre sopra alla curva $I$. In questo modo,
per $H=0$ si ha $B>0$ e il valore di $M=B/\mu_0$ corrispondente si chiama \textit{magnetizzazione residua}.
Cambiando ora il segno di $H$ e facendo crescere il suo valore assoluto, $B$ decresce ed esiste quindi sicuramente
un valore $H_c$ detto \textit{campo magnetico di coercizione} per cui $B=0$. In corrispondenza di questo valore
$M=H_c>0$. Superato questo punto $B$ diventa negativo e continua a decrescere fino al valore $-H_m$.
La curva completa si assesta su una curva ciclica simmetrica rispetto all'origine
detta \textit{curva di isteresi}. Se si restringe l'intervallo
all'interno del quale si fa variare $H$ si ottengono dei cicli sempre più piccoli sempre simmetrici rispetto all'origine
e prossimi alla curva di prima magnetizzazione. Eseguendo dei cicli in cui gradualmente si riduce l'ampiezza
dell'intervallo $B$ converge verso l'origine ed è possibile in questo modo smagnetizzare un materiale ferromagnetico.
Si può dimostrare che per produrre una variazione $\dd{H}$ del campo magnetico è necessario compiere
un lavoro $\dd{L}=B\dd{H}$, così che il lavoro complessivo dissipato dal ciclo di isteresi è
uguale all'area della superficie contenuta dalla curva.

È anche possibile relizzare dei cicli non simmetrici rispetto all'origine portando il sistema in una
posizione $(\bar{H},\bar{B})$ e spostandosi da questa di una piccola quantità $\Delta H$. Se questa variazione
è molto piccola il ciclo si riduce ad un segmento, in cui il lavoro dissipato è praticamente nullo: si parla di
\textit{ciclo elementare reversibile}.

Risulta quindi chiaro che fissato un valore di $H$ non è possibile ricavare un valore di $B$ senza conoscere il
resto della curva. In particolare perde di senso il parametro $\mu$ e bisogna quindi cercare dei parametri
alternativi adatti alla descrizione del fenomeno. Quando il ciclo è molto stretto una buona scelta è
la \textit{permeabilità magnetica differenziale assoluta}
\[
    \mu_d=\dv{B}{H}
\]

Un utile risultato riguarda il fatto che, superata una certa temperatura $T_c$ il materiale si comporta come
un materiale paramagnetico con suscettività magnetica
\[
    \chi_m=C\frac{\rho}{T-T_c}
\]


\section{Precessione di Larmor}
Per concludere si vuole tornare all'aspetto atomico della magnetizzazione
per descrivere un fenomeno che prende il nome di \textit{precessione di Larmor}.
Si consideri un elettrone orbitante attorno ad un nucleo con momento angolare
$\vb{L}$ e momento magnetico $\vb{m}_0=(-e/2m_e)\vb{L}$. Si supponga inoltre
che sull'elettrone agisca un campo magentico locale $\vb{B}_l=\mu_0\vb{H}_l$.
Nell'ipotesi che il campo magnetico perturbi di poco il moto dell'elettrone
allora, il momento angolare compie un moto di precessione attorno alla direzione
del campo. Si ha infatti, indicando con $\vb{M}$ il momento meccanico
\[
\dv{\vb{L}}{t}=\vb{M}=\vb{m}_0\cp\vb{B}_l=-\frac{e}{2m_e}\vb{L}\cp\vb{B}_l
\]
Chiaramente quindi la derivata di $\vb{L}$ è ortogonale a $\vb{L}$ e a $\vb{B}_l$.
Da questo segue che nel tempo cambia la direzione di $\vb{L}$ ma non il suo modulo
e che inoltre la componente $L_z$ di $\vb{L}$ parallela a $\vb{B}$ non cambia.
Perciò si ha effettivamente un moto di precessione che disegna un cono il cui asse
coincide con $\vb{B}$.

Resta da determinare la velocità di precessione, detta
\textit{velocità angolare di Larmor}. Dal corso di meccanica\footnote{Da leggersi
"non mi ricordo dell'esistenza di questa cosa, prendiamola per vera".} si ha
\[
\dv{\vb{L}}{t}=\vb{\omega}_L\cp\vb{L}
\]
Per cui si ha
\[
\vb{\omega}_L\cp\vb{L}=-\frac{e}{2m_e}\vb{L}\cp\vb{B}_l
\]
Ovvero per confronto
\[
\vb{\omega}_L=\frac{e}{2m_e}\vb{B}_l
\]
La velocità angolare ha lo stesso verso del campo di induzione magnetica, per cui la precessione
avviene in senso antiorario.

Questo moto comporta una \textit{corrente di Larmor}
\[
I_L=\frac{e}{T_L}=\frac{e\omega_L}{2\pi}=\frac{e^2B_L}{4\pi m_e}
\]
a cui, indicando con $S_z$ la proiezione della superficie dell'orbita sul piano $xy$, corrisponde
un momento magnetico diretto in verso opposto rispetto a $\vb{B}_L$ e in modulo
\[
m_L=I_L S_z=\frac{e^2B_L}{4\pi m_e} S_z
\]
$S_z$ è ottimamente approssimabile con $\pi(x^2+y^2)$, dove $x^2$ e $y^2$ rappresentano il valore
quadratico medio assunto dalle coordinate dell'elettrone mentre questo percorre l'orbita. Se
l'atomo è isotropicamente distribuito nello spazio si ha $x^2=y^2=z^2=r/3$ con $r$ il raggio
dell'orbita. Si può quindi riscrivere il momento magnetico di Larmor, tenendo conto di direzione e verso,
come
\[
\vb{m}_L=-\frac{e^2r^2}{6m_e} \vb{B}_L
\]
Se l'atomo ha $Z$ elettroni con raggio orbitale $r_i$, allora
\[
\vb{m}_L=-\frac{e^2r^2}{6m_e}\sum_{i=1}^Z r_i^2 \vb{B}_L
\]

