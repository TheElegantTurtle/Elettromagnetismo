Lo studio delle interazioni fra circuiti e campi magnetici è diviso in due categorie di fenomeni:
interazione fra un circuito percorso da corrente stazinaria con un campo magnetico esterno e
intrazione fra un circuito percorso da corrente stazionaria con un altro circuito percorso da corrente stazionaria.

\subsubsection{Interazione circuito-campo magnetico}
L'ipotesi fondamentale per lo studio di questi fenomeni è che il circuito in esame sia dotato di un generatore
di forza elettromotrice che mantenga costante nel tempo la corrente che attraversa il circuito.

Si vuole ora studiare il calcolo delle sollecitazioni meccaniche su una spira rigida percorsa da corrente $I$.
Si consideri quindi una spira $l$ percorsa da una corrente $I$ e immersa in un campo magnetico $\vb{B}$.
Si supponga che ogni tratto infinitesimo della spira $\dd{\vb{l}}$ compia uno spostamento infinitesimo
$\dd{\vb{s}}=\dd{\vb{s}}(\dd{\vb{l}})$, tale da portare la spira della configurazione $l'$.
Per far compiere alla spira questo spostamento senza che la sua energia cinetica vari è necessario applicare dall'esterno una forza
\[
    \dd{f}=-\dd{F}=-I\dd{\vb{l}}\cp\vb{B}
\]
Ovvero, è necessario compiere un lavoro, che risulta quindi in una variazione di energia potenziale
\[
    \dd{U}=\dd{L}=\oint\dd{\vb{f}}\vdot\dd{\vb{s}}=-\oint I\dd{\vb{l}}\cp\vb{B}\vdot\dd{\vb{s}}=
    -\oint I\dd{\vb{s}}\cp\dd{\vb{l}}\vdot\vb{B}=I\oint(\dd{\vb{l}}\cp\dd{\vb{s}})\vdot\vb{B}
\]
dove è stata usata la proprietà \eqref{app:eqn:vdot_cp}.
Il termine fra parentesi nell'ultimo membro della catena di uguaglianze rappresenta l'elemento di superficie $\dd{\vb{S}}$
della superficie laterale $\dd{\Sigma}$ di un solido le cui basi hanno spigoli $l$ ed $l'$.
L'integrale ottenuto è allora il flusso di $\vb{B}$ attraverso questa superficie laterale.
\[
    \dd{U}=\Phi_{\dd{\Sigma}}(\vb{B})
\]
Chiamando $\Sigma$ la base con spigolo $l$ e $\Sigma'$ la base con spigolo $l'$, siccome per il corollario \ref{cor:flusso_B}
il flusso di $\vb{B}$ attraverso una superficie chiusa deve essere nullo si ha
\[
    \Phi_{\dd{\Sigma}}=\Phi_\Sigma-\Phi_{\Sigma'}=-\dd{\Phi}
\]
Dove $\Phi_{\dd{\Sigma}}$ è il flusso entrante da $\Sigma$, mentre gli altri sono flussi uscenti dalle rispettive superfici.
Si ha quindi che a meno di una costante additiva arbitraria
\[
    U=-I\Phi(\vb{B})
\]
Dove il flusso è riferito alla superficie di una spira con contorno $l$ considerata positiva quando vede la corrente ruotare in sesno antiorario.
Ponendosi abbastanza lontano dalla spira in modo da poter considerare piccola la sua superficie
si può approssimare $\Phi(\vb{B})=\vb{B}\vdot\vb{S}$. In questo modo la relazione appena trovata diventa
\[
    U=-I\vb{S}\vdot\vb{B}=-\vb{m}\vdot\vb{B}
\]
Ma questa è un'espressione esattamente analoga a quanto trovato nel paragrafo sul dipolo elettrico \eqref{eqn:U_dipolo}.
Con passaggi identici si arriva allora a dimostrare
\begin{thm}
    Per una spira in un campo di induzione magnetica si ha che
    \begin{equation}
        \vb{F}=(\vb{m}\vdot\grad)\vb{B}
    \end{equation}
    \begin{equation}
        \vb{M}=\vb{m}\cp\vb{B}
    \end{equation}
\end{thm}
Si osservi quindi che esiste una completa analogia fra il dipolo in elettrostatica e la spira percorsa da
corrente in magnetostatica: sia la forma delle sollecitazioni meccaniche che quella del potenziale
(e di conseguenza del campo) prodotto
sono formalmente identiche a patto di usare $\vb{p}$ o $\vb{m}$ a seconda del contesto. Questo
risultato è detto \textit{teorema di equivalenza di Ampère} e sancisce la completa analogia
fra una spira da percorsa da corrente ed un "dipolo magnetico".\

Il calcolo del momento torcente può essere svolto in maniera immediata, e molto istruittiva, nel caso di una spira rettangolare
immersa in un campo magnetico costante.
\begin{example}
    Si supponga che il campo di induzione magnetica esterno $\vb{B}$ sia costante e che la spira sia rettangolare
    (con i lati 1 e 3 di lunghezza $b$ ed i lati 2 e 4 di lunghezza $a$), orientata in modo che i lati 2 e 4
    siano ortogonali a $\vb{B}$. I lati 1 e 3 sono soggetti ad una coppia di forze uguale ed opposta con braccio nullo;
    anche i lati 2 e 4 sono soggetti ad una coppia di forze uguale ed opposta,
    il braccio però vale $b\sin\theta$ con $\theta$ l'angolo fra $\vb{B}$ e la normale alla superficie $\vb{S}=ab\vu{n}$ della spira.
    Per la seconda legge di Laplace si ha $\vb{F}_2=\vb{F}_4=IaB$ per cui il momento meccanico sulla spira vale in modulo
    \[
        M=\frac{b}{2}F_2\sin\theta+\frac{b}{2}F_4\sin\theta=bIaB\sin\theta=ISB\sin\theta
    \]
    dove $\theta$ è l'angolo fra la forza ed il raggio-vettore che congiunge il punto di applicazione della forza al centro
    di massa.
    La relazione vettoriale è quindi
    \[
        \vb{M}=IS\vu{n}\cp\vb{B}=\vb{m}\cp\vb{B}
    \]
\end{example}



\subsubsection{Interazione circuito-circuito}
Dati due circuiti $l_1$ ed $l_2$ questi esercitano l'uno sull'altro delle azioni meccaniche in quanto
sorgenti di campi magnetici. La forza che $l_1$ esercita sull'elemento di filo $\dd{l_2}$ di
$l_2$ è\footnote{La distanza fra due punti presenti rispettivamente sul primo e sul secondo circuito
è stata indicata con $r_{12}$ anzichè con $\Delta r$}
\[
    \dd{\vb{F}_{12}}=I_2\dd{\vb{l}_2}\cp\vb{B}_1=I_2\dd{\vb{l}_2}\cp\oint_{l_1}\frac{\mu_0I_1}{4\pi}\frac{\dd{\vb{l}_1}\cp\vb{r}_{12}}{r^3_{12}}
\]
dove $\vb{r}_{12}$ è la distanza fra un punto di $l_1$ e l'elemento di filo $\dd{l_2}$.
Nel caso in cui i due circuiti siano rigidi si ha
\[
    \vb{F}_{12}=\oint_{l_2}I_2\dd{\vb{l}_2}\cp\oint_{l_1}\frac{\mu_0I_1}{4\pi}\frac{\dd{\vb{l}_1}\cp\vb{r}_{12}}{r^3_{12}}=
    \frac{\mu_0}{4\pi}I_1 I_2 \oint_{l_2}\oint_{l_1}\frac{\dd{\vb{l}_2}\cp\dd{\vb{l}_1}\cp\vb{r}_{12}}{r^3_{12}}
\]
Usando l'identità vettoriale  $\vb{a}\cp\vb{b}\cp\vb{c}=(\vb{a}\vdot\vb{c})\vb{b}-(\vb{b}\vdot\vb{a})\vb{c}$, si ottiene
\[
    \vb{F}_{12}=\frac{\mu_0}{4\pi}I_1 I_2 \Biggl[\oint_{l_2}\oint_{l_1}\frac{(\dd{\vb{l}_2}\vdot\vb{r}_{12})\dd{\vb{l}_1}}{r^3_{12}}-
    \oint_{l_2}\oint_{l_1}\frac{(\dd{\vb{l}_1}\vdot\dd{\vb{l}_2})\vb{r}_{12}}{r^3_{12}}\Biggr]
\]
Il primo di questi due integrali è nullo in quanto
\[
    \oint_{l_2}\oint_{l_1}\frac{(\dd{\vb{l}_2}\vdot\vb{r}_{12})\vb{l}_1}{r^3_{12}}=
    \oint_{l_1}\dd{\vb{l}_1}\oint_{l_2}\frac{(\dd{\vb{l}_2}\vdot\vb{r}_{12})}{r^3_{12}}=
    \oint_{l_1}\dd{\vb{l}_1}\oint_{l_2}\dd{\vb{l}_2}\vdot\grad(\rec{r_{12}})
\]
Ma la circuitazione di un campo conservativo è nulla.
Si ha allora
\begin{equation}
    \vb{F}_{12}=\frac{\mu_0}{4\pi}I_1 I_2 \oint_{l_2}\oint_{l_1}\frac{(\dd{\vb{l}_1}\vdot\dd{\vb{l}_2})\vb{r}_{12}}{r^3_{12}}
\end{equation}
L'equazione qui sopra mostra che scambiando gli indici si ottiene un cambio di segno, in accordo col terzo principio della dinamica.

Considerando due fili rettilinei infinitamente lunghi e paralleli, posti a distanza $\vb{r}_{12}$, si trova che
\[
    \dd{\vb{F}_{12}}=\frac{\mu_0}{2\pi r_{12}}I_1 I_2\vb{r}_{12} sgn(\dd{l_1}\vdot\dd{l_2})
\]
ovvero la forza è attrattiva o repulsiva in base al fatto che il verso delle correnti sia concorde o discorde.
Inoltre questa relazione fornisce una definizione operativa dell'ampère: si dice che due fili lunghi e sottili
sono attraversati da una corrente di $1A$ quando, posti ad un metro di distanza, risentono reciprocamente di
una forza pari a $\mu_0/2\pi$ per metro di filo.
