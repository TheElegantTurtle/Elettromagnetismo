In un conduttore percorso da corrente elettrica e immerso in un campo di induzione magnetica
esterno ortogonale alle linee di corrente, si genera tra bordi opposti una differenza di
potenziale
\begin{equation}
    \Delta V_H = v_d B a = R_H \frac{IB}{b} \quad\quad R_H=\rec{nq}
    \label{eqn:hall}
\end{equation}
dove $b$ rappresenta lo spessore del conduttore in direzione parallela a $\vb{B}$, a lo spessore
in direzione ortogonale e $n$ la densità
di portatori di carica. Questo effetto è chiamato \textit{Effetto Hall} e la costante $R_H$ è detta
\textit{costante di Hall}, con un valore tipico $\abs{R_H}\simeq 10^{-11}m^3/C$.

Per spiegare il fenomeno, si consideri una sbarretta conduttrice a forma di parallelepipedo la cui
sezione abbia base di lati $a$ e $b$ immersa in un campo magentico $\vb{B}$ parallelo al lato $b$.
Sia questa sbarretta percorsa da corrente $I$ distributia uniformemente sulla sezione. I portatori
di carica, sottoposti alla forza di Lorentz\footnote{Si osservi che il verso della forza di Lorentz
non dipende dal segno della carica dei portatori in quanto questo dipende dal segno del prodotto $qv_d$
che indipendente dalla carica.}, verranno deviati verso una delle facce della superficie
laterale della sbarretta sulla quale si avrà quindi un accumulo di carica. Come conseguenza di questo
accumulo, si deve avere sulla faccia opposta un accumulo della carica opposta.
Il campo elettrostatico $E_s$ generato da questi accumuli si oppone alla forza di Lorentz fino al punto di
saturazione $qE_s=qv_d B$, raggiunto il quale il moto dei portatori torna ad essere quello tipico.
Tra le due facce della sbarretta si genera un potenziale $\Delta V_H=E_sa=v_dBa$.
Indicando con $J$ la densità di corrente uniforme per ipotesi, si ha $I=Jab=nqv_dab$
Sostituendo $v_d$ nell'equazione del potenziale di Hall appena trovata si ha anche la seconda
espressione del potenziale.

Tipicamente la costante di Hall è negativa e questo conferma che i portatori di carica siano tipicamente
elettroni. Inoltre invertendo la formula si ricava che il numero di elettroni liberi per ogni atomo è di
$1-2$
