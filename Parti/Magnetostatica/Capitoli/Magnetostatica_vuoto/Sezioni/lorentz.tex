I risultati sperimentali esposti nell'introduzione al capitolo si spiegano ipotizzando che il 
campo di induzione magnetica determini una forza nella forma:
\begin{equation}
    \dd{\vb{F}}=I\dd{\vb{l}}\cp\vb{B}
\end{equation}
Questa equazione è detta \textit{seconda legge di Laplace} e consente di misurare $\vb{B}$ (ne costituisce quindi
la definizione operativa).

Si dimostra il seguente risultato
\begin{thm}[Forza di Lorentz]
    Data una carica puntifome $q$ che si muove con velocità $\vb{v}$ in un campo di induzione magnetica, questa carica risente di una forza
    \begin{equation}
        \label{eqn:f_lorentz}
        \vb{F}=q\vb{v}\cp\vb{B}
    \end{equation}
\end{thm}
\begin{proof}
    Si consideri un tratto di circuito di lunghezza $\dd{l}$, sezione $\dd{S}$ e -quindi- volume  
    $\dd{\tau}=\dd{l}\dd{S}$. Per il teorema \ref{thm:I_flusso_J}, la seconda legge di Laplace può essere riscritta come
    \[
        \dd{\vb{F}}=\vb{J}\dd{S}\dd{l}\cp\vb{B}=\vb{J}\dd{\tau}\cp\vb{B}
    \]
    Inoltre per la definizione di $\vb{J}$ si ha
    \[
        \vb{J}\dd{\tau}=nq\vb{v}_d\dd{\tau}=\dd{N}q\vb{v}_d
    \]
    dove $\dd{N}=n\dd{S}\dd{l}$ rappresenta il numero di portatori di carica nel tratto $\dd{l}$ del circuito. 
    Si ha quindi $\dd{\vb{F}}=\dd{N}q\vb{v}_d\cp\vb{B}$. Se si prende in esame una singola carica, 
    la velocità di deriva coincide con la velocità della carica. 
    Intregrando per ottenere la forza totale  e considerando che il numero totale di cariche è appunto $1$ si ottiene la tesi.
\end{proof}
La relazione di Lorentz è in realtà più generale rispetto alla seconda legge di Laplace: quest'ultima infatti vale
solo qualora $\vb{B}$ non vari significativamente nel tratto $\dd{l}$, mentre la legge di Lorentz è locale. Il prezzo
da pagare è una maggiore difficoltà nella misurazione del campo di induzione magnetica in quanto nel primo caso viene
usato come sonda un circuito percorso da corrente, nel secondo invece una carica in movimento.
\begin{cor}
    La forza di Lorentz non compie alcun lavoro.
\end{cor}
\begin{proof}
    Quando la carica è ferma, non risente della forza di Lorentz. Quando è in movimento, la forza di Lorentz è ortogonale alla velocità.
\end{proof}
Questa forza quindi cambia la direzione della carica in movimento ma non ne modifica la velocità (in modulo).

Grazie a quanto visto ora si può determinare l'unità di misura del campo di induzione magnetica, chiamata \textit{tesla}.
\[
    [B]=N\rec{Cm/s}=\frac{m}{m}N\rec{Cm/s}=\frac{Vs}{m^2}=T
\]
Il prodotto $Vs$ prende il nome di \textit{weber (Wb)}. 
Frequentemente, per il campo di induzione elettromagnetica si usa il \textit{gauss} $1T=10^4G$.
