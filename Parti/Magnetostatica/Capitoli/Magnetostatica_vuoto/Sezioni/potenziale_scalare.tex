Sorge spontaneo a questo punto chiedersi se e sotto quali condizioni sia possibile definire un
potenziale scalare $\phi$ per il campo magnetico, in modo tale che $-\grad\phi=\vb{B}$.
Condizione necessaria e sufficiente affinchè un campo vettoriale definito in un dominio $D$
sia esprimibile come gradiente di una funzione scalare è che il campo sia irrotazionale
e che il dominio sia semplicemente connesso.
La quarta equazione di Maxwell mostra chiaramente che il $\curl{\vb{B}}=0$ quando $\vb{J}=0$.
Quando la densità di corrente che genera il campo di induzione magnetica è localizzata
al finito in un dominio $D$, è possibile prendere un insieme $B$ che la contenga completamente:
sul dominio semplicemente connesso $D'=D\setminus B$ il campo è ora irrotazionale e può quindi
essere definito un potenziale scalare.
\begin{thm}
    Sotto le opportune ipotesi di dominio, $\vb{B}=-\grad\phi$ con
    \[
        \phi=-\frac{\mu_0 I}{4\pi}\Omega
    \]
    a meno di una costante additiva ($\Omega$ è l'angolo solido).
\end{thm}
\begin{proof}
    Dalla definizione di potenziale $-\dd{\phi}=\vb{B}\vdot\dd{\vb{l}}$. Esplicitando il campo
    \[
        -\dd{\phi}=\frac{\mu_0}{4\pi}\Biggl(\oint_l \frac{I\dd{\vb{l}'}\cp(\vb{r}-\vb{r}')}{\abs{\vb{r}-\vb{r}'}^3}\Biggr)\vdot\dd{\vb{l}}
    \]
    Uno spostamento $\dd{\vb{l}}$ di un osservatore nel campo è equivalente ad uno spostamento $\dd{\vb{s}}=-\dd{\vb{l}}$
    del circuito che genera il campo, da cui
    \[
        \dd{\phi}=\frac{\mu_0 I}{4\pi}\oint_l \dd{\vb{l}'}\cp\frac{(\vb{r}-\vb{r}')}{\abs{\vb{r}-\vb{r}'}^3}\vdot\dd{\vb{s}}
    \]
    per la \eqref{app:eqn:vdot_cp} l'integranda può essere riscritta
    \[
        \dd{\vb{l}'}\cp\frac{(\vb{r}-\vb{r}')}{\abs{\vb{r}-\vb{r}'}^3}\vdot\dd{\vb{s}}=
        \dd{\vb{l}'}\cp\dd{\vb{s}}\vdot\frac{-(\vb{r}-\vb{r}')}{\abs{\vb{r}-\vb{r}'}^3}=
        \dd{\vb{S}} \vdot \frac{-(\vb{r}-\vb{r}')}{\abs{\vb{r}-\vb{r}'}^3}
    \]
    dove $\dd{\vb{S}}$ è l'elemento di superficie spazzato dall'elemento di circuito $\dd{\vb{l}'}$ nello spostamento $\dd{\vb{s}}$.
    Il prodotto scalare che compare in questa espressione rappresenta la proiezione dell'elemento di superficie
    sul raggio-vettore che va da $\vb{r}'$ a $\vb{r}$, ovvero dalla posizione in cui si trova l'osservatore alla posizione
    dell'elemento di superficie. Ma allora l'integranda non è altro che l'angolo solido infinitesimo sotto al quale
    l'osservatore viene "visto" da $\dd{\vb{S}}$. L'integrale, che è esteso a tutto il circuito rappresenta dunque
    l'angolo solido $\dd{\Omega}$ sotto al quale l'osservatore viene visto dalla superficie $\dd{\Sigma}$ spazzata da tutto il circuito
    nello spostamento $\dd{\vb{s}}$. Per l'osservatore allora il circuito si sposta variando la propria posizione di un
    angolo solido $-\dd{\Omega}$. In conlcusione
    \[
        \dd{\phi}=-\frac{\mu_0 I}{4\pi}\dd{\Omega}
    \]
    da cui segue la tesi.
\end{proof}

\begin{cor}
    Nell'ipotesi in cui l'osservatore sia molto lontano dalla spira, indicando con $\vb{r}$ il raggio-vettore
    che va dal circuito all'osservatore, si ha
    \[
        \phi=\frac{\mu_0}{4\pi}\frac{\vb{m}\vdot\vb{r}}{r^3}
    \]
\end{cor}

\begin{proof}
    Nell'ipotesi in cui le dimensioni lineari della spira siano molto minori di $r$, si ha
    \[
        \Omega=-\frac{\vb{S}\vdot\vb{r}}{r^3}
    \]
    Sostituendo nell'espressione del potenziale scalare si ha la tesi.
\end{proof}
Si osservi la completa analogia che esiste fra questa espressione e l'espressione del potenziale prodotto da
un dipolo elettrico: come si vedrà nel paragrafo dedicato all'interazione fra campi magnetici e circuiti
questa analogia è molto più profonda.
È un puro esercizio di calcolo mostrare che questa approssimazione per il potenziale scalare e
l'approssimazione \eqref{eqn:potenziale_vettore_momento_magnetico} per il potenziale vettore
producono lo stesso campo.
