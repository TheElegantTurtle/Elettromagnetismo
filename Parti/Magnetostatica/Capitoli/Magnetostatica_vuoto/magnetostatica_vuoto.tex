\begin{obses}
    Si consideri un sistema fatto in questo modo:
    uno o più circuiti sono fermi e percorsi da corrente stazionaria;
    un circuito di prova ha un piccolo tratto rettilineo $\dd{\vb{l}}$, connesso al resto del circuito
    mediante connessioni flessibili, elettricamente neutro e percorso da corrente $I$.
    Si osserva (e si misura mediante un dinamometro) che $\dd{\vb{l}}$
    risente di una forza $\dd{\vb{F}}$ ad opera degli altri circuiti con le seguenti caratteristiche:
    \begin{enumerate}
        \item $\dd{F} \propto I\dd{l}$;
        \item la direzione di $\dd{\vb{F}}$ è ortogonale a quella di $\dd{\vb{l}}$;
        \item $\dd{\vb{F}}$ dipende dalla posizione e dall'orientamento di $\dd{\vb{l}}$. In particolare esiste sempre
            una direzione di $\dd{\vb{l}}$ tale per cui la forza è nulla e la direzione per cui la forza
            è massima risulta ortogonale a quest'ultima.
    \end{enumerate}
\end{obses}

Queste osservazioni sperimentali costituiscono il punto di partenza per lo sviluppo del magnetismo.


\section{Campo di induzione magnetica - prima legge di Laplace}
Tutta una serie di esperimenti porta a concludere che esista un campo, detto campo di induzione magnetica:
\begin{defn}[Campo di induzione magnetica]
    Si definisce campo di induzione magnetica $\vb{B}$ il responsabile delle forze sentite
    dal tratto di filo $\dd{\vb{l}}$. Questo campo è dipendente dalla posizione ed è
    generato da circuiti nei quali circoli corrente stazionaria.
\end{defn}
\begin{obses}
    Dato un circuito filiforme $l'$, detto $\dd{\vb{l'}}$ una porzione infinitesima di questo circuito
    in posizione $\vb{r'}$
    e posto l'osservatore in posizione $\vb{r}$, si ha che il campo $\vb{B}$ generato da questo circuito vale
    \begin{equation}
        \label{eqn:B}
        \vb{B}=\frac{\mu_0}{4\pi}\oint_{l'}\frac{I\dd{\vb{l'}}\cp(\vb{r}-\vb{r}')}{\abs{\vb{r}-\vb{r}'}^3}
    \end{equation}
    con $\Delta \vb{r}=\vb{r}-\vb{r'}$.
\end{obses}

Il seguente corollario rappresenta un'estrapolazione teorica della situazione sperimentale appena descritta.
\begin{cor}[Legge di Biot Savart / prima legge di Laplace]
    Il campo di induzione magnetica può essere calcolato come somma di contributi elementari prodotti
    dai singoli elementi $\dd{l'}$ del circuito:
    \begin{equation}
        \label{eqn:dB}
        \dd{\vb{B}}=\frac{\mu_0}{4\pi}\frac{I\dd{\vb{l'}}\cp(\vb{r}-\vb{r}')}{\abs{\vb{r}-\vb{r}'}^3}
    \end{equation}
\end{cor}

\begin{cor}
    Facendo cadere l'ipotesi di circuito filiforme si ottiene
    \begin{equation}
        \label{eqn:prima_laplace_non_filiforme}
        \vb{B}(\vb{r})=\frac{\mu_0}{4\pi}\int_{\tau}\frac{\vb{J}(\vb{r'})\cp(\vb{r}-\vb{r}')}{\abs{\vb{r}-\vb{r}'}^3}\dd{\tau}
    \end{equation}
\end{cor}
\begin{proof}
    Ponendo $I=\int_{S'}\vb{J}(\vb{r'})\vdot\dd{\vb{S'}}$ nella \eqref{eqn:B} si ottiene:
    \[
        \vb{B}(\vb{r})=\frac{\mu_0}{4\pi}\int_{l'}\Biggl[\int_S (\vb{J}(\vb{r'})\vdot\dd{\vb{S'}}) \frac{\dd{\vb{l'}}\cp(\vb{r}-\vb{r}')}{\abs{\vb{r}-\vb{r}'}^3}\Biggr]=
        \frac{\mu_0}{4\pi}\int_{\tau'}\frac{\vb{J}(\vb{r'})\cp(\vb{r}-\vb{r}')}{\abs{\vb{r}-\vb{r}'}^3}\dd{\tau'}
    \]
\end{proof}
È importante osservare come la sorgente di questo campo siano le cariche in movimento.


\section{Seconda equazione di Maxwell}
In analogia con quanto fatto col campo elettrico si vuole ora studiare
il flusso del campo di induzione magnetica attraverso una superficie chiusa.
\begin{thm}[Seconda equazione di Maxwell stazionaria nel vuoto]
    Nel vuoto il campo $\vb{B}$ generato da un circuito attraversato da corrente stazionaria è solenoidale, ovvero
    \begin{equation}
        \div{\vb{B}}=0
    \end{equation}
\end{thm}
\begin{proof}
    Per la \eqref{app:eqn:grad_r}
    \[
        \vb{B}(x,y,z)=-\frac{\mu_0}{4\pi}\int_\tau \vb{J}(\vb{r}')\cp\grad[\rec{\abs{\vb{r}-\vb{r}'}}]\dd{\tau'}
    \]
    Per la formula \eqref{app:eqn:curl_scalare_vettore}, l'integranda può essere riscritta come
    \[
        \vb{J}(\vb{r}')\cp\grad[\rec{\abs{\vb{r}-\vb{r}'}}]=
        \rec{\abs{\vb{r}-\vb{r}'}}[\curl{\vb{J}}(\vb{r}')]-\curl[\frac{\vb{J}(\vb{r}')}{\abs{\vb{r}-\vb{r}'}}]=
        -\curl[\frac{\vb{J}(\vb{r}')}{\abs{\vb{r}-\vb{r}'}}]
    \]
    in quanto il vettore densità di corrente dipende solo dalle coordinate $(x',y',z')$ mentre il gradiente opera sulle $(x,y,z)$.
    Per lo stesso motivo, si può affermare che
    \[
        \vb{B}(x,y,z)=\frac{\mu_0}{4\pi}\int_{\tau'} \curl[\frac{\vb{J}(\vb{r}')}{\abs{\vb{r}-\vb{r}'}}]=
        \curl[\frac{\mu_0}{4\pi}\int_{\tau'} \frac{\vb{J}(\vb{r}')}{\abs{\vb{r}-\vb{r}'}}]
    \]
    Ma la divergenza di un rotore è nulla, perciò segue immediatamente la tesi.
\end{proof}
\begin{cor}
    \label{cor:flusso_B}
    Nel vuoto, il flusso attraverso una superficie chiusa di $\vb{B}$, generato da un circuito attraversato da corrente stazionaria,
    è nullo.
\end{cor}
\begin{proof}
    Integrando sul volume la divergenza di $\vb{B}$
    e applicando il teorema della divergenza, si ha la tesi.
\end{proof}

La seconda equazione di Maxwell permette di ottenere alcune caratteristiche fondamentali del campo di induzione magnetica.
\begin{cor}
    Date due superfici orientate $S$ ed $S'$ aventi lo stesso contorno e stessa orientazione si ha
    \[
        \Phi_S(\vb{B})=\Phi_{S'}(\vb{B})
    \]
    Ovvero, il flusso di $\vb{B}$ dipende esclusivamente da contorno e orientamento.
\end{cor}
\begin{proof}
    Si consideri una superficie chiusa e la si divida in due superfici $S$ ed $S'$:
    queste due avranno orientazione opposta e condivideranno il contorno.
    Per la linearità del flusso e per il corollario appena dimostrato
    si ha $0=\Phi_{S\cup S'}(\vb{B})=\Phi_{S}(\vb{B})+\Phi_{S'}(\vb{B})$,
    ovvero $\Phi_{S}(\vb{B})=-\Phi_{S'}(\vb{B})$.
    Cambiando orientazione ad una delle due superfici si cambia il segno del relativo flusso ottenendo la tesi.
\end{proof}
Si parla quindi di \textit{flusso concatenato ad un contorno}, senza far riferimento alla superficie.
\begin{cor}
    Le linee di forza del campo di induzione magnetica sono chiuse.
\end{cor}
\begin{proof}
    Per assurdo, si cosideri una linea di forza di $\vb{B}$ non chiusa:
    deve esistere un punto sorgente per tale linea. È possibile prendere allora una superficie chiusa piccola a piacere
    attorno a questo punto sorgente attraverso la quale il flusso del campo è diverso da 0.
\end{proof}
Da questo segue immediatamente che i tubi di flusso per il campo $\vb{B}$ non hanno nè inizio nè fine.


\section{Il potenziale vettore e la quarta equazione di Maxwell}
\label{par:potenziale_vettore}
\input{Parti/Magnetostatica/Capitoli/Magnetostatica_vuoto/Sezioni/potenziale_vettore.tex}

\section{Il potenziale scalare}
Sorge spontaneo a questo punto chiedersi se e sotto quali condizioni sia possibile definire un
potenziale scalare $\phi$ per il campo magnetico, in modo tale che $-\grad\phi=\vb{B}$.
Condizione necessaria e sufficiente affinchè un campo vettoriale definito in un dominio $D$
sia esprimibile come gradiente di una funzione scalare è che il campo sia irrotazionale
e che il dominio sia semplicemente connesso.
La quarta equazione di Maxwell mostra chiaramente che il $\curl{\vb{B}}=0$ quando $\vb{J}=0$.
Quando la densità di corrente che genera il campo di induzione magnetica è localizzata
al finito in un dominio $D$, è possibile prendere un insieme $B$ che la contenga completamente:
sul dominio semplicemente connesso $D'=D\setminus B$ il campo è ora irrotazionale e può quindi
essere definito un potenziale scalare.
\begin{thm}
    Sotto le opportune ipotesi di dominio, $\vb{B}=-\grad\phi$ con
    \[
        \phi=-\frac{\mu_0 I}{4\pi}\Omega
    \]
    a meno di una costante additiva ($\Omega$ è l'angolo solido).
\end{thm}
\begin{proof}
    Dalla definizione di potenziale $-\dd{\phi}=\vb{B}\vdot\dd{\vb{l}}$. Esplicitando il campo
    \[
        -\dd{\phi}=\frac{\mu_0}{4\pi}\Biggl(\oint_l \frac{I\dd{\vb{l}'}\cp(\vb{r}-\vb{r}')}{\abs{\vb{r}-\vb{r}'}^3}\Biggr)\vdot\dd{\vb{l}}
    \]
    Uno spostamento $\dd{\vb{l}}$ di un osservatore nel campo è equivalente ad uno spostamento $\dd{\vb{s}}=-\dd{\vb{l}}$
    del circuito che genera il campo, da cui
    \[
        \dd{\phi}=\frac{\mu_0 I}{4\pi}\oint_l \dd{\vb{l}'}\cp\frac{(\vb{r}-\vb{r}')}{\abs{\vb{r}-\vb{r}'}^3}\vdot\dd{\vb{s}}
    \]
    per la \eqref{app:eqn:vdot_cp} l'integranda può essere riscritta
    \[
        \dd{\vb{l}'}\cp\frac{(\vb{r}-\vb{r}')}{\abs{\vb{r}-\vb{r}'}^3}\vdot\dd{\vb{s}}=
        \dd{\vb{l}'}\cp\dd{\vb{s}}\vdot\frac{-(\vb{r}-\vb{r}')}{\abs{\vb{r}-\vb{r}'}^3}=
        \dd{\vb{S}} \vdot \frac{-(\vb{r}-\vb{r}')}{\abs{\vb{r}-\vb{r}'}^3}
    \]
    dove $\dd{\vb{S}}$ è l'elemento di superficie spazzato dall'elemento di circuito $\dd{\vb{l}'}$ nello spostamento $\dd{\vb{s}}$.
    Il prodotto scalare che compare in questa espressione rappresenta la proiezione dell'elemento di superficie
    sul raggio-vettore che va da $\vb{r}'$ a $\vb{r}$, ovvero dalla posizione in cui si trova l'osservatore alla posizione
    dell'elemento di superficie. Ma allora l'integranda non è altro che l'angolo solido infinitesimo sotto al quale
    l'osservatore viene "visto" da $\dd{\vb{S}}$. L'integrale, che è esteso a tutto il circuito rappresenta dunque
    l'angolo solido $\dd{\Omega}$ sotto al quale l'osservatore viene visto dalla superficie $\dd{\Sigma}$ spazzata da tutto il circuito
    nello spostamento $\dd{\vb{s}}$. Per l'osservatore allora il circuito si sposta variando la propria posizione di un
    angolo solido $-\dd{\Omega}$. In conlcusione
    \[
        \dd{\phi}=-\frac{\mu_0 I}{4\pi}\dd{\Omega}
    \]
    da cui segue la tesi.
\end{proof}

\begin{cor}
    Nell'ipotesi in cui l'osservatore sia molto lontano dalla spira, indicando con $\vb{r}$ il raggio-vettore
    che va dal circuito all'osservatore, si ha
    \[
        \phi=\frac{\mu_0}{4\pi}\frac{\vb{m}\vdot\vb{r}}{r^3}
    \]
\end{cor}

\begin{proof}
    Nell'ipotesi in cui le dimensioni lineari della spira siano molto minori di $r$, si ha
    \[
        \Omega=-\frac{\vb{S}\vdot\vb{r}}{r^3}
    \]
    Sostituendo nell'espressione del potenziale scalare si ha la tesi.
\end{proof}
Si osservi la completa analogia che esiste fra questa espressione e l'espressione del potenziale prodotto da
un dipolo elettrico: come si vedrà nel paragrafo dedicato all'interazione fra campi magnetici e circuiti
questa analogia è molto più profonda.
È un puro esercizio di calcolo mostrare che questa approssimazione per il potenziale scalare e
l'approssimazione \eqref{eqn:potenziale_vettore_momento_magnetico} per il potenziale vettore
producono lo stesso campo.


\section{Forza di Lorentz - seconda legge di Laplace}
I risultati sperimentali esposti nell'introduzione al capitolo si spiegano ipotizzando che il 
campo di induzione magnetica determini una forza nella forma:
\begin{equation}
    \dd{\vb{F}}=I\dd{\vb{l}}\cp\vb{B}
\end{equation}
Questa equazione è detta \textit{seconda legge di Laplace} e consente di misurare $\vb{B}$ (ne costituisce quindi
la definizione operativa).

Si dimostra il seguente risultato
\begin{thm}[Forza di Lorentz]
    Data una carica puntifome $q$ che si muove con velocità $\vb{v}$ in un campo di induzione magnetica, questa carica risente di una forza
    \begin{equation}
        \label{eqn:f_lorentz}
        \vb{F}=q\vb{v}\cp\vb{B}
    \end{equation}
\end{thm}
\begin{proof}
    Si consideri un tratto di circuito di lunghezza $\dd{l}$, sezione $\dd{S}$ e -quindi- volume  
    $\dd{\tau}=\dd{l}\dd{S}$. Per il teorema \ref{thm:I_flusso_J}, la seconda legge di Laplace può essere riscritta come
    \[
        \dd{\vb{F}}=\vb{J}\dd{S}\dd{l}\cp\vb{B}=\vb{J}\dd{\tau}\cp\vb{B}
    \]
    Inoltre per la definizione di $\vb{J}$ si ha
    \[
        \vb{J}\dd{\tau}=nq\vb{v}_d\dd{\tau}=\dd{N}q\vb{v}_d
    \]
    dove $\dd{N}=n\dd{S}\dd{l}$ rappresenta il numero di portatori di carica nel tratto $\dd{l}$ del circuito. 
    Si ha quindi $\dd{\vb{F}}=\dd{N}q\vb{v}_d\cp\vb{B}$. Se si prende in esame una singola carica, 
    la velocità di deriva coincide con la velocità della carica. 
    Intregrando per ottenere la forza totale  e considerando che il numero totale di cariche è appunto $1$ si ottiene la tesi.
\end{proof}
La relazione di Lorentz è in realtà più generale rispetto alla seconda legge di Laplace: quest'ultima infatti vale
solo qualora $\vb{B}$ non vari significativamente nel tratto $\dd{l}$, mentre la legge di Lorentz è locale. Il prezzo
da pagare è una maggiore difficoltà nella misurazione del campo di induzione magnetica in quanto nel primo caso viene
usato come sonda un circuito percorso da corrente, nel secondo invece una carica in movimento.
\begin{cor}
    La forza di Lorentz non compie alcun lavoro.
\end{cor}
\begin{proof}
    Quando la carica è ferma, non risente della forza di Lorentz. Quando è in movimento, la forza di Lorentz è ortogonale alla velocità.
\end{proof}
Questa forza quindi cambia la direzione della carica in movimento ma non ne modifica la velocità (in modulo).

Grazie a quanto visto ora si può determinare l'unità di misura del campo di induzione magnetica, chiamata \textit{tesla}.
\[
    [B]=N\rec{Cm/s}=\frac{m}{m}N\rec{Cm/s}=\frac{Vs}{m^2}=T
\]
Il prodotto $Vs$ prende il nome di \textit{weber (Wb)}. 
Frequentemente, per il campo di induzione elettromagnetica si usa il \textit{gauss} $1T=10^4G$.


\section{Interazioni fra circuiti e campi magnetici}
Lo studio delle interazioni fra circuiti e campi magnetici è diviso in due categorie di fenomeni:
interazione fra un circuito percorso da corrente stazinaria con un campo magnetico esterno e
intrazione fra un circuito percorso da corrente stazionaria con un altro circuito percorso da corrente stazionaria.

\subsubsection{Interazione circuito-campo magnetico}
L'ipotesi fondamentale per lo studio di questi fenomeni è che il circuito in esame sia dotato di un generatore
di forza elettromotrice che mantenga costante nel tempo la corrente che attraversa il circuito.

Si vuole ora studiare il calcolo delle sollecitazioni meccaniche su una spira rigida percorsa da corrente $I$.
Si consideri quindi una spira $l$ percorsa da una corrente $I$ e immersa in un campo magnetico $\vb{B}$.
Si supponga che ogni tratto infinitesimo della spira $\dd{\vb{l}}$ compia uno spostamento infinitesimo
$\dd{\vb{s}}=\dd{\vb{s}}(\dd{\vb{l}})$, tale da portare la spira della configurazione $l'$.
Per far compiere alla spira questo spostamento senza che la sua energia cinetica vari è necessario applicare dall'esterno una forza
\[
    \dd{f}=-\dd{F}=-I\dd{\vb{l}}\cp\vb{B}
\]
Ovvero, è necessario compiere un lavoro, che risulta quindi in una variazione di energia potenziale
\[
    \dd{U}=\dd{L}=\oint\dd{\vb{f}}\vdot\dd{\vb{s}}=-\oint I\dd{\vb{l}}\cp\vb{B}\vdot\dd{\vb{s}}=
    -\oint I\dd{\vb{s}}\cp\dd{\vb{l}}\vdot\vb{B}=I\oint(\dd{\vb{l}}\cp\dd{\vb{s}})\vdot\vb{B}
\]
dove è stata usata la proprietà \eqref{app:eqn:vdot_cp}.
Il termine fra parentesi nell'ultimo membro della catena di uguaglianze rappresenta l'elemento di superficie $\dd{\vb{S}}$
della superficie laterale $\dd{\Sigma}$ di un solido le cui basi hanno spigoli $l$ ed $l'$.
L'integrale ottenuto è allora il flusso di $\vb{B}$ attraverso questa superficie laterale.
\[
    \dd{U}=\Phi_{\dd{\Sigma}}(\vb{B})
\]
Chiamando $\Sigma$ la base con spigolo $l$ e $\Sigma'$ la base con spigolo $l'$, siccome per il corollario \ref{cor:flusso_B}
il flusso di $\vb{B}$ attraverso una superficie chiusa deve essere nullo si ha
\[
    \Phi_{\dd{\Sigma}}=\Phi_\Sigma-\Phi_{\Sigma'}=-\dd{\Phi}
\]
Dove $\Phi_{\dd{\Sigma}}$ è il flusso entrante da $\Sigma$, mentre gli altri sono flussi uscenti dalle rispettive superfici.
Si ha quindi che a meno di una costante additiva arbitraria
\[
    U=-I\Phi(\vb{B})
\]
Dove il flusso è riferito alla superficie di una spira con contorno $l$ considerata positiva quando vede la corrente ruotare in sesno antiorario.
Ponendosi abbastanza lontano dalla spira in modo da poter considerare piccola la sua superficie
si può approssimare $\Phi(\vb{B})=\vb{B}\vdot\vb{S}$. In questo modo la relazione appena trovata diventa
\[
    U=-I\vb{S}\vdot\vb{B}=-\vb{m}\vdot\vb{B}
\]
Ma questa è un'espressione esattamente analoga a quanto trovato nel paragrafo sul dipolo elettrico \eqref{eqn:U_dipolo}.
Con passaggi identici si arriva allora a dimostrare
\begin{thm}
    Per una spira in un campo di induzione magnetica si ha che
    \begin{equation}
        \vb{F}=(\vb{m}\vdot\grad)\vb{B}
    \end{equation}
    \begin{equation}
        \vb{M}=\vb{m}\cp\vb{B}
    \end{equation}
\end{thm}
Si osservi quindi che esiste una completa analogia fra il dipolo in elettrostatica e la spira percorsa da
corrente in magnetostatica: sia la forma delle sollecitazioni meccaniche che quella del potenziale
(e di conseguenza del campo) prodotto
sono formalmente identiche a patto di usare $\vb{p}$ o $\vb{m}$ a seconda del contesto. Questo
risultato è detto \textit{teorema di equivalenza di Ampère} e sancisce la completa analogia
fra una spira da percorsa da corrente ed un "dipolo magnetico".\

Il calcolo del momento torcente può essere svolto in maniera immediata, e molto istruittiva, nel caso di una spira rettangolare
immersa in un campo magnetico costante.
\begin{example}
    Si supponga che il campo di induzione magnetica esterno $\vb{B}$ sia costante e che la spira sia rettangolare
    (con i lati 1 e 3 di lunghezza $b$ ed i lati 2 e 4 di lunghezza $a$), orientata in modo che i lati 2 e 4
    siano ortogonali a $\vb{B}$. I lati 1 e 3 sono soggetti ad una coppia di forze uguale ed opposta con braccio nullo;
    anche i lati 2 e 4 sono soggetti ad una coppia di forze uguale ed opposta,
    il braccio però vale $b\sin\theta$ con $\theta$ l'angolo fra $\vb{B}$ e la normale alla superficie $\vb{S}=ab\vu{n}$ della spira.
    Per la seconda legge di Laplace si ha $\vb{F}_2=\vb{F}_4=IaB$ per cui il momento meccanico sulla spira vale in modulo
    \[
        M=\frac{b}{2}F_2\sin\theta+\frac{b}{2}F_4\sin\theta=bIaB\sin\theta=ISB\sin\theta
    \]
    dove $\theta$ è l'angolo fra la forza ed il raggio-vettore che congiunge il punto di applicazione della forza al centro
    di massa.
    La relazione vettoriale è quindi
    \[
        \vb{M}=IS\vu{n}\cp\vb{B}=\vb{m}\cp\vb{B}
    \]
\end{example}



\subsubsection{Interazione circuito-circuito}
Dati due circuiti $l_1$ ed $l_2$ questi esercitano l'uno sull'altro delle azioni meccaniche in quanto
sorgenti di campi magnetici. La forza che $l_1$ esercita sull'elemento di filo $\dd{l_2}$ di
$l_2$ è\footnote{La distanza fra due punti presenti rispettivamente sul primo e sul secondo circuito
è stata indicata con $r_{12}$ anzichè con $\Delta r$}
\[
    \dd{\vb{F}_{12}}=I_2\dd{\vb{l}_2}\cp\vb{B}_1=I_2\dd{\vb{l}_2}\cp\oint_{l_1}\frac{\mu_0I_1}{4\pi}\frac{\dd{\vb{l}_1}\cp\vb{r}_{12}}{r^3_{12}}
\]
dove $\vb{r}_{12}$ è la distanza fra un punto di $l_1$ e l'elemento di filo $\dd{l_2}$.
Nel caso in cui i due circuiti siano rigidi si ha
\[
    \vb{F}_{12}=\oint_{l_2}I_2\dd{\vb{l}_2}\cp\oint_{l_1}\frac{\mu_0I_1}{4\pi}\frac{\dd{\vb{l}_1}\cp\vb{r}_{12}}{r^3_{12}}=
    \frac{\mu_0}{4\pi}I_1 I_2 \oint_{l_2}\oint_{l_1}\frac{\dd{\vb{l}_2}\cp\dd{\vb{l}_1}\cp\vb{r}_{12}}{r^3_{12}}
\]
Usando l'identità vettoriale  $\vb{a}\cp\vb{b}\cp\vb{c}=(\vb{a}\vdot\vb{c})\vb{b}-(\vb{b}\vdot\vb{a})\vb{c}$, si ottiene
\[
    \vb{F}_{12}=\frac{\mu_0}{4\pi}I_1 I_2 \Biggl[\oint_{l_2}\oint_{l_1}\frac{(\dd{\vb{l}_2}\vdot\vb{r}_{12})\dd{\vb{l}_1}}{r^3_{12}}-
    \oint_{l_2}\oint_{l_1}\frac{(\dd{\vb{l}_1}\vdot\dd{\vb{l}_2})\vb{r}_{12}}{r^3_{12}}\Biggr]
\]
Il primo di questi due integrali è nullo in quanto
\[
    \oint_{l_2}\oint_{l_1}\frac{(\dd{\vb{l}_2}\vdot\vb{r}_{12})\vb{l}_1}{r^3_{12}}=
    \oint_{l_1}\dd{\vb{l}_1}\oint_{l_2}\frac{(\dd{\vb{l}_2}\vdot\vb{r}_{12})}{r^3_{12}}=
    \oint_{l_1}\dd{\vb{l}_1}\oint_{l_2}\dd{\vb{l}_2}\vdot\grad(\rec{r_{12}})
\]
Ma la circuitazione di un campo conservativo è nulla.
Si ha allora
\begin{equation}
    \vb{F}_{12}=\frac{\mu_0}{4\pi}I_1 I_2 \oint_{l_2}\oint_{l_1}\frac{(\dd{\vb{l}_1}\vdot\dd{\vb{l}_2})\vb{r}_{12}}{r^3_{12}}
\end{equation}
L'equazione qui sopra mostra che scambiando gli indici si ottiene un cambio di segno, in accordo col terzo principio della dinamica.

Considerando due fili rettilinei infinitamente lunghi e paralleli, posti a distanza $\vb{r}_{12}$, si trova che
\[
    \dd{\vb{F}_{12}}=\frac{\mu_0}{2\pi r_{12}}I_1 I_2\vb{r}_{12} sgn(\dd{l_1}\vdot\dd{l_2})
\]
ovvero la forza è attrattiva o repulsiva in base al fatto che il verso delle correnti sia concorde o discorde.
Inoltre questa relazione fornisce una definizione operativa dell'ampère: si dice che due fili lunghi e sottili
sono attraversati da una corrente di $1A$ quando, posti ad un metro di distanza, risentono reciprocamente di
una forza pari a $\mu_0/2\pi$ per metro di filo.


\section{Effetto Hall}
In un conduttore percorso da corrente elettrica e immerso in un campo di induzione magnetica
esterno ortogonale alle linee di corrente, si genera tra bordi opposti una differenza di
potenziale
\begin{equation}
    \Delta V_H = v_d B a = R_H \frac{IB}{b} \quad\quad R_H=\rec{nq}
    \label{eqn:hall}
\end{equation}
dove $b$ rappresenta lo spessore del conduttore in direzione parallela a $\vb{B}$, a lo spessore
in direzione ortogonale e $n$ la densità
di portatori di carica. Questo effetto è chiamato \textit{Effetto Hall} e la costante $R_H$ è detta
\textit{costante di Hall}, con un valore tipico $\abs{R_H}\simeq 10^{-11}m^3/C$.

Per spiegare il fenomeno, si consideri una sbarretta conduttrice a forma di parallelepipedo la cui
sezione abbia base di lati $a$ e $b$ immersa in un campo magentico $\vb{B}$ parallelo al lato $b$.
Sia questa sbarretta percorsa da corrente $I$ distributia uniformemente sulla sezione. I portatori
di carica, sottoposti alla forza di Lorentz\footnote{Si osservi che il verso della forza di Lorentz
non dipende dal segno della carica dei portatori in quanto questo dipende dal segno del prodotto $qv_d$
che indipendente dalla carica.}, verranno deviati verso una delle facce della superficie
laterale della sbarretta sulla quale si avrà quindi un accumulo di carica. Come conseguenza di questo
accumulo, si deve avere sulla faccia opposta un accumulo della carica opposta.
Il campo elettrostatico $E_s$ generato da questi accumuli si oppone alla forza di Lorentz fino al punto di
saturazione $qE_s=qv_d B$, raggiunto il quale il moto dei portatori torna ad essere quello tipico.
Tra le due facce della sbarretta si genera un potenziale $\Delta V_H=E_sa=v_dBa$.
Indicando con $J$ la densità di corrente uniforme per ipotesi, si ha $I=Jab=nqv_dab$
Sostituendo $v_d$ nell'equazione del potenziale di Hall appena trovata si ha anche la seconda
espressione del potenziale.

Tipicamente la costante di Hall è negativa e questo conferma che i portatori di carica siano tipicamente
elettroni. Inoltre invertendo la formula si ricava che il numero di elettroni liberi per ogni atomo è di
$1-2$

