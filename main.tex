\documentclass[a4paper]{book}
\usepackage[T1]{fontenc}
\usepackage[utf8]{inputenc}
\usepackage[italian]{babel}
\usepackage{physics}
\usepackage{fourier}
\usepackage{amsthm}

%------------------------------------------------------------

\theoremstyle{plain}
\newtheorem{thm}{Teorema}[section]
\newtheorem{lemma}[thm]{Lemma}
\newtheorem{cor}[thm]{Corollario}
\newtheorem{obs}[thm]{Osservazione}
\newtheorem{obses}[thm]{Osservazione sperimentale}

\theoremstyle{definition}
\newtheorem{defn}[thm]{Definizione}
\newtheorem{example}{Esempio}

%------------------------------------------------------------

\renewcommand{\epsilon}{\varepsilon}
\newcommand{\rec}[1]{\frac{1}{#1}}

%------------------------------------------------------------

\begin{document}
\author{Vallenzasca Davdide}
\title{Dispense di Elettromagnetismo}
\maketitle

\newpage

\tableofcontents

\newpage

\chapter* {Equazioni di Maxwell}
\subsubsection{Equazioni di Maxwell nel vuoto}
\begin{minipage}[t]{0.5\textwidth}
\[
\div{\vb{E}}=\frac{\rho}{\epsilon_0}
\]

\[
\curl{\vb{E}}=-\pdv{\vb{B}}{t}
\]

\end{minipage}
\begin{minipage}[t]{0.5\textwidth}
\[
\div{\vb{B}}=0
\]

\[
\curl{\vb{B}}=\mu_0\vb{J}+\epsilon_0\mu_0\pdv{\vb{E}}{t}
\]

\end{minipage}

\subsubsection{Equazioni di Maxwell in mezzi omogenei ed isotropi}
\begin{minipage}[t]{0.5\textwidth}
\[
\div{\vb{D}}=\rho
\]

\[
\curl{\vb{E}}=-\pdv{\vb{B}}{t}
\]

\end{minipage}
\begin{minipage}[t]{0.5\textwidth}
\[
\div{\vb{B}}=0
\]

\[
\curl{\vb{H}}=\vb{J}+\pdv{\vb{D}}{t}
\]

\end{minipage}

\[
\vb{D}=\epsilon_0\vb{E}+\vb{P}
\]

\[
\vb{H}=\rec{\mu_0}\vb{B}-\vb{M}
\]


\newpage
\part{Elettrostatica}
\chapter{Elettrostatica nel vuoto}
\begin{obses}[Conservazione della carica]
    Si osserva sperimentalmente che la somma algebrica delle cariche elettriche si conserva nel tempo.
\end{obses}

\section{Legge di Coulomb}
\begin{obses}[Legge di Coulomb]
  Prese due cariche $q_1$ e $q_2$ nel vuoto poste a distanza $\vb{r}$, la forza $\vb{f_{21}}$ che $q_2$ subisce ad opera di $q_1$ è
  \begin{equation}
      \vb{f}_{21}=\frac{1}{4\pi\epsilon_0}\frac{q_1\,q_2}{r^2}\vu{r}_{21}
  \end{equation}
\end{obses}
  Detta $\vb{r}_1$ la posizione di $q_1$ e $\vb{r}_2$ la posizione di $q_2$ si ha $\vb{r}_{21}=\vb{r}_2-\vb{r}_1$.
\begin{obses}[Principio di sovrapposizione]
    Sperimentalmente la forza di Coulomb subita dalla carica $q$ è pari alla somma vettoriale delle forze eserciate dalle $n$ cariche $Q_i \quad i=1 \text{...} n$.
\end{obses}

Ovviamente vale il terzo principio di Newton per cui $\vb{f}_{21}=-\vb{f}_{12}$.


\section{Il campo elettrico}
Prese due cariche $q$ e $Q$ la quantità $\vb{f}/q$ è indipendente da $q$.
\begin{defn}
    Si definisce campo elettrico $\vb{E}$ generato dalla carica Q il rapporto $\vb{f}/q$. Q viene detta sorgente del
    campo; q viene detta carica di prova.
\end{defn}
Esplicitamente si ha
\begin{equation}
    \vb{E}(\vb{r})=\frac{1}{4\pi\epsilon_0}\frac{q}{r^2}\vu{r}
    \label{eqn:E}
\end{equation}
Il modulo di $\vb{E}$ viene detto \textit{intensità} del campo elettrico.
È importante osservare che se le sorgenti sono disposte su un corpo esteso, la presenza di una carica di prova può
modificarne la distribuzione. Convenzionalmente allora si suole definire il campo elettrico come
\[
    \vb{E}=\lim_{q\to 0}\frac{\vb{f}}{q}
\]
Questa definizione è solo formale: essendo la carica elettrica quantizzata infatti, non si può parlare di passagio
al limite in senso matematico.
\begin{obs}
    Date $n$ sorgenti, il principio di sovrapposizione per la forza di Coulomb si estende banalmente al campo elettrico:
    \[
        \vb{E}(\vb{r})=\rec{4\pi\epsilon_0}\sum_{i=1}^n \frac{Q_i}{\abs{\vb{r}-\vb{r}_i}^3}(\vb{r}-\vb{r}_i)
    \]
\end{obs}

Quando si ha a che fare con un elevato numero di cariche puntiformi risulta utile introdurre il concetto di \textit
{distribuzione continua di carica} descritta da una \textit{densità di carica}:
\begin{defn}
    Si definisce \textit{densità di carica} una funzione $\omega(x_1,x_2\ldots x_n)$ tale che $\dd{q}=\omega\,\dd{\mu}$ con
    $\dd{\mu}=\dd{x_1}\,\dd{x_2}\,\ldots\,\dd{x_n}$
\end{defn}
Si ha allora
\begin{equation}
    \label{eqn:E_distribuzione}
    \vb{E}(\vb{r})=\frac{1}{4\pi\epsilon_0}\,\int\frac{\omega(\vb{r'})}{\abs{\vb{r}-\vb{r'}}^3}(\vb{r}-\vb{r'})d\mu'
\end{equation}
Nella pratica si parla di densità spaziale $\rho$, densità superficiale $\sigma$, densità lineare $\lambda$
(rispettivamente $n=3$, $n=2$, $n=1$).


\section{Il teorema di Gauss e la prima equazione di Maxwell}
\begin{thm}
    Data una qualunque superficie chiusa $S$ nel vuoto, il flusso del campo elettrostatico $\Phi_S(\vb{E})$
    dipende esclusivamente dalle cariche interne alla superficie secondo la legge
    \[
        \Phi_S(\vb{E})=\frac{Q_{TOT}^{int}}{\epsilon_0}
    \]
\end{thm}
\begin{proof}
    Si consideri il caso in cui all'interno della superficie S sia presente solo una carica puntiforme Q. Il flusso elementare è
    allora dato da
    \[
        d\Phi_S(\vb{E})=\vb{E}\vdot \dd{\vb{S}}=\frac{1}{4\pi\epsilon_0}\frac{Q}{r^2}\vu{r}\vdot\vu{n}\,\dd{S}
    \]
    Il prodotto scalare rappresenta la proiezione di $\vu{n}$ su $\vu{r}$, ovvero la proiezione di $d\vb{S}$ sull'elemento di
    superficie di una sfera di raggio r e centro in Q, ma allora il rapporto fra questa proiezione ed $r^2$ è per definizione
    l'angolo solido sotteso alla superficie infinitesima $d\Omega$. In conclusione si ha
    \[
        d\Phi_S(\vb{E})=\frac{Q}{4\pi\epsilon_0}\dd{\Omega}
    \]
    Il flusso totale è allora
    \[
        \Phi_S(\vb{E})=\int \dd{\Phi_S(\vb{E})}=\frac{Q}{4\pi\epsilon_0}\int\dd{\Omega}=\frac{Q}{4\pi\epsilon_0}4\pi=\frac{Q}{\epsilon_0}
    \]
    Per il principio di sovrapposizione e la linearità dell'integrale si arriva facilmente alla tesi.

    Si consideri ora il caso in cui sia presente una carica puntiforme $Q_e$ esterna alla superficie.
    Si considerino i coni che danno origine agli angoli solidi $d\Omega$ in ogni direzione dello spazio:
    di questi si è interessati solo a quelli che intercettano S. Il flusso di campo elettrico attraverso le due superfici
    originate dall'intersezione fra i coni e S vale in modulo $Q_e\,d\Omega/4\pi\epsilon_0 $per entrambe le superfici, con
    segno però opposto a causa del fatto che la proiezione avviene in un caso con un angolo inferiore a $\pi/2$ e nell'altro
    caso con un angolo superiore a $\pi/2$. Il contributo complessivo è quindi nullo.
\end{proof}

\begin{cor}
    Nel caso in cui si abbia una distribuzione continua di carica, invece che un insieme di cariche puntiformi,
    vale ancora la tesi del teorema di Gauss e il flusso è dato da
    \[
        \Phi_S(\vb{E})=\rec{\epsilon_0}\int \omega\,\dd{\mu}
    \]
\end{cor}

Come è evidente dalla dimostrazione, il teorema permette di calcolare il flusso indipendentemente dalla forma della superficie o
dalla posizione occupata dalle singole cariche interne. Si rivela dunque uno strumento estremamente utile per calcolare, noto
il campo elettrico, le cariche presenti in una porzione di spazio o viceversa, nota la distribuzione di carica, di
ricavare il campo elettrico. Si presenta tuttavia come una diretta conseguenza della legge di Coulomb\footnote{Si
osservi infatti come il punto chiave della dimostrazione, ovvero l'identificazione dell'angolo solido, dipenda direttamente dal
fatto che la leggge di Coulomb vada come $1/r^2$.} e non aggiunge quindi informazioni rispetto a quelle già note.
Il teorema di Gauss fornisce inoltre alcune indicazioni utili alla rappresentazione del campo elettrico.
\begin{defn}[Linea di forza del campo]
    Si definisce \textit{linea di forza del campo} una linea che è in ogni suo punto tangente alla direzione del campo.
\end{defn}
\begin{defn}[Tubo di flusso]
    Presa una linea chiusa, in ogni suo punto passa una linea di forza del campo. La superficie tubolare definita da queste
    linee di forza è detta tubo di flusso.
\end{defn}
Nel caso del campo elettrico $\vb{E}$ si consideri una porzione di tubo di flusso compresa fra due superfici $S_1$
ed $S_2$: assieme alla superficie laterale queste due superfici definiscono una superficie chiusa $S$. Per il teorema di Gauss,
se non sono presenti cariche interne ad S, il flusso totale $\Phi_S^{tot}(\vb{E})=0$. Inoltre, per definizione di linea
di forza il flusso attraverso la superficie laterale del tubo è nullo.
Si ha quindi, tenendo conto della diversa orientazione di $S_1$ ed $S_2$ rispetto a $S$, $\Phi_{S_1}(\vb{E})=\Phi_{S_2}(\vb{E})$.
Siccome quanto detto è indipendente dalla porzione di tubo scelta si ha che il flusso del campo elettrico è una caratteristica del
tubo di flusso. Da questo segue che la sezione del tubo di flusso si riduce nelle zone in cui il campo elettrico è maggiore
, ovvero le linee di forza si addensano, e viceversa.

Risulta conveniente espriere in forma locale il teorema di Gauss.
\begin{cor}[Prima equazione di Maxwell]
    Nell'ipotesi che sia possibile applicare il teorema della divergenza al campo elettrico
    \begin{equation}
        \label{eqn:prima_maxwell}
        \div{\vb{E}}=\frac{1}{\epsilon_0}\rho
    \end{equation}
\end{cor}
\begin{proof}
Applicando il teorema della divergenza alla legge di Gauss si ricava che
\[
    \Phi_S(\vb{E})=\int_\tau \div{\vb{E}} \dd{\tau}=\frac{1}{\epsilon_0}\int_\tau \rho\, \dd{\tau}
\]
Siccome il teorema di Gauss vale qualunque sia la superficie di integrazione, la formula appena ottenuta deve valere qualunque sia
il volume di intergazione, il che implica l'uguaglianza delle integrande, cioè la tesi.
\end{proof}

La limitazione di questa equazione rispetto a quella fornita dal teorema di Gauss è situata nel fatto che viene richiesta la
validità del teorema della divergenza, ovvero che il campo elettrico sia derivabile in ogni punto del dominio considerato. Il
vantaggio risiede però nel fatto che il teorema di Gauss lega la distribuzione di carica interna alla superficie col campo
sulla superficie (e quindi nel caso non stazionario, una variazione della distribuzione di carica non si traduce in una
simultanea variazione del campo) mentre l'equazione di Maxwell lega grandezze calcolate nella stessa posizione e può
dunque essere immediatamente generalizzata al caso non stazionario.


\section{La terza equazione di Maxwell ed il potenziale elettrico}
\label{par:potenziale_elettrico}
Si vuole studiare la forma differenziale ottenuta moltiplicando scalarmente la \ref{eqn:E} con l'elemento di linea $d\vb{l}$
\begin{obs}
    La forma differenziale $\vb{E}\cdot d\vb{l}$ è esatta.
\end{obs}
\begin{proof}
    Per l'identità \eqref{app:eqn:grad_r}
    \[
        \vb{E}\vdot \dd{\vb{l}}=
        \frac{Q}{4\pi\epsilon_0}\frac{\vu{r}}{\abs{\vb{r}-\vb{r}'}^2}\vdot \dd{\vb{l}}=
        -\frac{Q}{4\pi\epsilon_0}\grad{\rec{\abs{\vb{r}-\vb{r}'}}}\vdot \dd{\vb{l}}
    \]
    Ma $\grad{f}\vdot\dd{\vb{l}}=\dd{f}$: $\vb{E}\cdot d\vb{l}$ è il differenziale di una funzione
    ed è quindi una forma differenziale esatta.
\end{proof}

\begin{cor}[Terza equazione di Maxwell]
    \begin{equation}
        \label{eqn:terza_maxwell}
        \curl{\vb{E}}=0
    \end{equation}
\end{cor}
\begin{proof}
    La dimostrazione è immediata ricordando che una formula differenziale esatta è anche chiusa.
\end{proof}

Risulta spontaneo a questo punto dare la seguente definizione:
\begin{defn}
    Si definisce \textit{potenziale elettrico} generato dalla carica puntiforme Q la funzione scalare $V(\vb{r})$ tale che
    $-\dd{V}=\vb{E}\vdot\dd{\vb{l}}$, o in altre parole
    \begin{equation}
        \int_A^B\vb{E}\cdot d\vb{l}=V(A)-V(B)
    \end{equation}
\end{defn}
Integrando l'espressione del campo elettrico generato da carica puntiforme
si ottiene un'espressione esplicita per il potenziale:
\[
    V(\vb{r})=\frac{1}{4\pi\epsilon_0}\frac{Q}{\abs{\vb{r}}} + C
\]

La costante C è arbitraria. Solitamente si richiede che il potenziale all'infinito sia nullo, da cui segue $C=0$.
I risultati ottenuti generalizzando il campo elettrico generato da carica puntiforme al campo elettrico generato da distribuzioni
di carica continue o discrete possono essere estesi al potenziale. Emerge l'utilità del potenziale: note le distribuzioni di carica
è molto più semplice, piuttosto che calcolare direttamente il campo elettrico (che è una funzione vettoriale),
calcolare il potenziale e poi ricavare il campo facendone il gradiente.

Quando si lavora con distribuzioni di carica, si considera l'elemento di volume $d\tau'$ il cui vettore posizione è
$\vb{r'}$. Si ha allora la forma differenziale del potenziale
\[
    dV=\frac{1}{4\pi\epsilon_0}\frac{\rho\,d\tau'}{\abs{\vb{r}-\vb{r'}}}+dC
\]
Si impone ora che il potenziale sia nullo in una posizione $\vb{r_0}$, ovvero
\[
    \begin{split}
        & 0=\frac{1}{4\pi\epsilon_0}\frac{\rho\,d\tau'}{\abs{\vb{r_0}-\vb{r'}}}+dC \\
        & dC=-\frac{1}{4\pi\epsilon_0}\frac{\rho\,d\tau'}{\abs{\vb{r_0}-\vb{r'}}}
    \end{split}
\]
Per cui si ottiene
\begin{equation}
    \label{eqn:V_rho}
    V(\vb{r})=\frac{1}{4\pi\epsilon_0}\int\rho\,d\tau'\Biggl(\frac{1}{\abs{\vb{r}-\vb{r'}}} - \frac{1}{\abs{\vb{r_0}-\vb{r'}}}\Biggr)
\end{equation}


\section{Energia del campo elettrico}
Un sistema di cariche che interagiscono reciprocamente possiede una certa \textit{energia elettrostatica di interazione}
dovuta al lavoro necessario per portare le cariche nella configurazione
desiderata partendo da una condizione di assenza di interazione fra le cariche.\\
\begin{thm}[Energia elettrostatica per un sistema discreto di cariche puntiformi]
    L'energia elettrostatica di interazione per un sistema discreto di $N$ cariche puntiformi vale
    \begin{equation}
        \label{eqn:UE_puntiformi}
        U=\rec{2}\sum_{j\neq i}^N\frac{q_iq_j}{4\pi\epsilon_0 r_{ij}}
    \end{equation}
\end{thm}
\begin{proof}
    Immaginiamo che inizialmente tutte le cariche siano all'infinito, in modo che non sentano
    l'interazione reciproca. Il posizionamento nello spazio della prima carica viene effettuato compiendo lavoro nullo,
    perchè inizialmente non è presente nessun campo elettrico.
    Ora è presente nello spazio il campo $\vb{E_1}$ prodotto dalla prima carica $q_1$.
    Per portare quindi la seconda carica $q_2$ dall'infinito fino ad una distanza $r_{12}$ dalla prima carica, è necessario compiere in lavoro
    \[
        L_2=\int_{-\infty}^{r_{12}}-q_2\vb{E_1}\vdot d\vb{l}=-\frac{q_1q_2}{4\pi\epsilon_0}\int_{-\infty}^{r_{12}}\frac{dr}{r^2}
        =\frac{q_1q_2}{4\pi\epsilon_0}\rec{r_{12}}
    \]
    Dove il segno meno deriva dal fatto che il lavoro è quello associato alla forza esterna che
    sposta la carica opponendosi alla forza di coulomb.
    Il lavoro necessario per spostare la terza carica $q_3$ ad una distanza $r_{13}$ da $q_1$
    e ad una distanza $r_{23}$ da $q_2$,  per il principio di sovrapposizione è uguale al lavoro necessario
    a spostare $q_3$ nel campo generato dalla prima carica e nel campo generato dalla seconda. Si ha quindi
    \[
        L_3=\frac{q_1q_3}{4\pi\epsilon_0}\rec{r_{13}}+\frac{q_2q_3}{4\pi\epsilon_0}\rec{r_{23}}
    \]
    L'energia posseduta dal sistema di tre cariche è quindi
    \[
        U=L_1+L_2+L_3=0+\frac{q_1q_2}{4\pi\epsilon_0}\rec{r_{12}}+\Biggl(\frac{q_1q_3}{4\pi\epsilon_0}\rec{r_{13}}
        +\frac{q_2q_3}{4\pi\epsilon_0}\rec{r_{23}}\Biggr)=\rec{2}\sum_{j\neq i}^3\frac{q_iq_j}{4\pi\epsilon_0 r_{ij}}
    \]
    dove il termine $1/2$ è necessario in quanto la sommatoria comprende sia l'interazione di $q_i$ con $q_j$
    che l'interazione di $q_j$ con $q_i$. Proseguendo col ragionamento, si ottiene la formula per $N$ cariche.
\end{proof}
Una comoda riscrittura della formula appena dimostrata si ottiene considerando
\[
    U=\rec{2}\sum_{i=1}^Nq_i\sum_{i\neq j}^N\frac{q_j}{4\pi\epsilon_0}\rec{r_{ij}}
\]
E osservando che la seconda sommatoria corrisponde al potenziale $V_i$ sentito dalla carica
i-esima dovuto a tutte le altre $N-1$ cariche e quindi si ottiene
\begin{equation}
    U=\rec{2}\sum_{i=1}^N q_i V_i
\end{equation}
L'utilità di questa riscrittura è dovuta al fatto che consente di generalizzare quanto visto
al caso di una distribuzione continua di carica contenuta in un insieme $A$
\begin{equation}
    \label{eqn:energia_distribuzione}
    U=\rec{2}\int_{A} \omega V \dd{\mu}
\end{equation}
Le equazioni trovate esprimono l'energia potenziale elettrica in termini di posizione reciproca delle cariche, mettendo
in evidenza quindi l'interazione fra queste mediante forza di Coulomb. Un altro approccio consiste invece nell'
enfatizzare il ruolo del campo elettrico interpretando l'energia potenziale elettrica come quell'energia
"immagazzinata" dal campo elettrico -che è un modo altisonante per dire: l'energia necessaria a generare
il campo. Per farlo, si sfrutta la prima equazione di Maxwell.
\begin{thm}
    Data una distribuzione di carica $\rho$ l'energia elettrostatica vale
    \begin{equation}
        \label{eqn:UE}
        U=\frac{\epsilon_0}{2}\int_S V\vb{E}\vdot d\vb{S} + \frac{\epsilon_0}{2}\int_V E^2 d\tau
    \end{equation}
\end{thm}
\begin{proof}
    Tenuto conto della prima equazione di Maxwell \eqref{eqn:prima_maxwell} e della \eqref{eqn:energia_distribuzione} si ha che
    \[
        U=\rec{2}\int_{V} \rho V d\tau=\frac{\epsilon_0}{2}\int_V \div{\vb{E}} V d\tau
    \]
    Applicando la \eqref{app:eqn:div_scalare_vettore}
    ricordando che il gradiente del potenziale è il campo elettrico cambiato di segno, si ottiene
    \[
        U=\frac{\epsilon_0}{2}\int_V \div{(V\vb{E})}d\tau + \frac{\epsilon_0}{2}\int_V E^2 d\tau
    \]
    Per il teorema della divergenza infine si ha
    \[
        \int_V \div{(V\vb{E})}d\tau=\int_S V\vb{E}\vdot d\vb{S}
    \]
    ottenendo la tesi.
\end{proof}

\begin{cor}
    Se si prende in considerazione tutto lo spazio, allora l'energia vale
    \[
        U=\int ud\tau \quad\quad\quad \text{ con } u=\frac{\epsilon_0 E^2}{2}
    \]
\end{cor}
\begin{proof}
    Si considerino i risultati del teorema precedente: fissata la distribuzione di carica in una regione finita di spazio,
    all'allargarsi del volume di integrazione l'integrale di volume aumenta e di paripasso l'intergale di superficie
    diventa trascurabile. Si ha quindi la tesi.
\end{proof}
La $u$ che compare nel risultato del corollario prende il nome di \textit{densità di energia del campo elettrostatico}.\\
Il risultato a cui si giunge sembra in apparenza contraddittorio: $u$ è chiaramente positiva e di conseguenza lo è il
suo integrale, mentre non è difficile immaginare un caso in cui \eqref{eqn:UE_puntiformi} sia negativa -basta
prendere un sistema costituito da due cariche di segno opposto. La differenza risiede nel fatto che la
\eqref{eqn:energia_distribuzione} e tutte le equazioni che ne conseguono contengono un termine di auto-energia,
ovvero l'energia necessaria alla costruzione della distribuzione di carica. Si consideri infatti
il sistema già citato costituito da due cariche puntiformi elementari. L'energia fornita da \eqref{eqn:UE_puntiformi}
è
\[
    U=\rec{4\pi\epsilon_0} \frac{q_1 q_2}{r_{12}}
\]
Si vuole ora calcolare l'energia passando per la densità di energia.
\[
    u=\rec{32\pi^2\epsilon_0}\Biggl[\frac{q_1^2}{\abs{\vb{r}-\vb{r}_1}} + \frac{q_2^2}{\abs{\vb{r}-\vb{r}_2}}+
    2q_1 q_2 \frac{(\vb{r}-\vb{r}_1)\vdot(\vb{r}-\vb{r}_2)}{\abs{\vb{r}-\vb{r}_1}^3\abs{\vb{r}-\vb{r}_2}^3} \Biggr]
\]
Si integri quanto ottenuto. In particolare, ci si focalizzi sul terzo termine e si introduca il cambiamento di variabile
$\vb*{\rho}=(\vb{r}-\vb{r}_1)/\abs{\vb{r}_1-\vb{r}_2}$, $\dd{\vb{\tau}}=\abs{\vb{r}-\vb{r}_1}^3\dd{\vb*{\rho}}$. Chiamando
$\vu{r}=(\vb{r}_1-\vb{r}_2)/\abs{\vb{r}_1-\vb{r}_2}$ si ottiene
\[
    \frac{ q_1 q_2}{16\pi^2\epsilon_0}
    \int\frac{(\vb{r}-\vb{r}_1)\vdot(\vb{r}-\vb{r}_2)}{\abs{\vb{r}-\vb{r}_1}^3\abs{\vb{r}-\vb{r}_2}^3}\dd{\tau}=
    \frac{ q_1 q_2}{16\pi^2\epsilon_0}
    \int\rec{\abs{\vb{r}_1-\vb{r}_2}^3}\frac{\vb*{\rho}\vdot(\vb*{\rho}+\vu{r})}{\rho^3\abs{\vb*{\rho}+\vu{r}}^3}\dd{\vb*{\rho}}
\]
Usando la \eqref{app:eqn:grad_r} e di seguito la \eqref{app:eqn:div_scalare_vettore} l'integrale assume la forma
\[
    \frac{ q_1 q_2}{16\pi^2\epsilon_0}\rec{\abs{\vb{r}_1-\vb{r}_2}^3}
    \Biggl[-\int\div{\frac{\vb*{\rho}}{\rho^3\abs{\vb*{\rho}+\vu{r}}^3}\dd{\vb*{\rho}}}
    -\int\rec{\abs{\vb*{\rho}+\vu{r}}^3}{\div{\frac{\vb*{\rho}}{\rho^3}}\dd{\vb*{\rho}}}\Biggr]
\]
Per il teorema della divergenza, il primo è un integrale sulla frontiera del volume. Siccome il volume di integrazione
è tutto lo spazio, l'integrale sulla superficie va a $0$. Per quanto concerne il secondo integrale, usando nuovamente
la \eqref{app:eqn:grad_r}
\[
    \int\rec{\abs{\vb*{\rho}+\vu{r}}^3}\laplacian{\rec{\rho}}\dd{\vb*{\rho}}=
    \int\rec{\abs{\vb*{\rho}+\vu{r}}^3}4\pi\delta(\vb*{\rho})\dd{\vb*{\rho}}=
    \frac{4\pi}{\abs{\vu{r}}^3}=4\pi
\]
dove è stato fatto uso anche della \eqref{app:eqn:laplacian_r}. In conclusione, mettendo tutto insieme si ottiene che
questo terzo termine rappresenta l'energia di interazione mentre gli altri due termini rappresentano l'auto-energia.
Per quanto visto, siccome l'energia di interazione può essere sia positiva che negativa e l'energia
totale è positiva, l'auto-energia deve essere necessariamente positiva.


\section{Dipoli}
\begin{defn}[dipolo elettrico]
    Si definisce dipolo elettrico un sistema costituito da due cariche q uguali ed opposte poste
    ad una distanza $\delta$ fissa. Il vettore che collega le due cariche è chiamato $\vb*{\delta}$,
    ha intensità $\abs{\vb*{\delta}}=\delta$ ed è orientato dalla carica negativa alla positiva.
\end{defn}

Il dipolo è caratterizzato dal \textit{momento di dipolo}
\begin{equation}
    \label{eqn:momento_di_dipolo}
    \vb{p}=q\vb*{\delta}
\end{equation}
Nel caso di distribuzioni di carica, la definizione può essre estesa
\begin{equation}
    \label{defn:p_rho}
    \vb{p}=\int_\tau \rho\vb{r'} \dd{\tau}
\end{equation}

\begin{thm}
    A distanze molto maggiori delle dimensioni lineari del dipolo ($r>>\delta$) il potenziale generato dal dipolo vale
    \begin{equation}
        \label{eqn:V_dipolo}
        V(\vb{r})=\frac{1}{4\pi\epsilon_0}\frac{\vb{p}\vdot\vb{r}}{r^3}
    \end{equation}
\end{thm}
\begin{proof}
    Definiti:\\
    $r^+$ la distanza della carica positiva dall'osservatore; \\
    $r^-$ la distanza della carica negativa dall'osservatore\\
    Si ha che
    \[
        V(\vb{r})=\frac{q}{4\pi\epsilon_0}\Bigl(\frac{1}{r^+}-\frac{1}{r^-}\Bigr)=\frac{q}{4\pi\epsilon_0}\frac{r^--r^+}{r^+\,r^-}
    \]
    Nell'ipotesi in cui $\delta<<r$ valgono le segueti approssimazioni:
    \[
        \begin{split}
            & r^+\,r^- \cong r^2\\
            & r^+-r^- \cong \delta\cos\alpha
        \end{split}
    \]
    con $\alpha$ l'angolo che il raggio vettore $\vb{r}$ forma con $\vb{p}$.
    Segue che il potenziale può essere scritto come
    \[
        V(\vb{r})=\frac{q\delta\cos\alpha}{4\pi\epsilon_0\, r^2}=\frac{p\cos\alpha}{4\pi\epsilon_0\, r^2}=
        \frac{p\,r\,\cos\alpha}{4\pi\epsilon_0\, r^3}=\frac{\vb{p}\vdot \vb{r}}{4\pi\epsilon_0\, r^3}
    \]
\end{proof}
Si osservi come il potenziale del dipolo decresca come $1/r^2$
\begin{cor}
    Il campo elettrico generato dal dipolo giace esclusivamente nel piano pr ed è dotato di una componente radiale
    \[
        E_r=\frac{1}{4\pi\epsilon_0}\frac{2p\cos\theta}{r^3}
    \]
    e di una componente angolare
    \[
        E_\theta=\frac{1}{4\pi\epsilon_0}\frac{p\sin\theta}{r^3}
    \]
\end{cor}
\begin{proof}
    Noto il potenziale è possibile ricavare il campo elettrico semplicemente facendone il gradiente.
    In questo caso la soluzione più immediata consiste nell'applicare il gradiente in coordinate polari.
    Si sceglie l'asse $z$ coincidente con la direzione di $\vb{p}$ e l'angolo $\theta$ coincidente con l'angolo $\alpha$ definito sopra.
    In questo modo il potenziale può essere scritto come
    \[
        V(r,\theta,\phi)=\frac{1}{4\pi\epsilon_0}\frac{p\cos\theta}{r^2}
    \]
    Applicando il gradiente in coordinate polari, si ottiene la tesi.
\end{proof}
Le linee di forza partono dalla carica positiva e si chiudono sulla carica negativa.

Ci si pone ora come obiettivo quello di descrivere la dinamica di un dipolo immerso in un campo elettrico.
Le forze in gioco sono conservative, quindi possono essere dedotte da una funzione potenziale $U$.
Il problema di descrivere la dinamica del dipolo si traduce nel problema di determinare il potenziale delle forze.
\begin{lemma}
    Si consideri un dipolo con momento $\vb{p}$ immerso in un campo $\vb{E}$ uniforme, l'energia potenziale del dipolo vale
    \begin{equation}
        \label{eqn:U_dipolo}
        U=-\vb{E}\vdot\vb{p}
    \end{equation}
\end{lemma}
\begin{proof}
    Chiamando $U_+$ l'energia potenziale dovuta alla carica positiva del dipolo
    e $U_-$ l'energia potenziale dovuta alla carica negativa si ottiene
    \[
        U=U_++U_-=qV(\vb{r}+\vb{\delta})-qV(\vb{r})
    \]
    Ponendo ora $V(\vb{r}+\vb{\delta})=V(\vb{r})+\dd{V}$ si ottiene

    \[
        U=qV(\vb{r})+q\dd{V}-qV(\vb{r})=q\dd{V}
    \]
    Il differenziale del potenziale può essere scritto, in quanto forma differenziale,
    come $\grad{V}\vdot\dd{\vb{l}}$ e da ciò segue, osservando che $\dd{\vb{l}}=\vb*{\delta}$
    \[
        U=q\grad{V}\vdot\vb*{\delta}=\grad{V}\vdot q\vb*{\delta}
    \]
    ottenendo la tesi.
\end{proof}

\begin{thm}
    Per un dipolo si ha che
    \begin{equation}
        \vb{F}=-(\vb{p}\vdot\grad)\vb{E}
    \end{equation}
    \begin{equation}
        \vb{M}=\vb{p}\cp\vb{E}
    \end{equation}
\end{thm}
\begin{proof}
    Dalla dinamica si ha che il lavoro elementare vale
    \[
        \dd{L}=\vb{F}\vdot\dd{\vb{l}}+\vb{M}\vdot\dd{\vb*{\theta}}
    \]
    dove $\dd{\vb*{\theta}}=\vu{n}\dd{\theta}$, con $\vu{n}$ il versore dell'asse di rotazione.
    D'altra parte è vero che
    \[
        -\dd{L}=\dd{U}=\pdv{U}{l}\dd{l}+\pdv{U}{\theta}\dd{\theta}
    \]
    Per confronto si ha quindi che
    \[
        \begin{cases}
            & \vb{F}\vdot\dd{\vb{l}}=-\pdv{U}{l}\dd{l} \\
            & \vb{M}\vdot\dd{\vb*{\theta}}=-\pdv{U}{\theta}\dd{\theta}
        \end{cases}
    \]
    Per il lemma $U=-\vb{E}\vdot\vb{p}=-Ep\cos\theta$, dove la dipendenza dalla posizione la si ha
    solo in $E$ mentre la dipendenza dall'angolo la si ha solo nel coseno.
    Si ha quindi che, ricordando che $\vb{p}$ non dipende dalla posizione
    \[
        \begin{cases}
            & \vb{F}\vdot\dd{\vb{l}}=-\pdv{U}{l}\dd{l}=\dd{U}|_{\theta=cost}=\grad{U}\vdot\dd{\vb{l}}=
            \grad{(\vb{E}\vdot\vb{p})}\vdot\dd{\vb{l}}=(\vb{p}\vdot\grad)\vb{E}\vdot \dd{\vb{l}}\\
            & \vb{M}\dd{\vb{\theta}}=-\pdv{U}{\theta}\dd{\theta}=-Ep\sin\theta\dd{\theta}=
            -(\vb{E}\cp\vb{p})\dd{\vb{\theta}}=(\vb{p}\cp\vb{E})\dd{\vb{\theta}}
        \end{cases}
    \]
    Per confronto si ottiene la tesi.
\end{proof}


\section{Sviluppo in serie di multipoli}
Si consideri una distribuzione di carica in una regione limitata di spazio complessivamente neutra, ovvero tale che
\[
    q=\int\rho\,d\tau=0
\]
Considerata una superficie chiusa che racchiude la distribuzione di carica, per il teorema di Gauss il flusso del campo elettrico
uscente è nullo. In particolare, se la distribuzione ha simmetria sferica, il campo elettrico esterno alla distribuzione di carica
è ovunque nullo. Se la simmetria non è sferica invece non è detto che il campo su una superficie chiusa $S$ che racchiude la
distribuzione sia nullo, come in effetti dimostra l 'esempio del dipolo.
Risulta conveniente caratterizzare la distribuzione di carica con alcune proprietà globali che consentano di calcolare le caratteristiche
del campo generato. Per farlo si parte dal potenziale definito in \ref{eqn:V_rho}. Essendo la distribuzione concentrata in una
regione limitata di spazio si può scegliere il potenziale nullo all'infinito ($r_0\rightarrow +\infty$). Il termine al denominatore
in coordinate cartesiane è
$f(x',y',z')=\bigl[(x-x')^2+(y-y')^2+(z-z')^2\bigr]^{\frac{1}{2}}$. Si vuole sviluppare la funzione al primo ordine
\[
    f(x',y',z')=f(0)+\pdv{f}{x'}(0)+\pdv{f}{y'}(0)+\pdv{f}{z'}(0)
\]
e tenuto conto della definizione di $f$ si ha che
\[
    \begin{split}
        & f(0)=\bigl[(x)^2+(y)^2+(z)^2\bigr]^{-\frac{1}{2}} \\
        & \pdv{f}{x'}(0)=\eval{(x-x')\bigl[(x-x')^2+(y-y')^2+(z-z')^2\bigr]^{-\frac{3}{2}}}_{x',y',z'=0}=\frac{x}{r^3} \\
        & \pdv{f}{y'}(0)=\eval{(y-y')\bigl[(x-x')^2+(y-y')^2+(z-z')^2\bigr]^{-\frac{3}{2}}}_{x',y',z'=0}=\frac{y}{r^3} \\
        & \pdv{f}{z'}(0)=\eval{(z-z')\bigl[(x-x')^2+(y-y')^2+(z-z')^2\bigr]^{-\frac{3}{2}}}_{x',y',z'=0}=\frac{z}{r^3}
    \end{split}
\]
Sostituendo quindi si ottene
\[
    f(x',y',z')=\frac{1}{r}+\frac{1}{r^3}(xx'+yy'+zz')=\frac{1}{r}+\frac{\vb{r}\vdot\vb{r'}}{r^3}
\]
Da cui, ricordando che il potenziale all'infinito è nullo e la definizione \ref{defn:p_rho}, si ottiene
\begin{equation}
    V(\vb{r})=\rec{4\pi\epsilon_0}\int_{V}f\rho \dd{\tau'} =\frac{1}{4\pi\epsilon_0}\frac{q}{r}+\frac{1}{4\pi\epsilon_0}\frac{\vb{p}\vdot\vb{r}}{r^3}
\end{equation}
Si parla di \textit{primo e secondo termine di dipolo}. Il secondo termine di dipolo decresce più rapidamente del primo,
rappresentando una correzione spesso trascurabile. Quando però il sistema è elettricamente neutro, il secondo termine diventa quello
dominante.


\section{Conduttori}
\begin{defn}[Conduttore]
    Si definisce conduttore un oggetto indeformabile all'interno del quale vi sono elettroni liberi di muoversi
\end{defn}
Un'ipotesi fondamentale dell'elettrostatica è l'invarianza temporale delle grandezze in gioco.
In particolare questo vale per le distribuzioni di carica: è necessario supporre allora che le cariche libere nei conduttori
non si muovano, ovvero che \textit{all'interno dei conduttori il campo elettrico sia nullo}.
Quando un conduttore viene immerso in un campo elettrico le cariche libere si muovono
alla ricerca di una configurazione di equilibrio, che viene raggiunta quando la loro disposizione
è tale da annullare il campo elettrico che si è formato internamente al conduttore.
\begin{thm}
    Passando da un mezzo materiale ad un altro, la componente tangenziale del campo elettrico non può subire discontinuità.
    \label{lemma:discontinuità_E}
\end{thm}
\begin{proof}
    La dimostrazione segue dal fatto che il campo elettrico è conservativo. Si ha infatti che l'integrale di linea
    lungo un qualsiasi percorso chiuso deve essere nullo. Chiamando $\vb{E}_1$ il campo elettrico in un materiale
    e $\vb{E}_2$ il campo elettrico nell'altro, con a priori $\vb{E}_1\neq\vb{E}_2$, si consideri un percorso rettangolare
    prossimo alla superficie di separazione dei due materiali. Si chiamino $dl$ i segmenti paralleli alla superficie
    -uno in un mezzo ed uno nell'altro- e $dn$ i segmenti ortogonali alla superficie. Al tendere di $dn$ a 0,
    il contributo alla circuitazione dato dai tratti ortogonali è nullo. Si ha quindi che
    \[
        \vb{E}_1\vdot d\vb{l} - \vb{E}_2\vdot d\vb{l}=E_1^t\,dl-E_2^t\,dl=0
    \]
    Dove con $E^t$ si indica la componente del campo elettrico tangente alla superficie.
    Segue immediatamente la tesi.
\end{proof}

\begin{cor}
    \label{cor:ortogonalità}
    In prossimità di un conduttore il campo elettrico è ortogonale alla superficie del conduttore.
\end{cor}
\begin{proof}
    Nel lemma, si consideri uno dei due mezzi come un conduttore, sia questo il secondo dei due mezzi.
    Si ha quindi $\vb{E}_2=0$ e in particolare $E_2^t=0$.
    Segue allora che la componente tangenziale del campo esterno al conduttore è nulla.
\end{proof}

Dalle proprietà appena discusse del campo elettrico, seguono immediatamente quelle del potenziale
\begin{cor}
    Il volume interno e la superficie dei conduttori sono equipotenziali.
\end{cor}
\begin{proof}
    Ricordando che $\vb{E}=-\grad{V}$ il fatto che il volume sia equipotenziale segue immediatamente dal fatto che
    il campo elettrico interno è nullo.
    La seconda parte dell'asserto si ottiene considerando che il gradiente è ortogonale alle superfici equipotenziali
    e che il campo elettrico è ortogonale alla superficie del conduttore.
\end{proof}

Si vuole trovare ora una relazione fra il potenziale interno e quello esterno al conduttore. Si introduce la definizione:
\begin{defn}[lavoro di estrazione]
    Si definisce lavoro di estrazione $L_e$ il lavoro necessario per portare una carica di prova $q$ dall'interno
    all'esterno del conduttore, in prossimità della superficie.
    \begin{equation}
        \Delta V=V_i-V_e=-\frac{L_e}{q}
    \end{equation}
\end{defn}

\begin{obses}
    Si osserva che $V_i-V_e$ è una quantità positiva dipendente esclusivamente dal materiale che cotituisce il conduttore.
    Viene detta \textit{funione lavoro}.
\end{obses}

Si ha infine questo importante corollario:
\begin{cor}
    Le cariche presenti su un conduttore si dispongono tutte sulla superficie del conduttore.
\end{cor}
\begin{proof}
    Si consideri una superficie chiusa arbitraria $\Sigma$ interna al coduttore. Essendo il campo elettrico nullo,
    il flusso attraverso questa superficie è nullo. Per il teorema di Gauss la carica totale interna a $\Sigma$ è nulla.
    Essendo la superficie arbitraria, si può fare il limite per $\Sigma$ tendente ad un qualsiasi punto interno al conduttore.
    L'unica possibilità per una carica sul conduttore è quindi quella di disporsi sulla superficie esterna di quest'ultimo.
\end{proof}
Segue quindi che la distribuzione delle cariche di un conduttore viene descritta da una distribuzione di superficie e non di volume.

Un importante risultato relativo ai conduttori è il \textit{teorema di Coulomb}.
\begin{thm}[teorema di Coulomb]
    In un punto vicino ad un conduttore il campo elettrico vale
    \[
        E=\frac{\sigma}{\epsilon_0}
    \]
    diretto secondo la normale uscente se $\sigma>0$, entrante se $\sigma<0$
\end{thm}
\begin{proof}
    Si consideri un conduttore C ed un cilindretto di base $dS$ e altezza $dh$, in modo tale che questa sia un infinitesimo
    di ordine superiore rispetto alla base. L'altezza del cilindro è ortogonale alla superficie di C. Si vuole applicare
    al cilindretto il teorema di Gauss. Il flusso uscente, per il corollario \ref{cor:ortogonalità}, si riduce al
    flusso attraverso la base superiore. La carica interna è $dQ=\sigma dS$.
    Si ha quindi:
    \[
        \vb{E}\vdot d\vb{S}=\frac{\sigma dS}{\epsilon_0}
    \]
    Per il corollario a cui si è già fatto riferimento $\vb{E}=E\vu{n}$. D'altro canto $d\vb{S}=dS\vb{n}$, da cui la tesi.
\end{proof}

Un altro importante risultato emerge a partire dalla seguente considerazione intuitiva: le cariche libere si dispongono
lungo la superficie del conduttore formando uno strato con uno spessore nell'ordine dell'[Angstrom]. Queste cariche
esercitano una forza di mutua repulsione che si deve tradurre in una pressione elettrostatica. Vale infatti il seguente teorema:
\begin{thm}%RIVED QUAND USARE MODULI O VETTORI
    In un punto della superficie del conduttore la pressione elettrostatica è pari alla densità di energia del campo
    elettrostatico in vicinanza del punto.
\end{thm}
\begin{proof}
    Si consideri un campo elettrico attorno al quale sia presente un campo elettrostatico $\vb{E}$ generato dalla distribuzione
    di cariche sul conduttore e dalle cariche nello spazio circostante. Si supponga di modificare per una quantità
    infinitesima e in maniera quasistatica (in modo da trascurare l'energia cinetica dovuta al movimento) la geometria
    del conduttore, facendolo espandere verso l'esterno di un tratto $\delta \vb{x}$ ogni elemento di superficie
    $dS$ ortogonalmente a se stesso. Per farlo è necessario applicare una forza esterna $\delta \vb{F}^{(e)}$ che
    compirà un lavoro $\delta L^{(e)}=\delta\vb{F}^{(e)}\vdot\delta\vb{x}$. Gli spostamenti sono chiamati \textit{spostamenti virtuali};
    i lavori \textit{lavori virtuali}. In assenza di energia cinetica il lavoro corrisponde alla variazione di energia elettrostatica del sistema
    \[
        \delta U=\delta L^{(e)}=\delta\vb{F}^{(e)}\vdot\delta\vb{x}
    \]
    Considerando che essendo la trasformazione quasistatica la forza esterna deve essere uguale e opposta alla forza elettrostatica si ha, in modulo:
    \[
        \delta\vb{F}^{(e)}=-\delta\vb{F}=\frac{\delta U}{\delta x}
    \]
    L'elemento di volume $dS\delta x$ aveva prima dello spostamento energia $\delta U_i=u\,ds\delta x$ e dopo lo spostamento
    ha energia $\delta U_f=0$, in quanto tutto l'elemento di volume si trova dentro al conduttore dove il campo elettrico
    è nullo. Si ha quindi $\delta U=\delta U_f -\delta U_i=-u\,dS\delta x$. Da questo segue che
    \[
        \delta F=-\frac{\delta U}{\delta x}=u\,dS
    \]
    Dividendo per l'elemento di superficie, dalla definizione di pressione, si ottiene la tesi.
\end{proof}
Il metodo dei lavori virtuali può essere usato per calcolare qualcunque sollecitazione meccanica agente sul
conduttore. A titolo di esempio, si consideri un conduttore spostato rigidamente di un tratto $\delta \vb{x}$.
La risultante delle forze elettrostatiche $R_x$ compie un lavoro $R_x \delta x=-\delta U$ con $U$ l'energia elettrostatica,
da cui si ha che $R_x=-\delta U/\delta x$. Una trattazione più approfondita delle problematiche relative a questo
metodo verrà svolta nel capitolo sulla corrente stazionaria.

Con quanto appena visto è possibile osservare due effetti non intuitivi caratteristici dei conduttori.

\subsubsection{Schermo elettrostatico}
Si consideri un conduttore cavo, carico con una carica $q$. La carica, per quanto visto, si dispone sulla superficie
del conduttore. Sia per assurdo $\vb{E}$ il campo interno alla cavità. Si consideri un percorso chiuso in parte interno
alla cavità ed in parte interno al conduttore: lungo questo percorso l'integrale di linea del campo elettrico non
sarebbe nullo, infatti la parte di percorso interna al conduttore darebbe contributo nullo, ma non si potrebbe dire
lo stesso per la parte interna alla cavità. Si viola quindi la conservatività del campo elettrico. Questo ragionamento
può essere ripetuto identicamente considerando anche una distribuzione di carica esterna al conduttore. Il conduttore
cavo funziona quindi da schermo elettrostatico.

\subsubsection{Potere delle punte}
Si immaginino due sfere con diverso raggio in cui tutte le grandezze fisiche caratteristiche abbiano simmetria sferica
(e quindi, sufficientemente lontane affinchè la distribuzione di carica di una non perturbi la distribuzioe di carica
dell'altra). I due potenziali allora valgono:
\[
    \begin{split}
        & V_1=\frac{1}{4\pi\epsilon_0}\frac{Q_1}{R_1}\\
        & V_2=\frac{1}{4\pi\epsilon_0}\frac{Q_2}{R_2}\\
    \end{split}
\]
Se ora queste due sfere sono collegate da un filo in modo che costituiscano un unico condutore, si deve avere $V_1=V_2$, ovvero
\[
    \frac{Q_1}{Q_2}=\frac{R_1}{R_2}
\]
La carica si distribuscie sulle due sfere in maniera proporzionale ai raggi.
Considerando ora invece la distribuzione superficiale di carica
\[
    \sigma=\frac{Q}{4\pi R^2}
\]
Si ottiene
\[
    \frac{\sigma_1}{\sigma_2}=\frac{Q_1}{R_1^2}\frac{R_2^2}{Q_2}=\frac{R_2}{R_1}
\]
Ovvero la distribuzione di carica è inversamente proporzionale ai raggi e quindi per il teorema di Coulomb il campo
elettrico è più intenso in prossimità della sfera più piccola. Generalizzando si può affermare che \textit{in vicinanza
di un conduttore il campo elettrico è tanto più intenso tanto più è piccolo il raggio di curvatura}.


\section{Condensatori}
Dipendendo il potenziale linearmente dalla carica, il rapporto $Q/V$ è costante ed è allora giustificata la seguente definizione
\begin{defn}[Capacità]
    Si definisce capacità C il rapporto
    \[
        C=\frac{Q}{V}
    \]
\end{defn}

Si considerino due conduttori $S_1$ ed $S_2$ sufficientemente vicini, affinchè interagiscano per induzione. Fissata la configurazione
geometrica, venga fornita a $S_1$ una carica $Q_1$, mantenendo pari a zero la carica di $S_2$. Moltiplicando per un fattore
$a$ la carica $Q_1$ si dimostra che anche i potenziali generati dai due conduttori, $V'_1$ e $V'_2$, vengono motliplicati per lo
stesso fattore: è quindi presente una relazione di proporzionalità diretta esprimibile dal sistema
\[
    \begin{cases}
        & V'_1=p_{11}Q_1 \\
        & V'_2=p_{21}Q_1
    \end{cases}
\]
Ripetendo lo stesso ragionamento, fornendo questa volta una carica $Q_2$ ad $S_2$ e mantenendo nulla la carica su
$S_1$ si ottiene
\[
    \begin{cases}
        & V''_1=p_{12}Q_2 \\
        & V''_2=p_{22}Q_2
    \end{cases}
\]
Per il principio di sovrapposizione, se ad ambedue i conduttori viene fornita una carica si ottiene il sistema
\begin{equation}
    \label{eqn:sist_p}
    \begin{cases}
        & V_1=p_{11}Q_1 + p_{12}Q_2\\
        & V_2=p_{21}Q_1 + p_{22}Q_2
    \end{cases}
\end{equation}

I coefficenti $p_{ij}$ si chiamano \textit{coefficenti di potenziale} e dipendono solo dalla geometria del sistema. La
situazione può essere generalizzata al caso con N conduttori ottenendo
\[
    V_i=\sum_{j=1}^N p_{ij}\,Q_j
\]
Fissate le cariche, i potenziali sono univocamente determinati. Da questo segue che la matrice rappresentativa del sistema
deve essere non singolare, ovvero $\det{p_{ij}}\neq 0$. In virtù di questo fatto, il sistema può essere invertito ottenendo che
\[
    Q_i=\sum_{j=1}^N c_{ij}\,V_j
\]
I coefficenti $c_{ij}$ si chiamano \textit{coefficenti di induzione}. Le matrici dei coefficenti sono ovviamente legate dalla relazione:
\[
    \Bqty{c_{ij}}=\Bqty{p_{ij}}^{-1}
\]
Le matrici di induzione e di potenziale sono simmetriche.

Si immagini ora un sistema di due conduttori disposti in una configurazione tale che il fenomeno di induzione che un condensatore
esercita sull'altro sia completo, ovvero fornita una carica $Q$ al primo conduttore il secondo si carica per induzione con una
carica $-Q$. Quando l'induzione è completa tutte le linee di campo uscenti da un conduttore terminano sull'altro conduttore.
\begin{defn}[Condensatore]
    Si definisce condensatore un sistema di due conduttori che goda della proprietà di cui sopra.
\end{defn}
Esistono tre tipi di condensatori:
\begin{description}
    \item[Condensatori sferici] Un conduttore è contenuto nella cavità di un altro conduttore;
    \item[Condensatori cilindrici] Un conduttore si sviluppa lungo una linea ed è avvolto da un secondo conduttore con struttura
        tubolare - l'induzione completa si realizza quando la lunghezza del condensatore è molto maggiore rispetto alle dimensioni trasversali;
    \item[Condensatori piani] Due conduttori si affacciano l'un l'altro in modo tale che le dimensioni lineari della superficie siano
        molto maggiori rispetto alla distanza - l'induzione completa si realizza nel limite di superficie infinita.
\end{description}
I due conduttori che formano il condensatore sono detti \textit{armature}. Un condensatore viene caricato quando si stabilisce
una differenza di potenziale fra le armature e su di esse si distribuiscono carice uguali in modulo e di segno opposto
(la carica totale di un condensatore carico è quindi nulla).

Fornendo una carica $Q_1$ all'armatura interna di un condensatore e lasciando la seconda elettricamente isolata,
su quest'ultima le cariche si ridistribuiscono per induzione completa in modo che la faccia interna sia dotata di carica $-Q_1$
e quella esterna $+Q_1$. Collegandola ora a terra, l'armatura esterna resta dotata di carica $-Q_1$.
Con un ragionamento analogo ci si rende conto che lo stesso avviene nel caso in cui quella collegata a terra fosse l'armatura interna.
\begin{thm}
    In un condensatore, il modulo della carica sulle maglie e la differenza di potenziale sono direttamente proporzionali.
    \[
        Q=C\Delta V \quad\quad C=\frac{1}{p_{11}+p_{22}-2p_{12}}
    \]
\end{thm}
\begin{proof}
    Si consideri il sistema \ref{eqn:sist_p} per il caso considerato, ricordando la proprietà di simmetria:
    \[
        \begin{split}
            & V_1=p_{11}Q-p_{12}Q=(p_{11}-p_{12})Q\\
            & V_2=p_{21}Q-p_{22}Q=(p_{21}-p_{22})Q
        \end{split}
    \]
    Sottraendo membro a memebro, la tesi. La seconda espressione per $C$ si ottiene facilmente invertendo la matrice $\Bqty{p_{ij}}$.
\end{proof}

È immediato da quanto visto ricavare l'energia immagazzianata da un condensatore. Ulteriori considerazioni verrano svolte
nel capitolo sulla corrente stazionaria, dopo aver introdotto il concetto di forza elettromotrice.
\begin{thm}
    Per un condensatore con capacità $C$, caricato fino ad avere una differenza di potenziale $\Delta V$ fra le armature
    l'energia elettrostatica è
    \begin{equation}
        U=\rec{2}Q\Delta V=\rec{2}C(\Delta V)^2=\rec{2}\frac{Q^2}{C}
        \label{eqn:U_condensatore}
    \end{equation}
\end{thm}
\begin{proof}
    La distribuzione di carica è una distribuzione superficiale. Si indichino con $S_a$ ed $S_b$ le superfici delle due armature.
    Usando l'espressione per l'energia elettrostatica \ref{eqn:energia_distribuzione}, si ha
    tenendo a mente
    \[
        U=\rec{2} \int_{S_a \cup S_b} \sigma V \dd{S}=\frac{V_a}{2}\int_{S_a} \sigma\dd{S}+\frac{V_b}{2}\int_{S_b} \sigma\dd{S}=
        \frac{V_a Q + V_b (-Q)}{2}=\rec{2} Q\Delta V
    \]
    Dalla definizione di capacità si ottengono le altre forme dell'asserto.
\end{proof}

Si vogliono ora ricavare delle formule specifiche per le capacità delle tre tipologie di condensatori.
\begin{obs}[Capacità del condensatore piano]
    Per un condensatore piano con maglie di superficie $S$ poste a distanza $d$, con le dimesioni lineari di $S$ molto maggiori di $d$ vale la formula
    \begin{equation}
        \label{eqn:capacità_piano}
        C=\epsilon_0\frac{S}{d}
    \end{equation}
\end{obs}
\begin{proof}
    Quella presente sulla superficie delle maglie è una distribuzione di carica superficiale che nelle ipotesi può considerarsi infinita,
    per cui il campo elettrico generato è
    \[
        \vb{E}=\frac{\sigma}{\epsilon_0}\vu{n}=\frac{Q}{S\epsilon_0}\vu{n}
    \]
    Da questo segue che la differenza di potenziale vale
    \[
        \Delta V=\int_1^2 \vb{E}\vdot d\vb{l}=\int_1^2\frac{Q}{S\epsilon_0}\vu{n}\vdot d\vb{l}=\frac{d}{S\epsilon_0}Q
    \]
    dato che la somma della proiezione di tutti gli elementi di linea sulla normale equivale alla distanza $d$ fra le superfici.
    Dalla definizione di capacità come rapporto fra differenza di potenziale e carica, la tesi.
\end{proof}

\begin{obs}[Capacità del condensatore sferico]
    Per un condensatore sferico con maglie di raggio $R_1$ ed $R_2$  poste a distanza $d=R_2-R_1$ vale la formula
    \begin{equation}
        \label{eqn:capacità_sferico}
        C=4\pi\epsilon_0\frac{R_1R_2}{d}
    \end{equation}
    Se vale la condizione $R_1\cong R_2$ allora la formula si riduce a:
    \[
        C=\epsilon_0\frac{S}{d}
    \]
\end{obs}
\begin{proof}
    Nella zona tra i due conduttori il campo elettrico vale
    \[
        \vb{E}(r)=\frac{Q}{4\pi\epsilon_0}\frac{\vu{r}}{r^2} \quad\quad\quad R_1<r<R_2
    \]
    Da questo segue che la differenza di potenziale vale
    \[
        \Delta V=\int_{R_1}^{R_2} \frac{Q}{4\pi\epsilon_0}\frac{\vu{r}}{r^2} \vdot d\vb{l}=\frac{Q}{4\pi\epsilon_0}\frac{R_2-R_1}{R_2R_1}
    \]
    Scegliendo come percorso, ad esempio, il segmento giacente sul raggio delle sfere. Dalla definizione di capacità, si
    ha la prima parte dell'asserto.
    La seconda parte dell'asserto si ottiene considerando che la condizione $R_2\cong R_1$ implica $R_1R_2\cong R^2$. Si ha quindi
    \[
        C=4\pi\epsilon_0\frac{R^2}{d}=\epsilon_0\frac{S}{d}
    \]
\end{proof}

\begin{obs}[Capacità del condensatore cilindrico]
    Per un condensatore cilindrico con lunghezza $l$ molto maggiore dei raggi $R_1$ ed $R_2$ delle maglie ($R_1<R_2$) vale la formula
    \begin{equation}
        \label{eqn:capacità_cilindirco}
        C=2\pi\epsilon_0\frac{l}{\ln(R_2/R_1)}
    \end{equation}
    Se vale la condizione $R_1\cong R_2$ allora la formula si riduce a:
    \[
        C=\epsilon_0\frac{S}{d}
    \]
\end{obs}
\begin{proof}
    Il ragionamento è analogo a quello precedente, ricordando che nella zona tra i due conduttori il campo elettrico vale
    \[
        \vb{E}(r)=\frac{\lambda}{2\pi\epsilon_0}\frac{\vu{r}}{r} \quad\quad\quad R_1<r<R_2
    \]

    La seconda parte dell'asserto si ottiene nuovamente considerando che la condizione $R_2\cong R_1$ implica $R_2-R_1=d<<R_2,R_1$.
    Si ha quindi, sviluppando
    \[
        \ln\frac{R_2}{R_1}=\ln\frac{R_1+d}{R_1}=\ln\Biggl(1+\frac{d}{R_1} \Biggr)\cong \frac{d}{R_1}
    \]
    Da cui segue
    \[
        C=2\pi\epsilon_0\frac{l}{\ln(R_2/R_1)}=\epsilon_0\frac{2\pi R_1 l}{d}=\epsilon_0\frac{S}{d}
    \]
\end{proof}



Si volgiono ora studiare invece i sistemi di condensatori. Si danno le seguenti definizioni:
\begin{defn}[Condensatori in parallelo]
    Si dicono in parallelo due consensatori collegati in modo tale che la prima armatura del primo sia collegata alla prima armatura
    del secondo e che la seconda armatura del primo sia collegata alla seconda armatura del secondo.
\end{defn}

\begin{defn}[Condensatori in serie]
    Si dicono in serie due consensatori collegati in modo tale che la seconda armatura del primo sia collegata alla prima armatura del secondo.
\end{defn}

Si osserva che i condensatori collegati in parallelo costituiscono di fatto un sistema di due conduttori. I condensatori collegati in
serie invece, costituiscono un sistema di tre conduttori%aggiungo immagine ??
\begin{thm}[Capacità dei condensatori in parallelo]
    Dati $n$ condensatori collegati in parallelo la capacità del sistema è equivalente a quella di un unico condensatore con capacità
    \[
        C=\sum_{i=1}^n C_i
    \]
\end{thm}
\begin{proof}
    Si consideri per semplicità un sistema di due condensatori - le generalizzazione è banale. Considerando separatamente i due condensatori si ha
    \[
        \begin{cases}
            & Q_1=C_1\Delta V_1 \\
            & Q_2=C_2\Delta V_2
        \end{cases}
    \]
    In virtù del collegamento in parallelo, le differenze di potenziale ai capi delle armature dei condensatori sono uguali. Sommando membro a membro
    \[
        Q_1+Q_2=(C_1+C_2)\Delta V
    \]
    Siccome $Q_1+Q_2$ rappresenta la carica totale del sistema, dalla definizione di capacità si  ha l'asserto.
\end{proof}

\begin{thm}[Capacità dei condensatori in serie]
    Dati $n$ condensatori collegati in serie la capacità del sistema è equivalente a quella di un unico condensatore con capacità
    \[
        \frac{1}{C}=\sum_{i=1}^n\frac{1}{C_i}
    \]
\end{thm}
\begin{proof}
    Si consideri nuovamente un sistema di due condensatori. Considerando separatamente i due condensatori si ha
    \[
        \begin{cases}
            & \Delta V_1=\frac{ Q_1 }{C_1}\\
            & \Delta V_2=\frac{ Q_2 }{C_2}\\
        \end{cases}
    \]
    In virtù del collegamento in serie, per l'induzione completa la carica sulla seconda armatura del primo condensatore sarà uguale
    in modulo e opposta alla carica sulla prima armatura. Un ragionamento analogo vale per il secondo condensatore, tenendo
    però conto anche del fatto che la carica sulla prima armatura del secondo condensatore sarà uguale alla carica sulla seconda
    armatura del primo. A questo punto quindi, sottraendo membro a membro le equazioni del sistema si ottiene:
    \[
        \Delta V_1 - \Delta V_2 = \Biggl(\frac{1}{C_1} + \frac{1}{C_2} \Biggr) Q
    \]

    Siccome $\Delta V_1 - \Delta V_2 $ rappresenta la caduta di potenziale ai capi del sistema di condensatori, dalla definizione di
    capacità si  ha l'asserto.
\end{proof}


\section{Il problema generale dell'elettrostatica nel vuoto}
Il problema dell'elettrostatica consiste nel calcolare, note le cariche totali possedute da dei conduttori,
il campo elettrico generato e la distribuzione delle cariche sui conduttori stessi.
Con distribuzioni di carica note, la soluzione è concettualmente semplice: basta applicare la \eqref{eqn:E_distribuzione}
per ottenere il campo elettrico e di seguito tutte le grandezze cercate. In presenza di conduttori
la distribuzione di cariche libere è però influenzata dalla reciproca interazione. Per riuscire a risolvere il problema
risultano allora di fondamentale importanza le due equazioni di Maxwell trovate.
\begin{thm}
Il potenziale soddisfa la seguente equazione
\begin{equation}
\laplacian V=-\frac{\rho}{\epsilon_0}
\end{equation}
equivalente alla prima equazione di Maxwell \eqref{eqn:prima_maxwell} ed alla terza \eqref{eqn:terza_maxwell}.
\end{thm}
L'equazione presentata nel teorema si chiama \textit{equazione di poisson}
\begin{proof}
  Per definizione $\vb{E}=-\grad{V}$. Dalla prima equazione di Maxwell si ha che:
  \[
  \frac{\rho}{\epsilon_0}=\div{\vb{E}}=\div{(-\grad{V})}=-\laplacian{V}
  \]
  dimostrando in questo modo sia l'equazione di Poisson che l'equivalenza fra quest'ultima e la prima equazione di Maxwell.
    L'ultima parte della tesi si ottiene facilmente: essendo $\vb{E}$ espresso come gradiente di $V$ allora
    la terza equazione di Maxwell è verificata e viceversa, se la terza equazione di Maxwell è verificata
    allora il campo elettrico è conservativo e può quindi essere espresso in termini di gradiente di una funzione scalare.
\end{proof}

\begin{thm}[Esistenza e unicità della soluzione all'equazione di Poisson]
Fissata la funzione $\rho$ localizzata in una porzione finita di spazio, l'equazione di Poisson ammette
    una ed una sola soluzione che soddisfi le condizioni al contorno del dominio di definizione.
\end{thm}
Questo risultato è importante nel caso in cui siano presenti dei conduttori: in caso contrario infatti
il potenziale e di conseguenza il campo elettrico sono forniti direttamente dalla \ref{eqn:V_rho}.
Segue direttamente l'importante corollario:
\begin{cor}
  L'equazione di Poisson caratterizza completamente il potenziale.
\end{cor}
Si può allora dire che il problema generale dell'elettrostatica consista nel risolvere l'equazione di Poisson
con determinate condizioni al contorno.

Si elencano ora le tre situazioni che si presentano più di frequente.

\textit{Problema di Dirichlet - Non sono presenti cariche localizzate; il campo è generato
da un sistema di conduttori di geometria nota, di cui si conoscono i potenziali.}
L'equazione di Poisson si riduce in questo caso all'equazione di Laplace
\[
    \laplacian{V}=0
\]
Il problema è definito dal punto di vista matematico, essendo note le condizioni al contorno ($V=0$ all'infinito,
$V=V_i$ sulla superficie dell'i-esimo conduttore). I passi risolutivi sono i seguenti:
\begin{enumerate}
    \item si risolve l'equazione di Poisson ottenendo il potenziale nello spazio circostante i conduttori;
    \item dal potenziale si può ricavare il campo;
    \item il valore del campo in prossimità dei conduttori consente di calcolare la densità di carica superficiale;
    \item integrando la densità di ogni conduttore sulla sua superficie si ottine la carica disposta sui conduttori;
    \item avendo le cariche ed essendo i potenziali noti è possibile ricavare i coefficienti di capacità.
\end{enumerate}

\textit{Non sono presenti cariche localizzate; il campo è generato
da un sistema di conduttori di geometria nota, di cui si conoscono le cariche.}
Questo problema è l'inverso del problema di Dirichlet
I passi risolutivi sono i seguenti:
\begin{enumerate}
    \item si scelgono in modo arbitrario i potenziali dei conduttori;
    \item si risolve il problema di Dirichlet relativo;
    \item ricavati i coefficenti di capacità, si ottenogno i potenziali veri a partire dalle cariche note;
    \item si risolve il problema di Dirichlet relativo ai potenziali trovati.
\end{enumerate}

\textit{Sono presenti cariche localizzate con distribuzione $\rho$ nota; è inoltre presente
un sistema di conduttori di geometria nota, di cui si conoscono le cariche.}
Il potenziale dell'i-esimo conduttore, per il principio di sovrapposizione, è dato da
\[
    V_i=V_i(\rho)+\sum_{j=1}^N p_{ij}Q_j
\]
Di questa equazione sono noti i coefficienti di potenziale determinati dalla geometria del
sistema, mentre non si conoscono i potenziali generati dalla distribuzione di carica.
I passi risolutivi sono i seguenti:
\begin{enumerate}
    \item si scelgono in modo arbitrario i potenziali dei conduttori;
    \item si risolve il problema di Dirichlet relativo;
    \item ricavate le cariche relative, si può invertire l'equazione per i potenziali dei conduttori
        in modo da ricavare i potenziali dovuti alla distribuzione di carica;
    \item si risolve l'equazione per i potenziali dei conduttori usando le cariche note.
\end{enumerate}

Si elencano ora tre casi notevoli in cui il problema generale dell'elettrostatica può
essere risolto senza ricorrere all'equazione di Poisson.

\subsubsection{Metodo delle cariche immagine}
Si consideri un conduttore $S$ collegato a terra, in modo che il suo potenziale sia 0, e che siano presenti
N cariche puntiformi $Q_i$. Se la geometria del conduttore è semplice, si può immaginare di eliminare il
conduttore posizionando dalla parte opposta delle $Q_i$ rispetto al conduttore, M cariche $Q'_i$, dette cariche immagie, in modo tale
che il potenziale complessivo su $S$ sia
comunque nullo. Nella porzione di spazio $\tau$ che contiene le $Q_i$ la distribuzione di carica e le condizioni
al contorno sono le stesse sia in presenza delle cariche immagine che della distribuzione di carica sul conduttore
indotta dalle $Q_i$, perciò per il
teorema di unicità la configurazione di potenziale in $\tau$ è la stessa in entrambi i casi.

Il grande vantaggio di questo metodo è che introducendo le cariche immagine il potenziale è calcolabile usando
l'espressione per il potenziale generato da un numero discreto di cariche.

\subsubsection{Equazione di Laplace undimensionale}
Se il sistema è costituito da piani omogeneamente carichi, infiniti e paralleli fra loro, il potenziale
dipende esclusivamente dalla coordinata $x$ ortogonale ai piani. L'equazione di Laplace si riduce allora a
\[
    \dv[2]{V}{x}=0
\]
Di conseguenza $V(x)=ax+b$, con le costanti determinate dalle condizioni al contorno.

\subsubsection{Separazione di variabili}
L'equazione di Laplace nel caso tridimensionale può essere risolta ipotizzando che il potenziale sia nella forma
\[
    V(x,y,z)=X(x)Y(y)Z(z)
\]
Questo è possibile solo se la decomposizione è compatibile con le equazioni al contorno assegnate.
Sostituendo nell'equazione di Laplace e dividendo per $XYZ$ si ottiene:
\[
    \rec{X}\dv[2]{X}{x}+\rec{Y}\dv[2]{Y}{y}+\rec{Z}\dv[2]{Z}{z}=0
\]
Affinchè la somma sia nulla per ogni valore di $X$, $Y$, $Z$ è necessario chee ciascuno dei
termini della somma sia costante, ovvero
\[
    \begin{split}
        & \rec{X}\dv[2]{X}{x} = K^2_1 \\
        & \rec{Y}\dv[2]{Y}{y} = K^2_2 \\
        & \rec{Z}\dv[2]{Z}{z} = -K^2_3
    \end{split}
\]
con $K^2_1+K^2_2-K^2_3=0$.
La soluzione delle tre equazioni è
\[
    \begin{split}
        & X =A_1\cos(K_1x+B_1) \\
        & Y =A_2\cos(K_2y+B_2) \\
        & Z =A_3\cos((K_1+K_2)^{\rec{2}}z+B_3) \\
    \end{split}
\]
I vaori delle costanti si trovano imponendo le condizioni al contorno.



\chapter{Elettrostatica nei dielettrici}
Si vogliono ora studiare gli effetti della presenza di cariche elettriche in uno spazio non più vuoto ma riempito con materiali dielettrici (o isolanti).

\section{Costante dielettrica}
\label{par:costante_dielettrica}
\begin{obses}
Preso un condensatore a geometria piana e riempito lo spazio fra le armature con un materiale dielettrico omogeneo e isotropo,
lasciando inalterata la geometria del consensatore, si osserva che a parità di carica la differenza di potenziale diminuisce.
\end{obses}
Dalla definizione di capacità, questo significa che la capacità è aumentata.
Ma dalla \eqref{eqn:capacità_piano} si osserva che ciò può essere dovuto esclusivamente
ad una variazione della costante dielettrica\footnote{Lo stesso discorso si può condurre per condensatori sferici e cilindici,
infatti per quanto ben più complicate, le espressioni della capacità dipendono anche in quel caso esclusivamente
dalla costante dielettrica e dalla geometria del condensatore.}.
\begin{defn}[Costante dielettrica relativa]
  Si definisce costante dielettrica relativa $\epsilon_r=C/C_0>1$,
  dove $C$ è la capacità del condensatore immerso nel dielettrico e $C_0$ la capacità del condensatore nel vuoto.
\end{defn}

\begin{defn}[Costante dielettrica assoluta]
  Si definisce costante dielettrica assoluta $\epsilon=\epsilon_0\epsilon_r>\epsilon_0$.
\end{defn}
In questo modo si ha che
\[
C=\epsilon_r C_0=\epsilon_r\,\epsilon_0\frac{S}{d}=\epsilon\frac{S}{d}
\]
Pragmaticamente quindi, l'effetto della presenza di un dielettrico in tutto lo spazio interessato da un potenziale
generato da un sistema di cariche è quello di diminuire il potenziale secondo un fattore $\epsilon_r$.


\section{Polarizzazione}
Evidetentemente i dielettrici si comportano in modo molto diverso dai conduttori. Questa differenza di comportamento
a livello macroscopico è ricoducibile a differenze di comportamento microscopico. I conduttori tipicamente sono metalli,
caratterizzati da una struttura cristallina in cui ogni atomo si trova al vertice di un poliedro.
Questa struttura fa si che gli elettroni più esterni di ciascun atomo siano liberi
- ovvero l'energia di interazione di questi elettroni col nucleo sia inferiore all'energia di agitazione termica.
L'applicazione di un campo elettrico induce su questi elettroni un movimento ordinato. Nei dielettrici
gli elettroni sono invece fortemente legati al nucleo e vengono strappati dalla loro posizione
solo in seguito a forze localizzate molto intense (come lo strofinio). Complessivamente quindi il dielettrico
può considerarsi neutro. Nonostante questo però esso produce un campo elettrico quando possiede un momento di dipolo diverso da zero,
che viene indotto da un campo elettrico esterno. Questo fenomeno è chiamato
\textit{polarizzazione elettrica} e può essere di due tipi: per deformazione e per orientamento.
Nella descrizione del fenomeno di polarizzazione si fa riferimento al campo elettrico locale $\vb{E}_l$,
ovvero il campo agente sui singoli atomi (o sulle singole molecole) dovuto sia alle cariche libere localizzate
che generano il campo elettrico esterno che al campo generato dagli atomi (molecole) del dielettrico meno quello considerato.
\begin{description}
    \item[Polarizzazione per deformazione]
        L'atomo è schematizzabile come un sistema elettricamente neutro costituito da un nucleo puntiforme con carica $Q_+=Ze$
        e da una distribuzione di carica a simmetria sferica variabile con carica totale $Q_-=-Q_+$ dovuta alla nube elettronica.
        A causa della simmetria sferica
        il campo elettrico generato dall'atomo è nullo. Se però interviene un campo elettrico esterno
        il nucleo risente di una forza $\vb{f}_p=Ze\vb{E}_{l}$ e il baricentro della nube elettronica risente di una forza $\vb{f}_e=-Ze\vb{E}_{l}$.
        Ne consegue che il nucleo ed il baricentro della nube elettronica
        si allontanano di una sistanza $\vb{r}$ e si attraggono quindi con una forza $\vb{f'}$.
        Si raggiunge una situazione di equilibrio quando $\abs{\vb{f'}}=\abs{\vb{f}_p}\bigl(=\abs{\vb{f}_e}\bigr)$: in questa situazione si hanno due
        cariche uguali e opposte ad una distanza $\vb{r}$, ovvero un dipolo.
        Immaginando un modello in cui gli elettroni sono legati elasticamente ai nuclei è naturale aspettarsi che,
        per campi elettrici esterni non troppo intensi\footnote{Questa condizione è largamente rispettata
        per tutti i campi elettrici realizzabili nella pratica},
        $\vb{r}$ sia proporzionale all'intensità del campo elettrico.
        Ricordando la \eqref{eqn:momento_di_dipolo} il momento di dipolo dovuto alla deformazione può quindi essere scritto come
        \[
            \vb{p}=\alpha_d\vb{E}_{l}
        \]
        $\alpha_d$ viene detta polarizzabilità elettronica.
        Un fenomeno analogo lo si può osservare anche nelle molecole poliatomiche:
        oltre alla polarizzazione elettronica è presente quella che viene detta \textit{polarizzazione atomica}.

    \item[Polarizzazione per orientamento]
        Molte molecole non sono simmetriche e perciò possiedono un momento di dipolo. Poichè solitamente le molecole sono orientate casualmente,
        il momento di dipolo medio è nullo e perciò non si osservano effetti a livello macroscopico.
        Quando però viene applicato un campo elettrico esterno i momenti di dipolo
        tendono ad orientarsi parallelamente a questo, e quindi il valor medio risulta diverso da zero.
        \begin{lemma}
            Dato un insieme di dipoli praticamente liberi, per i quali siano trascurabili le mutue interazioni, il momento di dipolo medio vale
            \begin{equation}
                \expval{\vb{p}}=\alpha_0\vb{E}_l \quad\quad\quad \alpha_0=\frac{p_0^2}{3KT}
            \end{equation}
        \end{lemma}
        \begin{proof}
            In presenza del campo elettrico locale ogni dipolo è sottoposto ad un momento meccanico $\vb{M}=\vb{p}\cp\vb{E}_l$
            che tende ad orientarlo come $\vb{E}_l$; l'agitazione termica invece favorisce l'orientamento casuale.
            L'equilibrio statistico fra queste due tendenze è descritto dalla funzione di Boltzmann
            \[
                P(U) = A e^{-\frac{U}{k_bT}}
            \]
            dove $A$ è una costante di normalizzazione, $U$ è l'energia del dipolo, $k_b$
            è la costante di Boltzmann e $T$ è la temperatura in gradi kelvin.
            Fissato il momento di dipolo ed il modulo del campo elettrico locale, si ha per la \ref{eqn:U_dipolo}
            $U=-p_0E_l\cos\theta=U(\theta)$, con $\theta$ l'angolo fra i due vettori.
            Ambientando il problema in un sistema di riferimento con l'asse $z$ rivolto nello stesso verso del campo $\vb{E}_l$,
            la probabilità che il dipolo sia orientato entro un angolo solido $\dd{\Omega}=\dd{\theta}\dd{\phi}$ è
            \[
                \dd{P}=P(U(\theta))\sin\theta \dd{\theta}\dd{\phi}=A e^{\frac{pE_l\cos\theta}{KT}}\sin\theta \dd{\theta}\dd{\phi}
            \]
            Nell'ipotesi di essere lontani dallo zero assoluto, e che l'intensità del campo locale non sia
            troppo alta, l'esponenziale può essere sviluppato al primo ordine ottenendo:
            \[
                \dd{P}=A \Biggl(1 + \frac{pE_l\cos\theta}{KT}\Biggr) \sin\theta \dd{\theta}\dd{\phi}
            \]
            È conveniente ora cambiare variabile: $x=\cos\theta$, $\dd{x}=-\sin\theta \dd{\theta}$.
             Si ottiene quindi
            \[
                \dd{P}=A \Biggl(1+\frac{pE_l}{KT}x\Biggr)\dd{x}\dd{\phi}
            \]
            La costante di normalizzazione si ottiene imponendo che la probabilità su tutto l'angolo
            solido\footnote{$\phi$ va da $0$ a $2\pi$, $x$ da $1$ a $-1$.} sia 1
            da cui segue, svolgendo l'integrale, $A=1/(4\pi)$.
            Si osservi ora come $\vb{p}$ abbia simmetria cilindrica attorno a $\vb{E}_l$ e la componente del momento di dipolo ortogonale
            al campo elettrico sia dunque, in media, nulla: $\expval{\vb{p}}$ è orientato come $\vb{E}_l$.
            In modulo il valor medio del momento di dipolo vale allora
            \[
                \abs{\expval{\vb{p}}}=\abs{\expval{\vb{p}_{//}}}=\int_0^{2\pi}\int_0^{\pi} p \cos{\theta}\dd{P}=
                \int_{+1}^{-1}-\rec{2}\Biggl(px + \frac{p^2E_l }{KT}x^2 \Biggr)\dd{x}=\frac{p^2}{3k_b T}E_l
            \]
            ovvero la tesi.
        \end{proof}
        $\alpha_0$ viene detta polarizzabilità molecolare.
        Il risultato ottenuto ha evidenti analogie con quello dedotto nel caso della polarizzazione per deformazione, con la sola differenza che
        in questo caso non è direttamente il momento di dipolo ma il suo valor medio ad essere proporzionale al campo elettrico.

\end{description}
Sulla base di queste considerazioni ci si aspetta che un dielettrico immerso in un campo elettrico
possegga un momento di dipolo medio $\expval{\vb{p}}$ non nullo, orientato come il campo elettrico
esterno.

L'effetto macroscopico dei fenomeni appena elencati può essere descritto introducedo il vettore polarizzazione elettrica.
\begin{defn}[Vettore polarizzazione elettrica]
    Si definisce il vettore polarizzazione elettrica $\vb{P}$ come il momento di dipolo elettrico per unità di volume posseduto dal dielettrico
    \[
        \vb{P}=\lim_{\tau\to 0}\frac{\sum \vb{p}_i}{\tau}=\frac{\dd{N}  \expval{\vb{p}}}{\dd{\tau}}
    \]
\end{defn}
Con $\dd{N}$ il numero di molecole contenute nell'elemento di volume $\dd{\tau}$.
Dalla definizione si ottiene che l'elemento di volume possiede un momento di dipolo $\dd{\vb{p}}=\vb{P}\dd{\tau}$.
Sebbene quindi le cariche in un dielettrico non siano libere di muoversi, il fatto che un dielettrico sia polarizzato
può essere schematizzato con la presenza sulla sua superficie di cariche aggiuntive con una distribuzione $\sigma_p$
e nel suo volume di cariche aggiuntive con distribuzione $\rho_p$.
\begin{thm}
    In un dielettrico polarizzato si ha che
    \begin{equation}
        \sigma_p=\vb{P}\vdot\vb{n}
    \end{equation}
    \begin{equation}
        \label{eqn:relazione_rhop_P}
        \rho_p=-\div{\vb{P}}
    \end{equation}
\end{thm}
\begin{proof}
    Si vuole calcoalre il campo elettrico generato da un dielettrico che occupi un volume $\tau$ e dotato di polarizzazione $\vb{P}(x',y',z')$. Dalla \eqref{eqn:V_dipolo} si ha che l'elemento di volume $\dd{\tau}$ in posizione $\vb{r'}$ porta al potenziale $V(x,y,z)=V(\vb{r})$ il contributo infinitesimo
    \[
        \dd{V(\vb{r})}=\rec{4\pi\epsilon_0}\frac{\vb{P}\vdot(\vb{r}-\vb{r'})}{\abs{\vb{r}-\vb{r'}}^3}\dd{\tau}
    \]
    Di conseguenza il potenziale su tutto lo spazio vale
    \[
        V(\vb{r})=\rec{4\pi\epsilon_0}\int_{\tau} \frac{\vb{P}(\vb{r'})\vdot(\vb{r}-\vb{r'})}{\abs{\vb{r}-\vb{r'}}^3}\dd{\tau'}
    \]
    Si può scrivere per la \eqref{app:eqn:grad_r}
    \[
        V(\vb{r})=\rec{4\pi\epsilon_0}\int_{\tau} \vb{P}\vdot \grad'{\rec{\abs{\vb{r}-\vb{r'}}}}\dd{\tau'}
    \]
    e, ricordando la \eqref{app:eqn:div_scalare_vettore}
    \[
        V(\vb{r})=\rec{4\pi\epsilon_0}\int_{\tau} \grad'\vdot{\Biggl(\frac{\vb{P}}{\abs{\vb{r}-\vb{r'}}}\Biggr)}\dd{\tau'}-\rec{4\pi\epsilon_0}\int_{\tau}{\frac{ \grad'\vdot\vb{P}}{\abs{\vb{r}-\vb{r'}}}}\dd{\tau'}
    \]
    Grazie al teorema della divergenza il primo integrale può essere riscritto, indicando con $S$ la superficie che racchiude il volume
    \[
        \int_{\tau} \grad'\vdot{\Biggl(\frac{\vb{P}}{\abs{\vb{r}-\vb{r'}}}\Biggr)}\dd{\tau'}=\int_{S} \frac{\vb{P}\vdot\vb{n}}{\abs{\vb{r}-\vb{r'}}}\dd{S}
    \]
    Siccome la soluzione al problema dell'elettrostatica deve essere unica e il potenziale deve anche essere uguale a
    \[
        V(\vb{r})=\rec{4\pi\epsilon_0}\int_{S} \frac{\sigma(\vb{r}')}{\abs{\vb{r}-\vb{r'}}}\dd{S'}+\rec{4\pi\epsilon_0}\int_{\tau}\frac{\rho(\vb{r}')}{\abs{\vb{r}-\vb{r'}}}\dd{\tau'}
    \]
    Per confronto fra le integrande si ottiene la tesi.
\end{proof}
Come è intuitivo, se il vettore polarizzazione è uniforme (da leggersi "è indipendente dalla posizione") il volume del dielettrico
è complessivamente neutro e le cariche di polarizzazione si manifestano solo in superficie.

Si vuole ora trovare una relazione fra il vettore polarizzazione e il campo elettrco macroscopico che agisce internamente al dielettrico.
Per farlo, è necessario prima puntualizzare alcuni aspetti del campo elettrico locale.
Il campo agente sulla singola molecola è generato sia dalle cariche libere che dalle molecole circostanti quella considerata:
quest'ultimo contributo dipende fortemente sia dalla posizione che dal tempo.
Si tratta però di posizioni e tempi molto piccoli rispetto al sistema macroscopico
e perciò si sceglie di considerare regioni di spazio piccole rispetto al sistema macroscopico
ma comunque grandi rispetto a lunghezze e tempi atomici: in questo modo la media dei campi elettrici
su queste porzioni di spazio dipende in maniera regolare dalla posizione.
Questa media viene indicata con $\vb{E}_l$ e ci si riferirà sempre a quest'ultima
quando si parlerà di campo elettrico locale.


Per quanto visto, ci si aspetta che il vettore polarizzazione sia proporzionale al campo elettrico locale, ovvero che abbia una forma
\[
    \vb{P}=n\alpha\vb{E}_l \quad\quad\quad n=\dv{N}{\tau},\,\alpha=\alpha_0+\alpha_d
\]
In generale la dipendenza del vettore polarizzazione dal campo elettrico può essere espressa nella forma
\[
    \begin{cases}
        &  P_x=\alpha_{11}E_x+\alpha_{12}E_y+\alpha_{13}E_z \\
        &  P_y=\alpha_{21}E_x+\alpha_{22}E_y+\alpha_{23}E_z \\
        &  P_z=\alpha_{31}E_x+\alpha_{32}E_y+\alpha_{33}E_z
    \end{cases}
\]
la matrice dei coefficenti $\alpha$, non necessariamente costanti, si chiama \textit{tensore di polarizzazione}.
Si dice \textit{perfetto} un dielettrico con matrice di polarizzazione costante.
Questa struttura del vettore polarizzazione diventa particolarmenete utile
nel momento in cui si trattano \textit{solidi cristallini anisotropi}. In alcuni casi questi dielettrici,
che prendono quando ciò si verifica il nome di \textit{ferroelettrici}, possono presentare una polarizzazione elettrica permanente
che segue una \textit{curva di isteresi} in funzione di $\vb{E}$.
In questi materiali si parla di \textit{piezoelettricità}, ovvero la polarizzazione
dipende dalle sollecitazioni meccaniche a cui il cristallo è sottopsto.

Si introduce ora
\begin{defn}[Suscettibilità elettrica]
    Si definisce suscettibilità elettrica
    \[
        \chi=\frac {P}{\epsilon_0E}
    \]
\end{defn}

\begin{obs}
    Per sostanze a bassa densità
    \[
        \chi=\frac{n}{\epsilon_0}\alpha=\frac{n}{\epsilon_0}\Biggl(\alpha_d+\frac{p^2}{3KT}\Biggr)
    \]
\end{obs}
\begin{proof}
    L'ipotesi di bassa densità implica che le interazioni reciproche fra le molecole siano trascurabili, per cui $\vb{E}_l\simeq\vb{E}$. Da questo segue
    \[
        \epsilon_0\chi\vb{E}=\vb{P}=n\Biggl(\alpha_d+\frac{p^2}{3KT}\Biggr)\vb{E}_l \simeq n\Biggl(\alpha_d+\frac{p^2}{3KT}\Biggr)\vb{E}
    \]
    Si ha quindi la tesi.
\end{proof}

Per i liquidi densi invece l'interazione fra molecole non è trascurabile. Si da il seguente lemma di cui viene omessa la dimostrazione.
\begin{lemma}[relazione di Lorentz]
    Nell'ipotesi in cui il campo generato dalle molecole sia puramente dipolare, la distribuzione dei dipoli sia uniforme,
    il momento dei dipoli sia parallelo al campo esterno e che i momenti di dipolo siano tutti uguali fra loro si ha:
    \[
        \vb{E}_l=\vb{E}+\frac{\vb{P}}{3\epsilon_0}
    \]
\end{lemma}

\begin{thm}[relazione di Clausius-Mossotti]
    Per un dielettrico perfetto, nelle ipotesi del lemma
    \[
        \alpha=\frac{3\epsilon_0}{n}\frac{\epsilon_r-1}{\epsilon_r-2}
    \]
\end{thm}
\begin{proof}
    Il vettore polarizzazione vale
    \[
        \vb{P}=n\alpha\Biggl(\vb{E}+\frac{\vb{P}}{3\epsilon_0}\Biggr)
    \]
    da cui
    \[
        \vb{P}=\Biggl(\frac{n\alpha}{1-\frac{n\alpha}{3\epsilon_0}}\Biggr)\vb{E}
    \]
    Per confronto con la definizione di suscettibilità elettrica si ha
    \[
        \chi=\frac{n\alpha}{1-\frac{n\alpha}{3\epsilon_0}}\rec{\epsilon_0}=\frac{3n\alpha}{3\epsilon_0-n\alpha}
    \]
    Si introduce per i dielettrici perfetti l'uguaglianza simbolica $\chi=\epsilon_r-1$,
    che più avanti nel capitolo troverà una giustificazione\footnote{Si vedano le considerazioni in calce
    al teorema \ref{thm:problema-elettrostatica-dielettrici}}.
    \[
        \epsilon_r-1=\frac{3n\alpha}{3\epsilon_0-n\alpha}
    \]
    Risolvendo l'equazione per $\alpha$ si trova la tesi.

\end{proof}


\section{Equazioni di Maxwell nei dielettrici}
\begin{defn}[Spostamento elettrico]
    Si definisce vettore di spostamento elettrico
    \[
        \vb{D}=\epsilon_0\vb{E}+\vb{P}
    \]
\end{defn}

\begin{obs}
    In dielettrici perfetti e isotropi, il vettore di spostamento elettrico ed il campo elettrico sono legati dalla relazione
    \begin{equation}
        \vb{D}=\epsilon\vb{E}
        \label{eqn:rel_D_E}
    \end{equation}
\end{obs}
\begin{proof}
    Nel caso di dielettrici perfetti isotropi dalla definizione si suscettibilità elettrica,
    coerentemente con l'uguaglianza simbolica introdotta prima
    $\vb{D}=\epsilon_0\vb{E}+\vb{P}=(\epsilon_0+\epsilon_0\chi)\vb{E}=\epsilon_0(1+\chi)\vb{E}=\epsilon_0\epsilon_r\vb{E=\epsilon\vb{E}}$.
\end{proof}

\begin{thm}
    In presenza di dielettrici la prima e la terza equazione di Maxwell assumono la seguente forma:
    \begin{equation}
        \div{\vb{D}}=\rho
    \end{equation}
    \begin{equation}
        \curl{\vb{E}}=0
    \end{equation}
\end{thm}
\begin{proof}
    In presenza di dielettrico la struttura del campo elettrico non cambia, per cui la sua circuitazione
    deve continuare ed essere nulla. Per quanto riguarda la prima equazione di Maxwell,
    introducendo la densità delle cariche di polarizzazione $\rho_p$, questa diventa
    \[
        \div{\vb{E}}=\frac{\rho+\rho_p}{\epsilon_0}
    \]
    Siccome la densità di carica di polarizzazione non è nota a priori, dalla \eqref{eqn:relazione_rhop_P},
    tenuto conto che $\epsilon_0$ è costante, si ottiene
    \[
        \begin{split}
            & \div{\epsilon_0 \vb{E}}=\rho-\div{\vb{P}}\\
            & \div{\epsilon_0 \vb{E}+\vb{P}=\rho}
        \end{split}
    \]
    Dalla definizione di vettore di spostamento elettrico, la tesi.
\end{proof}


\section{Problema generale dell'elettrostatica nei dielettrici}
Per semplicità verranno considerati solo dielettrici perfetti e isotropi.
Dalla prima delle equazioni di Maxwell in presenza di dielettrici si evince che
\[
    \Phi_S{\vb{D}}=\int_S\vb{D}\vdot\dd{\vb{S}}=Q_i
\]
ovvero l'equivalente del teorema di Gauss, dove $Q_i$ sono le cariche interne ad $S$ esclue quelle di polarizzazione che,
di fatto, non compaiono dell'equazione di Maxwell. In maniera analoga il teorema di Coulomb può  essere riformulato come
\[
    \vb{D}=\sigma\vu{n}
\]

Fatte queste premesse, si giunge al seguente importante risultato
\begin{thm}
    Nel caso in cui il dielettrico riempia interamente lo spazio, il problema dell'elettrostatica nei dielettrici è analogo a quello nel vuoto con
    \[
        \begin{split}
            &\vb{D}=\vb{D_0} \\
            &\vb{E}=\frac{\vb{E_0}}{\epsilon_r} \\
            &V=\frac{V_0}{\epsilon_r} \\
        \end{split}
    \]
    \label{thm:problema-elettrostatica-dielettrici}
\end{thm}
\begin{proof}
    Nell'ipotesi in cui il dielettrico riempia tutto lo spazio si ha che la \eqref{eqn:rel_D_E}.
    Moltiplicando allora la terza delle equazioni di Maxwell nei dielettrici per $\epsilon$ si ottiene:
    \[
        \curl{\vb{D}}=0
    \]
    Analogamente, dividendo la prima per $\epsilon$ si ottiene:
    \[
        \div{\vb{E}}=\frac{\rho}{\epsilon}
    \]
    Da cui si deduce che il vettore spostamento elettrico è anch'esso conservativo e che, nel caso dei dielettrici,
    il campo elettrico viene scalato semplicemente di un fattore $\epsilon_r$.
\end{proof}
Questo risultato giustifica finalmente l'uguaglianza simbolica $\chi=\epsilon_r-1$: a partire da questa si è dimostrato infatti
che il potenziale in un dielettrico è scalato di un fattore $\epsilon_r$ rispetto al caso nel vuoto e in particolare
questo è vero per un condensatore, come era stato dedotto in conclusione al paragrafo \ref{par:costante_dielettrica}.

Si immagini ora che il dielettrico non occupi tutto lo spazio: il caso più generale
è quello in cui tanti dielettrici diversi occupino diverse porzioni di spazio.
Localmente, ovvero su ciascun dielettrico, i risultati ottenuti col precedente teorema continuano a valere.
Sulle superfici di separazione i campi subiscono una discontinuità e quindi le derivate non sono più definite:
non valgono quindi più le equazioni di Maxwell. Fortunatamente, continua a valere il loro corrispettivo integrale.
Ne consegue che all'interno di ciascun dielettrico continua a valere l'equazione di Poisson
ma per risolvere il problema dell'elettrostatica è necessario determinare preliminarmente le condizioni di raccordo del campo elettrico
sulle superfici di separazione dei vari dielettrici.

\begin{thm}
    Attraversando l'interfaccia fra due dielettrici diversi le componenti tangenti di $\vb{E}$ e $\vb{D}$ sono legate dalla relazione
    \begin{equation}
        \label{eqn:Dn}
        D_{n1}=D_{n2}
    \end{equation}
    \begin{equation}
        \frac{E_{n1}}{E_{n2}}=\frac{\epsilon_2}{\epsilon_1}
    \end{equation}
\end{thm}
\begin{proof}
    Si consideri una superficie di separazione fra due dielettrici $\Sigma$ priva di cariche localizzate e un cilindretto con basi parallele a $\Sigma$ e altezza $\dd{h}$. Il flusso atraverso le superfici laterali può essere trascurato nel limite $\dd{h}\to0$. In assenza di cariche localizzate il teorema di Gauss assume la forma
    \[
        0=\Phi(\vb{D})=\dd{S}\vu{n}_1\vdot\vb{D}_1+\dd{S}\vu{n}_2\vdot\vb{D}_2
    \]
    Tenendo conto che le basi del cilindro sono parallele, per cui $\vu{n}_1=-\vu{n}_2$, e indicando con $D_{ni}$ [$i=1,2$] le proiezioni di $\vb{D}$ sulle normali, si ha:
    \[
        \dd{S}(D_{n1}-D_{n2})=0
    \]
    ovvero
    \[
        D_{n1}=D_{n2}
    \]
    Dalla \eqref{eqn:rel_D_E} si ottiene la tesi.
\end{proof}
Del seguente teorema si omette la dimostrazione in quanto analoga a quella del lemma \ref{lemma:discontinuità_E}.

\begin{thm}
    Attraversando l'interfaccia fra due dielettrici diversi le componenti tangenti di $\vb{E}$ e $\vb{D}$ sono legate dalla relazione
    \begin{equation}
        \label{eqn:Et}
        E_{t1}=E_{t2}
    \end{equation}
    \begin{equation}
        \frac{D_{t1}}{D_{t2}}=\frac{\epsilon_1}{\epsilon_2}
    \end{equation}
\end{thm}
Queste relazioni forniscono anche un metodo per la misurazione dei campo all'interno dei materiali:
è infatti sufficiente praticare nel dielettrico un sottile taglio parallelo (ortogonale) alle linee di
campo del campo elettrico (vettore spostamento elettrico) ed effettuare la misura all'interno della
cavità -il campo misurato sarà uguale al campo macroscopico presente all'interno del materiale.

\begin{thm}[legge di rifrazione delle linee di forza del campo elettrico]
    Chiamato $\theta_i$ l'angolo che $E_{ti}$ forma con la superficie di separazione (con $i=1,2$) si ha
    \[
        \frac{\tan\theta_1}{\tan\theta_2}=\frac{\epsilon_1}{\epsilon_2}
    \]
\end{thm}
\begin{proof}
    La dimostrazione è banale e si ottiene facendo il rapporto fra la \eqref{eqn:Et} e la \eqref{eqn:Dn}
\end{proof}
Una legge analoga vale per $\vb{D}$, considerando che è parallelo ad $\vb{E}$.


\section{Energia elettrostatica nei dielettrici}
L'energia elettrostatica nei dielettrici si ottiene in modo analgo a come si è ottenuta la \eqref{eqn:energia_distribuzione}
considerando però che la disposizione nello spazio delle cariche libere a partire dall'infinito
induce una ridistribuzione delle cariche di polarizzazione che modifica il potenziale.
Bisognerebbe calcolare il lavoro necessario sia a costruire la distribuzione di cariche libere
che il lavoro necessario a costruire la configurazione delle cariche di polarizzazione,
ma siccome il potenziale contiene sia l'informazione relativa all'interazione fra le cariche libere
che l'informazione relativa all'interazione fra cariche libere e di polarizzazione
la \eqref{eqn:energia_distribuzione} continua a definire l'energia elettrostatica
con l'unica differenza che $\rho$ soddisfa la prima equazione di Maxwell in presenza di dielettrici.
Con passaggi analoghi a quelli già visti nel caso del vuoto si può esprimere
\[
U=\int u\dd{\tau}
\]
con
\[
u=\frac{\vb{D}\vdot\vb{E}}{2}
\]

Nel caso in cui il dielettrico sia isotropo si ha
\[
u=\rec{2}\epsilon E^2=\rec{2}\frac{D^2}{\epsilon}
\]





\part{Corrente}
\chapter{Considerazioni generali}
\section{Concetti introduttivi}
I concetti esposti saranno formulati per concretezza nei conduttori metallici,
ma potranno essere estesi senza sforzo a qualsiasi altro tipo di conduttore.

Un conduttore metallico può essere visto come una struttura reticolare tridimensionale di atomi
con un gran numero di elettroni liberi. Per avere una stima di questo numero si consideri del rame,
con densità $\rho=8,9 g/cm^3$ e $A=63,5$ da cui si ricava immediatamente un valore di $8\,10^{22}$ elettroni liberi per centimetro cubo.
Le dimensioni degli elettroni sono molto più piccole delle sensibilità sperimentali oggi raggiunte.
Si possono vedere quindi gli elettroni liberi come un gas contenuto in un recipiente chiuso (il reticolo atomico).
Questi elettroni si muovono disordinatamente e urtano con gli ioni del reticolo portandosi
in equilibrio termico con questi ultimi. Applicando una differenza di potenziale si genera un campo elettrico.
L'effetto di questo campo è che un elettrone che dopo l'urto ha velocità $\vb{v}_T$ viene accelerato fino ad una velocità $\vb{v}'_T$.
Si ha
\[
\Delta\vb{v}=\frac{\vb{f}}{m}\Delta t=\frac{-e\vb{E}}{m}\Delta t
\]
dove l'intervallo di tempo è quello che intercorre fra due urti consecutivi. Complessivamente l'elettrone tra i due urti acquista una velocità di deriva
data dal valor medio di $\Delta\vb{v}$
\[
\vb{v}_d=\frac{\Delta\vb{v}}{2}=\Biggl(\frac{-e\Delta t}{2m} \Biggr)\vb{E}
\]
In linea di principio l'intervallo di tempo è dipendente dal campo elettrico.
In realtà si ha $v_d<<v_T$ (frazioni di millimetri al secondo contro centinaia di chilometri al secondo),
per cui $v_T\simeq v'_T$, ovvero la velocità termica non dipende sensibilmente dal campo elettrico.
Chiamando ora $l$ il libero cammino medio si ha $\Delta t\simeq l/v_T$ che non dipende quindi dal campo elettrico.
Ne segue che la velocità di deriva è costante e proporzionale al campo\footnote{In realtà la conduzione è un fenomeno sostanzialmente quantistico.
Analizzandolo però in ottica classica si può comunque avere un'idea qualitativa dei fenomeni che lo governano.}.

Come si è visto nei conduttori sono gli elettroni di conduzione le cariche che si muovono,
ma per ragioni storiche il fenomeno viene descritto dal punto di vista delle cariche positive fittizie che si muovono in direzione opposta.

\begin{defn} [Corrente elettrica]
Considerato un conduttore nel quale si abbia un movimento ordinato di
cariche, si definisce corrente elettrica
\[
I=\dv{Q}{t}
\]
\end{defn}
Nel sistema internazionale l'unità di misura è l'ampère.


\section{Densità di corrente}
\begin{defn}[densità di corrente]
Si definisce densità di corrente, detto $n$ il numero di portatori di carica $q$ per unità di volume, il vettore
\[
\vb{J}=nq\vb{v}_d
\]
\end{defn}
Si ha che
\[
[J]=\rec{m^3}C\frac{m}{s}=\frac{C}{s}\rec{m^2}=\frac{A}{m^2}
\]
È importante osservare come la densità di corrente risulti sempre parallela al campo elettrico, infatti
le velcità di deriva costituiscono un campo vettoriale parallelo o antiparallelo ad $\vb{E}$ a seconda che
la carica sia positiva o negativa e di conseguenza il prodotto $q\vb{v}_d$ è sempre parallelo ad $\vb{E}$.
In via del tutto generale fra il campo elettrico e la densità di corrente sussiste una relazione del tipo $\vb{J}=\vb{F}(\vb{E})$.
Nel caso particolare dei conduttori detti \textit{lineari} questa relazione diventa
\begin{equation}
    \label{eqn:rel_J_E}
    \vb{J}=\norm{\sigma}\vb{E}
\end{equation}
dove $\norm{\sigma}$ è una matrice detta \textit{tensore di conducibilità} i cui elementi dipendono in linea generale
non dall'intensità del campo elettrico ma dalla sua direzione. I materiali per cui vale esplicitamente questa relazione
sono detti \textit{anisotropi}. Si dicono invece \textit{isotropi}
i materiali per cui questa relazione non dipende dalla direzione del campo elettrico.

L'introduzione del vettore polarizzazione è giustificata dal seguente risultato
\begin{thm}
    \label{thm:I_flusso_J}
    Data una sezione $S$ del conduttore attraversata da corrente si ha che
    \[
        I=\int_S\vb{J}\vdot \dd{\vb{S}}=\Phi_S(\vb{J})
    \]
\end{thm}
\begin{proof}
    Dato un conduttore di sezione $S$ al cui interno sia presente un campo elettrico, le velocità di deriva delle cariche libere $\vb{v}_d$
    costituiscono un campo vettoriale. Si consideri un tubo di flusso elementare di questo campo con sezione $\dd{\vb{S}}$.
    Allora nel tempo $\dd{t}$ si ha che la carica che attraversa il tubo di flusso è
    \[
        \dd{q}=nq\vb{v}_d\vdot\dd{\vb{S}}\dd{t}=\vb{J}\vdot\dd{\vb{S}}\dd{t}
    \]
    Dalla definizione si ha che la corrente elettrica in questa porzione infinitesima di conduttore vale
    \[
        \dd{I}=\dv{q}{t}=\vb{J}\vdot\dd{\vb{S}}
    \]
    Integrando entrabi i membri dell'equazione si ha la tesi.
\end{proof}

\begin{cor}[Equazione di continuità della corrente]
    \begin{equation}
        \label{eqn:continuità}
        \div{\vb{J}}+\pdv{\rho}{t}=0
    \end{equation}
    Che dal punto di vista fisico significa che se in un conduttore si ha una variazione di carica nel tempo,
    questa deve essere dovuta alla carica che fluisce attraverso la superficie che racchiude il volume.
\end{cor}
\begin{proof}
    Si consideri una superficie chiusa $S$ in un conduttore in cui all'istante $t$ sia racchiusa la carica $Q(t)$.
    Se dopo un certo tempo la superficie racchiude una carica $Q(t)-\dd{Q}$ per la conservazione della carica,
    questa deve essere fluita fuori dalla superficie. Quindi, per il teorema appena dimostrato e per il teorema della divergenza
    \[
        -\dv{Q}{t}=\int_S\vb{J}\vdot\dd{\vb{S}}=\int_V\div{\vb{J}}\dd{\tau}
    \]
    D'altra parte $Q(t)=\int_V \rho(x,y,z,t)\dd{\tau}$. È possibile passare con la derivata sotto al segno di integrale, ottenendo
    \[
        \dv{Q}{t}=\int_V\pdv{\rho}{t}\dd{\tau}
    \]
    Uguagliando le espressioni ottenute e ricordando che il ragionamento fatto vale indipendentemente dalla scelta del volume $V$,
    si ha l'uguaglianza degli integrandi, ovvero la tesi.
\end{proof}



\chapter{Corrente stazionaria}
\input{Parti/Corrente/Capitoli/Stazionaria/Stazionaria.tex}

%\chapter{Corrente alternata}


\part{Magnetostatica}
\chapter{Magnetostatica nel vuoto}
\begin{obses}
    Si consideri un sistema fatto in questo modo:
    uno o più circuiti sono fermi e percorsi da corrente stazionaria;
    un circuito di prova ha un piccolo tratto rettilineo $\dd{\vb{l}}$, connesso al resto del circuito
    mediante connessioni flessibili, elettricamente neutro e percorso da corrente $I$.
    Si osserva (e si misura mediante un dinamometro) che $\dd{\vb{l}}$
    risente di una forza $\dd{\vb{F}}$ ad opera degli altri circuiti con le seguenti caratteristiche:
    \begin{enumerate}
        \item $\dd{F} \propto I\dd{l}$;
        \item la direzione di $\dd{\vb{F}}$ è ortogonale a quella di $\dd{\vb{l}}$;
        \item $\dd{\vb{F}}$ dipende dalla posizione e dall'orientamento di $\dd{\vb{l}}$. In particolare esiste sempre
            una direzione di $\dd{\vb{l}}$ tale per cui la forza è nulla e la direzione per cui la forza
            è massima risulta ortogonale a quest'ultima.
    \end{enumerate}
\end{obses}

Queste osservazioni sperimentali costituiscono il punto di partenza per lo sviluppo del magnetismo.


\section{Campo di induzione magnetica - prima legge di Laplace}
Tutta una serie di esperimenti porta a concludere che esista un campo, detto campo di induzione magnetica:
\begin{defn}[Campo di induzione magnetica]
    Si definisce campo di induzione magnetica $\vb{B}$ il responsabile delle forze sentite
    dal tratto di filo $\dd{\vb{l}}$. Questo campo è dipendente dalla posizione ed è
    generato da circuiti nei quali circoli corrente stazionaria.
\end{defn}
\begin{obses}
    Dato un circuito filiforme $l'$, detto $\dd{\vb{l'}}$ una porzione infinitesima di questo circuito
    in posizione $\vb{r'}$
    e posto l'osservatore in posizione $\vb{r}$, si ha che il campo $\vb{B}$ generato da questo circuito vale
    \begin{equation}
        \label{eqn:B}
        \vb{B}=\frac{\mu_0}{4\pi}\oint_{l'}\frac{I\dd{\vb{l'}}\cp(\vb{r}-\vb{r}')}{\abs{\vb{r}-\vb{r}'}^3}
    \end{equation}
    con $\Delta \vb{r}=\vb{r}-\vb{r'}$.
\end{obses}

Il seguente corollario rappresenta un'estrapolazione teorica della situazione sperimentale appena descritta.
\begin{cor}[Legge di Biot Savart / prima legge di Laplace]
    Il campo di induzione magnetica può essere calcolato come somma di contributi elementari prodotti
    dai singoli elementi $\dd{l'}$ del circuito:
    \begin{equation}
        \label{eqn:dB}
        \dd{\vb{B}}=\frac{\mu_0}{4\pi}\frac{I\dd{\vb{l'}}\cp(\vb{r}-\vb{r}')}{\abs{\vb{r}-\vb{r}'}^3}
    \end{equation}
\end{cor}

\begin{cor}
    Facendo cadere l'ipotesi di circuito filiforme si ottiene
    \begin{equation}
        \label{eqn:prima_laplace_non_filiforme}
        \vb{B}(\vb{r})=\frac{\mu_0}{4\pi}\int_{\tau}\frac{\vb{J}(\vb{r'})\cp(\vb{r}-\vb{r}')}{\abs{\vb{r}-\vb{r}'}^3}\dd{\tau}
    \end{equation}
\end{cor}
\begin{proof}
    Ponendo $I=\int_{S'}\vb{J}(\vb{r'})\vdot\dd{\vb{S'}}$ nella \eqref{eqn:B} si ottiene:
    \[
        \vb{B}(\vb{r})=\frac{\mu_0}{4\pi}\int_{l'}\Biggl[\int_S (\vb{J}(\vb{r'})\vdot\dd{\vb{S'}}) \frac{\dd{\vb{l'}}\cp(\vb{r}-\vb{r}')}{\abs{\vb{r}-\vb{r}'}^3}\Biggr]=
        \frac{\mu_0}{4\pi}\int_{\tau'}\frac{\vb{J}(\vb{r'})\cp(\vb{r}-\vb{r}')}{\abs{\vb{r}-\vb{r}'}^3}\dd{\tau'}
    \]
\end{proof}
È importante osservare come la sorgente di questo campo siano le cariche in movimento.


\section{Seconda equazione di Maxwell}
In analogia con quanto fatto col campo elettrico si vuole ora studiare
il flusso del campo di induzione magnetica attraverso una superficie chiusa.
\begin{thm}[Seconda equazione di Maxwell stazionaria nel vuoto]
    Nel vuoto il campo $\vb{B}$ generato da un circuito attraversato da corrente stazionaria è solenoidale, ovvero
    \begin{equation}
        \div{\vb{B}}=0
    \end{equation}
\end{thm}
\begin{proof}
    Per la \eqref{app:eqn:grad_r}
    \[
        \vb{B}(x,y,z)=-\frac{\mu_0}{4\pi}\int_\tau \vb{J}(\vb{r}')\cp\grad[\rec{\abs{\vb{r}-\vb{r}'}}]\dd{\tau'}
    \]
    Per la formula \eqref{app:eqn:curl_scalare_vettore}, l'integranda può essere riscritta come
    \[
        \vb{J}(\vb{r}')\cp\grad[\rec{\abs{\vb{r}-\vb{r}'}}]=
        \rec{\abs{\vb{r}-\vb{r}'}}[\curl{\vb{J}}(\vb{r}')]-\curl[\frac{\vb{J}(\vb{r}')}{\abs{\vb{r}-\vb{r}'}}]=
        -\curl[\frac{\vb{J}(\vb{r}')}{\abs{\vb{r}-\vb{r}'}}]
    \]
    in quanto il vettore densità di corrente dipende solo dalle coordinate $(x',y',z')$ mentre il gradiente opera sulle $(x,y,z)$.
    Per lo stesso motivo, si può affermare che
    \[
        \vb{B}(x,y,z)=\frac{\mu_0}{4\pi}\int_{\tau'} \curl[\frac{\vb{J}(\vb{r}')}{\abs{\vb{r}-\vb{r}'}}]=
        \curl[\frac{\mu_0}{4\pi}\int_{\tau'} \frac{\vb{J}(\vb{r}')}{\abs{\vb{r}-\vb{r}'}}]
    \]
    Ma la divergenza di un rotore è nulla, perciò segue immediatamente la tesi.
\end{proof}
\begin{cor}
    \label{cor:flusso_B}
    Nel vuoto, il flusso attraverso una superficie chiusa di $\vb{B}$, generato da un circuito attraversato da corrente stazionaria,
    è nullo.
\end{cor}
\begin{proof}
    Integrando sul volume la divergenza di $\vb{B}$
    e applicando il teorema della divergenza, si ha la tesi.
\end{proof}

La seconda equazione di Maxwell permette di ottenere alcune caratteristiche fondamentali del campo di induzione magnetica.
\begin{cor}
    Date due superfici orientate $S$ ed $S'$ aventi lo stesso contorno e stessa orientazione si ha
    \[
        \Phi_S(\vb{B})=\Phi_{S'}(\vb{B})
    \]
    Ovvero, il flusso di $\vb{B}$ dipende esclusivamente da contorno e orientamento.
\end{cor}
\begin{proof}
    Si consideri una superficie chiusa e la si divida in due superfici $S$ ed $S'$:
    queste due avranno orientazione opposta e condivideranno il contorno.
    Per la linearità del flusso e per il corollario appena dimostrato
    si ha $0=\Phi_{S\cup S'}(\vb{B})=\Phi_{S}(\vb{B})+\Phi_{S'}(\vb{B})$,
    ovvero $\Phi_{S}(\vb{B})=-\Phi_{S'}(\vb{B})$.
    Cambiando orientazione ad una delle due superfici si cambia il segno del relativo flusso ottenendo la tesi.
\end{proof}
Si parla quindi di \textit{flusso concatenato ad un contorno}, senza far riferimento alla superficie.
\begin{cor}
    Le linee di forza del campo di induzione magnetica sono chiuse.
\end{cor}
\begin{proof}
    Per assurdo, si cosideri una linea di forza di $\vb{B}$ non chiusa:
    deve esistere un punto sorgente per tale linea. È possibile prendere allora una superficie chiusa piccola a piacere
    attorno a questo punto sorgente attraverso la quale il flusso del campo è diverso da 0.
\end{proof}
Da questo segue immediatamente che i tubi di flusso per il campo $\vb{B}$ non hanno nè inizio nè fine.


\section{Il potenziale vettore e la quarta equazione di Maxwell}
\label{par:potenziale_vettore}
Ancora in analogia col campo elettrico, si vuole vedere se sia possibile descrivere il campo di induzione magnetica
come l'applicazione di un'operatore differenziale ad una opportuna funzione.
In generale non è detto che $\curl{\vb{B}}=0$ e che quindi il campo di induzione magnetica sia esprimibile come
gradiente di una funzione scalare.
Si introduce quindi il seguente potenziale
\begin{defn}
    Si definisce potenziale vettore quella funzione vettoriale $\vb{A}$ che soddisfi la seguente equazione
    \begin{equation}
        \label{eqn:def_potenziale_vettore}
        \curl{\vb{A}}=\vb{B}
    \end{equation}
\end{defn}
Ricordando che la divergenza di un rotore è identicamente nulla,
si ha che la divergenza del rotore di $\vb{A}$ è nulla, ovvero condizione necessaria affinchè sia valida la
definizione è che la divergenza di $\vb{B}$ sia nulla. Questo è garantito dalla seconda equazione di Maxwell.
\begin{obs}[Trasformazione di gauge]
    Il potenziale vettore è definito a meno del gradiente di una funzione scalare.
    Ovvero se $\vb{A}$ è un potenziale vettore per il campo $\vb{B}$, anche $\vb{A'}=\vb{A}+\grad{f}$ lo è, con $f$ una funzione scalare.
\end{obs}
\begin{proof}
    La dimostrazione segue direttamente dal fatto che il rotore del gradiente è nullo.
\end{proof}
Ovvaiamente, il potenziale vettore è definito anche a meno di una costante additiva.

È possibile fare la seguente osservazione
\begin{obs}
    Dato un potenziale vettore $\vb{A}$ è possibile definire a partire da questo un nuovo potenziale vettore $\vb{A}'$
    con divergenza nulla come indicato nella precedente osservazione. La condizione affinchè ciò avvenga è che
    \[
        \laplacian{f}=\div{\vb{A}}
    \]
\end{obs}
\begin{proof}
    La dimostrazione si ottiene imponendo che la divergenza di $\vb{A'}=\vb{A}+\grad{f}$ sia $0$.
\end{proof}

Il potenziale vettore gode di alcune proprietà generali.
\begin{thm}
    Dato un potenziale vettore $\vb{A}$, l'equazione definitoria locale \eqref{eqn:def_potenziale_vettore} ha una corrispondente integrale
    \[
        \int_S \vb{B}\vdot\vu{n}\dd{S}=\oint_l\vb{A}\vdot\dd{\vb{l}}
    \]
    Dove $S$ è una superficie aperta orientata e $l$ è il suo contorno. Questa relazione mostra come la circuitazione di $\vb{A}$
    sia uguale al flusso concatenato di $\vb{B}$.
\end{thm}
\begin{proof}
    La dimostrazione è immediata e si ottiene integrando sulla superficie $S$ entrambi i membri della
    \eqref{eqn:def_potenziale_vettore} e applicando il teorema di Stokes all'integrale del rotore di $\vb{A}$
\end{proof}
Il fatto che il potenziale vettore abbia divergenza nulla implica che le proprietà dedotte per $\vb{B}$ valgano
anche per $\vb{A}$.

Il seguente risultato fornisce un'espressione esplicita per il potenziale vettore, con divergenza nulla.
\begin{thm}
    La seguente espressione esplicita per il potenziale vettore ha divergenza nulla
    \begin{equation}
        \vb{A}(\vb{r})=\frac{\mu_0}{4\pi}\int\frac{\vb{J}(\vb{r}')}{\abs{\vb{r}-\vb{r}'}}\dd{\tau'}
        \label{eqn:A_I}
    \end{equation}
\end{thm}
\begin{proof}
    Per quanto visto nella dimostrazione della seconda equazione di Maxwell $\vb{B}$ può essere scritto
    come rotore di una funzione: dalla definizione di potenziale vettore segue quindi immediatamente
    la prima parte della tesi.
    Resta da dimostrare che, nella forma trovata, il potenziale vettore ha divergenza nulla. Per la \eqref{app:eqn:div_scalare_vettore}
    \[
        \div{\vb{A}}=\frac{\mu_0}{4\pi}\int_{\tau'} \div[\frac{\vb{J}(\vb{r}')}{\abs{\vb{r}-\vb{r}'}}]\dd{\tau'}=
        \frac{\mu_0}{4\pi}\int_{\tau'} \rec{\abs{\vb{r}-\vb{r}'}}\div{\vb{J}(\vb{r}')}\dd{\tau'}+
        \frac{\mu_0}{4\pi}\int_{\tau'} \grad{\rec{\abs{\vb{r}-\vb{r}'}}}\vdot\vb{J}(\vb{r}')\dd{\tau'}
    \]
    Il primo integrale è nullo perchè la divergenza opera su $\vb{r}$ mentre $\vb{J}$ è riferito $\vb{r'}$.
    Per quanto riguarda il secondo integrale invece, per le \eqref{app:eqn:grad_r} si può sostituire $\grad$
    con $-\grad'$. Invertendo la \eqref{app:eqn:div_scalare_vettore} si ottiene
    \[
        \div{\vb{A}}=-\frac{\mu_0}{4\pi}\int_{\tau'}\grad'\vdot\Biggl[\frac{\vb{J}(\vb{r}')}{\abs{\vb{r}-\vb{r}'}}\Biggr]-
        \rec{\abs{\vb{r}-\vb{r}'}}\grad'\vdot\vb{J}(\vb{r}') \dd{\tau'}
    \]
    Nel caso stazionario la divergenza di $\vb{J}$ è nulla. Per il teorema della divergenza si ha però
    \[
        \div{\vb{A}}=-\frac{\mu_0}{4\pi}\int_{S'}\frac{\vb{J}(\vb{r}')\vdot\vu{n}\dd{S'}}{\abs{\vb{r}-\vb{r}'}}
    \]
    Nell'intergale di volume, $\tau'$ deve essere abbastanza grande da contenere tutti i circuiti sui quali la densità di
    corrente sia diversa da $0$. Se tutte le linee di corrente sono al finito, e quindi la superficie contiene tutte le linee,
    l'integrale è nullo e di conseguenza la divergenza del potenziale vettore è nullo.
\end{proof}
Se la densità di corrente è localizzata solo su conduttori filiformi costituenti un circuito
si ha $\vb{J}\dd{\tau'}=JS\dd{\vb{l}}'=I\dd{\vb{l}}'$ e allora il potenziale vettore diviene
\begin{equation}
    \label{eqn:potenziale_vettore_circuito_lineare}
    \vb{A}(\vb{r})=\frac{\mu_0}{4\pi}\oint_{l'}\frac{I\dd{\vb{l}'}}{\abs{\vb{r}-\vb{r}'}}
\end{equation}
Nel caso di una spira piana posta molto lontano dall'osservatore, il potenziale vettore assume una forma particolarmente
semplice. È necessario introdurre preliminarmente la seguente definizione, che rivestirà un ruolo
centrale nei prossimi paragrafi.
\begin{defn}[Momento magnetico]
    Si definisce il vettore momento magnetico come
    \[
        \vb{m}=I\vb{S}
    \]
    dove il versore che definisce il verso di $\vb{S}$ è definito positivo quando vede girare la corrente in senso
    antiorario.
\end{defn}
\begin{cor}
Il potenziale vettore generato da una spira piana percorsa da corrente stazionaria $I$ posta molto
lontano dall'osservatore, in modo tale che la distanza sia molto maggiore delle dimensioni lineari
della spira, è
    \begin{equation}
        \label{eqn:potenziale_vettore_momento_magnetico}
        \vb{A}(\vb{r})=\frac{\mu_0}{4\pi}\frac{\vb{m}\cp\vb{r}}{r^3}
    \end{equation}
\end{cor}
\begin{proof}
    Il fatto che la distanza $r$ fra spira ed osservatore sia molto maggiore delle dimensioni lineari della spira
    implica che nella \eqref{eqn:potenziale_vettore_circuito_lineare} $\abs{\vb{r}-\vb{r}'}\simeq r$. Allora,
    usando la \eqref{app:eqn:grad_cp}
    \[
        \vb{A}(\vb{r})=\frac{\mu_0 I}{4\pi}\oint_l \frac{\dd{\vb{l}'}}{r}=
        \frac{\mu_0 I}{4\pi}\int_S \grad(\rec{r})\cp\dd{\vb{S}}\simeq
        \frac{\mu_0 I}{4\pi} \grad(\rec{r})\cp\vb{S}
    \]
    dove nell'ultimo passaggio si è fatto uso del fatto che la spira è piccola. Esplicitando il gradiente
    \[
        \vb{A}=\frac{\mu_0 I}{4\pi}\frac{-\vb{r}}{r^3}\cp\vb{S}=
        \vb{A}=\frac{\mu_0 }{4\pi}\frac{I\vb{S}\cp\vb{r}}{r^3}=
        \vb{A}=\frac{\mu_0 }{4\pi}\frac{\vb{m}\cp\vb{r}}{r^3}
    \]
\end{proof}
In queste espressioni la costante arbitraria di integrazione è scelta in modo che il potenziale sia nullo all'infinito.

\begin{thm}[Quarta equazione di Maxwell stazionaria nel vuoto]
    Nel vuoto vale che
    \begin{equation}
        \curl{\vb{B}}=\mu_0\vb{J}
    \end{equation}
\end{thm}
\begin{proof}
    Si può esprimere il rotore di $\vb{B}$ come rotore del rotore di $\vb{A}$. Per la \eqref{app:eqn:curl_curl},
    ricordando che il potenziale vettore può essere scelto con divergenza nulla
    \[
        \curl{\vb{B}}=\curl\curl{\vb{A}}=-\laplacian{A}+\grad(\div{\vb{A}})=-\laplacian{A}
    \]

    Quindi
    \[
        \curl{\vb{B}}=-\laplacian{\frac{\mu_0}{4\pi}\int_{\tau'} \frac{\vb{J}(\vb{r}')}{\abs{\vb{r}-\vb{r}'}}}\dd{\tau'}=
        -\frac{\mu_0}{4\pi}\int_{\tau'} \vb{J}(\vb{r}')\laplacian[\rec{\abs{\vb{r}-\vb{r}'}}]\dd{\tau'}
    \]
    Per la \eqref{app:eqn:laplacian_r} si ha la tesi.

\end{proof}
Questa equazione è valida solo nel caso stazionario, infatti applicando la divergenza ad entrambi i membri si ha $\mu_0\div{\vb{J}}=0$:
affinchè la quarta equazione di Maxwell sia valida quindi è necessario che la divergenza di $\vb{J}$ sia nulla, ovvero
è necessario trovarsi nel caso stazionario.

Per esprimere la forma intergrale della quarta equazione di Maxwell è necessario introdurre due definizioni.
\begin{defn}[Corrente concatenata ad un contorno]
    Data una linea chiusa $l$, si dicono correnti concatenate a $l$ le correnti $I_i$ che intersecano qualunque superficie che abbia $l$ come contorno
\end{defn}
\begin{defn}[Grado di concatenazione]
    Si definisce grado di concatenazione $n_i$ il numero di volte che una linea chiusa $l$ gira intorno ad una corrente concatenata $I_i$
\end{defn}

\begin{cor}[Teorema della circuitazione di Ampère]
    Prese delle correnti concatenate $I_i$ con segno positivo quando vedono girare il contorno orientato $l$ in senso antiorario
    \begin{equation}
        \oint_l \vb{B}\vdot\dd{\vb{l}}=\mu_0 \sum I_i n_i
    \end{equation}
    \label{teo:Ampère}
\end{cor}
\begin{proof}
    Si consideri una curva chiusa semplice $l$ ed una superficie $S$ che abbia $l$ come contorno,
    orientata in modo da vedere il verso di $l$ antiorario. Si calcoli il flusso di ambo i membri
    della quarta equazione di Maxwell attraverso questa superficie.
    Il primo membro, per il teorema di Stokes
    \[
        \int_S \curl{\vb{B}}\vdot\vu{n}\dd{S}=\oint_l\vb{B}\vdot\dd{\vb{l}}
    \]
    Per quanto riguarda il secondo membro invece è necessario osservare che il flusso di $\vb{J}$ è non nullo
    solo in quelle porzioni di superficie, indicate con $\Delta S_i$, attraversate dai condutori
    che portano le correnti che generano $\vb{J}$. Si ha quindi
    \[
        \mu_0 \int_S \vb{J}\vdot\dd{\vb{S}}=\mu_0 \sum_i \Biggl[\int_{\Delta S_i} \vb{J}_i \vdot\dd{\vb{S}} \Biggr]=\mu_0 \sum_i I_i
    \]
    Se la curva non fosse semplice bisognerebbe tenere conto delle concatenazioni multiple delle correnti mediante il grado di concatenazione.
\end{proof}

Grazie a quanto visto fin'ora è possibile dedurre tre equazioni analoghe a quella di Poisson
per il potenziale vettore nel caso statico, una per ogni componente. A causa di questa analogia,
i metodi risolutivi sono identici a quelli visti nel caso dell'equazione di Poisson elettrostatica.
\begin{thm}[Equazione generale del potenziale vettore statico]
    Dato un potenziale vettore $\vb{A}$ con divergenza nulla,
    \begin{equation}
        \laplacian \vb{A}=-\mu_0\vb{J}
    \end{equation}
\end{thm}
\begin{proof}
    Per la quarta equazione di Maxwell
    \[
        \curl{\vb{B}}=\curl\curl{\vb{A}}=\mu_0\vb{J}
    \]
    Per la \eqref{app:eqn:curl_curl}, nell'ipotesi di divergenza nulla, si ha $\curl\curl{\vb{A}}=-\laplacian{\vb{A}}$
    Sostituendo, si ottiene la tesi.
\end{proof}
Questa equazione è equivalente alla seconda ed alla quarta equazione di Maxwell in quanto, come già osservato,
è necessario che la divergenza di $\vb{B}$ sia zero affinchè il campo possa essere scritto come rotore di
un potenziale.

\begin{example}
    Si consideri un solenoide infinito, rettilineo, a spire serrate e uniformemente avvolte, posto in modo tale
    che il suo asse coincida con l'asse $x$. Si indichi con $I$ la corrente circolante nel solenoide
    e con $n$ il numero di spire per untià di lunghezza. Il sistema ha simmetria cilindrica e questo comporta,
    tenendo conto del fatto che il solenoide essendo infinito non ha estremità, che $\vb{B}$ internamente
    al solenoide sia indipendente dalla coordinata $x$ e che le sue linee di forza siano parallele all'asse $x$.
    Questo non è in contraddizione col fatto che le linee di forza del campo devono essere chiuse: il solenoide
    infinito è un'astrazione di un sitema costituito da un solenoide molto lungo - che quindi ha estremità e
    di conseguenza linee di forza non perfettamente parallele all'asse $x$.
    Si consideri ora un percorso rettangolare interno al solenoide, con due lati $AB$ e $CD$ di lunghezza $l$ paralleli
    all'asse $x$: per il teorema di Ampère la circuitazione di $\vb{B}$ vale
    \[
        \oint_{ABCD}\vb{B} \vdot \dd{\vb{l}}=B_{AB}l-B_{CD}l=\mu_0\sum I^{conc}=0
    \]
    in quanto nessuna corrente è concatenata al percorso. Da questo segue che $B_{AB}=B_{CD}$, a prescindere dalla
    posizione e dall'orientamento del percorso: il campo di induzione magnetica all'interno del solenoide non dipende
    nemmeno dalla distanza rispetto all'asse ed è perciò uniforme.

    Si consideri ora il percorso posizionato in modo tale da avere il lato $AB$ interno al solenoide ed il lato
    $CD$ esterno. La circuitazione in questo caso vale
    \[
        \oint_{ABCD}\vb{B} \vdot \dd{\vb{l}}=B_{AB}l-B_{CD}l=\mu_0\sum I^{conc}=\mu_0 nl I
    \]
    Facendo una sezione del solenoide si ottengono due file parallele di conduttori: la corrente nei conduttori
    appartententi ad una di queste file è entrante rispetto al piano che realizza la sezione, la corrente nei
    conduttori dell'altra fila è uscente. Ponendosi ad una distanza tale da poter trascurare il diametro del
    solenoide, i contrubuti che le correnti in queste due file di conduttori apportano a $\vb{B}$ sono uguali in
    modulo ma opposte in segno\footnote{Con un ragionamento analogo si comprende intuitivamente come mai
    all'interno del solenoide le linee di forza siano parallele all'asse $x$. Si ha infatti che i contributi
    al campo magnetico hanno in questo caso lo stesso segno.}.
    Ne segue che il campo magnetico esterno al solenoide è molto piccolo. Trascurandone
    il contrubuto si ha quindi che $B_{AB}=B\simeq\mu_0 n I$.
\end{example}


\section{Il potenziale scalare}
Sorge spontaneo a questo punto chiedersi se e sotto quali condizioni sia possibile definire un
potenziale scalare $\phi$ per il campo magnetico, in modo tale che $-\grad\phi=\vb{B}$.
Condizione necessaria e sufficiente affinchè un campo vettoriale definito in un dominio $D$
sia esprimibile come gradiente di una funzione scalare è che il campo sia irrotazionale
e che il dominio sia semplicemente connesso.
La quarta equazione di Maxwell mostra chiaramente che il $\curl{\vb{B}}=0$ quando $\vb{J}=0$.
Quando la densità di corrente che genera il campo di induzione magnetica è localizzata
al finito in un dominio $D$, è possibile prendere un insieme $B$ che la contenga completamente:
sul dominio semplicemente connesso $D'=D\setminus B$ il campo è ora irrotazionale e può quindi
essere definito un potenziale scalare.
\begin{thm}
    Sotto le opportune ipotesi di dominio, $\vb{B}=-\grad\phi$ con
    \[
        \phi=-\frac{\mu_0 I}{4\pi}\Omega
    \]
    a meno di una costante additiva ($\Omega$ è l'angolo solido).
\end{thm}
\begin{proof}
    Dalla definizione di potenziale $-\dd{\phi}=\vb{B}\vdot\dd{\vb{l}}$. Esplicitando il campo
    \[
        -\dd{\phi}=\frac{\mu_0}{4\pi}\Biggl(\oint_l \frac{I\dd{\vb{l}'}\cp(\vb{r}-\vb{r}')}{\abs{\vb{r}-\vb{r}'}^3}\Biggr)\vdot\dd{\vb{l}}
    \]
    Uno spostamento $\dd{\vb{l}}$ di un osservatore nel campo è equivalente ad uno spostamento $\dd{\vb{s}}=-\dd{\vb{l}}$
    del circuito che genera il campo, da cui
    \[
        \dd{\phi}=\frac{\mu_0 I}{4\pi}\oint_l \dd{\vb{l}'}\cp\frac{(\vb{r}-\vb{r}')}{\abs{\vb{r}-\vb{r}'}^3}\vdot\dd{\vb{s}}
    \]
    per la \eqref{app:eqn:vdot_cp} l'integranda può essere riscritta
    \[
        \dd{\vb{l}'}\cp\frac{(\vb{r}-\vb{r}')}{\abs{\vb{r}-\vb{r}'}^3}\vdot\dd{\vb{s}}=
        \dd{\vb{l}'}\cp\dd{\vb{s}}\vdot\frac{-(\vb{r}-\vb{r}')}{\abs{\vb{r}-\vb{r}'}^3}=
        \dd{\vb{S}} \vdot \frac{-(\vb{r}-\vb{r}')}{\abs{\vb{r}-\vb{r}'}^3}
    \]
    dove $\dd{\vb{S}}$ è l'elemento di superficie spazzato dall'elemento di circuito $\dd{\vb{l}'}$ nello spostamento $\dd{\vb{s}}$.
    Il prodotto scalare che compare in questa espressione rappresenta la proiezione dell'elemento di superficie
    sul raggio-vettore che va da $\vb{r}'$ a $\vb{r}$, ovvero dalla posizione in cui si trova l'osservatore alla posizione
    dell'elemento di superficie. Ma allora l'integranda non è altro che l'angolo solido infinitesimo sotto al quale
    l'osservatore viene "visto" da $\dd{\vb{S}}$. L'integrale, che è esteso a tutto il circuito rappresenta dunque
    l'angolo solido $\dd{\Omega}$ sotto al quale l'osservatore viene visto dalla superficie $\dd{\Sigma}$ spazzata da tutto il circuito
    nello spostamento $\dd{\vb{s}}$. Per l'osservatore allora il circuito si sposta variando la propria posizione di un
    angolo solido $-\dd{\Omega}$. In conlcusione
    \[
        \dd{\phi}=-\frac{\mu_0 I}{4\pi}\dd{\Omega}
    \]
    da cui segue la tesi.
\end{proof}

\begin{cor}
    Nell'ipotesi in cui l'osservatore sia molto lontano dalla spira, indicando con $\vb{r}$ il raggio-vettore
    che va dal circuito all'osservatore, si ha
    \[
        \phi=\frac{\mu_0}{4\pi}\frac{\vb{m}\vdot\vb{r}}{r^3}
    \]
\end{cor}

\begin{proof}
    Nell'ipotesi in cui le dimensioni lineari della spira siano molto minori di $r$, si ha
    \[
        \Omega=-\frac{\vb{S}\vdot\vb{r}}{r^3}
    \]
    Sostituendo nell'espressione del potenziale scalare si ha la tesi.
\end{proof}
Si osservi la completa analogia che esiste fra questa espressione e l'espressione del potenziale prodotto da
un dipolo elettrico: come si vedrà nel paragrafo dedicato all'interazione fra campi magnetici e circuiti
questa analogia è molto più profonda.
È un puro esercizio di calcolo mostrare che questa approssimazione per il potenziale scalare e
l'approssimazione \eqref{eqn:potenziale_vettore_momento_magnetico} per il potenziale vettore
producono lo stesso campo.


\section{Forza di Lorentz - seconda legge di Laplace}
I risultati sperimentali esposti nell'introduzione al capitolo si spiegano ipotizzando che il 
campo di induzione magnetica determini una forza nella forma:
\begin{equation}
    \dd{\vb{F}}=I\dd{\vb{l}}\cp\vb{B}
\end{equation}
Questa equazione è detta \textit{seconda legge di Laplace} e consente di misurare $\vb{B}$ (ne costituisce quindi
la definizione operativa).

Si dimostra il seguente risultato
\begin{thm}[Forza di Lorentz]
    Data una carica puntifome $q$ che si muove con velocità $\vb{v}$ in un campo di induzione magnetica, questa carica risente di una forza
    \begin{equation}
        \label{eqn:f_lorentz}
        \vb{F}=q\vb{v}\cp\vb{B}
    \end{equation}
\end{thm}
\begin{proof}
    Si consideri un tratto di circuito di lunghezza $\dd{l}$, sezione $\dd{S}$ e -quindi- volume  
    $\dd{\tau}=\dd{l}\dd{S}$. Per il teorema \ref{thm:I_flusso_J}, la seconda legge di Laplace può essere riscritta come
    \[
        \dd{\vb{F}}=\vb{J}\dd{S}\dd{l}\cp\vb{B}=\vb{J}\dd{\tau}\cp\vb{B}
    \]
    Inoltre per la definizione di $\vb{J}$ si ha
    \[
        \vb{J}\dd{\tau}=nq\vb{v}_d\dd{\tau}=\dd{N}q\vb{v}_d
    \]
    dove $\dd{N}=n\dd{S}\dd{l}$ rappresenta il numero di portatori di carica nel tratto $\dd{l}$ del circuito. 
    Si ha quindi $\dd{\vb{F}}=\dd{N}q\vb{v}_d\cp\vb{B}$. Se si prende in esame una singola carica, 
    la velocità di deriva coincide con la velocità della carica. 
    Intregrando per ottenere la forza totale  e considerando che il numero totale di cariche è appunto $1$ si ottiene la tesi.
\end{proof}
La relazione di Lorentz è in realtà più generale rispetto alla seconda legge di Laplace: quest'ultima infatti vale
solo qualora $\vb{B}$ non vari significativamente nel tratto $\dd{l}$, mentre la legge di Lorentz è locale. Il prezzo
da pagare è una maggiore difficoltà nella misurazione del campo di induzione magnetica in quanto nel primo caso viene
usato come sonda un circuito percorso da corrente, nel secondo invece una carica in movimento.
\begin{cor}
    La forza di Lorentz non compie alcun lavoro.
\end{cor}
\begin{proof}
    Quando la carica è ferma, non risente della forza di Lorentz. Quando è in movimento, la forza di Lorentz è ortogonale alla velocità.
\end{proof}
Questa forza quindi cambia la direzione della carica in movimento ma non ne modifica la velocità (in modulo).

Grazie a quanto visto ora si può determinare l'unità di misura del campo di induzione magnetica, chiamata \textit{tesla}.
\[
    [B]=N\rec{Cm/s}=\frac{m}{m}N\rec{Cm/s}=\frac{Vs}{m^2}=T
\]
Il prodotto $Vs$ prende il nome di \textit{weber (Wb)}. 
Frequentemente, per il campo di induzione elettromagnetica si usa il \textit{gauss} $1T=10^4G$.


\section{Interazioni fra circuiti e campi magnetici}
Lo studio delle interazioni fra circuiti e campi magnetici è diviso in due categorie di fenomeni:
interazione fra un circuito percorso da corrente stazinaria con un campo magnetico esterno e
intrazione fra un circuito percorso da corrente stazionaria con un altro circuito percorso da corrente stazionaria.

\subsubsection{Interazione circuito-campo magnetico}
L'ipotesi fondamentale per lo studio di questi fenomeni è che il circuito in esame sia dotato di un generatore
di forza elettromotrice che mantenga costante nel tempo la corrente che attraversa il circuito.

Si vuole ora studiare il calcolo delle sollecitazioni meccaniche su una spira rigida percorsa da corrente $I$.
Si consideri quindi una spira $l$ percorsa da una corrente $I$ e immersa in un campo magnetico $\vb{B}$.
Si supponga che ogni tratto infinitesimo della spira $\dd{\vb{l}}$ compia uno spostamento infinitesimo
$\dd{\vb{s}}=\dd{\vb{s}}(\dd{\vb{l}})$, tale da portare la spira della configurazione $l'$.
Per far compiere alla spira questo spostamento senza che la sua energia cinetica vari è necessario applicare dall'esterno una forza
\[
    \dd{f}=-\dd{F}=-I\dd{\vb{l}}\cp\vb{B}
\]
Ovvero, è necessario compiere un lavoro, che risulta quindi in una variazione di energia potenziale
\[
    \dd{U}=\dd{L}=\oint\dd{\vb{f}}\vdot\dd{\vb{s}}=-\oint I\dd{\vb{l}}\cp\vb{B}\vdot\dd{\vb{s}}=
    -\oint I\dd{\vb{s}}\cp\dd{\vb{l}}\vdot\vb{B}=I\oint(\dd{\vb{l}}\cp\dd{\vb{s}})\vdot\vb{B}
\]
dove è stata usata la proprietà \eqref{app:eqn:vdot_cp}.
Il termine fra parentesi nell'ultimo membro della catena di uguaglianze rappresenta l'elemento di superficie $\dd{\vb{S}}$
della superficie laterale $\dd{\Sigma}$ di un solido le cui basi hanno spigoli $l$ ed $l'$.
L'integrale ottenuto è allora il flusso di $\vb{B}$ attraverso questa superficie laterale.
\[
    \dd{U}=\Phi_{\dd{\Sigma}}(\vb{B})
\]
Chiamando $\Sigma$ la base con spigolo $l$ e $\Sigma'$ la base con spigolo $l'$, siccome per il corollario \ref{cor:flusso_B}
il flusso di $\vb{B}$ attraverso una superficie chiusa deve essere nullo si ha
\[
    \Phi_{\dd{\Sigma}}=\Phi_\Sigma-\Phi_{\Sigma'}=-\dd{\Phi}
\]
Dove $\Phi_{\dd{\Sigma}}$ è il flusso entrante da $\Sigma$, mentre gli altri sono flussi uscenti dalle rispettive superfici.
Si ha quindi che a meno di una costante additiva arbitraria
\[
    U=-I\Phi(\vb{B})
\]
Dove il flusso è riferito alla superficie di una spira con contorno $l$ considerata positiva quando vede la corrente ruotare in sesno antiorario.
Ponendosi abbastanza lontano dalla spira in modo da poter considerare piccola la sua superficie
si può approssimare $\Phi(\vb{B})=\vb{B}\vdot\vb{S}$. In questo modo la relazione appena trovata diventa
\[
    U=-I\vb{S}\vdot\vb{B}=-\vb{m}\vdot\vb{B}
\]
Ma questa è un'espressione esattamente analoga a quanto trovato nel paragrafo sul dipolo elettrico \eqref{eqn:U_dipolo}.
Con passaggi identici si arriva allora a dimostrare
\begin{thm}
    Per una spira in un campo di induzione magnetica si ha che
    \begin{equation}
        \vb{F}=(\vb{m}\vdot\grad)\vb{B}
    \end{equation}
    \begin{equation}
        \vb{M}=\vb{m}\cp\vb{B}
    \end{equation}
\end{thm}
Si osservi quindi che esiste una completa analogia fra il dipolo in elettrostatica e la spira percorsa da
corrente in magnetostatica: sia la forma delle sollecitazioni meccaniche che quella del potenziale
(e di conseguenza del campo) prodotto
sono formalmente identiche a patto di usare $\vb{p}$ o $\vb{m}$ a seconda del contesto. Questo
risultato è detto \textit{teorema di equivalenza di Ampère} e sancisce la completa analogia
fra una spira da percorsa da corrente ed un "dipolo magnetico".\

Il calcolo del momento torcente può essere svolto in maniera immediata, e molto istruittiva, nel caso di una spira rettangolare
immersa in un campo magnetico costante.
\begin{example}
    Si supponga che il campo di induzione magnetica esterno $\vb{B}$ sia costante e che la spira sia rettangolare
    (con i lati 1 e 3 di lunghezza $b$ ed i lati 2 e 4 di lunghezza $a$), orientata in modo che i lati 2 e 4
    siano ortogonali a $\vb{B}$. I lati 1 e 3 sono soggetti ad una coppia di forze uguale ed opposta con braccio nullo;
    anche i lati 2 e 4 sono soggetti ad una coppia di forze uguale ed opposta,
    il braccio però vale $b\sin\theta$ con $\theta$ l'angolo fra $\vb{B}$ e la normale alla superficie $\vb{S}=ab\vu{n}$ della spira.
    Per la seconda legge di Laplace si ha $\vb{F}_2=\vb{F}_4=IaB$ per cui il momento meccanico sulla spira vale in modulo
    \[
        M=\frac{b}{2}F_2\sin\theta+\frac{b}{2}F_4\sin\theta=bIaB\sin\theta=ISB\sin\theta
    \]
    dove $\theta$ è l'angolo fra la forza ed il raggio-vettore che congiunge il punto di applicazione della forza al centro
    di massa.
    La relazione vettoriale è quindi
    \[
        \vb{M}=IS\vu{n}\cp\vb{B}=\vb{m}\cp\vb{B}
    \]
\end{example}



\subsubsection{Interazione circuito-circuito}
Dati due circuiti $l_1$ ed $l_2$ questi esercitano l'uno sull'altro delle azioni meccaniche in quanto
sorgenti di campi magnetici. La forza che $l_1$ esercita sull'elemento di filo $\dd{l_2}$ di
$l_2$ è\footnote{La distanza fra due punti presenti rispettivamente sul primo e sul secondo circuito
è stata indicata con $r_{12}$ anzichè con $\Delta r$}
\[
    \dd{\vb{F}_{12}}=I_2\dd{\vb{l}_2}\cp\vb{B}_1=I_2\dd{\vb{l}_2}\cp\oint_{l_1}\frac{\mu_0I_1}{4\pi}\frac{\dd{\vb{l}_1}\cp\vb{r}_{12}}{r^3_{12}}
\]
dove $\vb{r}_{12}$ è la distanza fra un punto di $l_1$ e l'elemento di filo $\dd{l_2}$.
Nel caso in cui i due circuiti siano rigidi si ha
\[
    \vb{F}_{12}=\oint_{l_2}I_2\dd{\vb{l}_2}\cp\oint_{l_1}\frac{\mu_0I_1}{4\pi}\frac{\dd{\vb{l}_1}\cp\vb{r}_{12}}{r^3_{12}}=
    \frac{\mu_0}{4\pi}I_1 I_2 \oint_{l_2}\oint_{l_1}\frac{\dd{\vb{l}_2}\cp\dd{\vb{l}_1}\cp\vb{r}_{12}}{r^3_{12}}
\]
Usando l'identità vettoriale  $\vb{a}\cp\vb{b}\cp\vb{c}=(\vb{a}\vdot\vb{c})\vb{b}-(\vb{b}\vdot\vb{a})\vb{c}$, si ottiene
\[
    \vb{F}_{12}=\frac{\mu_0}{4\pi}I_1 I_2 \Biggl[\oint_{l_2}\oint_{l_1}\frac{(\dd{\vb{l}_2}\vdot\vb{r}_{12})\dd{\vb{l}_1}}{r^3_{12}}-
    \oint_{l_2}\oint_{l_1}\frac{(\dd{\vb{l}_1}\vdot\dd{\vb{l}_2})\vb{r}_{12}}{r^3_{12}}\Biggr]
\]
Il primo di questi due integrali è nullo in quanto
\[
    \oint_{l_2}\oint_{l_1}\frac{(\dd{\vb{l}_2}\vdot\vb{r}_{12})\vb{l}_1}{r^3_{12}}=
    \oint_{l_1}\dd{\vb{l}_1}\oint_{l_2}\frac{(\dd{\vb{l}_2}\vdot\vb{r}_{12})}{r^3_{12}}=
    \oint_{l_1}\dd{\vb{l}_1}\oint_{l_2}\dd{\vb{l}_2}\vdot\grad(\rec{r_{12}})
\]
Ma la circuitazione di un campo conservativo è nulla.
Si ha allora
\begin{equation}
    \vb{F}_{12}=\frac{\mu_0}{4\pi}I_1 I_2 \oint_{l_2}\oint_{l_1}\frac{(\dd{\vb{l}_1}\vdot\dd{\vb{l}_2})\vb{r}_{12}}{r^3_{12}}
\end{equation}
L'equazione qui sopra mostra che scambiando gli indici si ottiene un cambio di segno, in accordo col terzo principio della dinamica.

Considerando due fili rettilinei infinitamente lunghi e paralleli, posti a distanza $\vb{r}_{12}$, si trova che
\[
    \dd{\vb{F}_{12}}=\frac{\mu_0}{2\pi r_{12}}I_1 I_2\vb{r}_{12} sgn(\dd{l_1}\vdot\dd{l_2})
\]
ovvero la forza è attrattiva o repulsiva in base al fatto che il verso delle correnti sia concorde o discorde.
Inoltre questa relazione fornisce una definizione operativa dell'ampère: si dice che due fili lunghi e sottili
sono attraversati da una corrente di $1A$ quando, posti ad un metro di distanza, risentono reciprocamente di
una forza pari a $\mu_0/2\pi$ per metro di filo.


\section{Effetto Hall}
In un conduttore percorso da corrente elettrica e immerso in un campo di induzione magnetica
esterno ortogonale alle linee di corrente, si genera tra bordi opposti una differenza di
potenziale
\begin{equation}
    \Delta V_H = v_d B a = R_H \frac{IB}{b} \quad\quad R_H=\rec{nq}
    \label{eqn:hall}
\end{equation}
dove $b$ rappresenta lo spessore del conduttore in direzione parallela a $\vb{B}$, a lo spessore
in direzione ortogonale e $n$ la densità
di portatori di carica. Questo effetto è chiamato \textit{Effetto Hall} e la costante $R_H$ è detta
\textit{costante di Hall}, con un valore tipico $\abs{R_H}\simeq 10^{-11}m^3/C$.

Per spiegare il fenomeno, si consideri una sbarretta conduttrice a forma di parallelepipedo la cui
sezione abbia base di lati $a$ e $b$ immersa in un campo magentico $\vb{B}$ parallelo al lato $b$.
Sia questa sbarretta percorsa da corrente $I$ distributia uniformemente sulla sezione. I portatori
di carica, sottoposti alla forza di Lorentz\footnote{Si osservi che il verso della forza di Lorentz
non dipende dal segno della carica dei portatori in quanto questo dipende dal segno del prodotto $qv_d$
che indipendente dalla carica.}, verranno deviati verso una delle facce della superficie
laterale della sbarretta sulla quale si avrà quindi un accumulo di carica. Come conseguenza di questo
accumulo, si deve avere sulla faccia opposta un accumulo della carica opposta.
Il campo elettrostatico $E_s$ generato da questi accumuli si oppone alla forza di Lorentz fino al punto di
saturazione $qE_s=qv_d B$, raggiunto il quale il moto dei portatori torna ad essere quello tipico.
Tra le due facce della sbarretta si genera un potenziale $\Delta V_H=E_sa=v_dBa$.
Indicando con $J$ la densità di corrente uniforme per ipotesi, si ha $I=Jab=nqv_dab$
Sostituendo $v_d$ nell'equazione del potenziale di Hall appena trovata si ha anche la seconda
espressione del potenziale.

Tipicamente la costante di Hall è negativa e questo conferma che i portatori di carica siano tipicamente
elettroni. Inoltre invertendo la formula si ricava che il numero di elettroni liberi per ogni atomo è di
$1-2$



\chapter{Magnetostatica nei dielettrici}
Si vogliono ora studiare i fenomeni magnetici in uno spazio non più vuoto ma riempito con dei materiali.

\section{Aspetti atomici del magnetismo}
Si vuole preliminarmente studiare il comportamento di un atomo immerso in un campo di induzione
magnetica esterno. Per questo studio ci si limiterà all'approssimazione del modello planetario
di Rutherford, che consente di ottenere comunque informazioni interssanti.

L'idea di fondo è che un elettrone in orbita attorno al nucleo può essere visto come una spira
percorsa da corrente. Vogliamo quindi ricavare il momento magnetico, ovvero ricavare la superficie della spira
$S=\pi r^2$ e la corrente $I=q_e/T$, dove $T$ è il periodo.
Detto $r$ il raggio dell'orbita e $T$ il periodo di rivoluzione, si ha dalla legge di Coulomb
e dal secondo postulato della dinamica di Newton
\[
\rec{4\pi\epsilon_0}\frac{e^2}{r^2}=m_e\omega^2 r=\frac{(2\pi)^2}{T^2}m_e r
\]
da cui
\[
T=\frac{4\pi}{e}\sqrt{\pi\epsilon_0 m_e r^3}
\]
Il raggio dell'orbita può essere valutato a partire dal lavoro di prima ionizzazione,
ovvero dal lavoro necessario per strappare l'elettrone dalla sua orbita e portarlo
all'infinito. All'infinito infatti l'elettrone può essere considerato fermo e ha
dunque energia nulla e dunque il lavoro di ionizzazione equivale all'energia posseduta
dall'elettrone in orbita, cambiata di segno.
\[
L=-\Biggl(\rec{2}m_e v^2 - \rec{4\pi\epsilon_0}\frac{e^2}{r} \Biggr)=-\Biggl(\rec{2}m_e (\omega r)^2 - \rec{4\pi\epsilon_0}\frac{e^2}{r} \Biggr)=
-\Biggl(\rec{8\pi\epsilon_0} \frac{e^2}{r} - \rec{4\pi\epsilon_0}\frac{e^2}{r} \Biggr)
\]
E quindi
\[
L=\rec{8\pi\epsilon_0} \frac{e^2}{r} \Longrightarrow  r=\frac{e^2}{8\pi\epsilon_0}\rec{L}
\]

Inserendo i dati per l'atomo di idrogeno ($L=13.5 eV$) si ottiene una corrente $I=1mA$ e
un momento magnetico $m=9.35\vdot 10^{-24} Am^2$, in buon accordo coi dati sperimentali.

Il momento magnetico è proporzionale al momento angolare orbitale dell'elettrone rispetto al
nucleo, in quanto entrambi i vettori giacciono sull'asse perpendicolare alla spira. Siccome la carica dell'elettrone
è negativa, i due vettori sono antiparalleli. Il loro rapporto dipende solo da caratteristiche
intrinseche dell'elettrone e vale
\[
\frac{m}{L}=\frac{e^-}{2m_e}=g_0
\]
Viene chiamato \textit{fattore giromagnetico orbitale g}.
Oltre al momento orbitale è necessario considerare anche il momento orbitale di spin
$s=\hbar/2$, uguale per protone, neutrone ed elettrone, che da luogo al fattore giromagnetico
intrinseco
\[
g=C\frac{e^-}{2m}
\]
dove $C$ è una costante che vale $2$ per l'elettrone, $2.79$ per il protone e $1.91$ per il neutrone.
Il momento magnetico totale si ottiene come somma vettoriale del momento magnetico orbitale e di qiello di
spin\footnote{Bisogna in realtà tenere conto anche del principio di esclusione di Pauli e della quantizzazione
del momento angolare}, ma siccome la massa di neutrone e protone è di tre ordini di grandezza più grande rispetto a quella dell'elettrone
il loro contributo al momento magnetico dell'atomo può spesso essere trascurato.

Atomi con simmetria sferica mostrano avere momento di dipolo magnetico nullo. In generale però
anche quando il momento di dipolo di ogni singolo atomo non è nullo, è nullo il momento complessivo
perchè i momenti di dipolo dei singoli atomi sono disposti casualmente. Un campo magnetico esterno
ha l'effetto di indurre un momento magnetico non nullo sul materiale, in maniera che però è diversa a seconda
della tipologia del materiale come verrà discusso in un paragrafo dedicato.


\section{Vettore polarizzazione}
I fenomeni di polarizzazione magnetica possono essere descritti macroscopicamente intoducendo il vettore
polarizzazione magnetica.
\begin{defn}[Vettore polarizzazione magnetica]
    Si definisce il vettore polarizzazione magnetica $\vb{M}$ come il momento
    di dipolo magnetico per unità di volume posseduto dal materiale
    \[
        \vb{M}=\lim_{\tau\to 0}\frac{\sum \vb{m}_i}{\tau}=\frac{\dd{N}  \expval{\vb{m}}}{\dd{\tau}}
    \]
\end{defn}
Con $\dd{N}$ il numero di dipoli magentici contenute nell'elemento di volume $\dd{\tau}$.
Dalla definizione si ottiene che l'elemento di volume possiede un momento di dipolo $\dd{\vb{m}}=\vb{M}\dd{\tau}$.
Dei ragionamenti qualitativi preliminari ad una dimostrazione rigorosa permettono di comprendere in che modo
questo vettore sia legato alle correnti amperiane, cioè le correnti atomiche microscopiche.
Si consideri a tal fine un cilindro di materiale polarizzato lungo l'asse del cilindro stesso (ovvero tale che
$\vb{M}$ giaccia sull'asse del cilindro).
Dato che gli atomi sono sempre circondati da elettroni in movimento
l'unica possibilità per avere $M=0$ è che le spire microscopiche siano disposte casualmente
e che quindi il momento magnetico medio sia $0$. Quando $M\neq0$ significa che le spire microscopiche
sono prevalentemente orientate nella stessa direzione, ovvero giacciono sul piano ortogonale a $\vb{M}$.
Se $\vb{M}$ fosse indipendente dalla posizione, allora internamente al
materiale il valor medio delle correnti micorscopiche sarebbe nullo: infatti, prendendo un piano ortogonale
all'asse del cilindo, prendendo un punto $P$ su questo piano, si avrebbe in media perfetta compensazione fra
le correnti dei dipoli disposti simmetricamente a $P$. Non sarebbe nulla però la corrente sulla superficie
laterale del cilindro, descritta da un vettore densità $\vb{J}_{mS}$ che può essere definito come
\[
    \dd{I_{mS}}=\vb{J}_{mS}\vdot \dd{h}\vu{n}
\]
dove $Q_m$ è la carica microscopica media che flusce attraverso un segmento di lunghezza $\dd{h}$ ($\vu{n}$ è il
versore normale al segmento).
Se Se $\vb{M}$ non fosse indipendente dalla posizione, allora alla corrente microscopica superficiale andrebbe sommato
il contributo di una corrente microscopica interna al materiale descritta da una densità $\vb{J}_{mV}$
\[
    \dd{I_{mV}}=\vb{J}_{mV}\vdot \dd{\vb{S}}
\]

\begin{thm}
    Le relazioni fondamentali tra il vettore di polarizzazione e le densità di corrente
    di superficie e di volume in un materiale sono
    \begin{equation}
        \vb{J}_{mS}=\vb{M}\cp\vu{n}
        \label{eqn:M_JmS}
    \end{equation}
    \begin{equation}
        \vb{J}_{mV}=\curl{\vb{M}}
        \label{eqn:M_JmV}
    \end{equation}
    con $\vu{n}$ il versore normale alla superficie del materiale.
\end{thm}
\begin{proof}
    Si consideri un certo materiale di forma qualunque che occupi un volume $\tau$ delimitato da una superficie $S$.
    Sia $\vb{M}=\vb{M}(\vb{r}')$ il vettore di polarizzazione magnetica. Allora si ha, dalla definizione di $\vb{M}$
    \[
        \vb{A}(\vb{r})=\frac{\mu_0}{4\pi} \int_{\tau}\frac{\dd{\vb{m}}\cp(\vb{r}-\vb{r'})}{\abs{\vb{r}-\vb{r'}}^3}=
        \frac{\mu_0}{4\pi} \int_{\tau}\frac{\vb{M}(\vb{r'})\cp(\vb{r}-\vb{r'})}{\abs{\vb{r}-\vb{r'}}^3}\dd{\tau'}
    \]
    Ricordando la dimostrazione del teorema [REF] e usando l'equazione \eqref{app:eqn:curl_scalare_vettore}
    \[
        \vb{A}(\vb{r})=\frac{\mu_0}{4\pi} \int_{\tau}\frac{\curl'\vb{M}(\vb{r'})}{\abs{\vb{r}-\vb{r'}}}-
        \frac{\mu_0}{4\pi} \int_{\tau}\grad'\Biggl[\frac{\vb{M}(\vb{r'})}{\abs{\vb{r}-\vb{r'}}}\Biggr]\dd{\tau}
    \]
    Grazie alla seconda identità di green
    \[
        \vb{A}(\vb{r})=\frac{\mu_0}{4\pi} \int_{\tau}\frac{\curl'\vb{M}(\vb{r'})}{\abs{\vb{r}-\vb{r'}}}+
        \frac{\mu_0}{4\pi} \int_{S}\frac{\vb{M}(\vb{r'})\cp\vu{n}'\dd{S'}}{\abs{\vb{r}-\vb{r'}}}
    \]
    Ma per la \ref{eqn:A_I} il potenziale vettore può anche essere scritto come
    \[
        \vb{A}(\vb{r})=\frac{\mu_0}{4\pi}\int\frac{\vb{J}_{mV}(\vb{r}')}{\abs{\vb{r}-\vb{r}'}}\dd{\tau'}+
        \frac{\mu_0}{4\pi}\int\frac{\vb{J}_{mS}(\vb{r}')}{\abs{\vb{r}-\vb{r}'}}\dd{S'}
    \]
    Per confronto fra le integrande si ottiene la tesi.
\end{proof}


\section{Equazioni di Maxwell}
\begin{defn}[Campo Magnetico]
    Si definisce vettore campo magnetico
    \[
        \vb{H}=\rec{\mu_0}\vb{B}-\vb{M}
    \]
\end{defn}

\begin{thm}
    In presenza di materiali la seconda e la quarta equazione di Maxwell assumono la seguente forma:
    \begin{equation}
        \div{\vb{B}}=0
    \end{equation}
    \begin{equation}
        \curl{\vb{H}}=\vb{J}
    \end{equation}
\end{thm}
\begin{proof}
    L'osservazione di partenza è che la presenza dei materiali impone una semplice modifica sulle
    equazioni di Maxwell per il campo di induzione magnetica, ovvero nella quarta equazione di Maxwell
    oltre alla densità di corrente macroscopica va considerata anche la densità delle correnti atomiche
    presenti nel materiale. Formalmente
    \[
        \curl{\vb{B}}=\mu_0(\vb{J}+\vb{J}_m)
    \]
    In generale, sulle superfici di separazione fra materiali, le equazioni di Maxwell non sono definite
    in quanto $\vb{B}$ subisce una discontinuità. Ha senso limitarsi a considerare quindi sono l'interno
    di un materiale dove $\vb{J}_m=\vb{J}_{mV}$. Di conseguenza, per la \ref{eqn:M_JmV}
    \[
        \curl{\vb{B}}=\mu_0 \vb{J}+ \mu_0\curl{\vb{M}}
    \]
    Da cui segue che
    \[
        \curl(\frac{\vb{B}-\mu_0\vb{M}}{\mu_0})=\vb{J}
    \]
    Ovvero dalla definizione di $\vb{H}$, la tesi.
\end{proof}
All'interno delle due equazioni non compare lo stesso campo e quindi queste non ammettono soluzione univoca a meno di
conoscere la relazione fra $\vb{H}$ e $\vb{B}$ -ovvero, dalla definizione di campo magnetico, fra $\vb{B}$ ed $\vb{M}$.
Questo aspetto sarà approfondito nel paragrafo sui tipi di materiali magnetici.

La forma integrale della quarta equazione di Maxwell fornisce una versione per il teorema della circuitazione di
Ampère valida nei materiali.
\begin{cor}[Teorema della circuitazione di Ampère]
    Prese delle correnti
    concatenate $I_i$ con segno positivo quando vedono girare il contorno orientato $l$
    in senso antiorario
    \begin{equation}
        \oint_l\vb{H}\vdot\dd{\vb{l}}=\sum \vb{I}_i
    \end{equation}
\end{cor}
\begin{proof}
    La dimostrazione è analoga a quella del teorema di Ampère nel vuoto corollario \ref{teo:Ampère}
\end{proof}


Come accennato, le equazioni di Maxwell per il magnetismo nella materia valgono solo
all'interno di un materiale. Per caratterizzare i fenomeni magnetici in tutto lo spazio
è quindi necessario risolvere le equazioni all'interno di ogni materiale che riempie
lo spazio e poi usare delle condizioni di raccordo sulla superficie che separa un
materiale dall'altro.
\begin{thm}
    Passando da un mezzo materiale all'altro, la componente normale di $\vb{B}$ non subisce
    alcuna discontinuità, così come la componente tangenziale di $\vb{H}$. In formule
    \[
        \vb{B}_{n1}=\vb{B}_{n2}
    \]
    \[
        \vb{H}_{t1}=\vb{H}_{t2}
    \]
\label{teo:condizioni_raccordo_campomagnetico}
\end{thm}
\begin{proof}
    Per dimostrare la prima equazione si consideri un cilindretto $C$ con le basi parallele alla
    superficie di separazione fra due materiali; Per la seconda si consideri invece un piccolo
    percorso rettangolare $l$ con due lati paralleli e due normali alla superficie di separazione.
    Il flusso di $\vb{B}$ uscente dalla superficie di $C$ deve essere nullo, così come la circuitazione
    di $\vb{H}$ su $l$ -in quanto non sono presenti correnti macroscopiche concatenate al percorso, ma solo correnti
    microscopiche. Analogamente a quanto fatto per [REF] si ha la tesi.
\end{proof}
Per misurare quindi $B$ e $H$ all'interno di un materiale è sufficiente misurare $B$ e $H$ in un taglio
sulla superficie del materiale rispettivamente ortogonale o parallelo alle linee di forza del campo.
Le equazioni di Maxwell ricavate in questo paragrafo  non dipendono dalle correnti microscopiche. Data la
facililità  di misurazione del campo magnetico $\vb{H}$ questa riscrittura è estremamente vantaggiosa.


\section{Materiali dia-, para-, ferromagnetici}
Si vuole ora cercare di capire quale sia la relazione fra $\vb{B}$ e $\vb{H}$ in un materiale.
Nel vuoto evidentemente, essendo $\vb{M}=0$ si ha
\[
    \vb{H}=\rec{\mu_0}\vb{B}
\]

È allora sensato, considerando la linearità della teoria finora sviluppata pensare di scrivere
\[
    \vb{B}=\norm{\mu}\vb{H}
\]
dove $\norm{\mu}$ è un tensore.
\begin{obses}
    Nei mezzi materiali isotropi e omogenei $\vb{B}$ e $\vb{H}$ risultano sperimentalmente paralleli fra loro.
\end{obses}
Questo significa in particolare che $\vb{M}$ risulta essere parallelo o antiparallelo a $\vb{B}$ e che quindi
si può scrivere
\begin{equation}
    \vb{H}=\rec{\mu_0\mu_r}\vb{B}=\rec{\mu}\vb{B}
    \label{eqn:H_B}
\end{equation}
dove $\mu$ e $\mu_r$ sono delle costanti dette rispettivamente \textit{permeabilità megnetica} e
\textit{permeabilità megnetica relativa} del materiale in esame. Ovvero, nei materiali omogenei e isotropi
il tensore si riduce ad una costante.
\begin{thm}[Legge di rifrazione delle linee di forza]
    Presa la superficie di separazione fra due materiali ($1$ e $2$) isotropi e omogenei
    con permeabilità magnetica $\mu_1$ e $\mu_2$, detti $\vb{H}_1$ il campo nel materiale $1$ e
    $\vb{H}_2$ il campo nel materiale $2$ (analogamente per $\vb{B}$), detti $\theta_1$ e $\theta_2$ gli angoli che una
    linea di forza del campo forma con la normale alla superficie nei due materiali, si ha
    \[
        \frac{H_{t1}/{H_{n1}}}{H_{t2}/{H_{n2}}}=\frac{B_{t1}/{B_{n1}}}{B_{t2}/{B_{n2}}}=\frac{\tan{\theta_1}}{\tan{\theta_2}}=\frac{\mu_1}{\mu_2}
    \]
\end{thm}
\begin{proof}
    Riscrivendo la \eqref{eqn:H_B} per le componenti normali, ricordando i risultati del
    teorema \ref{teo:condizioni_raccordo_campomagnetico}, si ha
    \[
        \begin{cases}
            & H_{t1}=H_{t_2} \\
            & \mu_1 H_{n1}=\mu_2 H_{n2} \\
        \end{cases}
    \]
    \[
        \begin{cases}
            & B_{n1}=B_{n_2} \\
            & \mu_1 B_{t1}=\mu_2 B_{t2} \\
        \end{cases}
    \]
    Facendo i rapporti membro a membro si ottiene la tesi.
\end{proof}

Sono le caratteristiche di $\mu_r$ che permettono una classificazione dei materiali in:
diamagneti, paramagneti e ferromagneti.
Prima di addentrarsi nella descrizione di questi materiali è necessario introdurre la seguente quantità
\begin{defn}[Suscettività magnetica]
    Si definisce suscettività magnetica di un materiale
    \[
        \chi_m=\mu_r-1
    \]
\end{defn}
Per comprendere l'utilità di questa definizione bisogna osservare che, dalla definizione di campo magnetico,
per i materiali isotropi
\begin{equation}
    \vb{M}=\chi_m\vb{H}
    \label{eqn:M_H}
\end{equation}
Ovvero la suscettibilità magnetica rappresenta il fattore di proporzionalità fra il campo magnetico e
il vettore polarizzazione magnetica.


\subsubsection{Diamagneti}
Nelle sostanze diamagnetiche la permeabilità magnetica relativa è costante ed è prossima a $1$ in maniera tale che
$\chi_m<0$: il momento magnetico indotto nel materiale è di verso opposto rispetto al campo che lo induce. Generalmente
per queste sostanze si ha $\chi_m \in [-10^{-6},-10^{-9}]$.Per questi materiali
la suscettività magnetica è indipendente dalla temperatura. Inoltre, queste sostanze non presentano alcun tipo di
saturazione: le relazioni \eqref{eqn:H_B} e \eqref{eqn:M_H} valgono a prescindere dai valori assunti dai campi in esame.
Per quanto detto, riscrivendo la \eqref{eqn:H_B} come $\vb{B}=\mu_0(1+\chi_m)\vb{H}$, è chiaro come la perturbazione
causata dalla presenza di materiali diamagnetici sia di norma trascurabile.

\subsubsection{Paramagneti}
Anche nelle sostanze paramagnetiche la permebalitià magnetica relativa è costante e prossima a $1$, ma in questo caso
si ha $\chi_m>0$, ovvero il momento magnetico indotto nel materiale ha lo stessso verso del campo che lo induce.
Di norma per queste sostanze si ha $\chi_m \in [10^{-6},10^{-2}]$. La suscettività magnetica varia in funzione della
temperatura secondo la legge di curie
\begin{equation}
    \chi_m=C\frac{\rho}{T}
\end{equation}
dove $\rho$ è la densità del materiale e $C$ è una costante scritta in funzione delle grandezze atomiche.
All'avvicinarsi della temperatura allo 0 assoluto i campi in questi materiali sono molto diversi a quelli che
si hanno nel vuoto, ma in condizioni standard anche in questo caso la perturbazione è trascurabile.

\subsubsection{Ferromagneti}
Il comportamento dei ferromagneti è il più complesso: le relazioni fra campi e momenti di magentizzazione
non sono nè lineari nè univoche. Inoltre, le proprietà di questi materiali sono fortemente dipendenti dalle
loro caratteristiche chimico-fisiche.
Effettuando misure su un materiale isotropo si osserva che $\vb{B}(\vb{H})$ e $\vb{M}(\vb{H})$ assumono
sempre lo stesso verso di $\vb{H}$, per cui ci si può limitare a considerarle come relazioni scalari.
In particolare è istruttivo descrivere l'andamento di $B$ in funzione di $H$.

Partendo da $(H,B)=(0,0)$ all'aumentare di $H$ aumenta $B$ seguendo una curva $I$ detta
\textit{curva di prima magnetizzazione}. Raggiunto un valore $H_m\simeq 10^5 As/m$, da lì in poi $B$ aumenta
solo proporzionalmente ad $H$: dalla definizione di campo magnetico questo significa che $M$
ha raggiunto il valore di saturazione, oltre al quale non aumenta più.
Facendo diminuire $H$ a partire dal valore $H_m$ si ha che per un primo tratto $B$ segue la curva $I$, ma una
volta superato un valore $H_s$ si ha che $B$ scende mantenendosi sempre sopra alla curva $I$. In questo modo,
per $H=0$ si ha $B>0$ e il valore di $M=B/\mu_0$ corrispondente si chiama \textit{magnetizzazione residua}.
Cambiando ora il segno di $H$ e facendo crescere il suo valore assoluto, $B$ decresce ed esiste quindi sicuramente
un valore $H_c$ detto \textit{campo magnetico di coercizione} per cui $B=0$. In corrispondenza di questo valore
$M=H_c>0$. Superato questo punto $B$ diventa negativo e continua a decrescere fino al valore $-H_m$.
La curva completa si assesta su una curva ciclica simmetrica rispetto all'origine
detta \textit{curva di isteresi}. Se si restringe l'intervallo
all'interno del quale si fa variare $H$ si ottengono dei cicli sempre più piccoli sempre simmetrici rispetto all'origine
e prossimi alla curva di prima magnetizzazione. Eseguendo dei cicli in cui gradualmente si riduce l'ampiezza
dell'intervallo $B$ converge verso l'origine ed è possibile in questo modo smagnetizzare un materiale ferromagnetico.
Si può dimostrare che per produrre una variazione $\dd{H}$ del campo magnetico è necessario compiere
un lavoro $\dd{L}=B\dd{H}$, così che il lavoro complessivo dissipato dal ciclo di isteresi è
uguale all'area della superficie contenuta dalla curva.

È anche possibile relizzare dei cicli non simmetrici rispetto all'origine portando il sistema in una
posizione $(\bar{H},\bar{B})$ e spostandosi da questa di una piccola quantità $\Delta H$. Se questa variazione
è molto piccola il ciclo si riduce ad un segmento, in cui il lavoro dissipato è praticamente nullo: si parla di
\textit{ciclo elementare reversibile}.

Risulta quindi chiaro che fissato un valore di $H$ non è possibile ricavare un valore di $B$ senza conoscere il
resto della curva. In particolare perde di senso il parametro $\mu$ e bisogna quindi cercare dei parametri
alternativi adatti alla descrizione del fenomeno. Quando il ciclo è molto stretto una buona scelta è
la \textit{permeabilità magnetica differenziale assoluta}
\[
    \mu_d=\dv{B}{H}
\]

Un utile risultato riguarda il fatto che, superata una certa temperatura $T_c$ il materiale si comporta come
un materiale paramagnetico con suscettività magnetica
\[
    \chi_m=C\frac{\rho}{T-T_c}
\]


\section{Precessione di Larmor}
Per concludere si vuole tornare all'aspetto atomico della magnetizzazione
per descrivere un fenomeno che prende il nome di \textit{precessione di Larmor}.
Si consideri un elettrone orbitante attorno ad un nucleo con momento angolare
$\vb{L}$ e momento magnetico $\vb{m}_0=(-e/2m_e)\vb{L}$. Si supponga inoltre
che sull'elettrone agisca un campo magentico locale $\vb{B}_l=\mu_0\vb{H}_l$.
Nell'ipotesi che il campo magnetico perturbi di poco il moto dell'elettrone
allora, il momento angolare compie un moto di precessione attorno alla direzione
del campo. Si ha infatti, indicando con $\vb{M}$ il momento meccanico
\[
\dv{\vb{L}}{t}=\vb{M}=\vb{m}_0\cp\vb{B}_l=-\frac{e}{2m_e}\vb{L}\cp\vb{B}_l
\]
Chiaramente quindi la derivata di $\vb{L}$ è ortogonale a $\vb{L}$ e a $\vb{B}_l$.
Da questo segue che nel tempo cambia la direzione di $\vb{L}$ ma non il suo modulo
e che inoltre la componente $L_z$ di $\vb{L}$ parallela a $\vb{B}$ non cambia.
Perciò si ha effettivamente un moto di precessione che disegna un cono il cui asse
coincide con $\vb{B}$.

Resta da determinare la velocità di precessione, detta
\textit{velocità angolare di Larmor}. Dal corso di meccanica\footnote{Da leggersi
"non mi ricordo dell'esistenza di questa cosa, prendiamola per vera".} si ha
\[
\dv{\vb{L}}{t}=\vb{\omega}_L\cp\vb{L}
\]
Per cui si ha
\[
\vb{\omega}_L\cp\vb{L}=-\frac{e}{2m_e}\vb{L}\cp\vb{B}_l
\]
Ovvero per confronto
\[
\vb{\omega}_L=\frac{e}{2m_e}\vb{B}_l
\]
La velocità angolare ha lo stesso verso del campo di induzione magnetica, per cui la precessione
avviene in senso antiorario.

Questo moto comporta una \textit{corrente di Larmor}
\[
I_L=\frac{e}{T_L}=\frac{e\omega_L}{2\pi}=\frac{e^2B_L}{4\pi m_e}
\]
a cui, indicando con $S_z$ la proiezione della superficie dell'orbita sul piano $xy$, corrisponde
un momento magnetico diretto in verso opposto rispetto a $\vb{B}_L$ e in modulo
\[
m_L=I_L S_z=\frac{e^2B_L}{4\pi m_e} S_z
\]
$S_z$ è ottimamente approssimabile con $\pi(x^2+y^2)$, dove $x^2$ e $y^2$ rappresentano il valore
quadratico medio assunto dalle coordinate dell'elettrone mentre questo percorre l'orbita. Se
l'atomo è isotropicamente distribuito nello spazio si ha $x^2=y^2=z^2=r/3$ con $r$ il raggio
dell'orbita. Si può quindi riscrivere il momento magnetico di Larmor, tenendo conto di direzione e verso,
come
\[
\vb{m}_L=-\frac{e^2r^2}{6m_e} \vb{B}_L
\]
Se l'atomo ha $Z$ elettroni con raggio orbitale $r_i$, allora
\[
\vb{m}_L=-\frac{e^2r^2}{6m_e}\sum_{i=1}^Z r_i^2 \vb{B}_L
\]





\part{Elettrodinamica}
\chapter{Campi variabili}
\begin{obses}
  Si consideri un circuito costituito da una linea chiusa di materiale conduttore, in serie al quale sia disposto un galvanometro. Nei seguenti casi, il galvanometro indica il passaggio di corrente elettrica nel circuito:
  \begin{enumerate}
    \item il circuito si trova in prossimità di un altro circuito percorso da corrente variabile nel tempo;
    \item il circuito si trova in moto relativo rispetto ad un altro circuito percorso da corrente costante nel tempo;
    \item il circuito si trova in moto relativo rispetto ad un magnete permanente;
    \item il circuito viene deformato all'interno di un campo di induzione magnetica.
  \end{enumerate}
\end{obses}

\section{Legge di Faraday-Neumann}
Per tenere conto delle osservazioni sperimentali, si introduce la seguente legge.
\begin{obses} [Legge di Faraday-Neumann]
    Dato un circuito immerso in un campo di induzione magnetica $\vb{B}$ e detto $\Phi(\vb{B})$
    il flusso con del campo concatenato al circuito, allora nel circuito si genera una forza elettromotrice indotta
    \begin{equation}
        f_i=-\dv{\Phi(\vb{B})}{t}
    \end{equation}
\end{obses}
La derivata totale non può essere portata all'interno dell'intergale che esprime il flusso
in quanto la forma del circuito, e quindi il cammino di integrazione, dipende dal tempo.
\begin{obs}
    La forza elettromotrice indotta implica la presenza di un campo elettromotore indotto $\vb{E}_i$
    \begin{equation}
        \label{eqn:fi}
        \vb{E}_i=\vb{E}+\vb{v}_T\cp\vb{B}
    \end{equation}
    Dove $\vb{E}$ e $\vb{B}$ sono i campi elettrico e di induzione magnetica in cui il circuito è immerso
    e $\vb{v}_T$ è la velocità di trascinamento con cui si muove ciascun elemento infinitesimo di circuito.
\end{obs}
\begin{proof}
    Dalla definizione $f_i=\oint_l \vb{E}_i\vdot\dd{\vb{l}}$, che mostra la presenza del campo elettromotore indotto.
    Nel fenomeno dell'induzione i portatori di carica $q$ che danno orgine alla corrente nel circuito sono messi in moto
    da una forza
    \[
        \vb{F}=q\vb{E}+q\vb{v}_{tot}\cp\vb{B}
    \]
    Il campo elettromotore può quindi essere visto come rapporto fra questa forza e la carica dei portatori, ovvero
    \[
        \vb{E}_i=\vb{E}+\vb{v}_{tot}\cp\vb{B}=\vb{E}+(\vb{v}_T+\vb{v}_d)\cp\vb{B}
    \]
    dove $\vb{v}_T$ è la velocità di trascinamento e $\vb{v}_d$ la velocità di deriva delle cariche.
    Si ha quindi, considerando che la velocità di deriva è parallela punto per punto all'elemento $\dd{\vb{l}}$
    di volta in volta considerato
    \[
        f_i=\oint_l \vb{E}_i\vdot\dd{\vb{l}}=\oint_l(\vb{E}+\vb{v}_T\cp\vb{B})\vdot\dd{\vb{l}}
    \]
    Siccome la forza elettromotrice indotta è l'unica osservabile attraverso la quale sia possibile individuare
    il campo elettromotore indotto, allora si può tranqullamente identificare quest'ultimo con il secondo integrando.
\end{proof}


La corrente che circola nel circuito a sua volta genera un campo magnetico indotto $\vb{B}_i$.
\begin{cor}[Legge di Lenz]
    Il campo magnetico indotto $\vb{B}_i$ è tale che il suo verso si opponga
    a quello del campo magnetico che genera la forza elettromotrice indotta.
\end{cor}
\begin{proof}
    La dimostrazione è immediata, per il fatto che nella legge di Faraday-Neumann compare il segno meno
\end{proof}


\section{Il flusso tagliato}
L'obiettivo di questo paragrafo è duplice: dare un significato fisico più
concreto alla legge di Faraday-Neumann attraverso tre esempi significativi
e esporre il concetto di flusso tagliato, utile per i risultati
espressi nei prossimi paragrafi.

\begin{defn}[Flusso tagliato]
    Il flusso attraverso la superficie spazzata da un circuito in movimento è chiamato flusso tagliato.
\end{defn}


\subsubsection{Circuito variabile, B costante nel tempo}
$\vb{B}$ costante nel tempo significa che le caratteristiche delle sorgenti del campo non cambiano nel tempo
e che queste sorgenti sono in quiete nel sistema di riferimento scelto per descrivere il fenomeno.
\begin{thm}
    Dato un circuito di forma generica, il cui generico elemento $\dd{l}$ sia in moto con velocità $\vb{v}_T$,
    immerso in un campo $\vb{B}$ costante nel tempo, la variazione del flusso concatenato al circuito è
    pari in modulo e opposta in segno al flusso tagliato.
\end{thm}
\begin{proof}
    Detta $\Sigma$ la superficie orientata attraverso la quale viene calcolato il flusso concatenato all'istante $t_0$
    (indicato con $\Phi_i$), $\Sigma'$ la superficie attraverso cui viene calcolato il flusso $\Phi_f$ all'istante
    $t+\dd{t}$ e $\dd{\Sigma}$ la superficie spazzata dal circuito nel suo spostamento, si ha che l'unione di queste tre superfici
    è una superficie chiusa: il flusso totale -ovvero la somma dei flussi uscenti- deve essere nullo\footnote{Questo è vero solo
    nell'ipotesi di campo costante nel tempo, in quanto le superfici $\Sigma$ e $\Sigma'$ sono appoggiate al circuito
    in tempi diversi}. Si scelga convenzionalmente come verso di percorrenza positivo quello per cui la normale alla superficie del
    circuito sia parallela alla velocità complessiva del circuito. Allora
    \[
        -\Phi_i+\Phi_f+\Phi_{\dd{\Sigma}}=0
    \]
    dove il segno meno al primo termine è dovuto al fatto che con la convenzione scelta $\Phi_i$ è
    il flusso entrante nella superficie chiusa, mentre la somma è nulla quando riferita ai flussi uscenti.
    Si ha quindi che
    \[
        \dd{\Phi} \equiv \Phi_f-\Phi_i=-\Phi_{\dd{\Sigma}}
    \]
    ovvero la tesi.
\end{proof}
L'utilità di questo risultato risiede nel fatto che è possibile trovare un'espressione esplicita per il flusso
tagliato come mostrato nel seguente teorema.
\begin{thm}
    Dato un circuito di forma generica, il cui generico elemento $\dd{l}$ sia in moto con velocità $\vb{v}_T$,
    immerso in un campo $\vb{B}$ costante nel tempo, il flusso tagliato vale
    \[
        \Phi_{\dd{\Sigma}}=\dd{t}\oint_l(\vb{v}_T\cp\vb{B})\vdot\dd{l}
    \]
\end{thm}
\begin{proof}
    Si vuole calcolare $\Phi_{\dd{\Sigma}}$. Per farlo, siccome $\vb{B}$ è noto per ipotesi, bisogna esprimere
    $\dd{\vb{S}}$ in termini di grandezze note. Questo si ottiene considerando che
    ogni elemento di circuito compie uno spostamento $\dd{\vb{s}}=\vb{v}_T\dd{t}$ e nel
    farlo spazza la superficie $\dd{\vb{S}}=\dd{\vb{l}}\cp\dd{\vb{s}}=\dd{\vb{l}}\cp\vb{v}_T\dd{t}$.
    \[
        \Phi_{\dd{\Sigma}}=\int_{\dd{\Sigma}}\vb{B}\vdot\dd{\vb{S}}=\oint_l\vb{B}\vdot(\dd{\vb{l}}\cp\vb{v}_T\dd{t})
    \]
    Usando l'identità \eqref{app:eqn:vdot_cp} si ha la tesi.
    % \[
    %     \oint_l\vb{B}\vdot(\dd{\vb{l}}\cp\vb{v}_T\dd{t})=\oint_l(\vb{v}_T\dd{t}\cp\vb{B})\vdot\dd{\vb{l}}
    % \]
\end{proof}

\begin{cor}
    Nel caso di circuito in moto con velocità $\vb{v}_T$ in un campo $\vb{B}$ costante, il campo elettromotore indotto è
    \[
        \vb{E}_i=\vb{v}_T\cp\vb{B}
    \]
\end{cor}
\begin{proof}
    Per i due risultati appena dimostrati $\dv*{\Phi}{t}=-\Phi_{\dd{\Sigma}}/\dd{t}=-\oint_l(\vb{v}_T\dd{t}\cp\vb{B})\vdot\dd{\vb{l}}$.
    Confrontando il primo e l'ultimo membro dell'uguaglianza con la legge di Faraday-Neumann si ha la tesi.
\end{proof}
L'interpretazione fisica del fenomeno dell'induzione elettromagnetica è quindi immediata nelle ipotesi del corollario:
le cariche costrette a muoversi nel campo di induzione magnetica per via del moto del circuito, sono sottoposte alla forza di Lorentz.
Nonostante la forza di Lorentz non compia lavoro la forza elettromotice può essere diversa da $0$:
il lavoro dissipato dalla corrente nel circuito è compiuto dalla forza esterna che mantiene $\vb{v}_T$ costante
o è compiuto a spese dell'energia cinetica, per cui il circuito rallenta fino a fermarsi.



\subsubsection{Circuito rigido, sorgenti di B stazionarie in moto}
Si consideri un circuito $C$ in quiete nel sistema di riferimento inerziale $sr=Oxyz$. Supponendo che le
sorgenti di $\vb{B}$ siano in uno stato stazionario, ovvero con caratteristiche costanti nel tempo,
una variazione del flusso può avvenire solo se queste cariche si muovono rispetto a $sr$ con una velocità $\vb{v}$.
Nell'ipotesi che ci sia una sola sorgente del campo il cui moto sia
di pura traslazione con $v$ costante, è possibile considerare un sistema di riferimento inerziale
$sr'=O'x'y'z'$ rispetto al quale la sorgente è in quiete. $sr$ si muove rispetto ad $sr'$ con
velocità $\vb{v}'=-\vb{v}$. $sr'$ si trova allora nelle stesse condizioni discusse nel caso del circuito variabile e
campo costante nel tempo: le cariche risentono della forza di Lorentz ($\vb{F}'_l=q\vb{v}'\cp\vb{B}'$) come nel caso precedente.
In $sr$, dove il circuito è fermo, non si ha nessuna forza di Lorentz. A partire da considerazioni relativistiche però si può dedurre che
il moto delle sorgenti di $\vb{B}$ provochi l'insorgere di un campo $\vb{E}_l$ che esercita sui
portatori di carica del circuito una forza $\vb{F}_l=q\vb{E}_l$ equivalente alla forza di Lorentz $\vb{F}'_l$.
Per $v<<c$ si ha $\vb{B}\simeq\vb{B}'$, da cui $\vb{E}_l=\vb{v}'\cp\vb{B}'=-\vb{v}\cp\vb{B}$.

In virtù dell'additività di $\vb{B}$ il ragionamento fatto è estendibile al caso di più sorgenti.



\subsubsection{Circuito rigido, sorgenti di B ferme non stazionarie}
Il circuito $C$ e le sorgenti del campo di induzione magnetica non sono in moto relativo fra loro.
Le sorgenti di $\vb{B}$ non stazionarie sono dei circuiti nei quali le correnti di alimentazione
non sono costanti. L'effetto è che in ogni punto di $C$ il campo $\vb{B}$ sia variabile nel tempo,
esattamente come visto del caso del circuito in moto relativo con le sorgenti del campo.
È ragionevole perciò aspettarsi quindi che anche in questo caso si abbia come effetto un campo elettromotore,
sebbene questo non sia causato dalla forza di Lorenz. In effetti questo campo è presente ed è una
conseguenza della generalizzazione della terza equazione di Maxwell al caso non stazionario che
verrà presentata nel prossimo paragrafo.


\section{Equazioni di Maxwell -  caso non stazionario}
\subsection{Terza equazione di Maxwell}
\input{Parti/Elettrodinamica/Capitoli/Campi_variabili/Sezioni/terza_maxwell.tex}

\subsection{Prima e seconda equazione di Maxwell}
\input{Parti/Elettrodinamica/Capitoli/Campi_variabili/Sezioni/prima_seconda_maxwell.tex}

\subsection{Quarta equazione di Maxwell}
\input{Parti/Elettrodinamica/Capitoli/Campi_variabili/Sezioni/quarta_maxwell.tex}


\section{L'autoinduzione}
L'analisi proposta in questo e nel prossimo paragrafo è fatta per materiali omogenei ed isotropi in cui $\mu$ sia costante.

Si consideri un circuito in condizioni quasi-stazionarie, il che implica che la corrente abbia
lo stesso valore lungo tutto il circuito, percorso da una corrente variabile nel tempo $I(t)$.
Questa corrente genera nello spazio circostante un campo di induzione magnetica $\vb{B}(t)$ il cui
flusso concatenato al circuito è in generale diverso da zero e variabile.
Per la legge di Faraday-Neumann si genera nel circuito una forza elettromotrice detta \textit{autoindotta}.
\begin{obs}
    \begin{equation}
        \label{eqn:induttanza}
        \Phi(\vb{B})=LI
    \end{equation}
\end{obs}
\begin{proof}
    per la \eqref{eqn:dB} il campo di induzione magnetica è proporzionale alla corrente che percorre il circuito.
    Poichè il flusso elementare di $\vb{B}$ attraverso ogni elemento di superficie $\dd{\vb{S}}$ è
    proporzionale a $\vb{B}$, anche il flusso concatenato deve essere proporzionale ad $I$.
\end{proof}
\begin{defn}[Induttanza]
    La costante di proporzionalità che compare nell'osservazione precedente si chiama induttanza.
\end{defn}
L'unità di misura per l'induttanza è detta \textit{henry}
\[
    [L]=\frac{w}{A}=\frac{V \vdot s}{A}=\Omega \vdot s = H
\]

\begin{obs}
    L'induttanza per un solenoide nell'approssimazione di solenoide infinito vale
    \[
        L=\mu n^2 lS=\mu N^2 \frac{S}{l}
    \]
\end{obs}
\begin{proof}
    Il campo di induzione magnetica per un solenoide vale in modulo
    \[
        B=\mu nI
    \]
    ed è diretto assialmente. Il flusso concatenato ad una singola spira è $=SB$
    e quindi il flusso concatenato a tutto il circuito vale $\Phi(\vb{B})=NS\mu nI$.
    Ma allora l'induttanza vale
    \[
        L=\frac{\Phi(\vb{B})}{I}=\mu n^2 lS=\mu N^2 \frac{S}{l}
    \]
\end{proof}
Nelle ipotesi\footnote{Quasi-stazionarietà e materiale omogeneo e isotropo}, dalla legge di Faraday-Neumann si ha immediatamente
l'espressione per la forza elettromotrice autoindotta: $f_a=-L\dd{I}/\dd{t}$.
Se il circuito con resistenza $R$ è sede di una corrente variabile nel tempo, l'equazione del circuito è
\begin{equation}
    \label{eqn:RL}
    RI=f+f_a=f-L\dv{I}{t}
\end{equation}
Quando la geometria del circuito è semplice il termine $f_a$ risulta trascurabile, riducendosi quindi al caso stazionario.
Quando questo termine aggiuntivo non è trascurabile (ad esempio, per la presenza nel circuito di un solenoide) si indica
esplicitamente la presenza di un componente autoinduttivo.
\begin{example}
    Dopo aver chiuso  un circuito LR (circuito RL in fase di carica) la corrente nel tempo varia secondo la legge
    \[
        I=I_m\Bigl(1-e^{-\frac{t}{\tau}}\Bigr)
    \]
    con $I_m=f/R$ e $\tau=L/R$.
    Una volta chiuso il circuito quindi la corrente aumenta passando da $0$ a $I_m$, ovvero il valore
    che verrebbe raggiungo istantaneamente in assenza di induttanza.

    Per arrivare a questo risultato bisogna semplicemente risolvere l'equazione del circuito con $f$ costante
    e condizione iniziale $I(0)=0$.
    Con la notazione introdotta l'equazione può essere riscritta come
    \[
        I=I_m-\tau \dv{I}{t}
    \]
    Con la sostituzione $I\rightarrow I_m-I=u$ si ha $\dv*{I}{t}=-\dv*{u}{t}$ e $u(0)=I_m$. Di conseguenza
    \[
        \frac{\dv{u}{t}}{u}=-\rec{\tau}
    \]
    Integrando ambo i membri da $0$ a $t$
    \[
        \ln(u)-\ln(u(0))=-\frac{t}{\tau}
    \]
    Mediante sostituzione inversa si ottiene l'equazione cercata.
    Se ora in \eqref{eqn:RL} si isola $f_a$ e si inserisce l'andamento appena trovato per la corrente, si ottiene che
    la differenza di potenziale ai capi dell'induttanza vale $f_a=-fe^{-\frac{t}{\tau}}$, ovvero
    $f_a$ si oppone quindi ad $f$ ma diminusce col tempo fino ad azzerarsi.
\end{example}

\begin{thm}[Legge di Felici]
    \label{teo:felici}
    Sia data una spira con area $S$ e costituita da $N$ avvolgimenti, di resistenza $R$ e immersa in un campo
    di induzione magnetica costante nel tempo. Detta $Q$ la carica totale che attraversa il circuito
    \[
        Q=\frac{\Phi_i-\Phi_f}{R}
    \]
    ovvero la carica totale che attraversa il circuito dipende solo dallo stato iniziale e dallo stato finale.
    In particolare, se la spira viene portata in una zona priva di campo magnetico
    \[
        Q=\frac{NS[B]}{R}
    \]
    con $[B]$ il valor medio del campo di induzione magnetica sulla superficie della spira.
\end{thm}
\begin{proof}
    Dimostriamo la prima parte dell'asserto.
    In assenza di generatori, l'unica forza elettromotrice nel circuito è quella autoindotta la cui espressione
    è data dalla legge di Faraday-Neumann. Per la prima legge di Ohm la corrente nel circuito allora vale
    \[
        I=\rec{R}f_i=-\rec{R}\pdv{\Phi_S(\vb{B})}{t}
    \]
    La carica totale che scorre nel circuito dall'istante iniziale al tempo $t$ allora vale
    \[
        Q=\int_0^t I\dd{t}=-\rec{R}\int_0^t\pdv{\Phi_S(\vb{B})}{t}\dd{t}=\frac{\Phi_i-\Phi_f}{R}
    \]
    Per la seconda parte, si ha che la spira è inizialmente posta in quiete nel campo
    \[
        \Phi_i(\vb{B})=N\int_S \vb{B}\vdot\dd{\vb{S}}=NS\Biggl(\rec{S}\int_S \vb{B}\vdot\dd{\vb{S}} \Biggr)=NS[B]
    \]
    Se si porta la spira in una regione in cui il campo di induzione magnetica è nullo $\Phi_f(\vb{B})=0$ e quindi
    per quanto mostrato
    \[
        Q=\frac{NS[B]}{R}
    \]
\end{proof}


\section{L'induzione mutua}
Si considerino due circuiti $C_1$ e $C_2$ in condizioni quasi-stazionarie, immersi in un mezzo
omogeneo e isotropo,
e si supponga che $C_1$ sia percorso all'istante $t$ da una corrente $I_1(t)$.
Il circuito $C_1$ è quindi sorgente di un campo $\vb{B}_1(t)$ che per la \eqref{eqn:induttanza}
è proporzionale a $I_1(t)$. Allora anche il flusso di $\vb{B}_1$ concatenato a $C_2$
è proporzionale a $I_1$, ovvero $\Phi_2(\vb{B}_1)=M_{12}I_1$.
\begin{defn}[Induttanza mutua]
    Si definisce induttanza mutua il coefficente di proporzionalità nell'equazione precedente
\end{defn}
Ovviamente, mediante un ragionamento analogo si può definire il coefficente $M_{21}$.
Il seguente teorema garantisce la non ambiguità della definizione.
\begin{thm}
    \[
        M_{12}=M_{21}
    \]
\end{thm}
\begin{proof}
    Ricordando l'equazione definitoria del potenziale vettore \eqref{eqn:def_potenziale_vettore} si può scrivere
    \[
        \Phi_2(\vb{B}_1)=\int_{S_2}\vb{B_1}\vdot\dd{\vb{S}_2}=\int_{S_2}\curl{\vb{A}_1}\vdot\dd{\vb{S}_1}=\oint_{l_2}\vb{A}_1(\vb{r}_2)\vdot\dd{\vb{l}_2}
    \]
    Sostituendo la \eqref{eqn:potenziale_vettore_circuito_lineare}
    \[
        \Phi_2(\vb{B}_1)=I_1\frac{\mu}{4\pi}\oint_{l_2}\oint_{l_1}\frac{\dd{\vb{l}_1}\vdot\dd{\vb{l}_2}}{\abs{\vb{r}_2-\vb{r}_1}}
    \]
    Confrontando ora quanto ottenuto con la definizione di induzione mutua si ha
    \[
        M_{12}=\frac{\mu}{4\pi}\oint_{l_2}\oint_{l_1}\frac{\dd{\vb{l}_1}\vdot\dd{\vb{l}_2}}{\abs{\vb{r}_2-\vb{r}_1}}
    \]
    L'espressione è indipendente dall'ordine degli indici, si ha quindi la tesi.
\end{proof}


\section{L'energia magnetica}
Nel caso statico non si è affrontato il discorso relativo all'energia magnetica. Questo perchè la creazione
di una configurazione stazionaria di correnti (e campi magnetici associati) richiede che esista
un periodo iniziale di transizione in cui le correnti si vanno da $0$ al loro valore finale. Per quanto
visto in questo capitolo, in questa fase sul sistema vengono indotte delle forze elettromotrici.
Siccome l'energia posseduta dal campo magnetico è l'energia che è stata necessaria alla sua produzione
questo contributo non può essere tralasciato.

L'oggetto attraverso il quale si passa per studiare l'energia magnetica è l'induttanza.
Si consideri quindi un circuito $RL$, descritto dall'equazione \eqref{eqn:RL}. Moltiplicando per la quantità $\dd{Q}=I\dd{t}$ si ottiene
\[
    f\dd{Q}=RI^2\dd{t}+LI\dd{I}
\]
Dalla definizione di forza elettromotrice emerge che il primo membro dell'equazione è l'energia erogata dal
generatore nell'intervallo di tempo $\dd{t}$. Per la legge sull'effetto Joule
\eqref{eqn:effetto_joule} il primo termine del secondo membro rappresenta l'energia dissipata nella resistenza.
Il secondo termine al secondo membro è quello invece che non ci sarebbe se non ci fosse induttanza:
lo si può quindi interpretare come l'energia da fornire all'induttanza affinchè la corrente circolante
in essa si porti nel tempo $\dd{t}$ dal valore $I$ al valore $I+\dd{I}$.
L'energia necessaria allora affinchè la corrente circolante nell'induttanza si porti da $0$ ad $I$ è
\begin{equation}
    U_L=\int_0^I LI\dd{I}=\rec{2}LI^2
    \label{eqn:U_induttore}
\end{equation}
Questa interpretazione è confermata dal seguente calcolo
\begin{obs}
    L'energia $U_L$ è proprio l'energia posseduta da un'induttanza $L$ percorsa da corrente $I$
\end{obs}
\begin{proof}
    Si consideri un circuito percorso da una corrente $I_m=f/R$. All'instante iniziale,
    il generatore viene staccato dal circuito il quale, viene chiuso in corto circuito. L'equazione del circuito è
    diventa quindi
    \[
        RI=-L\dv{I}{t}, \quad\quad I(0)=I_m
    \]
    la cui soluzione è
    \[
        I=I_me^{t/\tau}
    \]
    con $\tau=L/R$. La potenza dissipata per effetto Joule è $RI^2$ e quindi l'energia totale dissipata dalla resistenza è
    \[
        U_R=\int_0^\infty RI^2\dd{t}=\int_0^\infty \frac{f^2}{R}e^{-2t/\tau}\dd{t}=\frac{f^2}{2R^2}L=\rec{2}LI_m^2
    \]
    Per conservazione dell'energia, questa non può che essere l'energia posseduta dall'induttanza prima di chiudere il circuito.
\end{proof}
$U_L$ rappresenta l'energia potenziale immagazzinata nel dispositivo -ovvero il singolo circuito.
Si consideri ora un solenoide percorso da corrente costante all'interno del quale sia parzialmente inserito un nucleo di ferro.
Si verifica sperimentalmente che il nucleo viene risucchiato all'interno del solenoide. Questi fenomeno è in apparenza contraddittorio:
a nucleo completamente inserito infatti l'induttanza è maggiore\footnote{Si ricordi che l'induttanza di un solenoide è
$L=\mu_r\mu_0 N^2S/l$ e che per il ferro $\mu_r>0$.} e quindi lo è anche l'energia del solenoide - ovvero, apparentemente
il sistema si è spostato spontaneamente da una configurazione con energia minore ad una con energia maggiore. La spiegazione
si basa su un discorso analogo a quello svolto nel paragrafo \ref{par:corrente_correntestazionaria_complementi}
sulla forza fra le maglie di un condensatore: l'ipotesi che la corrente sia costante richiede implicitamente
che al circuito sia collegato un generatore in grado di soddisfare questa richiesta.

Si vuole ora calcolare l'energia potenziale $U_M$ immagazzinata in un insieme di più circuiti, soggetti quindi
a mutua induzione.
\begin{example}
    Si consideri una regione di spazio uniformemente riempita con un materiale
    con $B/H=\mu$ costante
    in cui sia presente un campo $\vb{B}$.
    Si consideri un circuito $RL$ costituito da un solenoide con una lughezza molto maggiore del suo diametro in fase di carica,
    descritto dall'equazione \eqref{eqn:RL}.
    Il campo interno al solenoide è uniforme e coincide col campo medio. Tenuto conto della \eqref{eqn:induttanza}
    e della legge di Felici \ref{teo:felici} si ottiene quindi
    \[
        L\dv{I}{t}=NS\dv{B}{t}
    \]
    Sostituendo quindi nella \eqref{eqn:RL} e moltiplicando per $I\dd{t}$, si ricava
    \[
        fI\dd{t}=RI^2\dd{t}+INS\dd{B}
    \]
    Per le considerazioni fatte all'inizio del paragrafo, l'energia fornita all'induttanza è, indicando con $n=N/l$
    \[
        \dd{U}_L=INS\dd{B}=SlnI\dd{B}=nVI\dd{B}
    \]
    dove $V=Sl$è il volume del solenoide. Dividendo per $V$ in modo da ricavare la densità di energia
    e tenendo presente la \eqref{eqn:H_solenoide}, si ottiene
    \[
        \dd{u}_L=\frac{\dd{U}_L}{V}=nI\dd{B}=H\dd{B}
    \]
    Fin'ora, non si è fatto uso dell'ipotesi di $\mu$ costante, quindi i risultati ottenuti sono indipendenti dal fatto che il
    materiale sia paramagnetico, ferromagnetico o diamagnetico\footnote{
        si noti invece l'ipotesi fortemente restrittiva di avere a che fare con un solenoide, ovvero con un campo uniforme su un volume
        di geometria semplice}.
    Considerando anche questa ipotesi
    \[
        u_M=u_L(B)=\int_0^B \dd{u}_L=\int_0^B H\dd{B}=\int_0^B \rec{\mu}B\dd{B}=\rec{2\mu}B^2=\rec{2}BH=\rec{2}\mu H^2
    \]


    la densità di energia magnetica vale in conclusione
    \begin{equation}
        u_M=\rec{2}\mu H^2
    \end{equation}
\end{example}

\begin{thm}
    L'energia magnetica per un numero $N$ di circuiti è
    \begin{equation}
        U_M=\rec{2}\sum_{i,j=1}^N M_{ij}I_iI_j
    \end{equation}
    con $M_{ii}=L_i$, o equivalentemente
    \begin{equation}
        \label{eqn:UM_flussi}
        U_M=\rec{2}\sum_{i=1}^N I_i\Phi_i
    \end{equation}
    dove $\Phi_i$ è il flusso di tutti i campi di induzione magnetica presenti nel sistema attraverso il circuito i-esimo.
\end{thm}
\begin{proof}
    Si parte dal caso più semplice di due soli circuiti e si arriva ad una generalizzazione ad $N$ circuiti.
    Il sistema di due circuiti è descritto da
    \[
        \begin{cases}
            &  f_1=L_1\dv{I_1}{t}+M_{12}\dv{I_2}{t}+R_1I_1 \\
            &  f_2=L_2\dv{I_2}{t}+M_{21}\dv{I_1}{t}+R_2I_2 \\
        \end{cases}
    \]
    Moltiplicando la prima equazione per $\dd{Q_1}=I_1\dd{t}$ e la seconda per $\dd{Q_2}=I_2\dd{t}$,
    ricordando che $M_{21}=M_{12}$, poi sommando membro a membro si ottiene
    \[
        (f_1I_1\dd{t}+f_2I_2\dd{t})=(I_1^2R_1+I_2^2R_2)\dd{t}+[L_1I_1\dd{I_1}+L_2I_2\dd{I_2}+M_{12}(I_1\dd{I_2}+I_2\dd{I_1})]
    \]
    Per confronto, il membro fra parentesi quadre rappresenta il differenziale di energia magnetica, ovvero l'energia da fornire
    per incrementare di $\dd{I}$ le correnti $I_1$ e $I_2$.
    \[
        \dd{U_M}=L_1I_1\dd{I_1}+L_2I_2\dd{I_2}+M_{12}(I_1\dd{I_2}+I_2\dd{I_1})=\dd(\rec{2}L_1I_1^2+\rec{2}L_2I_2^2+M_{12}I_1I_2)
    \]
    Integrando da $0$ ai valori finali $I_1$, $I_2$
    \[
        U_M=\rec{2}L_1I_1^2+\rec{2}L_1I_1^2+M_{12}I_2I_2
    \]
    dove i primi due termini rappresentano l'energia dei signoli circuiti, mentre il terzo rappresenta l'energia di mututa induzione.
    Per l'uguaglianza fra i coefficenti di mutua induzione è possibile scrivere $M_{12}I_1I_2=1/2(M_{12}I_1I_2+M_{21}I_2I_1)$.
    Chiamando quindi $L_1=M_{11}$ ed $L_2=M_{22}$ la formula ottenuta può essere scritta come
    \[
        U_M=\rec{2}(M_{11}I_1^2+M_{12}I_1I_2+M_{21}I_2I_1+M_{22}I_2^2)=\rec{2}\sum_{i,j=1}^2M_{ij}I_iI_j
    \]

    Raccogliendo $I_1$ per i primi due termini del secondo membro e $I_2$ per gli altri due, si ha,
    dalle definizioni di induttanza e di coefficente di mutua induttanza:
    \[
        U_M=\rec{2}I_1(M_{11}I_1+M_{12}I_2)+\rec{2}I_2(M_{21}I_1+M_{22}I_2)=\rec{2}(I_1\Phi_1+I_2\Phi_2)=\rec{2}\sum_{i=1}^2I_i\Phi_i
    \]

    Entrambe le espressioni trovate per $U_M$ sono immediatamente generalizzabili al caso di $N$ circuiti.
\end{proof}


\begin{thm}
    Le forze agenti su un circuito percorso da corrente $I$ immerso in un campo magnetico esterno costante
    $\vb{B}_{ext}$ sono date dal gradiente (non cambiato di segno) dell'energia potenziale, che può essere scritta
    nel caso di un singolo circuito come
    \[
        U=I\Phi(\vb{B}_{ext})
    \]
\end{thm}
\begin{proof}
    Il sistema può essere schematizzato come costituio da
    $N$ circuiti, $N-1$ dei quali sono le sorgenti di $\vb{B}_{ext}$. Sia il circuito k-esimo percoro da corrente $I_k$ e abbia
    resistenza complessiva $R_k$. Si immagini di applicare una traslazione virtuale a questo circuito
    mediante una forza esterna $\vb{F}^{(k)}$ che imprima sul circuito una veloctià $\vb{v}^{(k)}$
    piccola, in modo tale che sia trascurabile l'energia cinetica associata a questo movimento.
    L'energia del circuito, ovvero l'energia magnetica\footnote{L'energia magnetica è l'energia potenziale totale
    immagazzinata nel circuito in quanto non sono presenti condensatori.}, varierà come conseguenza di tre contributi:
    $\vb{F}^{(k)}\vdot\dd{\vb{x}}$ lavoro compiuto
    dalla forza esterna; $\dd{L}_G$ lavoro compiuto dai generatori nel circuito; $\dd{L}_{R}$ lavoro
    dissipato (e quindi negativo) per effetto Joule. Per la conservazione dell'energia, rapportando tutto all'unità di tempo
    \[
        \dv{U_M}{t}=\vb{F}^{(k)}\vdot\dd{\vb{v}} + \dv{L_G}{t} - \dv{L_R}{t}
    \]
    Ricordando che
    \[
        \begin{split}
            &\dv{L_G}{t}=\sum I_j f_j=\sum I_j(I_j R_j + \dv{\Phi_j}{t} ) \\
            &\dv{L_R}{t}=\sum I^2_j R_j \\
            &\dv{U_M}{t}=\dv{t} \Biggl(\rec{2} \sum I_j \Phi_j \Biggr)=
            \rec{2} \sum I_j \dv{\Phi_j}{t}
        \end{split}
    \]
    Si ha
    \[
        \vb{F}^{(k)}\vdot\dd{\vb{v}}=-\rec{2}\sum I_j\dv{\Phi_j}{t}=-\dv{U_M}{t}
    \]
    Per garantire una traslazione virtuale a velocità trascurabile la forza esterna deve essere
    uguale in modulo e opposta alla forza $\vb{f}^{(k)}$ che il campo magentico esercita sul
    circuito k-esimo. Ne segue che $\vb{f}^{(k)}\vdot\dd{\vb{x}}=\dd{U_M}$, ovvero la forza è data
    dal gradiente di $U_M$.

    Svolgendo esplicitamente il gradiente dell'energia magnetica, se è costante la corrente ed il circuito è
    indeformabile (in modo che $L$ sia costante), le derivate dell'energia magnetica totale coincidono con le sole derivate dell'
    energia di accoppiamento, ovvero dell'energia dovuta all'interazione del circuito k-esimo con gli altri $N-1$ circuiti
    \[
        U_{acc}=\sum_{j\neq k} I_k \,M_{kj}I_j=I_k\Phi^{(k)}_{acc}(\vb{B})
    \]
    dove $\Phi^{(k)}_{acc}(\vb{B})$ è il flusso concatenato col k-esimo circuito, prodotto
    da tutti gli altri circuiti. Ma questo è proprio il flusso del campo esterno attraverso il
    circuito k-esimo, ovvero la tesi.
\end{proof}

Quelle ottenute sono espressioni di tipo integrale. È interessante cercare di trovare delle relazioni locali:
queste forme si riveleranno utili sia per generalizzare quanto visto fin'ora al caso non stazionario,
sia per trovare la densità di energia nella forma il più generale possibile.
\begin{thm}
    Dato sistema costituito da $N$ circuiti e un volume $\tau$ che racchiuda tutti i circuiti, l'energia magnetica del sistema è:
    \begin{equation}
        \label{eqn:UM_J_A}
        U_M= \rec{2}\int_\tau \Biggl(\vb{J}+\pdv{\vb{D}}{t}\Biggr)\vdot\vb{A}\dd{\tau}
    \end{equation}
\end{thm}
\begin{proof}
    In un sistema di $N$ circuiti si consideri nel dettaglio il circuito k-esimo.
    Questo sarà costituito da un conduttore di sezione non nulla $\sigma_k$ chiuso su se stesso
    e percorso da una corrente $I_k=\int_{\sigma_k}\vb{J}_k\vdot\dd{\vb{\sigma}_k}$.
    Si consideri la superficie $S_k$ che abbia $l_k$ come contorno. Per la \eqref{eqn:UM_flussi}
    \[
        \begin{split}
            U_M &=\rec{2}\sum_k I_k\Phi_k=\rec{2}\sum_k\Biggl[\int_{\sigma_k}\vb{J}_k\vdot\dd{\vb{\sigma}_k} \int_{S_k}\vb{B}\vdot\dd{\vb{S}_k} \Biggr]\\
            &=\rec{2}\sum_k\Biggl[\int_{\sigma_k}\vb{J}_k\vdot\dd{\vb{\sigma}_k} \int_{S_k}(\curl{\vb{A}})\vdot\dd{\vb{S}_k} \Biggr]=\\
            &=\rec{2}\sum_k\Biggl[\int_{\sigma_k}\vb{J}_k\vdot\dd{\vb{\sigma}_k} \oint_{l_k}\vb{A}\vdot\dd{\vb{l}_k} \Biggr]
        \end{split}
    \]
    Siccome $\dd{\vb{l}_k}$, $\vb{J}_k$, $\dd{\vb{\sigma}_k}$ sono paralleli, indicando con $\vu{l}_k$
    la direzioni di questi vettori, la relazione può essere riscritta come
    \[
        \begin{split}
            U_M&=\rec{2}\sum_k\Biggl[\int_{\sigma_k}J_k\dd{\sigma_k} \oint_{l_k}\vb{A}\vdot\vu{l}_k\dd{l_k} \Biggr]
            =\rec{2}\sum_k \int_{\sigma_k}\oint_{l_k}J_k\vu{l}_k \vdot \vb{A}\dd{\sigma_k}\dd{l_k} \\
            &=\rec{2}\sum_k \int_{\tau_k} \vb{J}_k \vdot \vb{A}\dd{\tau_k}
        \end{split}
    \]
    Siccome la densità di corrente è nulla al di fuori dei circuiti, si può elminiare la sommatoria
    trasformando l'integrale su $\tau_k$ in un integrale su un qualsiasi volume $\tau$ che racchiuda tutti i circuiti,
    ottenendo la tesi nel caso stazionario.

    Sostituendo la densità di corrente con la densità di corrente
    generalizzata, si ottiene la tesi nel caso generale.
\end{proof}

\begin{thm}
    Data una densità di corrente $J$ l'energia magnetica vale
    \begin{equation}
        \label{eqn:UM}
        U_M=\rec{2} \int_{S} (\vb{H}\cp\vb{A}) \vdot \dd{\vb{S}} +\rec{2} \int_{\tau}\vb{H}\vdot\vb{B}\dd{\tau}
    \end{equation}
\end{thm}
\begin{proof}
    Per la quarta equazione di Maxwell nel caso non stazionario, l'integrando nella \eqref{eqn:UM_J_A} può essere riscritto come
    \[
        \Biggl(\vb{J}+\pdv{\vb{D}}{t}\Biggr)\vdot\vb{A}=(\curl{\vb{H}})\vdot\vb{A}
    \]
    Tenuto conto dell'identità \eqref{app:eqn:div_cp}
    \[
        (\curl{\vb{H}})\vdot\vb{A}=\div(\vb{H}\cp\vb{A})+\vb{H}\vdot(\curl{\vb{A}})
    \]
    E indicando con $S$ la superficie che racchiude il volume $\tau$, $U_M$ può essere riscritta come
    \[
        \begin{split}
            U_M&=\rec{2} \int_{\tau} \div(\vb{H}\cp\vb{A})\dd{\tau} +\rec{2} \int_{\tau}\vb{H}\vdot(\curl{\vb{A}})\dd{\tau}\\
            &=\rec{2} \int_{S} (\vb{H}\cp\vb{A}) \vdot \dd{\vb{S}} +\rec{2} \int_{\tau}\vb{H}\vdot\vb{B}\dd{\tau}
        \end{split}
    \]
\end{proof}
L'equazione ottenuta è formalmente analoga alla \eqref{eqn:UE}: valgono tutte le considerazioni fatte in quel caso
e in particolare
\begin{cor}
    Se si prende in considerazione tutto lo spazio, allora l'energia vale
    \begin{equation}
        \label{eqn:uM_H_B}
        U_M=\int u\dd{\tau} \quad\quad\quad \text{ con } u=\frac{\vb{H}\vdot\vb{B}}{2}
    \end{equation}
\end{cor}
Un'osservazione su quanto ottenuto. Sia l'integranda della \eqref{eqn:UM_J_A} che la \eqref{eqn:uM_H_B}
godono della proprietà che il loro integrale esteso a tutto lo spazio sia pari all'energia magnetica.
Di fatto però, le due funzioni non sono necessariamente uguali punto per punto. Sono allora considerazioni di carattere puramente fisico
che portano ad interpretare la seconda come l'effettiva densità di energia.

Si conclude la sezione con un confronto fra l'energia magnetica e l'energia elettrostatica.
La prima analogia è fra le formule per l'energia di un'induttore \eqref{eqn:U_induttore} e quella per un condensatore \eqref{eqn:U_condensatore}.
In entrambi i casi la struttura della formula è quella di un prodotto fra l'integrale della densità della sorgente del campo
e la caratteristica del circuito. Inoltre, sia nel calcolo delle forze meccaniche dovute all'energia magnetica che di quelle dovute
all'energia elettrostatica si prendono le derivate positive dell'energia, contrariamente a quanto si fa nel caso meccanico.



\chapter{Il campo elettromagnetico}
Si vuole studaire qui cosa succede facendo cadere l'ipotesi di quasi stazionarietà.

\section{Equazioni di Maxwell e campo elettromagnetico}
I campi $\vb{E}$ e $\vb{B}$, così come i campi $\vb{D}$ e $\vb{H}$, sono legati alle caratteristiche
delle sorgenti dalle equazioni di Maxwell
%lasciare questa riga vuota

\begin{minipage}[t]{0.5\textwidth}
    \[
        \begin{split}
            & \div{\vb{D}}=\rho                    \\
            & \curl{\vb{E}}=-\pdv{\vb{B}}{t}       \\
        \end{split}
    \]
\end{minipage}
\begin{minipage}[t]{0.5\textwidth}
    \[
        \begin{split}
            & \div{\vb{B}}=0                       \\
            & \curl{\vb{H}}=\vb{J}+\pdv{\vb{D}}{t} \\
        \end{split}
    \]
\end{minipage}
Alcune considerazioni sulle equazioni di Maxwell. Le sorgenti del campo elettrico ($\rho$) e del campo magnetico ($\vb{J}$)
non sono fra loro indipendenti ma sono legate dall'equazione di continuità \eqref{eqn:continuità} e ciò riflette
il fatto che fisicamente le stesse cariche elettriche, che danno luogo ad una densità di carica $\rho$, quando
sono in movimento danno luogo ad una densità di corrente. Il fatto che le cariche siano ferme o in
movimento dipende dal sistema di riferimento scelto per descrivere il fenomeno e quindi è relativo il fatto
di avere a che fare con un campo elettrico o con un campo magnetico.
Nelle quattro equazioni di Maxwell inoltre è evidente una certa asimmetria: al secondo membro della prima compare la densità di carica, mentre
il secondo membro della seconda è sempre nullo; allo stesso modo nel secondo membro della terza non compare un addendo
equivalente alla densità di corrente presente nella quarta. Questo riflette il fatto che non c'è evidenza sperimentale dell'esistenza
del monopolo magnetico, l'equivalente magnetico della carica elettrica.
Infine, le equazioni di Maxwell mostrano come la derivata temporale di $\vb{E}$ sia una delle sorgenti
di $\vb{B}$ e viceversa. Questa circostanza giustifica una trattazione unificata dei due fenomeni mediante un unico campo detto
\textit{campo elettromagnetico}. Bisogna osservare che un campo elettromagnetico può essere presente in regioni di spazio
prive di sorgenti e, più  avanti, si dimostrerà che esso possiede quantità di moto, energia e momento angolare. Si
tratta quindi di un'entità fisica e non di un mero artificio introdotto per trattare i due fenomeni in maniera unificata.
Nella trattazione del campo elettromagnetico si faranno due fondamentali approssimazioni: si considereranno le distribuzioni di
carica come continue, sebbene a livello microscopico la carica sia quantizzata e si considereranno i campi come funzioni
continue (approssimazione classica) nonostante la meccanica quantistica prescriva che il campo elettromagnetico sia quantizzato
in fotoni. Queste approssimazioni sono ragionevoli per la maggior parte dei fenomeni macroscopici.

Dalle considerazioni fatte emerge chiaramente l'importanza di studiare le proprietà delle equazioni di Maxwell.
Queste sono otto equazioni scalari in dodici incognite ($\vb{E},\vb{D}$,$\vb{B},\vb{H}$), per cui è necessario
fornire delle relazioni aggiuntive affinchè siano risolubili: queste sono le relazioni fenomenologiche che legano
$\vb{E}$ a $\vb{D}$ e $\vb{B}$ a $\vb{H}$. In questo in modo le incognite si riducono a sei. Le otto equazioni
non sono perciò indipendenti.
\begin{obs}
    La seconda equazione di Maxwell può essere dedotta dalla terza, analogamente la prima può esere dedotta dalla quarta\footnote{
        la dimostrazione viene presentata usando le equazioni nel vuoto. Nel caso dei materiali si procede in modo analogo.}.
\end{obs}
\begin{proof}
    Applicando l'operatore divergenza alla terza equazione di Maxwell si ha
    \[
        -\div{\pdv{\vb{B}}{t}}=-\pdv{t}\div{\vb{B}}=\div(\curl{\vb{E}})=0
    \]
    dove si è usato il teorema di Schwarz ed il fatto che la divergenza del rotore è nulla.
    Questo implica che $\div{\vb{B}}=cost$. Prima che venissero realizzate le sorgenti di $\vb{B}$, il campo era nullo
    in tutto lo spazio e anche la sua divergenza doveva quindi essere nulla. Siccome tutte le equazioni usate sono
    indipendenti da ciò che accade alle sorgenti, la costante non può che essere $0$.
    In maniera analoga, applicando la divergenza alla quarta equazione di Maxwell si ha
    \[
        0=\div{\vb{J}} + \div{\pdv{\vb{D}}{t}}=-\pdv{\rho}{t} + \pdv{t}\div{\vb{D}}=\pdv{t}(-\rho + \div{\vb{D}})
    \]
    Avendo sostituito $\div{\vb{J}}$ con l'espressione fornita dall'equazione di continuità \eqref{eqn:continuità}.
    Ne segue che $-\rho + \div{\vb{D}}=cost$ dalla quale si arriva alla prima equazione di Maxwell
    per considerazioni analoghe a quelle fatte nel caso precedente.
\end{proof}
Ne segue quindi che tutte le proprietà dei campi elettrici e magnetici sono contenute nella terza
e nella quarta equazione di Maxwell, ovvero un sistema di sei equazioni in sei incognite.
Specificata la configurazione spazio-temporale delle sorgenti $\rho$ e $\vb{J}$ (configurazione che dovrà rispettare
l'equazione di continuità), assegnate le condizioni iniziali e le condizioni al contorno, le equazioni di Maxwell
corredate dalle opportune relazioni fenomenologiche consentono in linea di principio di
calcolare i campi, tramite i quali è immediatamente nota l'azione subita da una carica campione (grazie alla relazione
$\vb{F}=q(\vb{E}+\vb{v}\cp\vb{B})$).


Un caso in cui è particolarmente semplice risolvere le equazioni di Maxwell è quello del
campo elettromagnetico in cui un solo materiale isotropo e omogeneo riempia lo spazio.
In questo caso è necessario aggiungere alle sorgenti macroscopiche dei campi
anche le correnti e le cariche microscopiche che i campi inducono nel materiale, che non sono note a priori proprio perchè
indotte dal campo. Nell'ipotesi però che il materiale sia omogeneo ed isotropo (e che non sia ferromagnetico, a meno che
$B(H)$ sia una curva univoca e lineare),
è sufficiente considerare le equazioni di Maxwell analoghe a quelle nel vuoto, con $\epsilon$ e $\mu$
al posto di $\epsilon_0$ e $\mu_0$, ovvero introducendo $\vb{D}=\epsilon \vb{E}$  e $\vb{H}=\vb{B}/\mu$.
I risultati ottenuti nel vuoto restano quindi validi validi.
Nel caso in cui diverse porzioni dello spazio siano rimepite con materiali diversi,
il problema viene trattato risolvendo le equazioni di Maxwell in ciascuna porzione di spazio e imponendo poi le condizioni di raccordo

\begin{minipage}[t]{0.5\textwidth}
    \[
        \begin{cases}
            & E_{t1}=E_{t2}\\
            & D_{n1}=D_{n2}
        \end{cases}
    \]
\end{minipage}
\begin{minipage}[t]{0.5\textwidth}
    \[
        \begin{cases}
            & H_{t1}=H_{t2}\\
            & B_{n1}=B_{n2}
        \end{cases}
    \]
\end{minipage}
che ovviamente, possono essere semplificate come visto sopra nel caso in cui tutti i materiali siano isotropi e omogenei.

È possibile dalle equazioni di Maxwell ricavare informazioni sulle caratteristiche delle sorgenti miscroscopiche
indotte dai campi, riscrivendo le equazioni usando i vettori polarizzazione elettrica $\vb{P}$ e magnetica $\vb{M}$
al posto di $\vb{D}$ ed $\vb{H}$, per semplice inversione delle definizioni
\[
    \begin{split}
        & \vb{D}=\epsilon_0 \vb{E} + \vb{P} \\
        & \vb{H}=\rec{\mu_0}\vb{B} - \vb{M} \\
    \end{split}
\]
Si ottiene in questo modo
\[
    \begin{split}
        & \epsilon_0\div{\vb{E}}=\rho-\div{\vb{P}}\\
        & \div{\vb{B}}=0 \\
        & \curl{\vb{E}}=-\pdv{\vb{B}}{t}\\
        & \rec{\mu_0}\curl{\vb{B}}=\vb{J}+\epsilon_0\pdv{\vb{E}}{t}+\pdv{\vb{P}}{t}+\curl{\vb{M}}
    \end{split}
\]
Queste 8 equazioni (ricoducibili a 6 fra loro indipendenti) non sono sufficienti a ricavare le
12 grandezze scalari $\vb{E}$, $\vb{B}$, $\vb{P}$, $\vb{M}$ a meno che i vettori di polarizzazione
siano noti a priori, ovvero siano note a priori le condizioni che li legano ai relativi
campi. Questo non avviene praticamente mai ma una volta risolte le equazioni con $\vb{D}$ e $\vb{H}$ è possibile
determinare le caratteristiche delle sorgenti microscopiche per confronto.
Infatti, rispetto alle equazioni di Maxwell nel vuoto, le equazioni appena ricavate differiscono per tre termini aggiuntivi:
\[
    \rho'_p=-\div{\vb{P}} \quad\quad \vb{J}'_M=\curl{\vb{M}} \quad\quad \vb{J}'_p=\pdv{\vb{P}}{t}
\]
che rappresentano rispettivamente la densità di carica di polarizzazione, la densità di corrente di
magnetizzazione e la densità di corrente di polarizzazione elettrica, ovvero la densità di corrente dovuta al fatto che quando
un dielettrico si polarizza necessariamente c'è un movimento ordinato di cariche all'interno del materiale. Quest'ultimo
è un termine caratteristico del caso non stazionario a differenza dei primi due che invece erano presenti già nel caso stazionario.

Da qui in avanti si farà riferimento solo ed esclusivamente a mezzi isotropi e omogenei.


\section{Energia e vettore di Poynting}
I fenomeni elettromagnetici sono soggetti al principio di conservazione dell'energia. Questo
significa che l'energia del campo magnetico più l'energia dei sistemi con cui il campo
interagisce deve essere costante nel tempo.
Si introduce la seguente definizione
\begin{defn}[vettore di Poynting]
    Si definisce vettore di Poynting
    \begin{equation}
        \label{eqn:poynting}
        \vb{I}=\vb{E}\cp\vb{H}=\frac{\vb{E}\cp\vb{B}}{\mu}
    \end{equation}
\end{defn}
Si ha che $[I]=J/(m^2s)=W/m^2 $.
\begin{thm}[teorema di Poynting]
    Detta $S$ una superficie chiusa di forma costante che racchiude il campo elettromagnetico e detto $\tau$ il volume interno a questa
    superficie, si ha che la variazione di energia del campo magnetico vale
    \begin{equation}
        \label{eqn:dU_em}
        -\pdv{U}{t}=\int_S \vb{I}\vdot\dd{\vb{S}}+\int_{\tau} \vb{E}\vdot\vb{J}\dd{\tau}
    \end{equation}
\end{thm}
\begin{proof}
    L'energia posseduta dal campo elettromagnetico contenuto in $S$ è
    \[
        U=\int_{\tau}\rec{2}\vb{E}\vdot\vb{D}\dd{\tau}+\int_{\tau}\rec{2}\vb{H}\vdot\vb{B}\dd{\tau}
    \]
    Derivando rispetto al tempo, tenendo conto che $\vb{D}=\epsilon \vb{E}$ e $\vb{B}=\mu\vb{H}$ si ha
    \[
        \begin{split}
            \pdv{U}{t}&=\rec{2}\int_{\tau}\Biggl[\pdv{t} (E^2\epsilon  ) + (H^2\mu  ) \Biggr] \dd{\tau}
            =\rec{2}\int_{\tau}\Biggl[2\Biggl( \pdv{\vb{E}}{t} \Biggr)\vdot \vb{E}\epsilon  +
            2\Biggl(\pdv{\vb{H}}{t} \Biggr)\vdot \vb{H}\mu \Biggr]\dd{\tau} \\
            &=\int_{\tau}\Biggl[\Biggl( \pdv{\vb{D}}{t} \vdot \vb{E} \Biggr) +
            \Biggl( \pdv{\vb{B}}{t} \vdot \vb{H} \Biggr)\Biggr] \dd{\tau}
        \end{split}
    \]
    Sostituendo le derivate sotto al segno di integrale con le espressioni fornite dalla terza e dalla quarta equazione di Maxwell
    \[
        \pdv{U}{t}=\int_{\tau}\Biggl[ (\curl{\vb{H}}) \vdot \vb{E} -\vb{E}\vdot\vb{J} - (\curl{\vb{E}}) \vdot \vb{H} \Biggr]\dd{\tau}
    \]
    Per l'identità \eqref{app:eqn:div_cp}
    \[
        \pdv{U}{t}=-\int_{\tau}\Biggl[ \div(\vb{E} \cp \vb{H}) + \vb{E}\vdot\vb{J} \Biggr]\dd{\tau}
    \]
    Separando ora i due integrali e usando il teorema della divergenza si ottiene quindi
    \[
        -\pdv{U}{t}=\int_{S} (\vb{E} \cp \vb{H})\vdot\dd{\vb{S}} + \int_{\tau} \vb{E}\vdot\vb{J} \dd{\tau}=
        \int_{S} \vb{I} \vdot\dd{\vb{S}}+ \int_{\tau} \vb{E}\vdot\vb{J} \dd{\tau}
    \]
\end{proof}

Si vuole ora dare un'interpretazione fisica a questo risultato.
Il flusso del vettore di Poynting, se positivo, ovvero se $\vb{I}$ è "uscente" dal volume,
comporta una diminuzione di energia nel tempo. $\vb{I}$ rappresenta lo spostamento di
energia dentro o fuori dal volume preso in considerazione. Per afferrare il concetto, può essere
utile scrivere il risultato del teorema in forma locale
$-\pdv{t}\int_\tau u \dd{\tau}=\int_\tau \div{\vb{I}} \dd{\tau} +\int \vb{E}\vdot\vb{J}\dd{\tau}$
da cui segue che $\-\pdv{t}u=\div{\vb{I}}+\vb{E}\vdot{J}$. Se non fosse presente il termine
$\vb{E}\vdot\vb{J}$ si avrebbe un'equazione di continuità per l'energia:
\[
    \div{\vb{U}}+\pdv{u}=0
\]
Questa equazione afferma che se si ha una variazione di energia si deve avere anche una "corrente di energia"
che la porti da qualche parte. Emerge a questo punto il significato del termine $\vb{E}\vdot\vb{J}$ che
rappresenta una variazione dell'energia elettromagnetica per trasformazione da (o in) altre forme di energia: è
in pratica l'energia ceduta o ricevuta dalla materia per azione del campo elettromagentico sulle cariche che
la costituiscono,
Si considerino a tal proposito le cariche contenute in una
porzione infinitesima $\dd{\tau}$ del volume $\tau$. Il numero di cariche per unità di volume
è $n=\dd{N}/\dd{\tau}$. La forza esercitata dal campo elettromagnetico su $\dd{\tau}$ è
\[
    \dd{\vb{F}}=\dd{N}q(\vb{E+\vb{v}_d\cp\vb{B}})=nq(\vb{E+\vb{v}_d\cp\vb{B}})\dd{\tau}
\]
La potenza trasferita dal campo magetico alle cariche libere presenti nell'elemento di volume è dunque
\begin{equation}
    \dd{P}=\dd{F}\vdot\vb{v}_d=nq\vb{v}_d\vdot (\vb{E}+\vb{v}_d\cp\vb{B})\dd{\tau}=nq\vb{v}_d\vdot \vb{E} \dd{\tau}=\vb{E}\vdot\vb{J}\dd{\tau}
    \label{eqn:potenza_em}
\end{equation}
Risulta subito evidente dai passaggi come la forza causata dal campo magnetico
non compiendo alcun lavoro non contribusca alla variazione di potenza.

Complessivamente allora la \eqref{eqn:dU_em} dice che la variazione di energia è dovuta sia all'interazione
del campo con la materia contenuta nel volume che al flusso del vettore di Poynting attraverso la superifice
che delimita il volume. Il vettore di Poynting è allora quel
vettore il cui flusso rappresenta l'energia del campo elettromagnetico che sfugge attraverso $S$.


\section{Potenziali elettrodinamici}
Oltre che tramite le equazioni di Maxwell, tutte le caratteristiche del campo elettromagnetico possono essere ottenute fornendo i
quattro potenziali generalizzati $V,\vb{A}$. Il fatto che per ricavare i potenziali, da cui è deducibile il campo elettromagnetico,
servano quattro equazioni e che ne servano sei per dedurlo a partire dalle equazioni di Maxwell non è una contraddizione: le
equazioni di Maxwell infatti contengono al loro interno anche le condizioni affinchè il campo elettromagnetico ammetta potenziale
- condizioni che sono date per valide nel momento in cui si assume l'esistenza di un potenziale.

Il potenziale vettore $\vb{A}$ è definito da una relazione analoga a quella introdotta nel paragrafo \ref{par:potenziale_vettore}
in quanto la seconda equazione di Maxwell, condizione necessaria per la validità della definizione,
non varia fra caso stazionario e non stazionario. Per quanto riguarda invece il potenziale scalare
$V$, è necessaria una generalizzazione: nel caso stazionario l'esistenza del potenziale elettrico è garantita dal fatto che il
campo elettrico sia conservativo, ma nel caso non stazionario questo non è più vero.
Introducendo l'equazione definitoria del potenziale vettore \eqref{eqn:def_potenziale_vettore}
nella terza equazione di Maxwell
\[
    \curl{\vb{E}}=-\pdv{t}(\curl{\vb{A}})=-\curl(\pdv{\vb{A}}{t})
\]
Da cui
\[
    \curl(\vb{E}+\pdv{\vb{A}}{t})=0
\]
Il vettore fra parentesi è irrotazionale e ammette quindi potenziale
\begin{defn}[potenziale scalare]
    Si definisce potenziale scalare una funzione che soddisfi la condizione
    \begin{equation}
        \label{eqn:def_potenziale_scalare}
        -\grad{V}=\vb{E}+\pdv{\vb{A}}{t}
    \end{equation}
\end{defn}
Si osservi come nel caso stazionario questa definizione si riduca a quella fornita nel paragrafo \ref{par:potenziale_elettrico}.
Condizione necessaria all'introduzione dei potenziali è la validità della seconda e della terza equazione di
Maxwell, ovvero di quelle omogenee\footnote{In queste due equazioni infatti non compaiono i termini noti dovuti
alle sorgenti.}, che risultano quindi identicamente soddisfatte una volta sostituiti al loro interno i
campi con i potenziali:
\[
    \begin{split}
        &\div{\vb{B}}=\div(\curl{\vb{A}})=0\\
        &\curl{\vb{E}}+\pdv{\vb{B}}{t}=\curl(\vb{E}+\pdv{\vb{A}}{t})=-\curl{\grad{V}}=0
    \end{split}
\]
Per ricavare i potenziali è quindi necessario usare la prima e la quarta equazione di Maxwell. Sostituendo le equazioni
definitorie dei potenziali all'interno di queste ultime si ottiene
\[
    \begin{split}
        &\laplacian{V}+\pdv{t}(\div{\vb{A}})=-\frac{\rho}{\epsilon}\\
        &\laplacian{\vb{A}}-\epsilon\mu \pdv[2]{\vb{A}}{t}-\grad(\div{\vb{A}}+\epsilon\mu\pdv{V}{t})=-\mu\vb{J}
    \end{split}
\]
L'aver introdotto i potenziali non sembra aver semplificato di molto le equazioni da risolvere per trovare le caratteristiche del campo
elettromagnetico, dato che queste quattro equazioni non sono disaccoppiate. Si osservi che se i termini fra parentesi fossero
nulli, le equazioni risulterebbero disaccoppiate e assumerebbero una forma molto semplice.
Siccome i potenziali non sono univocamente definiti risulta allora
importante cercare la trasformazione più generale che lasci invariati i campi e vedere se tramite questa sia possibile
semplificare le equazioni.
\begin{thm}
    Dati due potenziali $\vb{A}$ e $V$ che soddisfano rispettivamente la \eqref{eqn:def_potenziale_vettore} e la
    \eqref{eqn:def_potenziale_scalare} si possono produrre dei potenziali $\vb{A}'$ e $V'$ tali che da entrambe le coppie di potenziali
    sia deducibile lo stesso campo elettromagnetico. Affinchè questo accada la trasformazione che lega le due coppie di potenziali deve
    essere del tipo
    \begin{equation}
        \label{eqn:trasformazioni_gauge}
        \begin{cases}
            & \vb{A}\to \vb{A}'=\vb{A}+\grad{\phi}\\
            & V\to V'=V-\pdv{\phi}{t}
        \end{cases}
    \end{equation}
    con $\phi$ una funzione scalare di classe $C^2$.
\end{thm}
\begin{proof}
    La prima delle due equazioni ha dimostrazione immediata, già affrontata a suo tempo nel paragrafo \ref{par:potenziale_vettore}.
    Per quanto riguarda la seconda invece
    \[
        \vb{E}'=-\grad{V'}-\pdv{\vb{A}'}{t}=-\grad{V}+\grad(\pdv{\phi}{t})-\pdv{\vb{A}}{t}-\pdv{t}(\grad{\phi})=-\grad{V}-\pdv{\vb{A}}{t}=\vb{E}
    \]
\end{proof}
La trasformazione \eqref{eqn:trasformazioni_gauge} è detta \textit{trasformazione di gauge} e la funzione $\phi$ è
detta \textit{funzione di gauge}. Un'opportuna trasformazione di gauge permette di trovare delle equazioni disaccoppiate
per i potenziali.

Se i potenziali elettrodinamici soddisfano la condizione, detta condizione di Lorentz,
    \begin{equation}
        \label{eqn:condizione_lorentz}
        \div{\vb{A}}+\epsilon\mu\pdv{V}{t}=0
    \end{equation}
il sistema di equazioni si riducono al sistema di quattro equazioni disaccoppiate
    \begin{equation}
        \label{eqn:potenziali_maxwell}
        \begin{split}
            &\laplacian{V}-\epsilon\mu \pdv[2]{V}{t}=-\frac{\rho}{\epsilon}\\
            &\laplacian{\vb{A}}-\epsilon\mu \pdv[2]{\vb{A}}{t}=-\mu\vb{J}
        \end{split}
   \end{equation}
Quando i potenziali soddisfano la condizione di Lorentz si dice che essi appartengono alla \textit{gauge di Lorentz}.
\begin{thm}
    Data una coppia di potenziali $\vb{A}$ e $V$ è sempre possibile trovare una funzione $\phi$ affinchè i potenziali $\vb
    {A}'$ e $V'$ generati a partire da questi due tramite trasformazione di gauge appartengano alla gauge di Lorentz.
\end{thm}
\begin{proof}
    Si vuole dimostrare che è possibile determinare $\phi$ attraverso un'equazione differenziale che ammette sempre soluzione.
    Bisogna avere
    \[
        0= \div{\vb{A}'}+\epsilon\mu\pdv{V'}{t}=\div{\vb{A}}+\epsilon\mu\pdv{V}{t}+\laplacian{\phi}-\epsilon\mu\pdv[2]{\phi}{t}
    \]
    Da cui segue
    \[
        \laplacian{\phi}-\epsilon\mu\pdv[2]{\phi}{t}= -\Biggl( \div{\vb{A}}+\epsilon\mu\pdv{V}{t} \Biggr)
    \]
    Per ipotesi il secondo membro è noto perchè sono noti i potenziali e quindi questa equazione ammette soluzione.
\end{proof}

La dimensione del prodotto $\epsilon\mu$ è quella dell'inverso di una velocità al quadrato ed è quindi possibile riscrivere
le equazioni per i potenziali usando l'operatore dalemebrtiano
\[
    \label{eqn:potenziali_maxwell}
    \begin{split}
        & \square V=-\frac{\rho}{\epsilon}\\
        & \square \vb{A}=-\mu\vb{J}
    \end{split}
\]
In questo modo diventa lampante il fatto che le equazioni per i potenziali nel caso non stazionario sono analoghe
a quelle del caso stazionario, con l'unica differenza di sostituire l'operatore laplaciano col dalembertiano: sono
equazioni di onde con sorgente. Questa gauge risulta particolarmente adatta a descrivere problemi radiativi.
In questa gauge una soluzione particolare e degna di nota,
in quanto immediata generalizzazione della soluzione per il caso stazionario, è quella dei \textit{potenziali ritardati}.
\begin{thm}[Potenziali ritardati]
    Se le sorgenti sono localizzate in una porzione di spazio finita le \eqref{eqn:potenziali_maxwell} ammettono come soluzione
    \begin{equation}
        \begin{split}
            &\vb{A}(\vb{r},t)=\frac{\mu}{4\pi}\int_{\tau}\frac{\vb{J}(\vb{r}', t-\Delta r/v)}{\Delta r}\dd{\tau'}\\
            &V(\vb{r},t)=\rec{4\pi\epsilon}\int_{\tau}\frac{\rho(\vb{r}', t-\Delta r/v)}{\Delta r}\dd{\tau'}
        \end{split}
        \label{eqn:potenziali_ritardati}
    \end{equation}
    avendo indicato $v=1/\sqrt{\epsilon\mu}$
\end{thm}
Il nome \textit{potenziali ritardati} assume senso nel momento in cui si osserva che la densità di corrente e la densità di
carica, in ogni posizione $\vb{r}'$ non vanno calcolate al tempo a cui si sta calcolando il potenziale ma ad un istante $t'=t-\Delta r/v$.
Si può intuire come questo tempo sia quello impegato dal segnale elettromagnetico per percorrere lo spazio
tra la posizione $\vb{r}'$ e la posizione $\vb{r}$ in cui si sta calcolando il potenziale.
La soluzione più generale possibile si ottiene sommando ai potenziali ritardati la soluzione generale delle equazioni omogenee corrispondenti
alle \eqref{eqn:potenziali_maxwell}. Nel prossimo capitolo si osserverà come queste soluzioni siano delle onde.
A grande distanza dalle sorgenti localizzate, i potenziali ritardati vanno a zero molto più velocemente delle soluzioni omogenee:
a grande distanza quindi permangono solo le onde elettromagnetiche.


%La scelta di $\phi$ non toglie arbitrarietà ai potenziali: infatti la funzione di gauge non è univocamente determinata
%dall'equazione che compare nella dimostrazione del teorema, ma a questa va sommata la soluzione generale dell'equazione
%omogenea associata $\phi_0$. Presi quindi due potenziali $\vb{A}'$ e $V'$ primo apparteneneti alla gauge di Lorentz,
%anche
%\[
%    \begin{cases}
%        & \vb{A}''=\vb{A}'+\grad{\phi_0}\\
%        & V''=V'-\pdv{\phi_0}{t}
%    \end{cases}
%\]
%appartengono alla gauge di Lorentz.


Per descrivere i problemi di interazione radiazione materia la gauge più utile è la \textit{gauge di Coulomb}.
Se i potenziali elettrodinamici soddisfano la condizione
    \begin{equation}
        \div{\vb{A}}=0
    \end{equation}
    allora le equazioni per i potenziali si riducono al sistema di quattro equazioni disaccoppiate
    \begin{equation}
        \begin{split}
            &\laplacian{V}=-\frac{\rho(\vb{r},t)}{\epsilon}\\
            &\laplacian{\vb{A}}-\epsilon\mu \pdv[2]{\vb{A}}{t}=\epsilon\mu \grad\pdv{V}{t}-\mu \vb{J}
        \end{split}
    \end{equation}
In questa gauge il potenziale scalare è quindi analogo al potenziale elettrico, con l'unica differenza che la densità di carica
dipende dal tempo
\[
    V(\vb{r},t)=\rec{4\pi\epsilon}\int\frac{\rho(\vb{r}',t)}{\abs{\vb{r}-\vb{r'}}}\dd{\tau'}
\]
La gauge di Coulomb è utile in assenza di sorgenti: il potenziale scalare è in questo caso nullo e l'equazione
per il potenziale vettore diventa $\square \vb{A}=0$; i campi diventano
\[
    \vb{B}=\curl{\vb{A}}\quad\quad \vb{E}=-\pdv{\vb{A}}{t}
\]


Ecco quindi che risulta giustificato il percorso seguito in questo paragrafo: per determinare i potenziali è sufficiente risolvere quattro
equazioni differenziali disaccoppiate. Una volta noti i potenziali è possibile risalire al campo elettromagnetico tramite semplici
operazioni di derivazione. Introducendo i potenziali l'unico ostacolo al calcolo del campo si trova nella struttura complicata delle
funzioni che entrano in gioco ma viene eliminato qualsiasi problema relativo alla struttura delle equazioni.



\chapter{Le onde elettromagnetiche}
\section{Equazioni delle onde}
Una delle soluzioni più importanti delle equazioni di Maxwell è quella delle onde elettromagnetiche.
\begin{obs}
    Dato un dielettrico neutro, illimitato, isotropo, omogeneo e perfetto, si hanno le equazioni
    \begin{equation}
        \label{eqn:diff_onde}
        \begin{split}
            &\laplacian{\vb{E}}-\epsilon\mu \pdv[2]{\vb{E}}{t}=0\\
            &\laplacian{\vb{B}}-\epsilon\mu \pdv[2]{\vb{B}}{t}=0
        \end{split}
    \end{equation}
\end{obs}
\begin{proof}
    L'ipotesi di dielettrico illimitato, isotropo e omogeneo permette di usare le equazioni di Maxwell in forma analoga a quelle
    nel vuoto previa sostistuzione di $\epsilon_0$ e $\mu_0$ con $\epsilon$ e $\mu$.
    Per l'ipotesi di dielettrico perfetto si ha l'assenza di correnti macroscopiche e quindi $\vb{J}=0$, mentre per l'ipotesi di
    neutralità $\rho=0$. Perciò si ha
\begin{minipage}[t]{0.5\textwidth}
\[
    \begin{split}
        & \div{\vb{E}}=0                    \\
        & \curl{\vb{E}}=-\pdv{\vb{B}}{t}       \\
    \end{split}
\]
\end{minipage}
\begin{minipage}[t]{0.5\textwidth}
\[
    \begin{split}
        & \div{\vb{B}}=0                       \\
        & \curl{\vb{B}}=\mu\epsilon\pdv{\vb{E}}{t} \\
    \end{split}
\]
\end{minipage}
    Applicando l'operatore rotore alla terza equazione di Maxwell, per la \eqref{app:eqn:curl_curl} ricordando che per la prima
    equazione di Maxwell la divergenza del campo elettrico è nulla,
    \[
        \curl{\curl{\vb{E}}}=-\laplacian{\vb{E}}=-\curl{\pdv{\vb{B}}{t}}=-\pdv{t}(\curl{\vb{B}})
    \]
    Derivando rispetto al tempo la quarta equazione di Maxwell si ottiene
    \[
        \pdv{t}(\curl{\vb{B}})=\epsilon\mu\pdv[2]{\vb{E}}{t}
    \]
    Per sostituzione si ha l'equazione nella tesi relativa al campo elettrico. Applicando il rotore
    alla quarta equazione di Maxwell e derivando temporalmente la terza, si ottiene in maniera del tutto analoga l'
    equazione relativa al campo magnetico.
\end{proof}

L'aver applicato l'operatore rotore rende le equazioni ottenute non equivalenti alle equazioni di Maxwell. Se un campo
$\vb{E}$ soddisfa le equazioni di Maxwell infatti, le \eqref{eqn:diff_onde} sono soddisfatte anche da un campo
$\vb{E}+\vb{E}'$, con $\vb{E}' $ un qualunque campo irrotazionale. Le equazioni ottenute possono così avere per soluzioni anche campi
a divergenza non nulla a differenza delle equazioni di Maxwell: la solenoidalità delle soluzioni deve essere imposta
come condizione aggiuntiva.
Per le proprietà delle onde, il coefficente
$v=1/\sqrt{\epsilon\mu}$ rappresenta la velocità di propagazione dell'onda. In effetti dimensionalmente questo
coefficente è proprio una velocità.
Nel vuoto si
ha $v=c=1/\sqrt{\epsilon_0\mu_0}$ che risulta essere sperimentalmente uguale alla velocità della luce. Questo dimostra che la
luce è un'onda elettromagnetica.

-----------------------------------------------------

Alla luce di queste considerazioni si può
dare un significato fisico concreto a quanto visto in conclusione al precedente capitolo sui potenziali elettrodinamici in quanto
le equazioni qui ottenute hanno la stessa forma delle \eqref{eqn:potenziali_maxwell}.
I risultati ottenuti nella gauge di Coulomb, ovvero
\[
    \vb{B}=\curl{\vb{A}}\quad\quad \vb{E}-\pdv{\vb{A}}{t} \quad\quad \laplacian\vb{A}-\epsilon\mu \pdv[2]{\vb{A}}{t}=0
\]
danno informazioni sulle onde elettromagnetiche: il campo elettrico ed il campo magnetico sono ortogonali fra loro
e ortogonali alla direzione di propagazione dell'onda; il rapporto fra i loro moduli vale $v=(\epsilon\mu)^{-\rec{2}}$.
[PERCHè?]


\section{Onde piane}
La forma più semplice della soluzione all'equazione delle onde è quella di onda piana, che corrisponde ad una configurazione piana
delle condizioni al contorno,
ovvero quella in cui $\vb{E}$ e $\vb{B}$ assumono lo stesso valore per tutti i punti di ogni piano ortogonale alla
direzione di propagazione\footnote{Fisicamente questa condizione non si verifica mai, tuttavia è un limite per molti
casi di interesse pratico come ad esempio lo studio di una piccola porzione di spazio molto lontana dalla sorgente
(approssimazione di sorgente puntiforme).}
che può essere presa, senza perdita di generalità, parallela all'asse $x$ - o,
in altri termini, ogni componente dei campi è indipendente da $y$ e $z$. In questo caso ciascuna delle
sei componenti del campo elettromagntico soddisfa un'equazione di D'Alembert, ovvero un'equazione nella forma
\[
    \pdv[2]{f}{x}-\epsilon\mu\pdv[2]{f}{t}=0
\]
La cui soluzione generale è quindi del tipo
\begin{equation}
    \psi(x,t)=f_1(x-vt)+f_2(x+vt)
\end{equation}

\begin{thm}
    I campi elettrico e magnetico di un'onda elettromagnetica piana che si propaga in un dielettrico illimitato, isotropo, omogeneo,
    perfetto e neutro sono legati dalla seguente relazione
    \[
        \vb{E}=\vb{B}\cp\vb{v}\quad\quad \frac{E}{B}=v
    \]
\end{thm}
Il teorema si dimostra banalmente grazie ai tre lemmi di seguito esposti, i quali derivano dal fatto che
per un dielettrico illimitato, omogeneo, isotropo, perfetto e neutro
le equazioni di Maxwell in forma scalare sono

\begin{minipage}[t]{0.5\textwidth}
    \[
        \begin{split}
            &\,\,\begin{cases}
                &\pdv{E_x}{x}=0 \\
            \end{cases} \\
            & \begin{cases}
                & \pdv{B_x}{t}=0 \\
                & \pdv{E_z}{x}=\pdv{B_y}{t} \\
                & \pdv{E_y}{x}=-\pdv{B_z}{t} \\
            \end{cases} \\
        \end{split}
    \]
\end{minipage}
\begin{minipage}[t]{0.5\textwidth}
    \[
        \begin{split}
            &\,\,\begin{cases}
                &\pdv{B_x}{x}=0 \\
            \end{cases} \\
            & \begin{cases}
                & \pdv{E_x}{t}=0 \\
                & \pdv{B_z}{x}=- \mu\epsilon \pdv{E_y}{t} \\
                & \pdv{B_y}{x}=\mu\epsilon \pdv{E_z}{t} \\
            \end{cases} \\
        \end{split}
    \]
\end{minipage}


\begin{lemma}
    In un dielettrico isotropo, illimitato, omogeneo e perfetto, le componenti dei campi parallele alla direzione di
    propagazione non contribuiscono alla propagazione del campo.
\end{lemma}
\begin{proof}
    Data l'equazione di D'Alembert, si vuole dimostrare che $E_x$, $B_x$ sono costanti nel tempo e uniformi nello spazio.
    Questo può essere visto facilmente osservando che nell'ipotesi di onda piana (campi indipendenti da $y$ e $z$) e di dielettrico neutro e perfetto
    le quattro equazioni di Maxwell danno, per la componente $x$ dei campi:
    \[
        \begin{split}
            I   &  \Longrightarrow \pdv{E_x}{x}=0 \\
            II  &  \Longrightarrow \pdv{B_x}{x}=0 \\
            III &  \Longrightarrow \pdv{B_x}{t}=\pdv{E_z}{y}-\pdv{E_y}{z}=0 \\
            IV  &  \Longrightarrow \epsilon\mu \pdv{E_x}{t}=\pdv{B_z}{y}-\pdv{B_y}{z}=0
        \end{split}
    \]
    Dato che non contribuiscono, le componenti dei campi in direzione $x$ possono essere considerate nulle.
\end{proof}
    Le onde elettromagnetiche sono fenomeni puramente trasversali.

\begin{lemma}
    In un'onda elettromagnetica piana campo elettrico e campo magnetico sono fra loro ortogonali.
\end{lemma}

\begin{proof}
    Dalle equazioni di Maxwell
    \[
        \begin{split}
            III & \Longrightarrow \pdv{E_z}{x}=\pdv{B_y}{t} \\
            III & \Longrightarrow \pdv{E_y}{x}=-\pdv{B_z}{t} \\
             IV & \Longrightarrow \pdv{B_z}{x}=- \mu\epsilon \pdv{E_y}{t} \\
             IV & \Longrightarrow \pdv{B_y}{x}=\mu\epsilon \pdv{E_z}{t} \\
        \end{split}
    \]
    Si osservi come se il campo elettromagnetico ha una componente $E_z$ deve avere anche una componente $B_y$
    e viceversa. Per il principio di sovrapposizione si può ottenere una soluzione generale
    sommando due soluzioni linearmente indipendenti, una con $\vb{E}$ diretto solo lungo $y$ e una con
    $\vb{E}$ diretto solo lungo $z$, opportunamente pesate.
    Si può allora portare avanti la discussione considerando il campo elettrico diretto solo in direzione
    una direzione, ad esempio $y$, ed il campo magnetico diretto di conseguenza solo in direzione $z$.
    \[
        E_z=0 \Longrightarrow \pdv{B_y}{x}=0,\pdv{B_y}{t}=0
    \]
    Ovvero, all'onda elettromagnetica non dà alcun contributo la componente del campo magnetico diretta lungo la direzione $y$
    e può quindi essere considerata nulla. Ma allora dato che per quanto dimostrato precedentemente era nulla la componente $x$
    dei campi, si ha che $\vb{E}=E_y$ e quindi $\vb{B}=B_z$. Siccome non si ha perdita di generalità nell'aver scelto una direzione
    fissa per i campi, si ha l'ortogonalità.
\end{proof}
Un'onda in cui i campi sono orientati in direzione fissa si dice avere polarizzazione lineare.

\begin{lemma}
    In un'onda elettromagntica piana si ha
    \[
        \frac{E_y}{B_z}=\frac{E}{B}=\pm v
    \]
\end{lemma}
\begin{proof}
    Per la dimostrazione di questo teorema sfrutta l'equazione di Maxwell
    \[
        III    \Longrightarrow \pdv{B_z}{x}= -\epsilon\mu \pdv{E_y}{t}
    \]
    Per il lemma precendente il campo elettrico è diretto solo in direzione $y$ mentre il campo magnetico è diretto solo in direzione $z$.
    L'equazione delle onde diventa perciò, per i due campi
    \[
        \begin{split}
            &\vb{E}(x\mp vt)=E_y(x\mp vt)=E_y(\xi)\\
            &\vb{B}(x\mp vt)=B_z(x\mp vt)=B_z(\xi)
        \end{split}
    \]
    Si osservi quindi come
    \[
        \pdv{E_y}{x}=\dv{E_y}{\xi}\quad \pdv{B_z}{t}=\dv{B_z}{\xi}\mp v
    \]
    L'equazione di Maxwell può essere scritta come
    \[
        \dv{E_y}{\xi}=\pm v \dv{B_z}{\xi}
    \]
    Che integrata, ponendo ragionevolmente a 0 la costante arbitraria dà $E_y=\pm v B_z$.
\end{proof}
Sostituendo $B=\mu H$ si ha che $E/H=\sqrt{\mu/\epsilon}=Z$. $Z$ ha le dimensioni di una resistenza e viene chiamata
\textit{impedenza caratteristica} del materiale. $Z_0=\sqrt{\mu_0/\epsilon_0}=377\Omega$.
Si osservi come la relazione che esprime il rapporto fra moduli del campo elettrico e del campo magnetico non costituisca una relazione
di confronto fisicamente significativa in quanto lega fra loro grandezze di dimensioni diverse.
Il seguente risultato, corollario del teorema appena dimostrato, fornisce un risultato relativo a questo confronto.
\begin{cor}
    In un'onda elettromagnetica piana, per ogni tempo e per ogni punto, le densità di energia associate a campo elettrico e
    campo magnetico sono uguali.
\end{cor}
\begin{proof}
    La dimostrazione è quasi immediata, grazie al teorema si ha infatti
    \[
        u_B=\rec{2}\frac{B^2}{\mu}=\rec{2}\frac{E^2}{v^2\mu}=\rec{2}\epsilon E^2=u_E
    \]
\end{proof}
Il teorema porta come conseguenza il fatto che campo elettrico e magnetico siano in fase.
Alla luce di questo risultato il vettore di Poynting può essere scritto come
\[
    \vb{I}=\rec{\mu} (\vb{B}\cp\vb{v})\cp\vb{B}=\frac{B^2}{\mu}\vb{v}=2u_M\vb{v}=(u_M+u_M)\vb{v}=(u_E+u_M)\vb{v}=u\vb{v}
\]
avendo indicato $u_E+u_M=u$: siccome $u$ è un'energia su un volume, questa è l'energia contenuta in un cilidro di
sezione unitaria perpendicolare alla direzione di propagazione dell'onda e con altezza pari alla velocità dell'onda,
il che conferma l'interpretazione fisica del vettore di Poynting fornita nel capitolo precedente:
il flusso del vettore di Poynting che come detto è l'energia del campo elettromagnetico che sfugge attraverso
una superficie $S$, è  di fatto l'energia che un'onda elettromagnetica trasporta nell'unità di tempo attraverso $S$.

%Un'altra forma significativa del vettore di Poynting è
%\[
%    \vb{I}=\epsilon E^2 \vb{v}=\epsilon E^2 \frac{\vu{v}}{\sqrt{\mu\epsilon}}=\frac{E^2}{Z}\vu{v}
%\]
%Da questa si ricava immediatamente il modulo
%\[
%    I(\vb{r},t)=E^2/Z=H^2Z
%\]
Si definisce \textit{intensità istantanea dell'onda} l'energia per unità di tempo e di superficie che fluisce
attraverso una superficie ortogonale alla direzione di propagazione della perturbazione.
\[
    \pdv{U}{t}=\pdv{u}{t}\dd{\tau}=\pdv{u}{t}\dd{l}\dd{S}=uv\dd{S}=I\dd{S}
\]
Ovvero, il modulo del vettore di Poynting è l'intensità istantanea.
Tipicamente, per la luce visibile si ha una frequenza di $10^{15}$Hz, cui corrisponde una lunghezza
d'onda di $10^3$[angstrom]: la maggioranza degli strumenti di misura (non ultimi i nostri occhi)
non hanno una risposta per tempi così piccoli. Di conseguenza è spesso più utile
parlare di intensità media $[I]\int_0^TI\dd{t}$.




Può essere utile descivere l'onda in un sistema di riferimento $sr$ in cui l'onda si propaga
in una direzione $\vu{n}$ qualunque, parallela a nessuno dei
vettori coordinati. Sia quindi $sr'$ un sistema di riferimento la cui origine coincide con l'origine di
$sr$ i cui assi sono però inclinati rispetto a quelli di $sr$ affinchè $\vu{n}$
sia parallelo ad $x'$ ($\vu{n}\equiv\vu{i}'$). La trasformazione fra i due sistemi di riferimento è data da una matrice di rotazione
\[
    \vb{r}'=\left(\begin{matrix}
                 \vu{i}'\vdot\vu{i} & \vu{i}'\vdot\vu{j} & \vu{i}'\vdot\vu{k} \\
                 \vu{j}'\vdot\vu{i} & \vu{j}'\vdot\vu{j} & \vu{j}'\vdot\vu{k} \\
                 \vu{k}'\vdot\vu{i} & \vu{k}'\vdot\vu{j} & \vu{k}'\vdot\vu{k} \\
            \end{matrix}\right)\vb{v}=
            R\vb{r}
\]
In $sr'$ l'onda piana ha forma $f(x'-vt)$. Esprimendo $x'$ in $sr$ si ha
\[f\bigl((\vu{i}'\vdot\vu{i})x + (\vu{i}'\vdot\vu{j})y + (\vu{i}'\vdot\vu{k})z \mp vt\bigr)=f(\vu{n}\vdot\vb{r}\mp vt)\]



Per quanto visto in appendice un'onda elettromagentica monocromatica che propaga in una direzione generica
è scritta nella forma
\[
    \begin{cases}
        & \vb{E}(\vb{r},t)=\vb{E}_0\cos(\vb{k}\vdot\vb{r}-\omega t)=\vb{E}_0e^{j(\vb{k}\vdot\vb{r}-\omega t)}\\
        & \vb{B}(\vb{r},t)=\vb{B}_0\cos(\vb{k}\vdot\vb{r}-\omega t)=\vb{B}_0e^{j(\vb{k}\vdot\vb{r}-\omega t)}
    \end{cases}
\]
 con $\omega=2\pi\nu$ e $\vb{k}=\vu{n}\omega/v$.


\section{Onde sferiche}
Questo paragrafo è dedicato a descrivere la forma di un'onda elettromagnetica con simmetria
sferica. Verrà usata $F(\vb{r},t)$ per indicare in modo generico il campo elettrico o il campo
magnetico.

Se l'onda è sferica $F(\vb{r},t)=F(r,t)$ e quindi le \ref{eqn:diff_onde} in coordinate polari
si riducono a
\[
    \rec{r} \pdv[2]{r}(rF)-\epsilon\mu\pdv[2]{F}{t}=0
\]
Moltiplicando e dividendo per $r$ il secondo membro, siccome $r$ non dipende esplicitamente dal tempo,
\[
    \rec{r} \pdv[2]{r}(rF)-\epsilon\mu\rec{r}\pdv[2]{rF}{t}=0
\]
Chiamando $u=rF$ si ottiene
\[
    \pdv[2]{r}(u)-\epsilon\mu\pdv[2]{u}{t}=0
\]
ovvero, $u$ soddisfa l'equazione di D'alembert. In conclusione
\begin{equation}
    F(r,t)=\rec{r}[f_1(x-vt)+f_2(x+vt)]
\end{equation}
Le onde sferiche hanno quindi un'ampiezza che si attenua come $1/r$.
La singolarità in $r=0$ non costituisce un problema in quanto in $r=0$ si trova
la sorgente, che non è descritta dalle equazioni di Maxwell per le onde.
Si osservi come $f_1$ ed $f_2$ siano dimensionalmente dei potenziali.
Nel limite di considerare
una porzione di spazio molto piccola e lontana dalla sorgente, le onde sferiche possono essere
approssimate con onde piane.

Per un'onda sferica monocromatica l'intensità istantanea vale
\[
    I(\vb{r},t)=\frac{E_0^2}{Zr^2}\cos[2](kr-\omega t)
\]
per cui l'intensità è
\[
    \bar{I}=\frac{E_0^2}{2Z}\rec{r^2}
\]
L'intesità decresce come $1/r^2$ è questo è concorde col principio di conservazione dell'energia:
siccome si sta considerando il caso di assenza di dissipazione, il flusso di energia attraverso una
qualunque superficie sferica  centrata nella sorgente (che cresce proporzionalmente ad $r^2$)
deve essere lo stesso (ovvero deve essere indipendente
da $r$) e questo può avvenire se e solo se l'intensità si attenua come $r^{-2}$.


\section{Sorgenti delle onde}
In questo paragrafo si vogliono studiare tre diverse sorgenti di onde elettromagnetiche.

%\subsection{Carica puntiforme in moto rettilineo uniforme}
%Si vuole calcolare il campo elettromagnetico generato da una carica in moto rettilineo uniforme. Per farlo
%è utile passare attraverso i potenziali ritardati. Ci si pone come primo obiettivo quindi quello di
%calcolare i potenziali ritardati per una particella carica puntiforme in qualsiasi stato di moto.
%\begin{thm}[Potenziali di Lienard-Wieckert]
%    I potenziali prodotti da una carica puntiforme $q$ in moto con velocità istantanea $\vb{v}$
%    in posizione istantanea $\vb{r}_s$ sono
%    \begin{equation}
%        \begin{split}
%            &V(\vb{r},t)=\rec{4\pi\epsilon_0}\frac{q}{\Delta r}\rec{1-\frac{v_r(t')}{c}}\\
%            &\vb{A}(\vb{r},t)=\frac{\mu_0}{4\pi}\frac{q\vb{v}(t')}{\Delta r}\rec{1-\frac{v_r(t')}{c}}=
%                V(\vb{r},t)\frac{\vb{v}(t')}{c^2}\\
%        \end{split}
%    \end{equation}
%    con $v_r$ la proiezione di $\vb{v}$ sulla direzione $\vb{r}-\vb{r}_s$,
%    $\Delta r=\abs{\vb{r}-\vb{r}_s}$ e $t'=t-\Delta r/c$.
%\end{thm}
%\begin{proof}
%Per la carica puntiforme le densità che compaiono nei potenziali ritardati assumono la forma
%    \[
%        \begin{split}
%            &\rho(\vb{r}',t')=q\delta^3 (\vb{r}'-\vb{r}_s(t'))\\
%            &\vb{J}(\vb{r}',t')=q\vb{v}_s(t')\delta^3 (\vb{r}'-\vb{r}_s(t'))\\
%        \end{split}
%    \]
%    dove $\delta^3$ è la delta di Dirac tridimensionale e $\vb{r}_s$, $\vb{v}_s$ sono posizione e velocità della particella.
%    I potenziali ritardati assumono quindi la forma
%    \[
%        \begin{split}
%            &V(\vb{r},t)=\frac{q}{4\pi\epsilon_0}\int_{\tau}\frac{\delta^3(\vb{r'}-\vb{r}_s(t'))}{\abs{\vb{r}-\vb{r}'}} \dd{\tau'}\\
%            &\vb{A}(\vb{r},t)=\frac{q}{4\pi\epsilon_0}\int_{\tau}\frac{\vb{v}_s(t')\delta^3(\vb{r'}-\vb{r}_s(t'))}{\abs{\vb{r}-\vb{r}'}} \dd{\tau'}\\
%        \end{split}
%    \]
%    Per poter calcolare gli integrali si usa il seguente artificio: si sostituisce $t'$ con $t^*$ e si integra tutto
%    sulla distribuzione $\delta(t^*-t')$
%    \[
%        \begin{split}
%            &V(\vb{r},t)=\frac{q}{4\pi\epsilon_0}\iint\frac{\delta^3(\vb{r'}-\vb{r}_s(t^*))}{\abs{\vb{r}-\vb{r}'}} \delta(t^*-t')\dd{t*}\dd{\tau}\\
%            &\vb{A}(\vb{r},t)=\frac{q}{4\pi\epsilon_0}\iint\frac{\vb{v}_s(t^*)\delta^3(\vb{r'}-\vb{r}_s(t^*))}{\abs{\vb{r}-\vb{r}'}} \delta(t^*-t')\dd{t*}\dd{\tau}\\
%        \end{split}
%    \]
%    Scambiando l'ordine di integrazione, l'integrale sul volume della delta di Dirac tridimensionale seleziona gli $\vb{r}'=\vb{r}_s(t*)$
%    \[
%        \begin{split}
%            &V(\vb{r},t)=\frac{q}{4\pi\epsilon_0}\int\rec{\abs{\vb{r}-\vb{r}_s(t*)}} \delta(t^*-t')\dd{t'}\\
%            &\vb{A}(\vb{r},t)=\frac{q}{4\pi\epsilon_0}\int\frac{\vb{v}_s(t^*)}{\abs{\vb{r}-\vb{r}_s(t^*)}} \delta(t^*-t')\dd{t'}\\
%        \end{split}
%    \]
%    Questo integrale non può essere ancora eseguito, perchè $t'$ è una funzione della distanza fra $\vb{r}$ e $\vb{r}_s(t^*)$
%    e quindi dipende da $t^*$.
%    È noto che
%    \[
%        \delta (f(t^*))=\sum_i\frac{\delta(t^*-t_i)}{\abs{f'(t_i)}}
%    \]
%    dove i $t_i$ sono zeri della funzione $f$. Fissato il sistema di coordinate e la traiettoria della sorgente
%    c'è un solo $t'$ possibile e quindi l'espressione per la delta si riduce a
%    \[
%        \begin{split}
%            &\delta(t^*-t')=\delta(t*-t')(\pdv{t*}(t*-t'))^{-1}=\delta(t*-t')[\pdv{t*}(t*-(t-\rec{c}\abs{\vb{r}-\vb{s}_s(t*)}))]^{-1}=\\
%            &=\delta(t*-t')(1+\rec{c}\frac{\vb{r}-\vb{r}_s(t')}{\abs{\vb{r}-\vb{r}_s(t')}}\vdot (-\vb{v}(t')))^{-1}\\
%            &=\delta(t*-t')(1-frac{\vb{v}}{c}\frac{\vb{r}-\vb{r}_s}{\abs{\vb{r}-\vb{r}_s}})^{-1}\\
%        \end{split}
%    \]
%
%
%    Sostituendo nell'integrale, si ha la tesi.
%\end{proof}
%
%Si immagini ora una carica puntiforme $q$ che si muove con velocità $\vb{v}=v\vu{x}$ costante
%($\vb{r'}=x'\vu{x}$).
%Si vogliono calcolare i potenziali generati dalla carica nel punto $\vb{r}$ al tempo $t$.
%Ponendo a $t=0$ il momento in cui la carica si trova in $x'=0$, al tempo $t$ la carica si trova in
%$x'=v=vt$. Per calcolare i potenziali è però necessario conoscere la distanza fra il punto in cui si vuole calcolare
%il potenziale e la posizione  in cui si trovava la carica al tempo $t'=t-\Delta r/c$. Chiaramente $x'=vt'$,
%da cui segue che $\Delta r=\sqrt{(x-vt')^2+y^2+z^2}$. Invertendo l'equazione per $t'$ si trova
%$\Delta r=c(t-t')$. Sostituendo ed elevando al quadrato si ottiene $c^2(t-t')^2=(x-v't)^2+x^2+y^2$.
%Svolgendo i quadrati e risolvendo per $t'$
%\[
%    \Biggl(1-\frac{v^2}{c^2}\Biggr)t'=t-\frac{vx}{c^2}-\rec{c}\sqrt{(x-vt)^2+\Biggl(1-\frac{v^2}{c^2}\Biggr)(y^2+z^2)}
%\]
%Che può essere scritta in forma più compatta chiamando $\gamma^{-2}=1-v^2/c^2$
%\[
%    \frac{t'}{\gamma^2}=t-\frac{vx}{c^2}-\rec{c}\sqrt{(x-vt)^2+\frac{y^2+z^2}{\gamma^2}}
%\]
%Si trova agevolmente $\Delta r$ sostituendo il valore ottenuto per $t'$ nell'espressione $\Delta r=c(t-t')$.
%Inserendo questo risultato nei potenziali di Lienard-Wieckert e raccogliendo un $\gamma^{-2}$ al denominatore
%\[
%    \begin{split}
%        &V(\vb{r},t)=\rec{4\pi\epsilon_0}\frac{q\gamma}{\sqrt{\gamma^2(x-vt)^2+y^2+z^2}}\\
%        &\vb{A}(\vb{r},t)=\frac{v}{c^2}V(\vb{r},t)\vu{x} \\
%    \end{split}
%\]
%Ora è possibile ricavare i campi (RIVEDO CONTI)
%\[
%    \begin{split}
%        &\vb{E}=-\grad{V}-\pdv{\vb{A}}{t}= \frac{q}{4\pi\epsilon_0}\frac{x\vu{x}+\gamma y\vu{y}+\gamma z \vu{z}}{(\Delta r)^3}\\
%        &\vb{B}=\curl{\vb{A}}=\pdv{A}{z}\vu{y} - \pdv{A}{y}\vu{z}=\frac{q\gamma v}{4\pi c^2}\rec{(\Delta r)^3}(z\vu{y}-y\vu{z})\\
%    \end{split}
%\]
%
%Per comprendere l'andamento qualitativo del campo elettromagnetico ci si limiti a considerare il piano $xy$. Qui
%$E_z=0$ e si ritrova facilmente la relazione $\vb{b}=(\vb{v}\cp\vb{E})/c^2$. Se si pone l'osservatore sull'asse $x$
%anche la componente $y$ del campo sarà nulla e l'espressione per il campo elettrico si riduce a
%\[
%    \vb{E}=\rec{\gamma}\frac{q}{4\pi\epsilon_0}\frac{x}{\abs{x-vt}}\vu{x}
%\]
%Siccome $\gamma$ cresce al crescere della velocità, più la particella va veloce più il campo elettrico viene soppresso. Viceversa
%ponendo l'osservatore perpendicolarmente al moto della particella, alla stessa altezza della particella, il campo si riduce a
%\[
%    \vb{E}= \gamma\frac{q}{4\pi\epsilon_0}\frac{x}{\abs{x-vt}}\vu{x}
%\]

\subsection{Dipolo oscillante}
Ci si ponga nel vuoto.
Un dipolo oscillante è costituito da due cariche uguali e opposte poste a distanza variabile nel tempo.
Equivalentemente però, si può consdierare la distanza fissa e le cariche variabili. Il dipolo può essere
quindi schematizzato con un segmento rettilineo di conduttore
percorso da corrente alternata. Fisicamente questa situaione può essere realizzata prendendo un conduttore rettilineo
con alle estremità due sfere costituenti le armature di un condensatore. Il circuito equivalente a tale dispositivo
è costituito da un generatore di corrente alternata, una resistenza ed un condensatore tutti disposti in serie.
Nell'ipotesi in cui
$\lambda=c\,2\pi/\omega$ (ovvero la lunghezza d'onda della radiazione emessa) sia molto maggiore della
lunghezza del conduttore, la corrente è indipendente dalla posizione sul conduttore. Si consideri inizialmente
per semplicità che la corrente vari in modo armonico $I(\vb{r},t)=I(t)=I_0\cos{\omega t}$. La carica presente sulle armature del
condensatore vale
\[
    q(t)=\int I\dd{t}=\frac{I_0}{\omega}\sin{\omega t}
\]
Sia $d$ la distanza fra le maglie del condensatore (ovvero la lunghezza del conduttore rettilineo),
orientato ortogonamente all'asse $z$ con il centro nell'origine del sistema di coordinate.
Le due sfere dotate di carica $+q$ e $-q$ costituiscono un dipolo con momento
\[
    \vb{p}=qd\vu{k}=\frac{I_0\,d}{\omega}\sin{\omega t}\vu{k}
\]
Ci si ponga nella gauge di Lorentz. Il potenziale vettore vale,
per le equazioni dei potenziali ritardati \eqref{eqn:potenziali_ritardati}
\[
    \vb{A}(\vb{r},t)=\frac{\mu_0}{4\pi}\int_\tau
    \frac{\vb{J}(\vb{r}',t-\abs{\vb{r}-\vb{r}'}/c)}{\abs{\vb{r}-\vb{r}'}}\dd{\tau'}
\]
Se $S$ è la sezione del conduttore, $\vb{J}(\vb{r},t)\vdot\vb{S}=I(t)\vu{k}$ dove per le ipotesi fatte
$I$ non dipende da $t$.
Inoltre siccome le coordinate con l'apice si riferiscono all'interno del conduttore $\dd{\tau'}=S\dd{z'}$.
Nell'ipotesi di essere molto lontani dal conduttore $\abs{\vb{r}-\vb{r}'}\simeq r$ si ha
\[
    \vb{A}(\vb{r},t)=\frac{\mu_0}{4\pi}\int\frac{I(t-r/c)}{r}\dd{z'}\vu{k}=\frac{\mu_0 I_0 d}{4\pi}
    \frac{\cos[\omega(t-r/c)]}{r}\vu{k}=
    \frac{\mu_0}{4\pi}\frac{\dot{\vb{p}}(t-r/c)}{r}
\]
Dalla condizione di Lorentz si può ricavare il potenziale scalare. Tenendo conto che $\vb{A}=A\vu{k}$ si ha
\[
    \pdv{V}{t}=-\rec{\epsilon_0\mu_0}\pdv{\vb{A}}{z}=\rec{4\pi\epsilon_0}\Biggl(\frac{\ddot{p}(t-r/c)}{cr}+
    \frac{\dot{p}(t-r/c)}{r^2} \Biggr)\frac{z}{r}
\]
Da cui, facendo un integrale indefinito rispetto al tempo e ponendo a zero la costante arbitraria
\[
    V(\vb{r},t)=\rec{4\pi\epsilon_0}\Biggl(\frac{\dot{p}(t-r/c)}{cr}+ \frac{p(t-r/c)}{r^2} \Biggr)\frac{z}{r}
\]

Si possono ora ricavare i campi a partire dalle \eqref{eqn:def_potenziale_scalare} e dalla
\eqref{eqn:def_potenziale_vettore}. Il calcolo è semplice se effettuato in coordinate sferiche, basta tenere
conto che $z/r=\cos{\theta}$ e che il versore $\vu{k}$ si scrive $\vu{k}=(\cos{\theta},-\sin{\theta},0)$.
Si ottiene infine
\[
    \begin{split}
        & B_r=0 \\
        & B_\theta=0 \\
        & B_\phi=\frac{\mu_0}{4\pi}\frac{\sin{\theta}}{r} \Biggl(\frac{\ddot{p}(t-r/c)}{c}+\frac{\dot{p}(t-r/c)}{r}\Biggr) \\
        & E_r=\rec{4\pi\epsilon_0}\frac{2\cos{\theta}}{r} \Biggl(\frac{\ddot{p}(t-r/c)}{c}+\frac{\dot{p}(t-r/c)}{r}\Biggr) \\
        & E_\theta=\rec{4\pi\epsilon_0}\frac{\sin{\theta}}{r} \Biggl(\frac{\ddot{p}(t-r/c)}{c^2}+\frac{\dot{p}(t-r/c)}{cr})+
        \frac{p(t-r/c)}{r^2}\Biggr) \\
        & E_\phi=0
    \end{split}
\]
Chiaramente $\vb{B}\vdot\vb{E}=0$, a conferma che i campi sono ortogonali. Le linee di forza del campo magnetico
sono circonferenze centrate sull'asse $z$, mentre il campo elettrico si trova nel piano $zr$. Tutti i termini che
contribuiscono ai campo hanno una dipendenza spazio-temporale del tipo $t-r/c$ e sono divisi per potenze di $r$:
i fronti d'onda sono dunque sfere che si propagano con velocità $c$ sulle quali l'ampiezza dipende da $\theta$
e si attenua all'aumentare di $r$.

Il vettore di Poynting vale, lasciando sottointesa la dipendenza di $p$ da $t-r/c$
\[
    \begin{split}
        \vb{I}=&\frac{\vu{r}}{16\pi^2\epsilon_0}\Biggl[ \Biggl(\frac{\ddot{p}}{c^6r}\Biggr)^2 \sin^2{\theta}-
        \Biggl(2\frac{\ddot{p}\dot{p}}{c^2r}+\frac{\dot{p}p}{r^3}+\frac{\ddot{p}p+\dot{p}^2}{cr^2}\Biggr)\frac{\sin^2{\theta}}{r^2}\Biggr]+ \\
        +&\frac{\vu{\theta}}{16\pi^2\epsilon_0}\Biggl[ \Biggl(\frac{\ddot{p}\dot{p}}{c^2r}+\frac{\dot{p}p}{r^3}+
        \frac{\ddot{p}p+\dot{p}^2}{cr^2}\Biggr)\frac{2\sin{\theta}\cos{\theta}}{r^2} \Biggr]
    \end{split}
\]
Si considerino ora le espressioni per $\vb{p}$ e le sue derivate. Chiamando $p_0=I_0d/\omega$, si ha:
$p=p_0\sin(\omega(t-r/c))$, $\dot{p}=\omega p_0\cos(\omega(t-r/c))$, $p=-\omega^2p_0\sin(\omega(t-r/c))$.
Chiaramente i termini $\ddot{p}\dot{p}$ e $\dot{p}p$ hanno media temporale nulla e inoltre, mediando su un
periodo $\ddot{p}p=-\dot{p}^2$, per cui $\ddot{p}p+\dot{p^2}=0$.
Mediando temporalmente il vettore di Poynting quindi si ottiene
\begin{equation}
    \vb{I}=\frac{\sin^2{\theta}}{16\pi^2\epsilon_0 c^3}\frac{\ddot{p}}{r^2}\vb{r}\Rightarrow
    [\vb{I}]\frac{\omega^4 p_0^2 \sin^2{\theta}}{32\pi^2\epsilon_0 c^3}\rec{r^2}\vb{r}
    \label{eqn:valormedio_poynting}
\end{equation}
ovvero restano solo i termini che decrescono come $r^{-2}$. Ne segue che il flusso attraverso una sfera centrata
nella sorgente non dipende da $r$. Questo flusso fornisce la potenza media irraggiata dal dipolo.
\[
    [P]=\int_S\bar{\vb{I}}\vdot\dd{\vb{S}}= \int_S \bar{\vb{I}}r^2\sin{\theta}\dd{\theta}\dd{\phi}=\frac{\omega^4 p_0^2}{2}\rec{6\pi\epsilon_0 c^3}=
\]
ovvero
\begin{equation}
    [P]=\frac{\omega^4 p_0^2}{2}\frac{\mu_0}{6\pi c}=\frac{\mu_0}{6\pi c}[p]
    \label{eqn:potenza_dipolo}
\end{equation}
Per il principio di sovrapposizione, la formula trovata per la potenza è applicabile ad un dipolo oscillante con legge
qualunque sviluppando la legge oraria in serie di Fourier o ad un insieme di dipoli oscillanti.
Per quanto riguarda i termini inversamente proporzionali a potenze di $r$ maggiori di due, questi sono nulli quando si
considera il flusso medio ma sono anche trascurabili quando si è lontani dal dipolo ($r>>\lambda$). Nel caso in cui
$r<<\lambda$ questi costituiscono il \textit{campo vicino}, un campo variabile nel tempo ma localizzato intorno alla
sorgente, al quale non è associato alcun trasporto di energia.

\subsection{Carica puntiforme in moto accelerato}
Si consideri una carica $q$ dotata di accelerazione $a$. È possibile usare gli stessi risultati ottenuti nel caso del
dipolo oscillante a patto di sostituire $\ddot{p}^2$ con $(qa)^2$ nella \eqref{eqn:valormedio_poynting}.
Con passaggi analoghi a quelli visti nel caso del dipolo, per velocità non relativistiche si ottiene la \textit{formula di Larmor}
\begin{equation}
    \bar{P}=\frac{\mu_0}{6\pi c}(qa)^2
\end{equation}


\section{Onde elettromagnetiche nei dielettrici}
Nei dielettrici $v=(\mu\epsilon)^{-\rec{2}}=(\mu_0\epsilon_0)^{-\rec{2}}(\mu_r\epsilon_r)^{-\rec{2}}=c\,(\mu_r\epsilon_r)^{-\rec{2}}$.
\begin{defn}[Indice di rifrazione]
    Si definisce indice di rifrazione $n$
    \[
        n=\frac{c}{v}=\sqrt{\epsilon_r\mu_r}\simeq \sqrt{\epsilon_r}
    \]
\end{defn}
I discorsi fatti sulle onde piane si appoggiavano all'ipotesi che la velocità dell'onda fosse indipendente dalla
frequenza. Se per $\mu$ l'approssimazione può essere buona, $\epsilon$ dipende dalla frequenza in maniera spesso marcata.
Quando si trattano onde non monocromatiche, condizione necessaria affinchè questa approssimazione sia buona è che
lo spettro delle frequenze nello sviluppo di Fourier dell'onda occupi un'intervallo ristretto. In questo
paragrafo verrà mostrato come questa condizione non sia suffuciente a causa del fenomeno della \textit{dispersione anomala}:
attorno a particolari frequenze, a piccole variazioni della frequenza corrispondono brusche variazioni dell'indice di rifrazione.


Si vuole quindi discutere la dipendenza di $\epsilon$ da $\nu$. Il punto di partenza per il calolcolo di $\epsilon_r$
in elettrostatica sono state le considerazioni sulla polarizzabilità $\alpha$:
quando sono sottoposti ad un campo elettrico, il nucleo e
il baricentro della nube elettronica degli atomi che costituiscono il dielettrico si allontanano e iniziano
a risentire di una forza attrattiva che può essere schematizzata con una forza elastica\footnote{Per piccole lunghezze d'onda
la polarizzazione per orientamento delle molecole non risente delle oscillazioni del campo. Alla fine del paragrafo
verrà discussa brevemente l'eventualità che anche le mmolecole possano oscillare.}.
Usando gli stessi simboli introdotti nel capitolo sui dielettrici all'equilibrio $-k\vb{r}+Ze\vb{E}_l=0$,
e quindi il momento di dipolo $\vb{p}=(Ze)\vb{r}=(Ze)^2/k\,\vb{E}_l$ da cui segue che la polarizzabilità $\alpha$,
ovvero il termine di proporzionalità fra $\vb{p}$ ed $\vb{E}_l$ è costante. Si noti come questo risultato
dipenda direttamente dal fatto che la deformazione sulla nube elettronica sia di tipo statico.
Se però il campo elettrico è oscillante ($\vb{E}(t)=\vb{E}_{0l}e^{j\omega t}$), $\vb{r}$
soddisfa l'equazione dell'oscillatore armonico forzato
\[
    m\ddot{\vb{r}} + m\gamma\dot{\vb{r}} + k\vb{r}=Ze\vb{E}_{0l}e^{j\omega t}
\]
dove  $m=Zm_e$ è la massa della nube elettronica e il termine $b\dot{\vb{r}}=m\gamma\dot{\vb{r}}$ tiene conto
dell'interazione della nube elettronica con gli atomi circostanti e dell'energia dissipata per irraggiamento
dalla carica oscillante: un campo elettrico oscillante comporta la presenza nel dielettrico di
dipoli oscillanti che, come si è visto, emettono energia.
\begin{obs}
    \[
        \vb{r}(t)=\frac{Ze\vb{E}_l(t)}{m(\omega_0^2 - \omega^2 + j\omega\gamma)} \quad\quad \omega_0^2=\frac{k}{m}
    \]
    \label{obs:r_dielettrici}
\end{obs}
\begin{proof}
    La soluzione all'equazione dell'oscillatore armonico è $\vb{r}(t)=\vb{R}_0e^{j\omega t}$.
    Sostituendo nell'equazione differenziale si trova
    \[
        \vb{R}_0=\frac{Ze\vb{E}_{0l}}{m(\omega_0^2 - \omega^2 + j\omega\gamma)}
    \]
    Ma $\vb{E}_{0l}e^{j\omega t}=\vb{E}_l$, da cui la tesi.
\end{proof}
\begin{cor}
    Nell'ipotesi che a livello macroscopico il materiale sia schematizzabile come un insieme di
    oscillatori armonici tutti fra loro identici (stesso $\omega_0$ e stesso $\gamma$)
    \[
        \alpha=\frac{(Ze)^2}{m}\rec{\omega_0^2 - \omega^2 + j\omega\gamma}
    \]
    o analogamente, $\alpha=\abs{\alpha}e^{j\delta}$ con
    \[
        \abs{\alpha}=\frac{(Ze)^2}{m\sqrt{(\omega_0^2-\omega^2)^2+\gamma^2\omega^2}} \quad\quad
        \tan{\delta}=\frac{\gamma\omega}{\omega_0^2-\omega^2}
    \]
\end{cor}
\begin{proof}
$\alpha=p/E_l=(Ze)^2r/E_l$. Per l'osservazione precedente, si ha la tesi.
\end{proof}
L'ipotesi è molto forte: si sta richiedendo che il dielettrico sia costituito da atomi fra loro tutti uguali e
indipendenti e che le nuvole elettroniche si comportino come sistemi rigidi.
Si noti come sia il modulo di $\alpha$ che la sua fase dipendano dalla pulsazione $\omega$ del campo elettrico locale.
Il fatto che la polarizzabilità sia complessa implica che $\vb{r}$ e quindi $\vb{p}$ hanno la stessa direzione e la stessa
pulsazione del campo elettrico locale, ma diversa fase.
\begin{obs}
Il momento di dipolo è sfasato di $\delta$ rispetto al campo elettrico locale.
\end{obs}
\begin{proof}
    La dimostrazione si ottiene considerando che la quantità fisicamente significativa è la parte reale di $\vb{p}$.
    Si ha quindi $\Re(\vb{p})=\Re(\alpha \vb{E}_l(t))=\Re(\abs{\alpha}e^{j\delta} E_{0l}e^{j\omega t})=\abs{\alpha}E_{0l}\cos(\omega t+\delta)$
\end{proof}

Si vogliono ora studiare le conseguenze di quanto detto su $\epsilon_r$ e $n$. Indicando con $N$ il numero di atomi per unità di
volume, esprimendo $\epsilon_r$ in funzione della suscettività elettrica e ricordando la relazione di Clausius-Mossotti, si ha
\[
    \epsilon_r=\chi+1=\frac{N\alpha}{\epsilon_0-N\alpha/3}+1=\frac{1+\frac{2N\alpha}{3\epsilon_0}}{1-\frac{N\alpha}{3\epsilon_0}}
\]
Quindi, anche $\epsilon_r$ e di conseguenza $n=\sqrt{\epsilon_r}$ sono numeri complessi.
Si può quindi porre $n=n_1-jn_2$\footnote{Si è scelto di usare il segno meno perchè, come si mostrerà più avanti,
la parte immaginaria di $n$ è negativa e in questo modo $n_2$>0.}.
Il significato della parte reale e della parte immaginaria dell'indice di rifrazione
appaiono evidenti qualora si consideri un'onda elettromagnetica che si propaga nel dielettrico. Per semplicità, si fa
riferimento alle onde piane
\[
    \vb{E}=\vb{E}_0e^{j\omega(t-\frac{x}{v})}=\vb{E}_0e^{j\omega(t-x\frac{n_1-jn_2}{c})}=
    \vb{E}_0e^{-\frac{\omega n_2 x}{c}}e^{j\omega(t-n_1\frac{x}{c})}=
    \vb{E}_0e^{-\beta x}e^{j\omega(t-n_1\frac{x}{c})}
\]
Si ha quindi un'onda che si propaga con velocità $v=c/n_1$ e la cui ampiezza si attenua secondo la legge
esponenziale $\vb{E}_0 e^{-\frac{\omega n_2 x}{c}}$. La quantità $\beta=\omega n_2/c$ è detta
\textit{coefficiente di assorbimento} del materiale; il suo inverso è detto \textit{cammino di attenuazione},
ha le dimensioni di una lunghezza e rappresenta la distanza che l'onda deve percorrere all'interno del
materiale prima che la sua ampiezza risulti ridotta di un fattore $e^{-1}$.

Per comprendere l'andamento di $n_1$ ed $n_2$ in funzione di $\omega$ si sviluppi $n=\sqrt{\epsilon_r}$ al primo
ordine in $\omega$ (nell'ipotesi che $N\abs{\alpha}/\epsilon_0<<1$). Si ottiene $n=1+N\alpha/2\epsilon_0$ che
quindi fornisce, sostituendo l'espressione esplicita di $\alpha$
\[
    \begin{split}
        & n_1=\Re(n)=1+\frac{N(Ze)^2}{2\epsilon_0 m} \frac{\omega_0^2 -\omega^2}{(\omega_0^2 -\omega^2)^2+\gamma^2\omega^2}\\
        & n_2=-\Im(n)=\frac{N(Ze)^2}{2\epsilon_0 m} \frac{\gamma \omega}{(\omega_0^2 -\omega^2)^2+\gamma^2\omega^2}
    \end{split}
\]
Per $\omega=0$ si torna nel caso del campo elettrostatico, dove non c'è fenomeno di attenuazione.
All'avvicinarsi di $\omega$ ad $\omega_0$, detta \textit{frequenza di risonanza}, $n_1$ passa dall'essere
maggiore di $1$ all'essere minore di $1$. Questo fatto, si vedrà, non è in contraddizione con il
principio di velocità limite. La zona in cui $\dd{n_1}/\dd{\omega}$ è positiva è detta
\textit{zona di dispersione normale}, quella in cui è negativa è detta \textit{zona di dispersione anomala}.
$n_2$ ha l'andamento di una lorenziana, con picco centrato sulla zona di dispersione anomala.
Al di fuori della zona immediatamente circostante al picco
l'attenuazione è trascurabile e il dielettrico è trasparente alla radiazione. Nelle zone di trasparenza, $n_1$
è crescente.

Facendo cadere l'ipotesi che il materiale sia costituito da oscillatori armonici identici, la polarizzabilità assume
la forma
\[
    \alpha=\sum_k\frac{q_k^2}{m_k(\omega_{0k}^2-\omega^2+j\gamma_k\omega)}
\]
dove $q_k$ ed $m_k$ sono la carica e la massa efficaci di ciascun oscillatore. In corrispondenza di ogni $\omega_{0k}$ si verifica
un fenomeno di risonanza che va sommato all'andamento non risonante degli altri oscillatori. L'andamento di $n_1$ è caratterizzato
dall'alternarsi di zone di dispersione anomala, in corrispondenza delle risonanze, e zone di dispersione normale.
È comune esprimere $n_1(\lambda)$: siccome $\lambda=2\pi c\omega$, nelle zone di dispersione normale $n_1(\lambda)$
ha un andamento decrescente, mentre ha un andamento crescente in corrispondenza delle zone di dispersione anomala.
In corrispondenza di ogni frequenza di risonanza $n_2$ ha un picco che corrisponde ad un aumento pronunciato del
coefficiente di assorbimento $\beta$. Questo spiega le righe di assorbimento che si osservano in spettroscopia.
Anche un modello più sofisticato prevede che $n_1$ possa essere minore di 1, specie per piccole lunghezze d'onda.
Questo effetto è in effetti osservato sperimentalmente. Per $\lambda$
elevati, nell'ordine dei micrometri, anche le molecole iniziano ad oscillare e comprtano la comparsa di righe di assorbimento che
possono diventare molto larghe, a costituire delle bande di assorbimento.


\section{Onde elettromagnetiche nei conduttori}
Si consideri un'onda elettromagnetica che incide su un conduttore. Gli elettroni liberi, sotto
l'effetto del campo elettromagnetico variabile
iniziano a muoversi con moto oscillatorio forzato, dissipando energia.
Ci si aspetta quindi che un'onda elettromagnetica in un conduttore si attenui e scaldi il conduttore.

Si consideri un conduttore ohmico. Nel caso stazionario si ha $\vb{J}=\sigma \vb{E}$ con $\sigma$ la
conducibilità elettrica. Siccome questa è una legge locale, ci si aspetta valga anche nel caso non
stazionario come effettivamente conferma l'esperienza (la conducibilità può però essere una funzione
della frequenza ed essere complessa).
La presenza di cariche in moto nel conduttore non permette di prendere $\vb{J}=0$
nella quarta equazione di Maxwell: questo è il motivo per cui le equazioni delle onde
nel conduttore assumono una forma diversa rispetto a quelle nel vuoto.
\begin{thm}
    In un conduttore ohmico, omogeneo e isotropo
    \[
        \begin{split}
            & \laplacian{\vb{H}}-\sigma\mu\pdv{\vb{H}}{t}-\epsilon\mu\pdv[2]{\vb{H}}{t}=0\\
            & \laplacian{\vb{E}}-\sigma\mu\pdv{\vb{E}}{t}-\epsilon\mu\pdv[2]{\vb{E}}{t}=0
        \end{split}
    \]
\end{thm}
\begin{proof}
    Tenuto conto della relazione fra $\vb{E}$ e $\vb{J}$ la quarta equazione di Maxwell assume la forma
    \[
        \curl{\vb{H}}=\sigma \vb{E} +\epsilon \pdv{\vb{E}}{t}
    \]
    Applicando l'operatore rotore ad ambo i membri
    \[
        -\laplacian{\vb{H}} +\grad(\div{\vb{H}})=\sigma(\curl{\vb{E}}) + \epsilon \pdv{t}(\curl{\vb{E}})
    \]
    Per l'ipotesi di omogeneità e isotropia $\vb{H}=\vb{B}/\mu$\footnote{Nel caso dei ferromagneti, questo
    è vero solo localmente.} e quindi per la seconda equazione di Maxwell
    $\div{\vb{H}}=\div{\vb{B}}/\mu=0$. Inoltre per la terza equazione di Maxwell
    $\curl{\vb{E}}=-\pdv*{\vb{B}}{t}=-\mu\pdv*{\vb{H}}{t}$. Sostituendo nella quarta equazione di Maxwell
    si ottiene
    \[
        \laplacian{\vb{H}}=\sigma\mu\pdv{\vb{H}}{t}+\epsilon\mu\pdv[2]{\vb{H}}{t}
    \]
    Applicando l'operatore rotore alla terza equazione di Maxwell e confrontandola con la quarta
    si ottiene un risultato analogo per il campo elettrico.
\end{proof}
La soluzione a queste equazioni viene proposta nel caso di un'onda piana che si propaga lungo l'asse $x$.
Le equazioni per il campo elettrico e magnetico assumono la forma
    \[
        \dv[2]{\phi}{x} - \sigma\mu\pdv{\phi}{t} - \epsilon\mu\pdv[2]{\phi}{t} =0
    \]
La derivata prima rappresenta il termine di smorzamento. Si noti come questo termine scompaia
quando $\sigma=0$, ovvero nel caso di un dielettrico perfetto.
\begin{thm}
L'onda piana che si propaga lungo l'asse $x$ in un conduttore omogeneo e isotropo ha forma
    \[
        \phi(x,t)=A e^{\gamma x} e^{j(\beta x + \omega t)}
    \]
    con
    \[
        \begin{split}
            &\beta=\omega\sqrt{\frac{\epsilon\mu}{2}\bigl[1\pm \sqrt{1+(\sigma/\omega\epsilon)^2}\bigr]}\\
            &\gamma= \frac{\omega\sigma\mu}{2\beta}
        \end{split}
    \]
\end{thm}
\begin{proof}
    Si pongano i campi nella forma
    \[
        \phi(x,t)=\phi{x}e^{-j\omega t}
    \]
    L'esponenziale complesso viene introdotto per comodità di calcolo. Siccome l'equazione differenziale
    è a coefficienti reali, il campo fisicamente significativo è costituito solo dalla parte reale
    della soluzione.
    Sostituendo nell'equazione delle onde si trova un'equazione differenziale per $\phi(x)$
    \[
        \dv[2]{\phi}{x}-j\omega\sigma\phi + \omega^2\epsilon\mu\phi=0
    \]
    È lecito prendere $\phi(x)$ nella forma $\phi(x)=A e^{j\alpha x}$ siccome l'equazione è lineare, ma siccome i coefficienti
    in questo caso sono complessi non si può più affermare che la parte fisicamente significativa sia solo
    quella reale. Sostituendo nell'equazione differneziale si trova un'equazione algebrica per $\alpha$:
    \[
        \alpha^2=\epsilon\mu\omega^2 + i\mu\sigma\omega
    \]
    $\alpha^2$ è un numero complesso e di conseguenza anche $\alpha$ è complesso che può essere preso
    nella forma $\alpha=\beta-j\gamma$. Inserendo nell'equazione precedente e uguagliando i vari termini
    si trova la tesi.
 \end{proof}
L'onda progressiva si ha scegliendo $\beta<0$ e di conseguenza
$\gamma<0$: $\abs{\gamma}$ svolge il ruolo di coefficiente di attenuazione.
Spesso $\sigma>>\epsilon\omega$\footnote{Ad esempio nel caso del rame $\sigma\simeq6\vdot10^7 \Omega^{-1}m^{-1}$.
Nel visibile $\omega\simeq10^{15}Hz$ e dunque $\epsilon\omega\simeq10^4<<10^7$.}
e i coefficienti $\beta$ e $\gamma$ assumono la semplice forma
\[
    \begin{split}
        &\beta=\sqrt{\frac{\omega\sigma\mu}{2}}\\
        &\gamma=-\sqrt{\frac{\omega\sigma\mu}{2}}\\
    \end{split}
\]

Il risultato appena trovato fornisce una giustificazione all'\textit{effetto pelle}: quando un conduttore è percorso
da una corrente alternata ad alta frequenza (e quindi $\omega$ grande) la corrente tende ad addensarsi sullo strato
superficiale del conduttore, riducendone la sezione utile ed aumentandone la resistenza.
Per valutare qualitativamente il fenomeno e capire quali entità fisiche entrano in gioco,
si consideri quindi un conduttore collegato ad un generatore di corrente alternata, ovvero un generatore nel quale gli
accumuli di carica sui morsetti varino nel tempo. Di conseguenza il campo elettrico prodotto varia nel tempo,
quindi varia la corrente circolante nel conduttore e anche il campo di induzione magnetica generato dal conduttore.
Si ha il fenomeno dell'autoinduzione, per cui si sviluppa un campo elettrico indotto $\vb{E}_i$ che si oppone alla variazione
temporale del campo elettrico che lo ha generato.
Concretamente, si prenda un conduttore cilindrico percorso da corrente variabile - ad esempio, crescente.
Per simmetria il campo $\vb{B}$ è costituito da cerchi centrati nell'asse del cilindo e cresce nel tempo
come la corrente dando luogo ad un campo indotto $\vb{E}_i$ orientato parallelamente all'asse del conduttore
con verso opposto a quello di $\vb{J}$. Si consideri ora una linea chiusa rettangolare $l$ di lati $h$ e $\dd{r}$,
che giaccia su un piano contenente l'asse del cilindro. Si prenda come verso positivo quello antiorario, in modo
che la normale alla superficie sia concorde in verso con $\vb{B}$ e che quindi il flusso sia positivo e crescente.
Per la legge di Faraday-Neumann
\[
    -\dv{t}\Phi_S(\vb{B}) =\oint_l \vb{E}_i\vdot\dd{\vb{l}}\simeq E_i(r+\dd{r})h - E_i(r)h
\]
Ma per come è stato impostato il problema $\vb{B}$ è crescente nel tempo, perciò la sua derivata è positiva.
Quindi $E_i(r)>E_i(r+\dd{r})$: $E_i$ aumenta all'aumentare della profondità del conduttore
e quindi ostacola maggiormente il passaggio di corrente negli strati profondi che negli strati superficiali.

%Una valutazione quantitativa del fenomeno può essere effettuata in modo semplice nel caso di un conduttore
%ohimico omogeneo infinito, delimitato da una superficie piana e immerso in un campo $\vb{E}(t)=\vb{E}_0e^{j\omega t}$.
%Si ponga il conduttore in un sistema di riferimento cartesiano in modo che occupi lo spazio $z<0$ e che il campo
%elettrico abbia solo la componente $y$ diversa da 0. La quarta equazione di Maxwell si scrive
%\[
%    \curl{\vb{H}}=\sigma\vb{E}_0e^{j\omega t} + \epsilon\pdv{\vb{E}_0e^{j\omega t}}{t}=
%    (\sigma+j\omega\epsilon)\vb{E}
%\]
%Applicando l'operatore divergenza ad entrambi i membri, nell'ipotesi che il conduttore sia omogeneo
%\[
%    0=\div(\curl{\vb{H}})=(\sigma+j\omega\epsilon)\div{\vb{E}}
%\]
%Questo implica che $\div{\vb{E}}=0$ e quindi, per la prima equazione di Maxwell nell'ipotesi di isotropia $\rho=0$.
%La densità di volume di cariche localizzate è quindi nulla.
%Il rapporto tra i moduli delle densità di corrente di spostamento e di conduzione vale
%\[
%    \frac{\epsilon\pdv{E}{t}}{J}=\frac{\abs{j\omega\epsilon E}}{\abs{\sigma e}}=\frac{\omega\epsilon}{\sigma}
%\]
%Sperimentalmente nei buoni conduttori $\omega\epsilon<<\sigma$ e quindi la corrente di spostamento può essere
%trascurata. Alla luce di quanto detto le equazioni di Maxwell utili al fine di caratterizzare il campo nel
%conduttore sono la terza e la quarta, che assumono la forma
%\[
%    \begin{split}
%        & \curl{\vb{E}}=-\pdv{\vb{B}}{t} \\
%        & \curl{\vb{B}}=-\mu\sigma \vb{E} \\
%    \end{split}
%\]
%Applicando il rotore alla prima di queste si ottiene: per il primo membro
%$\curl{\curl{\vb{E}}}=-\laplacian{\vb{E}}+\grad(\div{\vb{E}})=-\laplacian{\vb{E}}+\grad(\div{\vb{J}})/\sigma=-\laplacian{\vb{E}}$
%dove si è fatto uso del fatto che, essendo
%$\rho=0$, per l'equazione di continutità $\div{\vb{J}}=0$; per il secondo membro, usando la quarta equazione di Maxwell
%$-\pdv*{\curl{\vb{B}}}{t}=-\pdv*{\mu\sigma\vb{E}}{t}$.
%L'equazione per il campo elettrico diventa pertanto, ricordando che il campo elettrico ha solo componente $y$ dipendente da $z$(PERCHÈ??)
%\[
%    \begin{split}
%        & \pdv[2]{E_{0y}(z)e^{j\omega t}}{z}=\mu\sigma\pdv{E_{0y}(z)e^{j\omega t}}{t}\\
%        & \dv{E_{0y}(z)}{z}=j\omega\mu\sigma E_{0y}(z)\\
%    \end{split}
%\]
%che ha come soluzione
%\[
%    E_{0y}=Ae^{\alpha_1 z}+Be^{\alpha_2 z}
%\]
%con $\alpha_{1,2}$ soluzioni dell'equazione $\alpha^2-j\omega\mu\sigma=0$, ovvero $\alpha_{1,2}=\pm \beta(1+j)$
%(con $\beta=\sqrt{\omega\sigma\mu/s}$).
%La soluzione assume quindi la forma
%\[
%    E_{0y}=Ae^{\beta z}e^{j\beta z}+Be^{\beta z}e^{-j\beta z}
%\]
%Si osservi come per $z\to -\infty$ il secondo addendo diverga, portando ad una soluzione fisicamente inaccettabile:
%bisogna quindi imporre $B=0$.
%In conclusione si ottiene
%\[
%    E_y(z,t)=Ae^{\beta z}e^{j(\beta z +\omega t)}
%\]
%ovvero il campo elettrico (e quindi la densità di corrente, per la legge di Ohm locale) decresce esponenzialmente con
%la profondità ($-z$). La profondità caratteristica di penetrazione vale $1/\beta$.


\section{Onde stazionarie}
Quanto visto porta ad un fenomeno analogo a quello delle onde stazionarie meccaniche.
Si consideri un'onda piana che si propaga con verso positivo rispetto all'asse $z$ e incide su un
conduttore perfetto (ovvero con conducibilità infinita) posto in $z=0$.
Verranno ora trattati campo elettrico e magnetico separatamente, cominciando col campo elettrico.
L'onda è polarizzata linearmente e
il campo elettrico può essere immaginato lungo la direzione $x$. Per l'ipotesi di conducibilità
infinita è trascurabile la parte di campo che penetra nel conduttore, ma siccome il campo elettrico
è tangente alla superficie di separazione per la condizione di continuità della componente tangente
$E=0$ immediatamente fuori dal conduttore. Nel semispazio vuoto sono presenti
sia presente l'onda incidente\footnote{Siccome il campo è diretto solo lungo la direzione $x$ l'equazione
può essere posta in forma scalare.} $E_i=E_{(+)}e^{i(kz-\omega t)}$
che l'onda riflessa $E_r=E_{(-)}e^{i(kz+\omega t)}$ che si sovrappongono.
Dato che il conduttore è perfetto le due onde hanno uguale ampiezza ma segno opposto.
Il campo elettrico nello spazio vuoto vale:
\[
    \begin{split}
        E_x=&\Re\bigl(E_i+E_r\bigr)=\Re\bigl(E_{(+)}e^{jkz}(e^{-j\omega t}-e^{j\omega t})\bigr)=\\
        =&\Re\bigl(-2jE_{(+)}\sin(\omega t)e^{jkz}\bigr)=2E_{(+)}\sin(kz)\sin(\omega t)
    \end{split}
\]
Questa non è un'onda perchè non ha $kz\pm\omega t$ come argomento: è un'oscillazione armonica di pulsazione
$\omega$ con ampiezza $2E_{(+)}\sin(kz)$
che presenta dei nodi fissi per i valori di $z$ tali da annullare il seno. La distanza fra due
nodi adiacenti vale
\[
    \Delta z=\frac{\pi}{k}=\frac{\pi}{2\pi}\lambda=\frac{\lambda}{2}
\]

Per quanto concerne il campo magnetico invece, per le condizioni di ortogonalità questo ha solo componente $y$.
Siccome $E/H=\pm Z$, dove si ha segno positivo per onde progressive e segno negativo per onde regressive,
\[
    \begin{split}
        H=&\Re(\frac{E_{+}}{Z}e^{i(kz-\omega t)}+\frac{-E_{(+)}}{-Z}e^{i(kz+\omega t)})=\Re(2\frac{E_{(+)}}{Z}e^{jkz}\cos(\omega t))\\
        =&2\frac{E_{(+)}}{Z}\cos(kz)\cos(\omega t)
    \end{split}
\]
Il campo magnetico è perciò sfasato di $\pi/2$ rispetto al campo elettrico. Inoltre, quando uno dei due campi è
nullo l'altro è massimo e vale il doppio del campo elettromagnetico totale. Questo risultato non è in contraddizione
col fatto che in un'onda elettromagnetica $E$ ed $H$ sono in fase, perchè un'onda stazionaria è il risultato
della sovrapposizione di due onde che viaggiano in versi opposti.

Sulla base di quanto detto si può capire cosa succede qualora venga lanciata un'onda elettromagnetica
fra due piani conduttori paralleli posti a distanza $L$, in modo tale che incida ortogonalmente sui due piani.
Siccome il campo elettrico si deve annullare sui due piani, ovvero deve avere dei nodi sui due piani, la
lunghezza $L$ deve contenere un numero intero di lunghezze d'onda
\[
    L=\frac{\lambda}{2}n
\]


\section{Momenti di un'onda elettromagnetica}
In questo paragrafo $n=\dv*{N}{\tau}$ torna ad indicare il numero di cariche per unità di volume e non
l'indice di rifrazione.

Si consideri un'onda elettromagnetica incidente su un materiale. Per quanto visto nel
paragrafo sul vettore di Poynting il campo elettromagnetico esercita una forza sulla
superficie. Dalla \eqref{eqn:potenza_em} si ha che la potenza assorbita dal materiale per
unità di volume vale $w=\dv*{P}{\tau}=\vb{E} \vdot \vb{J}$.

\begin{obs}
    \[
        \vb{J}=\sigma\vb{E}
    \]
    con $\sigma$ complesso. Quindi $\vb{J}$ ha lo stesso andamento di $\vb{E}$, ampiezza proporzionale
    all'ampiezza di $\vb{E}$ ed è sfasato rispeto ad $\vb{E}$ della fase del coefficiente $\sigma$.
\end{obs}
\begin{proof}
    Se il materiale è un conduttore la tesi è evidente sulla base della relazione che lega $\vb{E}$ e $\vb{J}$
    nei conduttori. In linea del tutto generale, sotto l'effetto del campo oscillante le cariche si comportano
    come oscillatori smorzati che compiono un moto oscillatorio forzato: per quanto visto nella dimostrazione dell'osservazione
    \ref{obs:r_dielettrici} $\dot{\vb{r}}$ è proporzionale al campo elettrico mediante un coefficiente complesso
    e dunque lo è anche la sua derivata -e di conseguenza, la velocità di deriva.
    Ma siccome $\vb{J}=nq\vb{v}_d$, si ha la tesi.
\end{proof}
Alla luce di questo risultato,
\[
w=\sigma E^2
\]
Il valor medio della potenza su un periodo, detto $E_0$ l'ampiezza del campo elettrico
e $\alpha$ la fase di $\sigma$, vale quindi
\[
    [w]=\frac{E_0^2}{2}\abs{\sigma}\cos{\alpha}
\]

\begin{thm}
    La quantità di moto che l'onda trasferisce nell'unità di tempo all'untià di volume del materiale è diretta
    come la velocità di propagazione dell'onda e vale
    \[
        [\vb{q}]=\frac{[w]}{v}\vu{v}
    \]
\end{thm}
\begin{proof}
    La quantità di moto trasferita nell'unità di tempo all'unità di volume è data dall'impulso trasferito nell'unità di tempo,
    cioè la media temporale della forza impressa
    per untià di volume. Dal paragrafo sul vettore di Poynting si ha che questa forza vale
    \[
        \vb{f}=\dv{\vb{F}}{\tau}=nq(\vb{E}+\vb{v}_d\cp\vb{B})
    \]
    Siccome la media temporale del campo elettrico è nulla
    \[
        [\vb{q}]\equiv [\vb{f}]=[nq\vb{v}_d\cp\vb{B}]=[\vb{J}\cp\vb{B}]
    \]
    Dato che campo elettrico e magnetico sono ortogonali e $\vb{J}$ è parallelo a $\vb{E}$
    \[
        [\vb{q}]=[JB]\vu{v}
    \]
    Tenendo conto della relazione fra i moduli di $B$ ed $E$
    \[
        [\vb{q}]=\frac{[\vb{J\vdot\vb{E}}]}{v}\vu{v}
    \]
    Ovvero la tesi.
\end{proof}

Qualora ci si trovi nel caso di assorbimento totale, ovvero nel caso in cui l'onda trasferisca al materiale tutta la sua energia,
conviene far riferimento all'energia incidente sull'unità di superficie nell'unità di tempo, ovvero al modulo del vettore
di Poynting.
In questo caso al posto della potenza per unità di volume  si avrà l'intensità, ovvero potenza per unità di superficie,
e al posto della quantita di moto per unità di volume e unità di tempo
si avrà la quantità di moto trasferità all'unità di superficie normale all'onda nell'unità di tempo $\vb{p}$. Perciò
\[
    \vb{p}=\frac{I}{v}\vu{v}=\frac{\vb{I}}{v}
\]
Quindi, dalla definizione di vettore di Poynting
\begin{equation}
    \vb{p}=\frac{\vb{E}\cp\vb{B}}{\mu v}
\end{equation}
$p$ ha le dimensioni di una forza diviso una superficie ed è quindi effettivamente una pressione. La stessa pressione, in verso
opposto la subisce una sorgente che emetta un'onda con intensità $I$. Una superficie perfettamente riflettente investita
ortogonalmente dall'onda subisce una pressione doppia.

Oltre alla quantità di moto le onde elettromagnetiche trasportando anche un momento angolare. Dato un polo $O$ è evidente
infatti che le onde trasportino il momento
\[
    \vb{L}=\int_S\vb{r}\cp\dd{\vb{p}}
\]
dove $\vb{r}$ è la distanza fra il polo ed il punto dell'onda con momento $\dd{\vb{p}}$.
Nel caso di un'onda opportunamente collimata questa espressione si semplifica a $\vb{L}=\vb{r}\cp\vb{p}$.
Inoltre, la radiazione elettromagnetica possiede un momento angolare intrinseco quando è
polarizzata circolarmente, ovvero quando il campo elettrico ruota attorno alla direzione di
propagazione. Il momento angolare intrinseco vale
\[
    \vb{S}=\pm\omega\vb{I}
\]
dove il segno dipende dal verso di polarizzazione, destrorsa o sinistrorsa. Il momento angolare intrinseco è
longitudinale, ovvero diretto come la velocità dell'onda.
Un'onda con polarizzazione lineare, in cui il campo elettrico oscilli su un piano fisso, non possiede momento
angolare intrinseco, in quanto può essere ottenuta come sovrapposizione di due onde identiche con polarizzazione
circolare in un caso destrorsa e nell'altro sinistrorsa.


\section{Leggi di riflessione e rifrazione}
Uno dei grandi successi della teoria dell'elettromagnetismo di Maxwell è
la possibilità di dedurre le leggi dell'ottica.
Le leggi di riflessione e rifrazione in particolare, derivano dalle condizioni al
contorno fra due materiali.
Suppongo di avere a che fare con un'onda piana (il che è ragionevole nel caso
in cui la sorgente sia molto lontana) che incida sulla superficie di
separazione fra due materiali con $\epsilon_1$, $mu_1$ ed $epsilon_2$, $\mu_2$.
Per la \eqref{eqn:B_onde} noto il campo elettrico è noto anche il campo magnetico:
per le successive considerazioni ci si può quindi limitare a considerare il campo
elettrico. In generale, si avrà a che fare con tre diversi campi elettrici
\[
\begin{split}
&\vb{E}=\vb{E}_0 \exp {i(\vb{k}\vdot{r}-\omega t +\phi)} \quad\quad \text{(incidente)}\\
&\vb{E}'=\vb{E}'_0 \exp {i(\vb{k}'\vdot{r}-\omega' t +\phi')} \quad\quad \text{(riflesso)}\\
&\vb{E}''=\vb{E}''_0 \exp {i(\vb{k}''\vdot{r}-\omega'' t +\phi'')} \quad\quad \text{(rifratto)}\\
\end{split}
\]
Di questi, solo il campo incidente è noto.

Si immagini senza perdita di generalità che la propagazione avvenga solo nel piano $yz$ e si
ponga lo $0$ dell'asse $z$ al livello della superficie di separazione fra i due mezzi.
La condizione di raccordo per il campo elettrico $E_{t1}=E_{t2}$ implica l'uguaglianza
dei moduli e delle fasi dei campi incidente, rifletto e rifratto. L'uguaglianza delle fasi
permette di dedurre le leggi di riflessione e rifrazione; l'uguaglianza dei moduli,
oltre a queste due leggi, consente di ricavare informazioni sulla polarizzazione dell'onda
a scapito di un calcolo più lungo e tedioso. Ci si concentrerà solo sull'uguaglianza delle
fasi. Si ha dunque
\[
\vb{k}\vdot{r}+\omega t+\phi=\vb{k}'\vdot{r}+\omega' t+\phi'=\vb{k}''\vdot{r}+\omega'' t+\phi''
\]
Si tratta di un'uguaglianza fra polinomi nelle variabili $y,z,t$. L'uguaglianza è soddisfatta
per ogni valore di $y,z,t$ se e solo se sono uguali i coefficienti.
Banalmente $\phi=\phi'=\phi''$ e $\omega=\omega'=\omega''$: non si hanno sfasamenti
nel passaggio fra i mezzi e la pulsazione dipende esclusivamente dalla sorgente.
Il discorso è solo leggermente più complesso per l'uguaglianza
$\vb{k}\vdot\vb{r}=\vb{k}'\vdot\vb{r}=\vb{k}''\vdot\vb{r}$.
Per il sistema di riferimento scelto $k_x=0$ e $z=0$, da cui segue
$k_y y=k_x'x+k_y'y=k_x''x+k_y''y$. Uguagliando nuovamente termine a termine e
chiamando $\theta$, $\theta'$, $\theta''$ l'angolo
che i tre raggi formano con la perpendicolare alla superficie di separazione si ha
\[
\begin{split}
&0=k_x'=k_x''\\
&k y\sin{\theta}=k'y\sin{\theta'}=k''\sin{\theta''}\\
\end{split}
\]
La prima equazione afferma che se il raggio incidente propaga nel piano $yz$, anche i raggi
riflesso e rifratto propagano nello stesso piano. Per quanto concerne la seconda si osservi che
\[
\frac{n}{c}v=\lambda\nu=\lambda\frac{\omega}{2\pi}=\rec{k}\omega
\]
Per quanto visto $\omega$ non dipende dal mezzo, mentre $n$ dipende dal mezzo per definizione:
anche $k$ deve dipendere dal mezzo. Si ottiene quindi
\[
\begin{split}
&n_1 y \sin{\theta}=n_2 y \sin{\theta'}\\
&n_1 y \sin{\theta}=n_1 y \sin{\theta''}\\
\end{split}
\]
Dalla prima segue la nota legge di rifrazione $n_1\sin{\theta}=n_2\sin{\theta'}$;
la legge di riflessione $\theta=\theta''$


\section{Spettro della radiazione elettromagnetica}
A seconda della loro frequenza, le onde elettromagnetiche sono prodotte da tipi diversi di sorgente,
hanno proprietà diverse e diversi modi di interagire con la materia.

$10^3-10^9 Hz$ - \textit{Onde a radiofrequenza}.
Usate per le comunicazioni, sono prodotte da circuiti oscillanti accoppiati ad antenne.

$10^9-10^{11} Hz$ - \textit{Microonde}.
Usate per lo studio di strutture atomiche o molecolari e nelle comunicazioni, sono prodotte da circuiti oscillanti associati
a dispositivi meccanici come cavità risonanti  o guide d'onda.

$5\vdot 10^{11}-4\vdot 10^{14} Hz$ - \textit{Infrarossi}.
Viene ulteriormente divisa in \textit{lontano, medio, vicino infrarosso}. È spontaneamente emessa
dai corpi caldi.

$4\vdot 10^{14}-8\vdot 10^{14} Hz$ - \textit{Radiazione visibile}.
Viene emessa da atomi e molecole quando i relativi elettroni compiono una transizione da uno
stato eccitato allo stato fondamentale.

$8\vdot 10^{14}-3\vdot 10^{17} Hz$ - \textit{Radiazione ultravioletta}.
Anche questa è emessa da atomi e molecole, in particolare nei gas sottoposti ad una scarica elettrica.

$3\vdot 10^{17}-5\vdot 10^{19} Hz$ - \textit{Raggi X}.
Sono generati da cariche che subiscono una forte accelerazione, come un raggio catodico che viene
bruscamente fermato dall'impatto con un materiale soldio. La radiazione emessa (costituita non solo da
raggi X) prende il nome di \textit{bremsstrahlung}. Molte stelle o ammassi stellari sono sorgenti di
questo tipo di radiazione. Trovano applicazione in radiochimica e medicina grazie al diverso assorbimento
ad opera di materiali con diversa densità e consistenza.

$- 10^{18} Hz$ - \textit{Raggi gamma}.
La loro emissione si accompagna a molti processi nucleari. A queste frequenze la descrizione
dell'interazione fra campo elettromagnetico e materia non può prescindere dalla meccanica
quantistica.


\section{Effetto Doppler}
Nel caso delle onde sonore, se la sorgente, il mezzo materiale in cui si propaga l'onda
e l'osservatore sono in moto relativo si manifesta l'effetto Doppler: se l'osservatore si sta avvicindando alla sorgente
vede arrivare il suono a velocità maggiore e quindi incontra un numero maggiore di fronti d'onda a parità
di intervallo di tempo e quindi osserva una frequenza maggiore; viceversa se è in allontanamento.

Nel caso delle onde elettromagnetiche, a causa del principio di costanza della velocità della luce e del
fatto che non si propagano in un mezzo, ci si aspetta che l'effetto Doppler non sussista. In realtà,
a partire da considerazioni di carattere relativistico si può mostrare che effettivamente l'effetto
si manifesta anche per le onde elettromagnetiche. In particolari, per velocità molto minori della
velocità della luce la formula è analoga a quella dell'effetto Doppler acustico
\[
\nu'=\nu \Biggl(1\pm\abs{\frac{V}{c}}\Biggr)
\]
dove $\nu'$ e $\nu$ sono rispettivamente la frequenza osservata e la frequenza emessa dalla sorgente, $V$ è
la velocità relativa fra osservatore e sorgente e il segno deve essere scelto a seconda che
sorgente e osservatore siano in avvicinamento o in allontanamento.






\appendix
\chapter{Onde}
\section{Definizioni}
\begin{defn}[Onda]
    Si definisce onda una perturbazione che nasce da una sorgente e si propaga nel tempo e
    nello spazio.
\end{defn}

\begin{obs}
    Un'onda di ampiezza costante è descritta da una funzione che goda della proprietà
    \[
        f(x,t)=f(\xi(x,t))\equiv f(\xi)
    \]
    con $\xi(x,t)=x\mp vt$, $v$ costante positiva.
\end{obs}
\begin{proof}
    Se si considera la funzione $f(\xi)$ questa ha un ben definito profilo che rappresenta la perturbazione
    generata dalla sorgente. Questo profilo corrisponde al profilo di $f(x)$, con $t$ fissato.
    Si consideri un certo valore $\xi_0=x_0\mp v t_0$. Ci si chiede per quale valore $x_0+\Delta x$
    all'istante $t_0+\Delta t$ si abbia ancora il valore $\xi_0$. Si deve quindi risolvere l'equazione
    $x_0 \mp v t_0 = (x_0+\Delta x) \mp v(t_0+\Delta t)$ che porta alla condizione $\Delta x \mp \Delta t=0$
    ovvero $\Delta x / \Delta t = \mp v$: $f(\xi(x,t))$ rappresenta dunque una perturbazione che viaggia nello
    spazio con velocità $v$.
\end{proof}
$\xi$ viene detto fase dell'onda, $v$ è la velocità con cui si muove la fase dell'onda: l'onda resta costante nel tempo
se la si osserva da un sistema di riferimento con velocità $v$. Per questo motivo $v$ è detta velocità di fase.
L'onda si dice positiva o regressiva
a seconda del segno che compare nell'espressione di $\xi$.

\begin{defn}[Fronte d'onda]
    Si definisce fronte d'onda il luogo dei punti in cui $\xi$ assume lo stesso valore.
\end{defn}
Si parla di  onde piane, sferiche etc... in base alla forma dei fronti d'onda.

\begin{defn}[Onde periodiche]
    Si parla di onda periodica quando si ha a che fare con un'onda descritta da una
    funzione periodica in $\xi$.
\end{defn}
Una classe particolarmente importante di onde periodiche sono le onde sinusoidali.
\begin{defn}[Onde sinusoidali]
    Si definiscono onde sinusoidali le onde periodiche in cui $f(\xi)$ assume una delle seguenti
    espressioni fra loro equivalenti
    \[
        \begin{split}
            & A\sin\Biggl[\frac{2\pi}{\lambda}(x-vt)+\phi\Biggr]\\
            & A\sin\Biggl[\frac{2\pi}{T}\Biggl(\frac{x}{v}-t\Biggr)+\phi\Biggr]\\
            & A\sin\Biggl[2\pi\Biggl(\frac{x}{\lambda}-\frac{t}{T}\Biggr)+\phi\Biggr]\\
            & A\sin(kx-\omega t + \phi)\\
        \end{split}
    \]
    dove:
    \begin{itemize}
            \item $A$: ampiezza
            \item $\phi$: fase iniziale
            \item $\lambda$: lunghezza d'onda, distanza fra due picchi
            \item T: periodo, tempo necessario affinchè nel punto $x$ fissato l'onda assuma nuovamente lo stesso valore
            \item $\omega$: pulsazione, $\omega=2\pi/T$
            \item $k$: numero d'onda, $\k=2\pi/\lambda$
            \item $v=\lambda/T=\omega/k$
    \end{itemize}
\end{defn}


\section{Equazioni delle onde}
Le equazioni differenziali che danno come soluzione un'onda sono nella forma $\square\vb{F}=0$,
dove $\square$ è l'operatore dalembertiano
\begin{equation}
    \label{app:eqn:dalembertiano}
    \square=\laplacian - \rec{v^2}\pdv[2]{t}= \pdv[2]{x} + \pdv[2]{y} + \pdv[2]{z} - \rec{v^2}\pdv[2]{t}
\end{equation}
Chiaramente il dalemebertiano è un'operatore lineare e dunque le equazioni delle
onde sono equazioni differenziali lineari e omogenee. Vale perciò il principio di sovrapposizione:
ogni combinazione lineare di soluzioni dell'equazione delle onde è anch'essa una soluzione dell'equazione
delle onde.

Le soluzioni possono essere risolte sviluppando le soluzioni in serie di Fourier: per il principio di
sovrapposizione non si perde di generalità limitandosi a considerare solo soluzioni sinusoidali.



\chapter{Operazioni}
\section{Operazioni vettoriali}
\begin{equation}
  \label{app:eqn:vdot_cp}
  \vb{a}\vdot(\vb{b}\cp\vb{c})=\vb{c}\vdot(\vb{a}\cp\vb{b})=\vb{b}\vdot(\vb{c}\cp\vb{a})
\end{equation}


\begin{equation}
  \label{app:eqn:cp_cp}
  \vb{a}\cp(\vb{b}\cp\vb{c})=(\vb{a}\vdot\vb{c})\vb{b}-(\vb{a}\vdot\vb{b})\vb{c}
\end{equation}


\begin{equation}
  \label{app:eqn:cp_vdot_cp}
  (\vb{a}\cp\vb{b})\vdot(\vb{c}\cp\vb{d})=(\vb{a}\vdot\vb{c})(\vb{b}\vdot\vb{d})-(\vb{a}\vdot\vb{d})(\vb{b}\vdot\vb{c})
\end{equation}


\section{Operatori vettoriali}
\begin{equation}
\label{app:eqn:curl_gradiente}
    \curl{\grad{f}}=0
\end{equation}

\begin{equation}
\label{app:eqn:div_curl}
    \div(\curl{f})=0
\end{equation}

\begin{equation}
\label{app:eqn:curl_curl}
  \curl\curl{\vb{A}}=-\laplacian{A}+\grad(\div{\vb{A}})
\end{equation}

\begin{equation}
\label{app:eqn:div_scalare_vettore}
    \div(f\vb{v})=f\div{\vb{v}}+\grad{f}\vdot\vb{v}
\end{equation}

\begin{equation}
\label{app:eqn:curl_scalare_vettore}
    \curl(f\vb{v})=f\curl{\vb{v}}+\grad{f}\cp\vb{v}
\end{equation}

\begin{equation}
\label{app:eqn:div_cp}
  \div(\vb{u}\cp\vb{v})=(\curl{\vb{u}})\vdot\vb{v}-\vb{u}\vdot(\curl{\vb{v}})
\end{equation}


\section{Relazioni integrali}

\begin{equation}
\label{app:eqn:teo_div}
    \int_\tau \div{\vb{v}}\dd{\tau}=\int_S \vb{v}\vdot\dd{\vb{S}}
\end{equation}

\begin{equation}
\label{app:eqn:teo_stokes}
    \int_S \curl{f}\vdot\dd{\vb{S}}=\oint_l \vb{v}\vdot\dd{\vb{l}}
\end{equation}

\begin{equation}
    \int_S \grad{f}\cp\dd{\vb{S}}=-\oint_l f\dd{\vb{l}}
\end{equation}

\begin{equation}
\label{app:eqn:seconda_green}
    \int_\tau \curl{\vb{v}}\dd{\tau}=-\int_S \vb{v}\cp\dd{\vb{S}}
\end{equation}


\section{Proprietà operazioni sul raggio vettore}
\begin{equation}
  \label{app:eqn:grad_r}
    \frac{\vb{r}-\vb{r}'}{\abs{\vb{r}-\vb{r}'}^3}=-\grad[\rec{\abs{\vb{r}-\vb{r'}}}]=\grad'[\rec{\abs{\vb{r}-\vb{r'}}}]
\end{equation}

\begin{equation}
  \label{app:eqn:laplacian_r}
    \laplacian{\vb{r}-\vb{r}'}=4\pi\delta^3(\vb{r}-\vb{r}')
\end{equation}



\chapter{Unità di misura e costanti}
\section{Capitolo 1}
\subsection{Unità di misura}
Carica elettrica
    \begin{equation} [Q] = C = A \vdot s \label{}\end{equation}
Costante dielettrica del vuoto
    \begin{equation} [\epsilon_0] = \frac{C^2}{Nm^2} = \frac{F}{m} \label{}\end{equation}
Campo elettrico
    \begin{equation} [E] = \frac{N}{C} = \frac{V}{m} \label{}\end{equation}
Potenziale elettrico
    \begin{equation} [V] = V = \frac{J}{C} \label{}\end{equation}
Capacità
    \begin{equation} [C] = F = \frac{C}{V} \label{}\end{equation}

\subsection{Costanti}
$\epsilon_0$ - Costante dielettrica del vuoto
    \begin{equation} \epsilon_0 \simeq \rec{4\pi} \rec{9} 10^{-9} \frac{C^2}{Nm^2} \label{}\end{equation}


\section{Capitolo 2}
\subsection{Unità di misura}

    \begin{equation}  =  \label{}\end{equation}


\subsection{Costanti}

    \begin{equation}  =  \label{}\end{equation}


\section{Capitolo 3}
\subsection{Unità di misura}

    \begin{equation}  =  \label{}\end{equation}


\subsection{Costanti}

    \begin{equation}  =  \label{}\end{equation}


\section{Capitolo 4}
\subsection{Unità di misura}

    \begin{equation}  =  \label{}\end{equation}


\subsection{Costanti}

    \begin{equation}  =  \label{}\end{equation}


\section{Capitolo 5}
\subsection{Unità di misura}
Campo di induzione magnetica
    \begin{equation} [B]  = T \label{}\end{equation}

Potenziale vettore
    \begin{equation} [A]  = T \vdot m \label{}\end{equation}


\subsection{Costanti}
$\mu_0$ - Suscettività magnetica del vuoto
    \begin{equation} \mu_0 = 4\pi 10^{-7} \frac{N}{A^2} \label{}\end{equation}


\section{Capitolo 6}
\subsection{Unità di misura}
Polarizzazione magnetica
    \begin{equation} [M] = \frac{A}{m} \label{}\end{equation}


\subsection{Costanti}

    \begin{equation}  =  \label{}\end{equation}


\section{Capitolo 7}
\subsection{Unità di misura}

    \begin{equation}  =  \label{}\end{equation}


\subsection{Costanti}

    \begin{equation}  =  \label{}\end{equation}


\section{Capitolo 8}
\subsection{Unità di misura}

    \begin{equation}  =  \label{}\end{equation}


\subsection{Costanti}

    \begin{equation}  =  \label{}\end{equation}


\section{Capitolo 9}
\subsection{Unità di misura}

    \begin{equation}  =  \label{}\end{equation}


\subsection{Costanti}

    \begin{equation}  =  \label{}\end{equation}




\end{document}
